\documentclass[11pt]{article}

\usepackage{amsfonts}
\usepackage{amsmath,accents,amsfonts, amssymb, mathrsfs }
\usepackage{tikz-cd}
\usepackage{graphicx}
\usepackage{syntonly}
\usepackage{color}
\usepackage{mathrsfs}
\usepackage[spanish]{babel}
\usepackage[latin1]{inputenc}
\usepackage{fancyhdr}
\usepackage[all]{xy}
\usepackage[at]{easylist}
\usepackage[colorlinks=true,linkcolor=blue,urlcolor=black,bookmarksopen=true]{hyperref}

\usepackage{bookmark}

\topmargin-2cm \oddsidemargin-1cm \evensidemargin-1cm \textwidth18cm
\textheight25cm


\newcommand{\B}{\mathcal{B}}
\newcommand{\Cont}{\mathcal{C}}
\newcommand{\F}{\mathcal{F}}
\newcommand{\inte}{\mathrm{int}}
\newcommand{\A}{\mathcal{A}}
\newcommand{\C}{\mathbb{C}}
\newcommand{\Q}{\mathbb{Q}}
\newcommand{\Z}{\mathbb{Z}}
\newcommand{\inc}{\hookrightarrow}
\renewcommand{\P}{\mathcal{P}}
\newcommand{\R}{{\mathbb{R}}}
\newcommand{\N}{{\mathbb{N}}}
\newcommand\tq{~:~}
\newcommand{\dual}[1]{\left(#1\right)^{\ast}}
\newcommand{\ortogonal}[1]{\left(#1\right)^{\perp}}
\newcommand{\ddual}[1]{\left(#1^{\ast}\right)^{\ast}}
\newcommand{\x}[3]{#1_#2^#3}
\newcommand{\xx}[4]{#1_#3#2_#4}
\newcommand\dd{\,\mathrm{d}}
\newcommand{\norm}[1]{\left\lVert#1\right\rVert}
\newcommand{\abs}[1]{\left\lvert#1\right\rvert}
\newcommand{\ip}[1]{\left\langle#1\right\rangle}
\renewcommand\tt{\mathbf{t}}
\newcommand\nn{\mathbf{n}}
\newcommand\bb{\mathbf{b}}                      % binormal
\newcommand\kk{\kappa}
\newcommand{\sett}[1]{\left\lbrace#1\right\rbrace}
\newcommand{\interior}[1]{\accentset{\smash{\raisebox{-0.12ex}{$\scriptstyle\circ$}}}{#1}\rule{0pt}{2.3ex}}
\fboxrule0.0001pt \fboxsep0pt
\newcommand{\Bigcup}[2]{\bigcup\limits_{#1}{#2}}
\newcommand{\Bigcap}[2]{\bigcap\limits_{#1}{#2}}
\newcommand{\Bigprod}[2]{\prod\limits_{#1}{#2}}
\newcommand{\Bigcoprod}[2]{\coprod\limits_{#1}{#2}}
\newcommand{\Bigsum}[2]{\sum\limits_{#1}{#2}}
\newcommand{\BigsumA}[3]{ \sideset{}{^#2}\sum\limits_{#1}{#3}}
\newcommand{\Biglim}[2]{\lim\limits_{#1}{#2}}
\newcommand{\quotient}[2]{{\raisebox{.2em}{$#1$}\left/\raisebox{-.2em}{$#2$}\right.}}
\newcommand{\derivation}[1]{\left(#1\right)^\prime}
\newcommand{\constants}[1]{#1_C}
\DeclareMathOperator{\rank}{ran}
\DeclareMathOperator{\graf}{Gr}
\DeclareMathOperator{\ball}{ball}

\def \le{\leqslant}	
\def \ge{\geqslant}
\def\noi{\noindent}
\def\sm{\smallskip}
\def\ms{\medskip}
\def\bs{\bigskip}
\def \be{\begin{enumerate}}
	\def \en{\end{enumerate}}
\def\deck{{\rm Deck}}
\def\Tau{{\rm T}}

\newtheorem{mytheorem}{Theorem}
 %

\newtheorem{theorem}{Teorema}
\numberwithin{theorem}{subsection}
\newtheorem{lemma}[theorem]{Lema}

\newtheorem{proposition}[theorem]{Proposici\'on}

\newtheorem{corollary}[theorem]{Corolario}


\newenvironment{proof}[1][Demostraci\'on]{\begin{trivlist}
		\item[\hskip \labelsep {\bfseries #1}]}{\end{trivlist}}
\newenvironment{definition}[1][Definici\'on]{\begin{trivlist}
		\item[\hskip \labelsep {\bfseries #1}]}{\end{trivlist}}
\newenvironment{example}[1][Ejemplo]{\begin{trivlist}
		\item[\hskip \labelsep {\bfseries #1 }]}{\end{trivlist}}
\newenvironment{remark}[1][Observaci\'on]{\begin{trivlist}
		\item[\hskip \labelsep {\bfseries #1}]}{\end{trivlist}}
\newenvironment{declaration}[1][Afirmaci\'on]{\begin{trivlist}
		\item[\hskip \labelsep {\bfseries #1}]}{\end{trivlist}}


\newcommand{\qed}{\nobreak \ifvmode \relax \else
	\ifdim\lastskip<1.5em \hskip-\lastskip
	\hskip1.5em plus0em minus0.5em \fi \nobreak
	\vrule height0.75em width0.5em depth0.25em\fi}

\newcommand{\twopartdef}[4]
{
	\left\{
	\begin{array}{ll}
		#1 & \mbox{ } #2 \\
		#3 & \mbox{ } #4
	\end{array}
	\right.
}

\newcommand{\threepartdef}[6]
{
	\left\{
	\begin{array}{lll}
		#1 & \mbox{ } #2 \\
		#3 & \mbox{ } #4 \\
		#5 & \mbox{ } #6
	\end{array}
	\right.
}

\tikzset{commutative diagrams/.cd,
	mysymbol/.style={start anchor=center,end anchor=center,draw=none}
}
\newcommand\Center[2]{%
	\arrow[mysymbol]{#2}[description]{#1}}

\newcommand*\circled[1]{\tikz[baseline=(char.base)]{
		\node[shape=circle,draw,inner sep=2pt] (char) {#1};}}


\makeatletter
\newcommand{\xRightarrow}[2][]{\ext@arrow 0359\Rightarrowfill@{#1}{#2}}
\makeatother


\begin{document}
	
	\pagestyle{empty}
	\pagestyle{fancy}
	\fancyfoot[CO]{\slshape \thepage}
	\renewcommand{\headrulewidth}{0pt}
	
	
	
	\centerline{\bf \'Algebra 3}
	\centerline{\sc Final}
	\centerline{\sc Axel Sirota}
	
	
\section{Algebra Diferencial}

Sea $R$ un anillo conmutativo y $\delta: R \mapsto R$ un morfismo de grupos aditivo.

\begin{definition}
	Decimos que $\delta$ es una derivacion si para todo $r,s \in R$ vale:
	
	\begin{equation}
		\delta(rs) = (rs)\prime = rs' + sr'
	\end{equation}
	
\end{definition}

Notemos que por induccion vale que $\derivation{r^n} = nr^{n-1}\derivation{r}$ y que $\derivation{\frac{r}{s}} = \dfrac{r \derivation{s} - s\derivation{r}}{s^2}$. A un anillo $R$ con una derivacion le llamamos anillo diferencial y sus morfismos son los morfismos de anillos que conmutan con la derivacion.

Un ideal decimos que es un ideal diferencial si $\derivation{I} \subset I$.

\begin{remark}
	Sea $\phi$ un morfismo entonces si $x \in \ker \phi$ entonces $\phi\left(\derivation{x}_R\right) = \derivation{\phi(x)}_S = \derivation{0}_S = 0$, luego $\ker \phi$ es un ideal diferencial.
\end{remark}

Si $\derivation{r} = s$ decimos que $s$ es una derivada de $r$ y que $r$ es una primitiva de $s$

\begin{example}
	Ejemplos de anillos diferenciales son:
	
	\begin{enumerate}
		\item $\R[x]$ con $\delta = \dfrac{d}{dx}$ la derivacion usual
		\item $R$ anillo cualquiera y $\delta = 0$ la derivacion trivial
		\item $\C(x, \log x)$ con $\derivation{\log x} = \frac{1}{x}$
	\end{enumerate}

	Ejemplos de ideales diferenciales:
	
	\begin{enumerate}
		\item Si $K$ es cuerpo entonces el unico ideal diferencial no trivial es el nulo, pues si $I \subsetneq K[x]$ es ideal entonces existe $f$ tal que $I = \ip{f}$, luego $\deg\left(\derivation{f}\right) < \deg(f)$ entonces $\derivation{f} \not \in I$.
		\item Si $K$ es de caracteristica $p$ y perfecto entonces no admite derivacion no trivial. En efecto, si $\delta$ es una derivacion entonces sea $k \in K$ existe $a \in K$ tal que $k = a^p$, luego $\derivation{k} = p \derivation{a} a^{p-1} = 0$.
	\end{enumerate}

\end{example}

\section{Extensiones de anillos diferenciables}

Sea $R \subset S$ un subanillo tal que ambos son anillos diferenciables y supongamos que $\delta_S \vert_R = \delta_R$, luego decimos que $\quotient{R}{S}$ es una extension de anillos diferenciables.

\begin{definition}
	Definimos el subanillo de constantes de un anillo diferenciable $R$ como:
	
	\begin{equation}
		\constants{R} = \ker \delta_R
	\end{equation}
	
	Y notemos que trivialmente $\quotient{\constants{R}}{R}$ es una extension de anillos diferenciables. A su vez notemos que si $R$ es cuerpo entonces $\constants{R}$ incluye al cuerpo primo de $R$
	
\end{definition}

\begin{proposition}
	Sea $\quotient{R}{S}$ una extension de anillos diferenciables, entonces son equivalentes:
	
	\begin{itemize}
		\item $\constants{R} = \constants{S}$
		\item Dado $r \in R$, si existe $s \in R$ tal que $\derivation{s} =r$ entonces no existe $\tilde{s} \in S \setminus R$ tal que $\derivation{\tilde{s}} = r$
	\end{itemize}
	
	Y en ese caso decimos que la extension es sin nuevas constantes
	
\end{proposition}

\begin{proof}
	Si existen $s \in R$ y $t \in S \setminus R$ tal que $\derivation{s} = \derivation{t} = r$ entonces $\derivation{s-t} = 0$ por lo que $s-t \in \constants{S} \setminus \constants{R}$, absurdo.
	Por el otro lado, si $s \in \constants{S} \setminus \constants{R}$ entonces $0$ adminiute primitiva tanto en $R$ como $S \setminus R$. \qed
\end{proof}

\begin{example}
	Sea $K$ un cuerpo diferenciable, entonces $\quotient{\constants{K}}{K}$ es una extension sin nuevas constantes
\end{example}

\begin{example}
	Sea $z_0 \in U \subset \C$ un abierto y tomemos $K = \left(Mer(\C), \dfrac{d}{dz}\right)$ y $L = \left(Mer(U), \dfrac{d}{dz}\right)$, luego $\quotient{K}{L}$ es una extension sin nuevas constantes.
\end{example}

Ahora veamos los no ejemplos:

\begin{example}
	Sea $\quotient{\R[x]}{Hol(\C)}$ con la extensi\'on de la derivacion conocida de analisis complejo, luego $\constants{\R[x]} \simeq \R$ pero $i = \dfrac{\derivation{\exp ix}}{\exp ix} \in \constants{Hol(\C) \setminus \R[x]}$ por lo que $\constants{Hol(\C)} \simeq \C \not \R$ asiq ue esta extension agrega constantes.
\end{example}

\begin{example}
	Sea $\C(x, e^x, u)$ el cuerpo de funciones rac ionales en las tres componentes tal que $e^x, u$ son tracsendentes, donde ademas $\derivation{e^x} = e^x$ y $\derivation{u} =u$, luego $\derivation{\dfrac{e^x}{u}} =0$ por lo que si tomamos la extension $\quotient{\C(x)}{\C(x, e^x, u)}$ tenemos que es una extension que agrega constantes
\end{example}

Desde aca todos los anillos van a sewr cuerpos de caracteristica 0

\begin{proposition}
	Sea $\quotient{K}{L}$ una extension diferenciable sin nuevas constantes y $l \in L \setminus K$ tal que $\derivation{l} \in K$, luego:
	
	\begin{enumerate}
		\item $l$ es trascendente sobre $K$
		\item La derivada $\derivation{p(l)}$ de cualquier polinomio $p(l) \in K[l]$ tiene grado $n$ si y solo si $k_n \not \in \constants{K}$. En caso constrario tiene grado $n-1$.
	\end{enumerate}
	
\end{proposition}

\begin{proof}
	Sea $b = \derivation{l} \in K$, luego $b \neq 0$ por la hipotesis de nuevas constantes y supongamos que $l$ es algebraico. Sea $p = t^n + c_m t^m + \dots + c_0 \in K[t]$ el polinomio minimal de $l$ donde $m$ es el maximo indice tal que $c_i \neq 0$ que esta bien definido pues $l \not \in k$; entonces:
	
	\begin{equation}
	0 = \derivation{0} = \derivation{l^n + c_ml^m + \dots + c_0} = bnl^{n-1} + \derivation{c_m}l^m + bc_ml^{m-1} + \dots + \derivation{c_0}
	\end{equation}
	
	Separemos en lso tres casos correspondientes:
	
	\begin{itemize}
		\item[$n-1 > m$]
		
		Luego en este caso si dividimos por $bn$ obtenemos que $q = t^{n-1} + \frac{\derivation{c_m}}{bn} t^m + \frac{c_m}{n} t^{m-1} + \dots + \frac{\derivation{c_0}}{bn} \in K[t]$ anula a $l$ y tiene grado $n-1$, lo que contradice la minimalidad de $p$.
		
		\item[$n-1=m$ y $bn + \derivation{c_m} \neq 0$]
		
		Estamos en el mismo caso que antes pues dividimos por $bn + \derivation{c_m}$
		
		\item [$n-1=m$ y $bn + \derivation{c_m} = 0$]
		
		En este caso $0 = bn + \derivation{c_m} = \derivation{ln + c_m}$ por lo que  $ln + c_m \in \constants{L} \setminus \constants{K}$ lo que es absurdo. \qed
		
	\end{itemize}
	
	
\end{proof}

\begin{proposition}
	Sea $\quotient{K}{L}$ una extension diferenciable sin nuevas constantes tal que $l \in L \setminus K$ cumple $\frac{\derivation{l}}{l} \in K$, luego:
	
	\begin{itemize}
		\item $l$ es algebraico sobre $K$ si y solo si existe $n$ tal que $l^n \in K$
		\item Cuando $l$ es trascendente sobre $K$, para todo $p \in K[x]$ polinomio de grado $n$ con $k_n \neq 0$ vale que $\deg \derivation{p(l)} =n$ y $\derivation{p(l)} = K p(l)$ si y solo si  $p(l)$ es un monomio.
	\end{itemize}
\end{proposition}

\begin{proof}
	Notemos $b = \frac{\derivation{l}}{l} \in K$ y notemos nuevamente que $b \neq 0$.
	
	\begin{itemize}
		\item Si $l$ es algebraico y llamamos $p = t^n + \sum\limits_{i=0}^{m} c_it^i$ al polinomio minimal entonces nuevamente tenemos:
		
		\begin{equation}
			\begin{aligned}
			 0 = & l^n + c_ml^m + \dots + c_0 \\
			 0 = &  bnl^{n} + \left(\derivation{c_m} + bmc_m\right)l^m + \dots + \derivation{c_0}
			\end{aligned}
		\end{equation}
		
		De lo que concluimos que $q(l) = (bnc_m + \derivation{c_m} + bmc_m) l^m + \dots + bnc_0 + \derivation{c_0} = 0$ es un polinomio de grado menor que $n$ que lo anula; por la minimalidad de $p$ concluimos que $q = 0$. Como $\derivation{c_m} + bmc_m = bn c_m$ podemos llegar a que $\frac{\derivation{c_m}}{c_m} = (n-m)b$, luego:
		
		\begin{equation}
		\begin{aligned}
			\dfrac{\derivation{c_m l^{m-n}}}{c_m l^{m-n}} = & \dfrac{(m-n)c_m l^{n-m-1}bl + \derivation{c_m}l^{m-n}}{c_m l^{m-n}} \\
			= & \dfrac{(m-n)bc_ml^{m-n} + \derivation{c_m} l^{m-n}}{c_m l^{m-n}} \\
			= & (m-n)b + \dfrac{\derivation{c_m}}{c_m} \\
			=& 0
		\end{aligned}
		\end{equation}
		
		Luego $c_m l^{m-n} \in \constants{L} = \constants{K} \subset K$ por lo que $l^{m-n} \in K$.
		
		\item Sea $p = \sum k_n t^n$, luego:
		
		\begin{equation}
			\derivation{p(l)} = \left(\derivation{k_n }+ bnk_n\right) l^n + \dots + \derivation{k_0}
		\end{equation}
		
		Si $\derivation{k_n }+ bnk_n = 0$ entonces $ 0 = l^n \left(\derivation{k_n }+ bnk_n\right) = \derivation{k_n l^n}$ por lo que $k_nl^n \in \constants{L} = \constants{K} \subset K$ por lo que $l^n \in K$ que vimos que pasa si y solo si $l$ es algebraico, luego $\deg(p(l)) = n$.
		
		Si $p(l) = kl^n$ entonces es claro que $\derivation{p(l)} = \underbrace{\frac{\left(\derivation{k} + bnk\right)}{k}}_{C}kl^n = C p(l)$; reciprocamente supongamos que $\derivation{p(l)} = k p(l)$ y supongamos que $k_n l^n, k_ml^m$ son dos terminos de $p(l)$, ewntonces para $i=n,m$:
		
		\begin{equation}
		\derivation{k_i} + ik_ib = k k_i
		\end{equation}
		
		Luego:
		
		\begin{equation}
			\dfrac{\derivation{k_n} + nk_nb}{k_n} = \dfrac{\derivation{k_m} + mk_mb}{k_m}
		\end{equation}
		
		Lo que dice que:
		
		\begin{equation}
			a := (n-m)k_nk_mb + k_m\derivation{k_n} - k_n\derivation{k_m} = 0
		\end{equation}
		
		Luego:
		
		\begin{equation}
			\derivation{\dfrac{k_nl^n}{k_ml^m}} = \dfrac{a l^{n+m}}{\left(k_m l^m\right)^2} = 0
		\end{equation}
		
		Luego concluimos que $l^{n-m} \in \constants{K}$ lo que es absurdo pues $l$ es trascendente. \qed
		
	\end{itemize}
	
\end{proof}

\section{Extensiones de derivaciones}

\begin{proposition}
	Sea $\quotient{K}{L}$ una extension de cuerpos diferenciable y supongamos que $\delta: K \rightarrow K$ es una derivacion en $K$, entonces:
	
	\begin{enumerate}
		\item Sea $l \in L \setminus K$ trascendente sobre $K$ y $m \in L$ arbitrario, entonces existe una extension $\delta_L$ de $\delta$ tal que $\derivation{l}_L = m$
		\item Si $\quotient{K}{L}$ es algebraica, entonces existe una unica extension $\delta_L$ de $\delta$.
	\end{enumerate}
	
\end{proposition}

\begin{proof}
	Para la demostracion supongamos que la extension es simple, dado que sino es simplemente usar finitos pasos (en el caso algebraico) o el lema de Zorn.
	
	\begin{enumerate}
		\item [$l$ trascendente]
		
		Dado que si existe $\delta : K[l] \mapsto K[l]$ entonces existe uan unica extension a $K(l)$ dada por la regla del cociente, extendamos a $K[l]$. Sea $m \in L$ y definamos $\delta_L: K[l] \mapsto L$ dado por:
		
		\begin{equation}
			\derivation{\sum\limits_j a_jl^j} = \sum\limits_j \derivation{a_j}l^j + m \sum\limits_j j a_j l^{j-1}
		\end{equation}
		
		De la manera que fue armado, es claro que haciendo las cuentas va a daer que cumple la regla de Leibniz asi q	ue $\delta$ es la extension buscada.
		
		\item[$\quotient{K}{K[l]}$ es algebraica]
		
		Sea $p(t) = t^n + \sum\limits_{i=0}^{n-1} k_it^i \in K[t]$ el polinomio minimal de $l$, entonces vemos que si $\delta$ fuese una extension a $K[l]$ vale que:
		
		\begin{equation}
			\begin{aligned}
				0 = & \derivation{p(l)} \\
				=& nl^{n-1}\derivation{l} + \sum\limits_j j k_j l^{j-1} \derivation{l} + \sum\limits_j \derivation{k_j}l^j \\
				= & \derivation{l} \left( \derivation{p}(l)\right) + \sum\limits_j \derivation{k_j}l^j
			\end{aligned}
		\end{equation}
		
		Como $K$ es de caraceteristica $0$ y entonces seprabale sabemos que $\derivation{p}(l) \neq 0$ por lo que, si notamos $\check{D}q(t) = \sum\limits_j \derivation{k_j}t^j$: al polinomio con los coeficientes derivados para todo polinomio $q \in K[t]$, vale que:
		
		\begin{equation}
			\derivation{l} = \dfrac{- \check{D}p(l)}{\derivation{p}(l)}
		\end{equation}
		
		Luego , de existir, a lo sumo una extension puede ser definida y esta dada por la formula anterior. Formalmente, sea $\check{D}$ la derivacion en $K[t]$ dada por $m=0$ y como $l$ es algebraico existe un polinomio $s(l)$ tal que:
		
		\begin{equation}
		s(l) = \dfrac{-\check{D}p (l)}{\derivation{p}(l)} \in K[l]
		\end{equation}
		
		Definamos entonces la segunda derivacion $\widetilde{D} : K[t] \mapsto L$ dada por la eleccion $m = s(t)$, ie: $\widetilde{D}(q)(t) = \check{D}(q)(t) + s(t) \derivation{q}(t)$ y definamos $\eta: K[t] \mapsto L$ el morfismo de evaluacion tal que $\eta(q(t)) = q(l)$. Notemos entonces que $\ker(\eta)$ es un ideal principal generado por $p(t)$, luego si $q(t) \in \ker \eta$ entonces $q = p(t)r(t)$ y notemos que:
		
		\begin{equation}
		\begin{aligned}
			\eta \circ \widetilde{D} \left(q\right) (t) = & \eta \circ \widetilde{D} \left(p(t)r(t\right) \\
			= & \eta\left(p(t) \check{D}r(t) + \check{D}p(t)r(t) + s(t) \left(p(t) \derivation{r}(t) + \derivation{p}(t) r(t)\right) \right) \\
			= & p(l) \check{D}r(l) + \check{D}p(l)r(l) + s(l) \left(p(l) \derivation{r}(l) + \derivation{p}(l) r(l)\right) \\
			= & r(l) \left(\check{D}p (l) + s(l)\derivation{p}(l)\right) \\
			= & 0
		\end{aligned}
		\end{equation}
		
	Luego $\widetilde{D}\left(\ker \eta\right) \subset \ker \eta$ y por el primer teorema de isomorfismo $\widetilde{D}$ induce un morfismo de cuerpos diferenciales $D : K[l] \simeq \quotient{K[t]}{\ker \eta} \simeq  \quotient{K[t]}{\ip{p}} \mapsto K[l]$ dado por:
	
	\begin{equation}
		Dq(l) = \eta \left(\widetilde{D}q (t)\right)
	\end{equation}		
		
		Es facil ver que $D$ esta bien definido y que es una derivacion. \qed
		
	\end{enumerate}
	
\end{proof}

\begin{remark}
	La separabilidad de la extension es necesaria en la proposicion anterior. En efecto, sea $K := \mathbb{F}_2 (t)$ y tomemos $\sqrt{t}$ raiz de $x^2 -t$ en alguna clausura algebraica $F$, entonces afirmo que no existe una extension de la derivacion usual a $K(\sqrt{t})$; ya que $1 = \derivation{t} = \derivation{\sqrt{t}^2} = 2 \sqrt{t} \derivation{\sqrt{t}} = 0$ lo que es absurdo.
\end{remark}

\section{Integracion en terminos finitos}

\begin{theorem}
	Sea $K$ un cuerpo diferencial de caracteristica $0$ y sea $\alpha \in K$. Entonces $\alpha$ admite una primitva en uan extension de cuerpos elemental sin nuevas constantes si y solo si existen $c_1, \dots, c_m \in \constants{K}$ y elementos $\beta_1, \dots, \beta_m, \gamma \in K$ tal que $\beta_j \neq 0$ para $j = 1, \dots, m$ y:
	
	\begin{equation}
		\alpha = \sum\limits_{j=1}^{m} c_j \dfrac{\derivation{\beta_j}}{\beta_j} + \derivation{\gamma}
	\end{equation}
	
\end{theorem}

\begin{proof}
	Supongamos que existe una torre $K = K_0 \subset \dots \subset K_n$ de extensiones de cuerpos diferenciales sin nuevas constantes tal que $K_j = K_{j-1}[a]$ con $a$ un logaritmo, exponencial o algebraico. Ademas, existe un elemento $\rho \in K_n$ tal que $\derivation{\rho} = \alpha$; hagamos induccion en el largo de la torre $n$.
	
	Si $n=0$ entonces trivialmente vale por lo que supongamos que la ecuacion vale para $n-1$, entonces mirando la extension $\quotient{K_1}{K_n}$ existe $m \in \N, \ c_1, \dots, c_m \in {K_1}_C = \constants{K}$ y $\beta_1, \dots, \beta_m , \gamma \in K_1 = K[l]$ tal que vale la ecuacion. Luego basta probar que $\beta_1, \dots, \beta_m , \gamma \in K$ para los tres casos, ie: $l$ algebraico, logaritmo y exponencial.
	
	\begin{enumerate}
		\item [$l$ algebraico] Sea $F$ una clausura algebraica para $K$ conteniendo a $K[l]$ y sea $\sigma_i$ los diferentes morfismos de extension de cuerpos que permuta las raices del minimal de $l$, notemoslo $p$.
		
		Como $\quotient{K}{F}$ es algebraica, sabemos que existe una unica extension de la derivacion a $F$, y como $\sigma_i \circ \delta \circ \sigma_i^{-1}$ vale lo mismo en $l$, concluimos que $\sigma_i$ conmuta con la derivacion de $K$.
		
		Como $\beta_i, \gamma \in K(l) = K[l]$ pues $l$ es algebraico eligamos polinomios $q_i, \dots, q_m, r \in K[x]$ tal que:
		
		\begin{equation}
			\beta_i = q_i(l) \qquad \gamma = r(l)
		\end{equation}
		
		Luego tenemos que vale:
		
			\begin{equation}
		\alpha = \sum\limits_{j=1}^{m} c_j \dfrac{\derivation{q_j(l)}}{q_j(l)} + \derivation{r(l)}
		\end{equation}
		
		Y si aplicamos $\sigma_i$:
		
		\begin{equation}
		\begin{aligned}
			\alpha = & \sum\limits_{j=1}^{m} c_j \dfrac{\derivation{\sigma_i(q_j(l))}}{\sigma_i(q_j(l))} + \derivation{\sigma_i(r(l))} \\
			= & \sum\limits_{j=1}^{m} c_j \dfrac{\derivation{q_j(l_i)}}{q_j(l_i)} + \derivation{r(l_i)} 
		\end{aligned}
		\end{equation}
		
		Si llamamos $s$ a la cantidad de morfismos $\sigma_i$ y sumamos sobre $i$ y dividimos por $s$:
		
		\begin{equation}
		\begin{aligned}
		\alpha = & \sum\limits_{j=1}^{m} \frac{c_j}{s} \sum\limits_i{\dfrac{\derivation{q_j(l_i)}}{q_j(l_i)}} + \derivation{\frac{\sum\limits_i{r(l_i)}}{s}} \\
		=& \sum\limits_{j=1}^{m} \frac{c_j}{s} {\dfrac{\derivation{\prod\limits_i q_j(l_i)}}{\prod\limits_i q_j(l_i)}} + \derivation{\frac{\sum\limits_i{r(l_i)}}{s}}
		\end{aligned}
		\end{equation}
		
		Como cada termino en la suma es fijado por $\sigma_i$ para todo $i$, eso implica que todas las sumas estan en $K$.
		
		\item[$l$ es un logaritmo]
		
		Como $l$ es trascendente existen $q_1, \dots, q_m, r \in K(l)$ tal que:
		
		\begin{equation}
			\alpha = \sum\limits_{j=1}^{m} c_j \dfrac{\derivation{q_j(l)}}{q_j(l)} + \derivation{r(l)}
		\end{equation}
		
		Como $l$ es logaritmo entonces $\derivation{l} = \frac{\derivation{k}}{k} \in K$ para algun $k \in K$, luego tenemos que $\quotient{K}{K(l)}$ es extension de cuerpos diferencial. Asumamos el siguiente lema:
		
		\begin{lemma}
			Sea $\quotient{K}{K(l)}$ una extension de cuerpos diferencial con $l$ trascendente y tal que $\derivation{l} \in K[l]$, supongamos que $\alpha \in K$ y que $p(l)$ es un polinomio monico e irreducible tal que $\derivation{p(l)} \in K[l]$ no es dividible por $p(l)$. Luego si tenemos:
			
			\begin{equation}
			\alpha = \sum\limits_{j=1}^{m} c_j \dfrac{\derivation{q_j(l)}}{q_j(l)} + \derivation{r(l)}
			\end{equation}
			
			Entonces $q_j \neq p$ para todo $j$.
		\end{lemma}
		
		Sea $p(l)$ un polinomio irreducible, monico y no constante, luego sabemos que $\deg(\derivation{p(l)}) < \deg(p(l))$ por lo que $p(l) \not\vert \derivation{p(l)}$. Del lema podemos concluir que para todo $p(l)$ polinomio monico e irreducible, $q_j(l) \neq p$, luego como $K[l]$ es UFD concluimos que $q_j(l) \in K$ para todo $j$.
		
		Como $\alpha, \sum\limits_{j=1}^{m} c_j \dfrac{\derivation{q_j(l)}}{q_j(l)}  \in K$entonces $ \derivation{r(l)} \in K$ por lo que $r(l) = cl + \check{c}$ con $c \in \constants{K}$ y $\check{c} \in K$ y concluimos que vale lo pedido:
		
		\begin{equation}
			\alpha = \sum\limits_{j=1}^{m} c_j \dfrac{\derivation{q_j(l)}}{q_j(l)} + c \dfrac{\derivation{k}}{k} + \derivation{\check{c}}
		\end{equation}
		
		\item [$l$ es exponencial]
		
		En este caso como $\derivation{l} = l \derivation{k} \in K[l]$ concluimos que $\quotient{K}{K(l)}$ es una extension de cuerpos diferenciables. Si $p(l) \in K[l]$ es monico, irreducible, no constante y $p(l) \neq l$ tenemos que $p(l) \not\vert \derivation{p(l)} \in K[l]$; por lo mismo de antes entonces $q_j(l) \in K$ para todo $j$ y continuamos de la misma manera \qed
	\end{enumerate}
	
\end{proof}

\begin{corollary}
	Sea $\quotient{K}{K(e^g)}$ una extension de cuerpos diferenciales sin nuevas constantes con $g \in K$ y supongamos que $e^g$ es trascendente sobre $K$, entonces dado $f \in K$ arbitrario $fe^g$ admite una primitiva  en una extension elemental sin nuevas constantes si y solo si existe $a\in K$ tal que:
	
	\begin{equation}
		f = \derivation{a} + a\derivation{g}
	\end{equation}
\end{corollary}

\begin{proof}
	Por el teorema $fe^g$ admite primitva si y solo si existen $c_j \in \constants{K}$ y $\beta_j, \gamma \in K$ tal que:
	
	\begin{equation}
		fe^g = \sum\limits_j c_j \dfrac{\derivation{\beta_j}}{\beta_j} + \derivation{\gamma}
	\end{equation}
	
	Razonando igual que en la demostraciomn del caso exponencial notando $l = e^g$, sabemos que existen $q_j(l) \in K$, $r(l) = \sum\limits_{j=-b}^{b} {k)j l^j} \in K[l]$ tal que:
	
	\begin{equation}
	fe^g = \sum\limits_j c_j \dfrac{\derivation{q_j(l)}}{q_j(l)} + \derivation{\sum\limits_{j=-b}^{b} {k_j l^j}}
	\end{equation}
	
	Luego:
	
	
	\begin{equation}
	fl = c + \derivation{\sum\limits_{j=-b}^{b} {\derivation{k_j} l^j}} + \derivation{g} \sum\limits_{j=-b}^{b} {j k_j l^{j}} = c +  \sum\limits_{j=-b}^{b} {l^j \left(\derivation{k_j} + j \derivation{g} k_j\right)}
	\end{equation}
	
	Luego de igualar los coeficientes obtenemos:
	
	\begin{equation}
		f = \derivation{k_1} + k_1 \derivation{g}
	\end{equation}
	\qed
\end{proof}

\begin{corollary}
	No existe una extension diferencial elemental si nuevas constantes de $\R(x, e^{x^2})$ tal que $e^{x^2}$ tenga primitiva.
\end{corollary}

\begin{proof}
	En efecto, esta existe si y solo si existe $a \in \R(x)$ tal que $1 = \derivation{a} + 2ax$; luego escribamos $a = \frac{p}{q}$ con $p,q$ coprimos. Entonces:
	
	\begin{equation}
	1 = \dfrac{q\derivation{p} - p \derivation{q}}{q^2} + 2\dfrac{px}{q}
	\end{equation}
	
	de lo que:
	
	\begin{equation}
		q - 2px - \derivation{p} = - \dfrac{\derivation{q}p}{q}
	\end{equation}
	
	Y como $q \vert \derivation{q}p$ y $p,q$ son coprimos concluimos que $q \vert \derivation{q}$ y por las proposiciones anteriores llegamos a que $a = Cp \in \R[x]$ y comparando grados concluimos que no existe tal $a$. \qed
	
\end{proof}
\end{document}