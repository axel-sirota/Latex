\documentclass[10pt,a4paper,spanish]{article}

\usepackage{amsmath}
\usepackage{amsfonts}
\usepackage{amssymb}
\usepackage[spanish]{babel}
\usepackage[latin1]{inputenc}

\date{Primer cuatrimestre de 2015}
\title{Algebra III}

\newcounter{cont_ejer}[section]
\newenvironment{ejer}{\par\par\smallskip\noindent\addtocounter{cont_ejer}{1}{\bf Ejercicio \arabic{cont_ejer}.}}{\par\par\smallskip}

\renewcommand{\contentsname}{Indice}
\renewcommand{\refname}{Bibliograf\'\i a}

\newcommand{\car}{\mbox{car}}
\newcommand{\Gal}{\mbox{Gal}}
\newcommand{\Aut}{\mbox{Aut}}
\newcommand{\No}{\mbox{N}}
\newcommand{\Tr}{\mbox{Tr}}
\newcommand{\C}{\mathbb{C}}
\newcommand{\R}{\mathbb{R}}
\newcommand{\Q}{\mathbb{Q}}
\newcommand{\N}{\mathbb{N}}
\newcommand{\F}{\mathbb{F}}


\parindent = 0cm

\pagestyle{myheadings} \markright{{\footnotesize Departamento de Matem\'atica -
Facultad de Ciencias Exactas y Naturales - UBA}}

\addtolength{\hoffset}{-1.5cm} \addtolength{\textwidth}{3cm}

\begin{document}

\begin{center}
\begin{Large}\textbf{Algebra III}\end{Large}

\textbf{Pr\'actica 8 - Extensiones trascendentes}\\{\em 2do cuatrimestre 2015}
\end{center}

\bigskip

\begin{ejer}  Sea $E/K$ una extensi�n de cuerpos y sean $t, t', \alpha\in E$.
Determinar cu�les de las siguientes afirmaciones son verdaderas, justificando las respuestas:

\begin{enumerate}
\item Si $t$ y $t'$ son trascendentes sobre~$K$, entonces  $t\,t'$ y
$t+t'$ no son ambos algebraicos sobre~$K$.

\item Si $t$ es trascendente sobre~$K$ y $\alpha$ es algebraico sobre~$K$,
entonces $t+\alpha$ es trascendente sobre~$K$.

\item Si $t$ es trascendente sobre~$K$ y $\alpha$ es algebraico sobre~$K$,
entonces $\alpha t$ es trascendente sobre~$K$.


\item Si $t$ y $t'$ son trascendentes sobre~$K$, entonces el
conjunto~$\{t,t'\}$ es algebraicamente independiente sobre~$K$.
\end{enumerate}
\end{ejer}

\begin{ejer}
Sea $K$ un cuerpo y sean $\alpha$ algebraico sobre $K$ y $t$
trascendente sobre $K$. Probar que $f(\alpha,K)=f(\alpha,K(t))$ y que $[K[\alpha]:K]=[K(t)[\alpha]:K(t)]$.
\end{ejer}

\begin{ejer} Sea $E=\C(X)[Y]/\langle f(X,Y)\rangle$ con $f(X,Y)=Y^2-(X-a)(X-b)(X-c)$, para $a,b,c\in \C$.
Probar que tanto $\{X\}$ e $\{Y\}$ son bases de trascendencia de $E/\C$.
\end{ejer}

\begin{ejer}
Sean $L_{1}$ y $L_{2}$ dos subextensiones de $E/K$. Probar que:
$${\rm trdeg}(L_{1}L_{2}/K)\leqslant{\rm trdeg}(L_{1}/K)+{\rm
trdeg}(L_{2}/K).$$ Dar un ejemplo donde no valga la igualdad.
\end{ejer}

\begin{ejer}
Sea $E$ un cuerpo algebraicamente cerrado y sea $K$ un subcuerpo.
Sea $\varphi:E\to E$ un $K$-endomorfismo. Probar que si ${\rm
trdeg}(E/K)<\infty$ entonces $\varphi$ es un automorfismo. Mostrar
que esto no tiene por qu�  valer si ${\rm trdeg}(E/K)=\infty$.
\end{ejer}

\begin{ejer}
Sea $t$ trascendente sobre $\mathbb{C}$ y sea $K$ la clausura
algebraica de $\mathbb{C}(t)$. Probar que $K\simeq\mathbb{C}$.
\end{ejer}

\begin{ejer}?`Cu�l es el cardinal de una base de trascendencia de $\C/\Q$?
\end{ejer}

\begin{ejer} Probar que $\C$ tiene infinitos automorfismos de cuerpo. ?`Cu�l es el cardinal del conjunto  de automorfismos de $\C$?
\end{ejer}

\begin{ejer} Probar que $\C$ tiene infinitos subcuerpos propios que son isomorfos a �l.
\end{ejer}

\begin{ejer}
Sea $K$ un cuerpo y $E$ una $K$-\'algebra finitamente generada.
Probar que si $E$ es un cuerpo entonces $E/K$ es una extensi\'on
algebraica.
\end{ejer}

\end{document} 