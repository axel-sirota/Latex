\documentclass[10pt,a4paper,spanish]{article}

\usepackage{amsmath}
\usepackage{amsfonts}
\usepackage{amssymb}
\usepackage[spanish]{babel}
\usepackage[latin1]{inputenc}
\usepackage{multicol}


\def\N{\mathbb{N}}
\def\Z{\mathbb{Z}}
\def\R{\mathbb{R}}
\def\Q{\mathbb{Q}}
\def\C{\mathbb{C}}
\def\K{\mathbb{K}}
\def\F{\mathbb{F}}

\newcommand\Gal{\mbox{Gal}}
\newcommand\car{\mbox{car}}

\date{Primer cuatrimestre de 2006}
\title{Algebra III}

\newcounter{cont_ejer}[section]
\newenvironment{ejer}{\par\par\smallskip\noindent\addtocounter{cont_ejer}{1}{\bf Ejercicio \arabic{cont_ejer}.}}{\par\par\smallskip}

\renewcommand{\contentsname}{Indice}
\renewcommand{\refname}{Bibliograf\'\i a}

\parindent = 0cm

\pagestyle{myheadings} \markright{{\footnotesize Departamento de Matem\'atica -
Facultad de Ciencias Exactas y Naturales - UBA}}

\addtolength{\hoffset}{-1.5cm} \addtolength{\textwidth}{3cm}

\begin{document}

\begin{center}
\begin{Large}\textbf{Algebra III}\end{Large}

\textbf{Pr\'actica 3 - Extensiones normales y extensiones separables}\\\em{2do cuatrimestre 2015}
\end{center}

\bigskip

\begin{ejer}
Determinar si las siguientes afirmaciones son verdaderas o falsas. Justificar.
\begin{enumerate}
\item Todo polinomio no constante se factoriza linealmente sobre alg\'un cuerpo.
\item El cuerpo de descomposici\'on de un polinomio es \'unico, salvo isomorfismos.
\item Toda extensi\'on de grado finito es el cuerpo de descomposici\'on de alg\'un polinomio.
\item Sean $K\subseteq L\subseteq E$. Si $E$ es el cuerpo de descomposici\'on un polinomio $f\in K[X]$, entonces $E$ es el cuerpo de descomposici\'on de $f$ visto como polinomio en $L[X]$.
\end{enumerate}
\end{ejer}

\begin{ejer}
Exhibir cuerpos de descomposici\'on, determinando su grado y sistemas de generadores, para cada uno de los siguientes polinomios sobre los cuerpos indicados, y determinar sus grupos de Galois.
\begin{enumerate}
\item $X^{p}-a$, sobre $\mathbb{Q}$, con $p\in\mathbb{N}$ primo y $a\in\mathbb{Q}-\mathbb{Q}^{p}$.
\item $X^{3}-10$, sobre $\mathbb{Q}$ y $\mathbb{Q}[\sqrt{2}]$.
\item $X^{4}-5$, sobre $\mathbb{Q}$, $\mathbb{Q}[\sqrt{5}]$, $\mathbb{Q}[\sqrt{-5}]$ y $\mathbb{Q}[i]$.
\item $X^{4}+2$, sobre $\mathbb{Q}$ y $\mathbb{Q}[i]$.
\item $\prod_{i=1}^{n}(X^{2}-p_{i})$, sobre $\mathbb{Q}$, con $p_{1},\ldots,p_{n}\in\mathbb{N}$ primos distintos.
\item $X^{3}-2$, sobre $\F_{7}$.
\item $(X^{3}-2)(X^{3}-3)(X^{2}-2)$, sobre $\mathbb{Q}[\sqrt{-5}]$ y $\F_{5}$.
\item $X^{n}-t$, sobre $\mathbb{C}(t)$, con $t$ trascendente sobre $\mathbb{C}$ y $n\in\mathbb{N}$.
\item $X^{4}-t$, sobre $\mathbb{R}(t)$, con $t$ trascendente sobre $\mathbb{R}$.
\end{enumerate}
\end{ejer}

\begin{ejer}
(F\'acil, pero muy \'util) Sea $E=K[a]/K$ una extensi\'on normal. Sea $b\in\overline{K}$ una ra\'\i z de $f(a,K)$. Probar que $b\in E$ y que $E=K[b]$.
\end{ejer}

\begin{ejer}
Caracterizar  los cuerpos de descomposici\' on de los
polinomios $X^3+2X+1$ y \\$X^3+X^2+X+2$ sobre $\F_3$. Probar que
son isomorfos como extensiones de $\F_3$.\end{ejer}

\begin{ejer} Calcular los cuerpos de descomposici\' on de los polinomios
irreducibles de grado $2$ sobre $\F_5$. ?`Son isomorfos entre
ellos?
\end{ejer}

\begin{ejer}
Sea $E/K$ una extensi�n que es el cuerpo de descomposici\'on de un polinomio $f\in K[X]$, con ${\rm gr}(f)=n$. Probar que $[E:K]\, |\, n!$. Mostrar ejemplos donde se cumpla la igualdad y donde no se cumpla.
\end{ejer}

\begin{ejer}
Determinar si las siguientes afirmaciones son verdaderas o falsas. Justificar.
\begin{enumerate}
\item Toda extensi\'on finita es normal.
\item Toda extensi\'on finita est\'a contenida en una extensi\'on finita normal.
\item Toda extensi\'on de un cuerpo de caracter\'\i stica cero es normal.
\item Todo $K$-morfismo $f:L/K\to L/K$ es un $K$-automorfismo.
\item Si $L/K$ es algebraica, entonces todo $K$-morfismo $f:L\to L$ es un $K$-automorfismo.
\item Toda extensi\'on con grupo de Galois trivial es normal.
\item El grupo de Galois de una extensi\'on normal es c\'\i clico.
\end{enumerate}
\end{ejer}

\begin{ejer}
Sea $E/K$ una extensi\'on normal y sea $K\subseteq F\subseteq E$ una subextensi\'on. Probar que todo $K$-morfismo de $F$ en $E$ puede ser extendido a un $K$-automorfismo de $E$.
\end{ejer}

\begin{ejer}
Determinar cuales de las siguientes extensiones $E/K$ son normales. En cada caso, calcular ${\rm Gal}(E/K)$ y $\{\sigma:E\to \overline K, \ K-\mbox{morfismo}\}$.
\begin{enumerate}
\item $\mathbb{Q}[\sqrt[7]{5}]/\mathbb{Q}$.
\item $\mathbb{Q}[\sqrt[7]{5},\sqrt{5}]/\mathbb{Q}[\sqrt[7]{5}]$.
\item $\mathbb{Q}[\sqrt{2},\sqrt[3]{3}]/\mathbb{Q}$.
\item $\mathbb{Q}[\xi_{p}]/\mathbb{Q}$, con $p\in\mathbb{N}$ primo.
\item $\F_3[a]/\F_3$, con $a$ ra�z de $X^3+2X+1 $.
\end{enumerate}
\end{ejer}

\begin{ejer}
Sea $K$ un cuerpo, $n\in\mathbb{N}$ y $t$ trascendente sobre $K$. Probar que $K(t)/K(t^{n})$ es normal si y s\'olo si el polinomio $X^{n}-1$ se factoriza linealmente en $K[X]$.
\end{ejer}

\begin{ejer} \begin{enumerate}
\item  Probar que $\Q [\sqrt[4]{2}\,]/\Q[ \sqrt
2\,]$ y $\Q[ \sqrt 2\,] /\Q$ son normales, pero $\Q
[\sqrt[4]{2}\,] /\Q$ no lo es. \\Calcular $\Gal(\Q
[\sqrt[4]{2}\,] /\Q)$.
\item  Exhibir extensiones normales con subextensiones no normales.
\end{enumerate}
\end{ejer}

\begin{ejer}
Sea $H/K$ una extensi\'on algebraica y sean $E/K$ y $F/K$ subextensiones normales. Probar que $EF/K$ y $E\cap F/K$ son normales.
\end{ejer}

\begin{ejer}
Probar que toda extensi\'on $E/K$ generada por elementos de grado 2 es normal. ?`Para que valores de $n\in\mathbb{N}$ vale que toda extensi\'on de grado $n$ sobre $\mathbb{Q}$ es normal?
\end{ejer}

\begin{ejer}
Determinar el cuerpo de descomposici\'on de $X^{4}-10X^{2}+5$ y su grupo de Galois, sobre $\mathbb{Q}$, $\F_{3}$ y $\F_{7}$.
\end{ejer}

\begin{ejer}
\begin{enumerate}
\item  Sea $E/K$ una extensi\'on algebraica tal que todo polinomio no
constante en $K[X]$ se factoriza linealmente en $E[X]$. Probar que
$E$ es algebraicamente cerrado.
\item  Sea $K$ un cuerpo infinito y sea $E/K$ una extensi\' on algebraica tal
que todo polinomio no constante en $K[X]$ tiene una ra\'\i z en
$E$. Probar que $E$ es algebraicamente cerrado. (Nota: Vale tambi\'en si $K$ es finito.)
\end{enumerate}
\end{ejer}


\begin{ejer}
Sea $E/K$ una extensi\'on finita y separable. Probar las siguientes afirmaciones:
\begin{enumerate}
\item $\{\sigma:E\to\overline K, \ K-\mbox{morfismo}\}$ tiene $[E:K]$ elementos.
\item Si $a\in E$ satisface $\sigma_1(a)\ne \sigma_2(a)$ para todo $\sigma_1\ne\sigma_2$ $K$-morfismos, entonces $E=K[a]$.
\item Si $b\in E$, entonces $$\prod_{\sigma:E\to \overline K, K-\mbox{\scriptsize morfismo}}\!\!\!\!\!\!\!\!(X-\sigma(b))=f(b,K)^{[E:K[b]]}.$$
\item Existe $\theta\in E$ tal que $E=K[\theta]$.
\end{enumerate}
\end{ejer}

\begin{ejer} (Lema de Carlos)
Sean $f,g,h\in K[X]$ con $f$ irreducible separable, y $h(X)\, |\,f(g(X))$.
Sean $\alpha_{1},\ldots,\alpha_{n}\in\overline{K}$ las raices de $f$ y sean $\beta_{1},\ldots,\beta_{m}\in\overline{K}$ las raices de $h$. Probar que los conjuntos $A_{j}=\{ 1\leqslant i\leqslant m\; :\; g(\beta_{i})=\alpha_{j}\}$ tienen todos la misma cantidad de elementos.
\end{ejer}


%\begin{ejer} Determinar un elemento primitivo de $\Q[a,b]$ con $a,b\in \C$.\end{ejer}


\begin{ejer} Determinar elementos primitivos de $E/\Q$ donde $E$ es el cuerpo de descomposici�n del polinomio
\begin{multicols}{4}
\begin{enumerate}
\item $X^3-2$ \item $(X^2-2)(X^2-3)$ \item $X^4-2$ \item $(X^4+1)(X^2+5)$
\end{enumerate}
\end{multicols}
\end{ejer}

\begin{ejer} Sea $K$ un cuerpo de caracter�stica $p>0$ y sea $f\in K[X]$ un polinomio irreducible de grado $n$ con $p\nmid n$. Probar que el polinomio $f$ es separable.
\end{ejer}

\begin{ejer} Probar que el polinomio $X^n-1\in K[X]$ no es separable si y solo si $\car(K)=p$ y $p\mid n$.
\end{ejer}

\begin{ejer} Sea  $K$ es un cuerpo de caracter�stica $p>0$, Probar que
\begin{enumerate} \item la aplicaci�n $x\mapsto x^p$ es un $\F_p$-endomorfismo de $K$ (se llama {\em endomorfismo de Frobenius}).
\item la aplicaci�n $x\mapsto x^{p^e}$ es un $\F_p$-endomorfismo de $K$, $\forall\, e\in \N$.
\end{enumerate}
\end{ejer}

\begin{ejer} Probar que las condiciones siguientes sobre un cuerpo $K$ son equivalentes:
\begin{enumerate} \item Todo polinomio irreducible en $K[X]$ tiene ra�ces distintas (o sea es separable)
\item Toda extensi�n finita de $K$ es separable.
\item Si $\car(K)=p$ ($p=0$ o $p>0$), el conjunto $K^{p^{-\infty}}:=
\{x\in \overline K: \sigma(x)=x, \ \forall\, \sigma\in \Gal(\overline K/K)\}$ coincide con $K$.
\item O bien $\car(K)=0$ o si $\car(K)=p>0$,  cada elemento de $K$ es  potencia $p$-�sima de un elemento de $K$.
\item O bien $\car(K)=0$ o si $\car(K)=p>0$, el endomorfismo de Frobenius de $K$ es un automorfismo de $K$.
    \end{enumerate}
    En cualquiera de estos casos se dice que el cuerpo $K$ es {\em perfecto}.
\end{ejer}

\begin{ejer} Probar que todo cuerpo finito es perfecto. \end{ejer}

\end{document} 