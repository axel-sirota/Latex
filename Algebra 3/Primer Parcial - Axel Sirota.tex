\documentclass[11pt]{article}

\usepackage{amsfonts}
\usepackage{amsmath,accents,amsfonts, amssymb, mathrsfs }
\usepackage{tikz-cd}
\usepackage{graphicx}
\usepackage{syntonly}
\usepackage{color}
\usepackage{mathrsfs}
\usepackage[spanish]{babel}
\usepackage[latin1]{inputenc}
\usepackage{fancyhdr}
\usepackage[all]{xy}
\usepackage[at]{easylist}
\usepackage[colorlinks=true,linkcolor=blue,urlcolor=black,bookmarksopen=true]{hyperref}

\usepackage{bookmark}

\topmargin-2cm \oddsidemargin-1cm \evensidemargin-1cm \textwidth18cm
\textheight25cm


\newcommand{\B}{\mathcal{B}}
\newcommand{\Cont}{\mathcal{C}}
\newcommand{\F}{\mathcal{F}}
\newcommand{\inte}{\mathrm{int}}
\newcommand{\A}{\mathcal{A}}
\newcommand{\C}{\mathbb{C}}
\newcommand{\Q}{\mathbb{Q}}
\newcommand{\Z}{\mathbb{Z}}
\newcommand{\inc}{\hookrightarrow}
\renewcommand{\P}{\mathcal{P}}
\newcommand{\R}{{\mathbb{R}}}
\newcommand{\N}{{\mathbb{N}}}
\newcommand\tq{~:~}
\newcommand{\dual}[1]{\left(#1\right)^{\ast}}
\newcommand{\ortogonal}[1]{\left(#1\right)^{\perp}}
\newcommand{\ddual}[1]{\left(#1^{\ast}\right)^{\ast}}
\newcommand{\x}[3]{#1_#2^#3}
\newcommand{\xx}[4]{#1_#3#2_#4}
\newcommand\dd{\,\mathrm{d}}
\newcommand{\norm}[1]{\left\lVert#1\right\rVert}
\newcommand{\abs}[1]{\left\lvert#1\right\rvert}
\newcommand{\ip}[1]{\left\langle#1\right\rangle}
\renewcommand\tt{\mathbf{t}}
\newcommand\nn{\mathbf{n}}
\newcommand\bb{\mathbf{b}}                      % binormal
\newcommand\kk{\kappa}
\newcommand{\sett}[1]{\left\lbrace#1\right\rbrace}
\newcommand{\interior}[1]{\accentset{\smash{\raisebox{-0.12ex}{$\scriptstyle\circ$}}}{#1}\rule{0pt}{2.3ex}}
\fboxrule0.0001pt \fboxsep0pt
\newcommand{\Bigcup}[2]{\bigcup\limits_{#1}{#2}}
\newcommand{\Bigcap}[2]{\bigcap\limits_{#1}{#2}}
\newcommand{\Bigprod}[2]{\prod\limits_{#1}{#2}}
\newcommand{\Bigcoprod}[2]{\coprod\limits_{#1}{#2}}
\newcommand{\Bigsum}[2]{\sum\limits_{#1}{#2}}
\newcommand{\BigsumA}[3]{ \sideset{}{^#2}\sum\limits_{#1}{#3}}
\newcommand{\Biglim}[2]{\lim\limits_{#1}{#2}}
\newcommand{\quotient}[2]{{\raisebox{.2em}{$#1$}\left/\raisebox{-.2em}{$#2$}\right.}}
\DeclareMathOperator{\rank}{ran}
\DeclareMathOperator{\graf}{Gr}
\DeclareMathOperator{\ball}{ball}

\def \le{\leqslant}	
\def \ge{\geqslant}
\def\noi{\noindent}
\def\sm{\smallskip}
\def\ms{\medskip}
\def\bs{\bigskip}
\def \be{\begin{enumerate}}
	\def \en{\end{enumerate}}
\def\deck{{\rm Deck}}
\def\Tau{{\rm T}}

\newtheorem{mytheorem}{Theorem}
 %

\newtheorem{theorem}{Teorema}
\numberwithin{theorem}{subsection}
\newtheorem{lemma}[theorem]{Lema}

\newtheorem{proposition}[theorem]{Proposici\'on}

\newtheorem{corollary}[theorem]{Corolario}


\newenvironment{proof}[1][Demostraci\'on]{\begin{trivlist}
		\item[\hskip \labelsep {\bfseries #1}]}{\end{trivlist}}
\newenvironment{definition}[1][Definici\'on]{\begin{trivlist}
		\item[\hskip \labelsep {\bfseries #1}]}{\end{trivlist}}
\newenvironment{example}[1][Ejemplo]{\begin{trivlist}
		\item[\hskip \labelsep {\bfseries #1 }]}{\end{trivlist}}
\newenvironment{remark}[1][Observaci\'on]{\begin{trivlist}
		\item[\hskip \labelsep {\bfseries #1}]}{\end{trivlist}}
\newenvironment{declaration}[1][Afirmaci\'on]{\begin{trivlist}
		\item[\hskip \labelsep {\bfseries #1}]}{\end{trivlist}}


\newcommand{\qed}{\nobreak \ifvmode \relax \else
	\ifdim\lastskip<1.5em \hskip-\lastskip
	\hskip1.5em plus0em minus0.5em \fi \nobreak
	\vrule height0.75em width0.5em depth0.25em\fi}

\newcommand{\twopartdef}[4]
{
	\left\{
	\begin{array}{ll}
		#1 & \mbox{ } #2 \\
		#3 & \mbox{ } #4
	\end{array}
	\right.
}

\newcommand{\threepartdef}[6]
{
	\left\{
	\begin{array}{lll}
		#1 & \mbox{ } #2 \\
		#3 & \mbox{ } #4 \\
		#5 & \mbox{ } #6
	\end{array}
	\right.
}

\tikzset{commutative diagrams/.cd,
	mysymbol/.style={start anchor=center,end anchor=center,draw=none}
}
\newcommand\Center[2]{%
	\arrow[mysymbol]{#2}[description]{#1}}

\newcommand*\circled[1]{\tikz[baseline=(char.base)]{
		\node[shape=circle,draw,inner sep=2pt] (char) {#1};}}


\makeatletter
\newcommand{\xRightarrow}[2][]{\ext@arrow 0359\Rightarrowfill@{#1}{#2}}
\makeatother


\begin{document}
	
	\pagestyle{empty}
	\pagestyle{fancy}
	\fancyfoot[CO]{\slshape \thepage}
	\renewcommand{\headrulewidth}{0pt}
	
	
	
	\centerline{\bf \'Algebra 3}
	\centerline{\sc Primer Parcial}
	\centerline{\sc Axel Sirota}
	
\begin{enumerate}
	\item[Ejercicio 1] 
	
	\begin{itemize}
		\item 
	Primero recordemos el ejercicio 27 de la pr\'actica 3 que dice:
	
	\begin{equation*}
		Gal \left(K(t) / K\right) \simeq PGL_2(K)
	\end{equation*}
	
	Luego afirmo que en realidad $G = \ip{\tau, \sigma, i} = Gal \left(K(t) / K\right)$ de lo que el orden de $G$ va a ser el orden de $PGL_2(K) = q^3 - q$.
	
	En pos de eso, para un lado es claro que los tres generadores de $G$ estan en $Gal \left(K(t) / K\right)$, luego $G \subseteq Gal \left(K(t) / K\right)$. Rec\'iprocamente,  notemos que si $f \in Gal \left(K(t) / K\right)$ entonces ya sabemos que $f(t) = \dfrac{at+b}{ct+d}$ donde $ac-bd \neq 0$; por lo tanto podemos representarlo como $f(t) = \dfrac{A}{Ct + D} + B$.
	
	Luego, si notamos $\sigma_C(t) = Ct$, $t_B(t) = t + B$:
	
	\begin{equation*}
		\begin{aligned}
			f(t) = & \sigma_C\left(\dfrac{A}{t + D} + B\right) \\
			= & \sigma_C \circ \tau_D \left(\dfrac{A}{t} + B\right) \\
			= & \sigma_C \circ \tau_D \circ -i \left(At + B\right) \\
			= & \sigma_C \circ \tau_D \circ -i \circ \tau_B \circ \sigma_A(t)
		\end{aligned}
	\end{equation*}
	
	Para concluir, es claro que como $B \in K$ entonces $\tau_B = \tau^{B}$ y como $\left(8, 37\right) = 1$ y $37$ es primo (ac\'a usamos que $K = \mathbb{F}_{37}$) entonces existe $j_A, j_C \in \sett{1, \dots, 36}$ tal que $A = 8^{j_A}$ y $C = 8^{j_C}$. Luego tenemos que para $f \in Gal \left(K(t) / K\right)$ existe $j_A, B, D, j_C \in \mathbb{F}_{37}$ tal que $f = \sigma^{j_C} \circ \tau^{D} \circ -i \circ \tau^{B} \circ \sigma^{j_A}$; concluimos que $G = Gal \left(K(t) / K\right)$ y como $\abs{Gal \left(K(t) / K\right)} = q^3 - q$ ya sabemos el orden de $G$.
	
	\item Sean $p,q \in \mathbb{F}_{37}[t]$ coprimos y analicemos que tiene que pasar para que $\frac{f}{g} \in \mathbb{F}_{37}(t)^{\ip{h}}$ donde $h$ van a ser $\sigma, \tau, i$ respectivamente. 
	
	Para $\sigma$ notemos que si $u = t^{36}$ entonces para $f \in \mathbb{F}_{37}[t]$ vale:
	
	\begin{equation*}
		\begin{aligned}
		\sigma\left(f(t^{36})\right) = & \sum\limits_{i \in sop(f)}{\sigma\left(a_i (t^{36})^{i} \right)} \\
		= & \sum\limits_{i \in sop(f)} {a_i \left(\sigma(t^{36})\right)^i} \\
		= & \sum\limits_{i \in sop(f)} {a_i \left(\underbrace{8^{36}}_{\cong 1 \text{ mod }(37)}(t^{36})\right)^i} \\
		=  & f(t^{36})
		\end{aligned}
	\end{equation*}
	
	Luego si $u = t^{36}$ vimos que $\mathbb{F}_{37}(t^{36}) \subset E^{\ip{\sigma}}$. 
	
	Por un lado, como $ord(\sigma) = 37$ pues $mcd(8,37) = 1$ del teorema de Galois sabemos que $\left[E: E^{\ip{\sigma}}\right] = 36$; por el otro, como $f(x) = x^{36} - t^{36} \in \mathbb{F}_{37}(t^{36})[X]$ es m\'onico, irreducible (Einseinstein en $t^{36}$ que es primo) y anula a $t$ sabemos que $\left[\mathbb{F}_{37}(t^{36}): \mathbb{F}_{37}(t)\right] = 36$. Luego, juntando todo, tenemos la torre $\mathbb{F}_{37}(t^{36})  \subseteq E^{\ip{\sigma}} \subseteq \mathbb{F}_{37}(t) $ donde:

	\begin{equation*}
		\left[\mathbb{F}_{37}(t^{36}): E^{\ip{\sigma}}\right] = \dfrac{\left[\mathbb{F}_{37}(t^{36}): \mathbb{F}_{37}(t)\right]}{\left[\mathbb{F}_{37}(t): E^{\ip{\sigma}}\right]} = 1
	\end{equation*}
	
	De lo que concuimos que $E^{\ip{\sigma}} = \mathbb{F}_{37}(t^{36})$.
	
	Ahora vayamos a $i$! Si $u = t^2 + t^{-2}$ entonces para $f \in \mathbb{F}_{37}[t]$ vale:
	
	\begin{equation*}
	\begin{aligned}
	i\left(f(t^2 + t^{-2})\right) = & \sum\limits_{i \in sop(f)}{i\left(a_i (t^2 + t^{-2})^{i} \right)} \\
	= & \sum\limits_{i \in sop(f)} {a_i \left(i(t^2 + t^{-2})\right)^i} \\
	= & \sum\limits_{i \in sop(f)} {a_i \left(i(t)^2 + \left(\frac{1}{i(t)}\right)^{2}\right)^i} \\
	= & \sum\limits_{i \in sop(f)} {a_i \left(t^{-2} + t^{2})\right)^i} \\
	=  & f(t^2 + t^{-2})
	\end{aligned}
	\end{equation*}
	
	Luego si $u = t^{2} + t^{-2}$ vimos que $\mathbb{F}_{37}(u) \subset E^{\ip{i}}$. 
	
	A continuaci\'on notemos que en realidad $i = i_1 \circ i_2$ donde $i_1(t) = -t$ y $i_2(t) = \frac{1}{t}$ cumplen las relaciones $i_1^2 = i_2^2 = Id$ y $i_1 \circ i_2 = i_2 \circ i_1$; por lo tanto $\ip{i_1, i_2} = \ip{i} \simeq \Z_2 \times \Z_2$. Luego $4 = \abs{Gal \left(\quotient{\mathbb{F}_{37}(t)}{E^{\ip{i}}}\right)} = \left[E: E^{\ip{i}}\right]$ por el teorema de Galois.
	
	Por otro lado, si $p(x) = x^4 - x^2 \left(t^2 + t^{-2}\right) + 1$ entonces $p \in \mathbb{F}_{37}(t^2 + t^{-2})[X]$ es m\'onico y $p(t) = 0$, por lo que $f\left(t, \mathbb{F}_{37}(t^2 + t^{-2})\right) \vert p$ con lo que $\left[\mathbb{F}_{37}(u): \mathbb{F}_{37}(t)\right] \leq 4$.
	
	Luego, juntando todo, tenemos la torre $\mathbb{F}_{37}(u)  \subseteq E^{\ip{i}} \subseteq \mathbb{F}_{37}(t) $ donde:
	
	\begin{equation*}
	\left[\mathbb{F}_{37}(u): E^{\ip{i}}\right] = \dfrac{\left[\mathbb{F}_{37}(u): \mathbb{F}_{37}(t)\right]}{\left[\mathbb{F}_{37}(t): E^{\ip{i}}\right]} = 1
	\end{equation*}
	
	Pues $1 \leq \left[\mathbb{F}_{37}(u): E^{\ip{i}}\right] \leq 1$, de lo que concuimos que $E^{\ip{i}} = \mathbb{F}_{37}(t^{2} + t^{-2})$.
	
	Finalmente analicemos a $\tau$, si $u = t^{37} - t$ entonces para $f \in \mathbb{F}_{37}[t]$ vale:
	
	\begin{equation*}
	\begin{aligned}
	\tau\left(f(t^{37} - t)\right) = & \sum\limits_{i \in sop(f)}{\tau\left(a_i (t^{37} - t)^{i} \right)} \\
	= & \sum\limits_{i \in sop(f)} {a_i \left(\tau(t^{37} - t)\right)^i} \\
	= & \sum\limits_{i \in sop(f)} {a_i \left(\left(t + 1\right)^{37} - t - 1\right)^i} \\
	= & \sum\limits_{i \in sop(f)} {a_i \left(t^{37} + 1^{37} - t - 1\right)^i} \\
	=  & f(t^{37} - t)
	\end{aligned}
	\end{equation*}
	
	Luego si $u = t^{37} - t$ vimos que $\mathbb{F}_{37}(u) \subset E^{\ip{\tau}}$. 
	
	Por un lado, como $ord(\tau) = 37$ del teorema de Galois sabemos que $\left[E: E^{\ip{\tau}}\right] = 37$; por el otro, como $f(x) = x^{37} - x - t^{37} + t \in \mathbb{F}_{37}(t^{37} - t)[X]$ es m\'onico, irreducible (Einseinstein en $t^{37} - t$ que es primo) y anula a $t$ sabemos que $\left[\mathbb{F}_{37}(t^{37} - t): \mathbb{F}_{37}(t)\right] = 37$. Luego, juntando todo, tenemos la torre $\mathbb{F}_{37}(t^{37} - t)  \subseteq E^{\ip{\tau}} \subseteq \mathbb{F}_{37}(t) $ donde:
	
	\begin{equation*}
	\left[\mathbb{F}_{37}(t^{37} - t): E^{\ip{\tau}}\right] = \dfrac{\left[\mathbb{F}_{37}(t^{37} - t): \mathbb{F}_{37}(t)\right]}{\left[\mathbb{F}_{37}(t): E^{\ip{\tau}}\right]} = 1
	\end{equation*}
	
	De lo que concuimos que $E^{\ip{\tau}} = \mathbb{F}_{37}(t^{37} - t)$.
	
	Para concluir el punto notemos que ahora simplemente tenemos que juntar lo que fuimos descubriendo! Es decir es claro que:
	
	\begin{equation*}
		\begin{array}{rcl}
			E^{\ip{\sigma}} & = & \mathbb{F}_{37}\left(t^{36}\right) \\
			E^{\ip{\sigma, i}} & = & \mathbb{F}_{37}\left(\left(t^2 - t^{-2}\right)^{36}\right) \\
			E^{\ip{\sigma, \tau}} & = & \mathbb{F}_{37}\left(\left(t^{37} - t\right)^{36}\right) \\
			E^{\ip{\tau, i}} & = & \mathbb{F}_{37}\left(\left(t^{2} + t^{-2} \right)^{37} - t^{2} + t^{-2} \right)
		\end{array}
	\end{equation*}
	
	\item Afirmo que $f(t) = \dfrac{\left(t^{37^2} - t\right)^{38}}{\left(t^{37} - t\right)^{37^2 + 1}}$ cumple que $\mathbb{F}_{37}(t)^{\ip{\sigma, i , \tau}} = \mathbb{F}_{37}(t)^{Gal(\mathbb{F}_{37}(t))} = \mathbb{F}_{37}(f)$.
	
	Por un lado recordemos que $q^3 - q = \abs{Gal \left(\quotient{\mathbb{F}_{37}(t)}{E^{G}}\right)} = \left[E: E^{G}\right]$ por el teorema de Galois y el primer punto; y por el otro del ejercicio 19 de la pr\'actica 2 si $f = \frac{g}{h} \in E \setminus \mathbb{F}_{37}$ entonces $\left[E: E(f)\right] = \max \left\lbrace gr(g), gr(h) \right\rbrace$. Luego, si probamos que al reducir $f$ a factores coprimos $g,h$ vale que $\max \left\lbrace gr(g), gr(h) \right\rbrace= q^3 - q$ podemos concluir, ya que claramente $\mathbb{F}_{37}(f) \subseteq E^G$, que $E^G = \mathbb{F}_{37}(f)$.
	
	Notemos que:
	
	\begin{equation*}
		\begin{aligned}
			\dfrac{\left(t^{37^2} - t\right)^{38}}{\left(t^{37} - t\right)^{37^2 + 1}} = & \dfrac{t^{q+1}\left(t^{q^2 -1} - 1\right)^{q+1}}{t^{q^2 + 1}\left(t^{q-1} - 1\right)^{q^2 + 1}} \\
			= & \dfrac{\left( \left(t^{q -1} \right)^{q+1} - 1\right)^{q+1}}{t^{q^2 - q} \left(t^{q-1} - 1\right)^{q^2 + 1}} \\
			= & \dfrac{\left( \left(t^{q -1} \right) - 1\right)^{q+1} \left(\sum\limits_{r=0}^{q}{t^{(q-1)r}}\right)^{q+1}}{t^{q^2 - q} \left(t^{q-1} - 1\right)^{q^2 + 1}}  \\
			= & \dfrac{\left(\sum\limits_{r=0}^{q}{t^{(q-1)r}}\right)^{q+1}}{t^{q^2 - q} \left(t^{q-1} - 1\right)^{q^2 -q}}  \\
			= & \frac{g}{h} \qquad \text{pues } (g,h) = 1
		\end{aligned}
	\end{equation*}
	
	Y finalmente $\max \left\lbrace gr(g), gr(h) \right\rbrace = \max \sett{\underbrace{q^2 - q + \left(q-1\right)\left(q^2 - q\right)}_{q^3 - q^2}, \underbrace{(q-1)q(q+1)}_{q^3 - q}} = q^3 - q$, luego conlcuimos que $E^G = \mathbb{F}_{37}(f)$. \qed
	
	\end{itemize}
	
	\item[Ejercicio 2] 
	
	\begin{itemize}
		\item Notemos primero que $\beta^2 = 10 + 5\sqrt{2} + 2\sqrt{5} + \sqrt{10} \in \Q[\sqrt{2}, \sqrt{5 }]$; \textcolor{red}{
			 es m\'as, notemos que de la misma cuenta $\beta^2 \not\in  \Q,  \Q[\sqrt{2}],  \Q[\sqrt{5 }]$ que son (lo vimos en la pr\'actica analizando $\Q[\sqrt{p_1}, \sqrt{p_2}]$) las \'unicas subextensiones propias de $\Q[\sqrt{2}, \sqrt{5}]$ 
		 }. Luego como $\quotient{\Q[\sqrt{2}, \sqrt{5 }]}{\Q}$ es separable, de la te\'orica sabemos que para todo $\alpha \in \Q[\sqrt{2}, \sqrt{5 }]$ vale:
		
		\begin{equation*}
			f \left(\alpha, \Q\right) = \prod\limits_{i=1}^{4} \left(x - \sigma_i(\alpha)\right)
		\end{equation*}
		
		\textcolor{red}{Donde $\sett{\sigma_i(\alpha)}_i$ son los valores diferentes que toma $\sigma(\alpha)$ con  $\sigma \in Gal \left(\quotient{\Q[\sqrt{2}, \sqrt{5 }]}{\Q}\right)$.}
		
		Por otro lado, ya de la pr\'actica sabemos que $Gal \left(\quotient{\Q[\sqrt{2}, \sqrt{5 }]}{\Q}\right) \simeq \Z_2 \times \Z_2$ y los $4$ morfismos est\'an generados por las restricciones de la conjugaci\'on en $\quotient{\Q[\sqrt{5 }]}{\Q}$ y $\quotient{\Q[\sqrt{2}]}{\Q}$.
		
		\textcolor{red}{Como en nuestro caso $\beta^2 \not\in  \Q,  \Q[\sqrt{2}],  \Q[\sqrt{5 }]$ no est\'a en ninguna subextensi\'on de $\Q[\sqrt{2}, \sqrt{5}]$ entonces por el teorema de correspondencia de Galois (es claramente normal) $\beta^2$ no esta en ning\'un cuerpo fijo, lo que es equivalente a que $\sigma(\beta) \neq \beta$ para ninguno de estos 4 morfismos. Luego podemos concluir que:
			\begin{equation*}
			\begin{aligned}
			\sett{\sigma_i(\beta^2)}_i = & \left\lbrace \left(10 -5\sqrt{2} + 2\sqrt{5} - \sqrt{10}\right), \left(10 +5\sqrt{2} - 2\sqrt{5} - \sqrt{10}\right), \right.\\ & \left. \left(10 +5\sqrt{2} + 2\sqrt{5} + \sqrt{10}\right), \left(10 -5\sqrt{2} - 2\sqrt{5} + \sqrt{10}\right) \right\rbrace
			\end{aligned}
			\end{equation*}
			Es decir, $\beta^2$ evaluado en cada uno de los 4 morfismos que generan el grupo de Galois; pues si alguno fijara $\beta^2$ esto ser\'ia equivalente a que $\beta^2$ este en alg\'un cuerpo fijo, lo que ser\'ia equivalente a que $\beta^2$ pertenezca a alguna subextensi\'on propia de $\Q[\sqrt{2}, \sqrt{5}]$.
		} Por lo tanto:
		
		\begin{equation*}
			\begin{aligned}
				f \left(\beta^2, \Q\right) = & \left(x - \left(10 -5\sqrt{2} + 2\sqrt{5} - \sqrt{10}\right)\right)\left(x - \left(10 +5\sqrt{2} - 2\sqrt{5} - \sqrt{10}\right)\right) \\
				& \left(x - \left(10 +5\sqrt{2} + 2\sqrt{5} + \sqrt{10}\right)\right)\left(x - \left(10 -5\sqrt{2} - 2\sqrt{5} + \sqrt{10}\right)\right) \\
				= & x^4 - 40x^3 + 440x^2 - 1600 x +1600 \qquad \text{ si hice bien las cuentas}
			\end{aligned}
		\end{equation*}
		
		De yapa, como ya sabemos que $f$ es el minimal y tiene grado $4$, sacamos que $\Q[\sqrt{2}, \sqrt{5}] = \Q[\beta^2]$ pues  ya hab\'iamos visto una inclus\'on y sus grados sobre $\Q$ son iguales.
		
		Para continuar, entonces tenemos la torre $\Q \subsetneq \Q[\beta^2] \subset \Q[\beta]$ \textcolor{red}{, veamos que las inclusiones son estrictas!} 
		
		\textcolor{red}{En pos de esto supongamos que $\beta \in \Q[\beta^2] = \Q[\sqrt{2}, \sqrt{5}]$, luego:
		\begin{equation*}
			\beta = a + b\sqrt{2} + c\sqrt{5} + d\sqrt{10}
		\end{equation*}
		Notemos que las matrices de multiplicar por $\beta, \sqrt{2}, \sqrt{5}, \sqrt{10}$ en la base $\sett{1, \sqrt{2}, \sqrt{5}, \sqrt{10}}$ resultan respectivamente (esto es cuentitas):
		\begin{equation*}
		\begin{aligned}
		m_{\beta} = & \left( 
		\begin{array}{cccc}
		a & b & c & d \\
		2b & a & 2d & c \\
		5c & 5d & a & b \\
		10 & 5c & 2b & a
		\end{array}
		\right) \\
		m_{\sqrt{2}} = & \left(
		\begin{array}{cccc}
		0 & 1 & 0 & 0 \\
		2 & 0 & 0 & 0 \\
		0 & 0 & 0 & 1 \\
		0 & 0 & 2 & 0
		\end{array}
		\right) \\
		m_{\sqrt{5}} = & \left(
		\begin{array}{cccc}
		0 & 0 & 1 & 0 \\
		0 & 0 & 0 & 1 \\
		5 & 0 & 0 & 0 \\
		0 & 5 & 0 & 0
		\end{array}
		\right) \\
		m_{\sqrt{10}} = & \left(
		\begin{array}{cccc}
		0 & 0 & 0 & 1 \\
		0 & 0 & 2 & 0 \\
		0 & 5 & 0 & 0 \\
		10 & 0 & 0 & 0
		\end{array}
		\right) 
		\end{aligned}
		\end{equation*}
		Luego debe valer:
		\begin{equation*}
			Tr_{\quotient{\Q[\sqrt{2}, \sqrt{5}]}{\Q}}(\beta) = a + b Tr_{\quotient{\Q[\sqrt{2}, \sqrt{5}]}{\Q}}(\sqrt{2}) + c Tr_{\quotient{\Q[\sqrt{2}, \sqrt{5}]}{\Q}}(\sqrt{5}) + d Tr_{\quotient{\Q[\sqrt{2}, \sqrt{5}]}{\Q}}(\sqrt{10}) 
		\end{equation*}	
		De lo que concluimos que $a=0$. Apliquemos el mismo razonamiento para $\sqrt{2}\beta, \sqrt{5}\beta$ y $\sqrt{10}\beta$; no obstante, al solo interesarnos el t\'ermino de la diagonal vamos a escribir el resto de los coeficientes con * pues no afectan el c\'alculo de la traza. Continuando con $\sqrt{2}\beta$
		\begin{equation*}
		\begin{aligned}
		m_{\sqrt{2}\beta} = & \left( 
		\begin{array}{cccc}
		2b & * & * & * \\
		* & 2b & * & * \\
		* & * & 2b & * \\
		* & * & * & 2b
		\end{array}
		\right)
		\end{aligned}
		\end{equation*}
		De lo que concluimos que:
		\begin{equation*}
		Tr_{\quotient{\Q[\sqrt{2}, \sqrt{5}]}{\Q}}(\sqrt{2}\beta) = 2b + c Tr_{\quotient{\Q[\sqrt{2}, \sqrt{5}]}{\Q}}(\sqrt{10}) + 2d Tr_{\quotient{\Q[\sqrt{2}, \sqrt{5}]}{\Q}}(\sqrt{5}) 
		\end{equation*}
		De lo que concluimos que $b=0$. Continuando con $\sqrt{5}\beta$:
		\begin{equation*}
		\begin{aligned}
		m_{\sqrt{5}\beta} = & \left( 
		\begin{array}{cccc}
		5c & * & * & * \\
		* & 5c & * & * \\
		* & * & 5c & * \\
		* & * & * & 5c
		\end{array}
		\right)
		\end{aligned}
		\end{equation*}
		De lo que concluimos que:
		\begin{equation*}
		Tr_{\quotient{\Q[\sqrt{2}, \sqrt{5}]}{\Q}}(\sqrt{5}\beta) = 5c + 5d Tr_{\quotient{\Q[\sqrt{2}, \sqrt{5}]}{\Q}}(\sqrt{2})
		\end{equation*}
		De lo que concluimos que $c=0$. Finalizando con $\sqrt{10}\beta$:
		\begin{equation*}
		\begin{aligned}
		m_{\sqrt{10}\beta} = & \left( 
		\begin{array}{cccc}
		10d & * & * & * \\
		* & 10d & * & * \\
		* & * & 10d & * \\
		* & * & * & 10d
		\end{array}
		\right)
		\end{aligned}
		\end{equation*}
		De lo que concluimos que:
		\begin{equation*}
		Tr_{\quotient{\Q[\sqrt{2}, \sqrt{5}]}{\Q}}(\sqrt{10}\beta) = 10d
		\end{equation*}
		De lo que concluimos que $d=0$. Probamos que $\beta \not\in \Q[\beta^2]$ y las inclusiones eran estrictas. Como $f = x^2 - \beta^2$ es un polinomio m\'onico que anula a $\beta$ y $f \in \Q[\beta^2][X]$ podemos concluir que $1 < [\Q[\beta]:\Q[\beta^2]] \leq 2$, o sea que la extensi\'on es cuadr\'atica. 
		}   Por lo tanto por grados sabemos que $[\Q[\beta], \Q] = 8$.
		
		Para concluir este punto, recordemos adem\'as que como $\Q$ es de caracter\'istica 0 todas nuestras extensiones son separabales y eso implica que $Hom(\quotient{\Q[\beta]}{\Q}) = Hom\left(\quotient{\Q[\beta]}{\Q[\beta^2]}\right) \times Hom\left(\quotient{\Q[\beta^2]}{\Q}\right)$. \textcolor{red}{Similarmente al punto anterior, ya probamos que $\beta$ no se encuentra en ninguna subextensi\'on de $\quotient{\Q[\beta]}{\Q}$ (esto es consecuencia de la cuenta anterior) por lo que por el teorema de correspondencia de Galois no es fijado por ning\'un elemento de $Hom(\quotient{\Q[\beta]}{\Q})$. En consecuencia, $\sett{\sigma_i(\beta)}_i$ es exactamente $\sigma(\beta)$ para cada una de los 8 morfismos que son base de $Hom(\quotient{\Q[\beta]}{\Q})$} y entonces de lo que dedujimos en la te\'orica:
		
		\begin{equation*}
			f\left(\beta, \Q\right) = \prod\limits_{\psi \in Hom\left(\quotient{\Q[\beta^2]}{\Q}\right)}{\psi \left(f\left(\beta, \Q[\beta^2]\right)\right)}
		\end{equation*}
		
		Como hab\'iamos visto que $f\left(\beta, \Q[\beta^2]\right) = x^2 - \beta^2 = p$ entonces:
		
		\begin{equation*}
			\begin{aligned}
				\sigma_1\left(x^2 - \left(10 + 2\sqrt{5} + 5\sqrt{2} + \sqrt{10}\right)\right) = & \left(x^2 - \left(10 -5\sqrt{2} + 2\sqrt{5} - \sqrt{10}\right)\right) \\
				\sigma_2\left(x^2 - \left(10 + 2\sqrt{5} + 5\sqrt{2} + \sqrt{10}\right)\right) = & \left(x^2 - \left(10 +5\sqrt{2} - 2\sqrt{5} - \sqrt{10}\right)\right) \\
				\sigma_3\left(x^2 - \left(10 + 2\sqrt{5} + 5\sqrt{2} + \sqrt{10}\right)\right) = & \left(x^2 - \left(10 +5\sqrt{2} + 2\sqrt{5} + \sqrt{10}\right)\right)\\
				\sigma_4\left(x^2 - \left(10 + 2\sqrt{5} + 5\sqrt{2} + \sqrt{10}\right)\right) = & \left(x^2 - \left(10 -5\sqrt{2} - 2\sqrt{5} + \sqrt{10}\right)\right)
			\end{aligned}
		\end{equation*}
		
		De lo que deducimos que:
		
		\begin{equation*}
		\begin{aligned}
		f \left(\beta, \Q\right) = & \left(x^2 - \left(10 -5\sqrt{2} + 2\sqrt{5} - \sqrt{10}\right)\right)\left(x^2 - \left(10 +5\sqrt{2} - 2\sqrt{5} - \sqrt{10}\right)\right) \\
		& \left(x^2 - \left(10 +5\sqrt{2} + 2\sqrt{5} + \sqrt{10}\right)\right)\left(x^2 - \left(10 -5\sqrt{2} - 2\sqrt{5} + \sqrt{10}\right)\right) \\
		= & x^8 - 40x^6 + 440x^4 - 1600 x^2 +1600 \qquad \text{ si hice bien las cuentas}
		\end{aligned}
		\end{equation*}
		
		Y sabemos que es el minimal (adem\'as de por todos los teoremas) porque es m\'onico, anula a $\beta$ (se ve) y es del grado correcto.
		
		\item Si recopilamos un poco lo que fuimos calculando notemos que llegamos a la conclusi\'on que las 8 ra\'ices del minimal son $\pm\sqrt{\left(2 \pm \sqrt{2}\right)\left(5 \pm \sqrt{5}\right)}$, veamos que todas est\'an en $\Q[\beta]$ y para eso es claro que basta verlo para $\sqrt{\left(2 \pm \sqrt{2}\right)\left(5 \pm \sqrt{5}\right)}$.
		
		Sea $\alpha_1 = \sqrt{\left(2 - \sqrt{2}\right) \left(5 +\sqrt{5}\right)}$, luego $\alpha_1\beta = \sqrt{\left(2^2 - 2\right)\left(5 + \sqrt{5}\right)^2} = \sqrt{2}\left(5 + \sqrt{5}\right) \in \Q[\sqrt{2}, \sqrt{5}] = \Q[\beta]$. Similarmente sea $\alpha_2 = \sqrt{\left(2 + \sqrt{2}\right) \left(5 -\sqrt{5}\right)}$ y vemos que $\alpha_2\beta = 2\sqrt{5}\left(2+\sqrt{2}\right) \in \Q[\beta]$; y $\alpha_3 = \sqrt{\left(2 - \sqrt{2}\right) \left(5 -\sqrt{5}\right)}$ que se ve que $\alpha_3\beta = 2\sqrt{2}\sqrt{5} \in \Q[\beta]$. Luego conlcu\'imos que $\pm \alpha_1, \pm \alpha_2, \pm \alpha_3 \in \Q[\beta]$ de lo que deducimos que las $8$ ra\'ices de $f\left(\beta, \Q\right)$ est\'an en $\Q[\beta]$m, lo que dice que $\Q[\beta]$ es el cuerpo de descomposici\'on de $f$ y por ende es Galois.
		
		\item Sea $\sigma \in Gal\left(\quotient{\Q[\beta]}{\Q}\right)$ un automorfismo que mande $\beta$ a $\alpha_1 := \alpha$, luego por ser automorfismo sabemos que:
		
		\begin{equation*}
			\left(2 + \sigma(\sqrt{2})\right)\left(5 + \sigma(\sqrt{5})\right) = \sigma\left(\left(2 + \sqrt{2}\right)\left(5 + \sqrt{5}\right)\right) = \sigma(\beta^2) = \sigma(\beta)^2 = \alpha^2 = \left(2 - \sqrt{2}\right) \left(5 +\sqrt{5}\right)
		\end{equation*}
		
		Y concluimos que $\sigma(\sqrt{2}) = -\sqrt{2}$ y $\sigma(\sqrt{5}) = \sqrt{5}$. Luego:
		
		\begin{equation*}
			\sigma(\alpha)\alpha = \sigma(\alpha)\sigma(\beta) = \sigma(\alpha \beta) = \sigma(\sqrt{2})(5 + \sigma(\sqrt{5})) = - \sqrt{2}(5 + \sqrt{5}) = -\alpha\beta
		\end{equation*}
		
		Por lo que $\sigma(\alpha)= -\beta$	y tenemos el siguiente diagrama de la acci\'on de $\sigma$:
		
		\begin{equation*}
			\beta \rightarrow \alpha \rightarrow - \beta \rightarrow - \alpha \rightarrow \beta
		\end{equation*}
		
		Y $\sigma$ es un elemento de orden 4. Similarmente vamos a analizar $\tau\in Gal\left(\quotient{\Q[\beta]}{\Q}\right)$ tal que $\tau(\beta) = \alpha_2:= \omega = \sqrt{\left(2 + \sqrt{2}\right) \left(5 -\sqrt{5}\right)}$.
		
		\begin{equation*}
		\left(2 + \tau(\sqrt{2})\right)\left(5 + \tau(\sqrt{5})\right) = \tau\left(\left(2 + \sqrt{2}\right)\left(5 + \sqrt{5}\right)\right) = \tau(\beta^2) = \sigma(\beta)^2 = \omega^2 = \left(2 + \sqrt{2}\right) \left(5 -\sqrt{5}\right)
		\end{equation*}
		
		Y concluimos que $\tau(\sqrt{2}) = \sqrt{2}$ y $\tau(\sqrt{5}) = -\sqrt{5}$. Luego:
		
		\begin{equation*}
		\tau(\omega)\omega = \tau(\omega)\tau(\beta) = \tau(\omega\beta) = 2\tau(\sqrt{5})(2 + \tau(\sqrt{2})) = - 2\sqrt{5}(2 + \sqrt{2}) = - \omega\beta
		\end{equation*}
		
		Por lo que $\tau(\omega)= -\beta$	y tenemos el siguiente diagrama de la acci\'on de $\tau$:
		
		\begin{equation*}
		\beta \rightarrow \omega \rightarrow - \beta \rightarrow - \omega \rightarrow \beta
		\end{equation*}
		
		Y $\tau$ es un elemento de orden 4. A su vez notamos que $\sigma, \tau$ generan y que $\sigma^2 = \tau^2$ con lo que nos faltar\'ia ver si conmutan; en pos de eso
		
		\begin{equation*}
			\begin{aligned}
				\dfrac{\sigma(\beta)^2}{\beta^2} = & \dfrac{2 - \sqrt{2}}{2 + \sqrt{2}} = & \left(\dfrac{2 - \sqrt{2}}{\sqrt{2}}\right)^2 = & \left(\sqrt{2} - 1\right)^2 \\
				\dfrac{\tau(\beta)^2}{\beta^2} = & \dfrac{5 - \sqrt{5}}{5 + \sqrt{5}} = & \left(\dfrac{5 - \sqrt{5}}{2\sqrt{5}}\right)^2 = & \left(\dfrac{\sqrt{5} - 1}{2}\right)^2 \\				
			\end{aligned}
		\end{equation*}
		
		Lo que podemos deducir que (si notamos de igual manera a la extension):
		
		\begin{equation*}
			\begin{aligned}
				\sigma(\beta) = \left(\sqrt{2} - 1\right)\beta \\
				\tau(\beta) = \dfrac{\sqrt{5} - 1}{2} \beta
			\end{aligned}
		\end{equation*}
		
		Luego si vemos como actuan las composiciones en $\beta$:
		
		\begin{equation*}
			\begin{aligned}
				\beta \xrightarrow{\tau} \dfrac{\sqrt{5} - 1}{2} \beta \xrightarrow{\sigma} \dfrac{\sqrt{5} - 1}{2}\left(\sqrt{2} - 1\right)\beta \\
				\beta \xrightarrow{\sigma} \left(\sqrt{2} - 1\right)\beta \xrightarrow{\tau} \dfrac{\sqrt{5} - 1}{2}\left(\sqrt{2} - 1\right)\beta 
			\end{aligned}
		\end{equation*}
		
		Por lo que conclu\'imos que una presentaci\'on de $Gal\left(\quotient{\Q[\beta]}{\Q}\right)$ es
		
		 $\sett{\sigma, \tau \tq \sigma^4 = \tau^4 = Id \ , \ \sigma^2 = \tau^2 \ , \ \sigma\tau = \tau\sigma } \simeq \Z_4 \times \Z_2$
		
		\item Este punto es muy parecido a la pr\'actica asi que notemos que tenemos la torre de extensiones $\Q \subseteq \Q[\sqrt{10}] \subseteq \Q[\sqrt{10 + \sqrt{10}}]$ y veamos que son inclusiones estrictas.
		
		Es simple y ya sabemos que $\sqrt{10} \not \in \Q$, luego asumamos que $\sqrt{10 + \sqrt{10}} \in \Q[\sqrt{10}] $; entonces existen $a,b \in \Q$ tal que $\sqrt{10 + \sqrt{10}} = a + b \sqrt{10}$. Elevando al cuadrado e igualando t\'ermino a t\'ermino de la base $\sett{1, \sqrt{10}}$ de $\Q[\sqrt{10}]$ tenemos el sistema:
		
		\begin{equation*}
			\begin{aligned}
				2ab = 1 \\
				a^2 + 10b^2 = 10
			\end{aligned}
		\end{equation*}
		
		Lo que lleva a que $b = \frac{1}{2a}$ y $a$ cumpla $a^4 + \frac{10}{4}-10a^2 =0$ que podemos verificar que no tiene soluciones racionales; luego las inclusiones son estrictas.
		
		Pero como $p_1 = x^2 - 10 \in \Q[X]$ y $p_2 = x^2 - \left(10 + \sqrt{10}\right) \in \Q[\sqrt{10}]$ son polinomios m\'onicos y anulan a $\sqrt{10}, \sqrt{10+\sqrt{10}}$ respectivamente podemos concluir que ambas extensiones son cuadr\'aticas y esos son los polinomios minimales; es m\'as, tenemos en conclusi\'on que el grado de la extensi\'on $\quotient{\Q[\sqrt{10 + \sqrt{10}}]}{\Q}$ es $4$.
		
		Si hacemos lo mismo que en el primer punto, podemos verificar entonces que las 4 ra\'ices del polinomio minimal resultan $\pm \sqrt{10 \pm \sqrt{10}}$ y todas se encuentran en $\Q[\sqrt{10 + \sqrt{10}}]$ pues:
		
		\begin{equation*}
		\sqrt{10 + \sqrt{10}} \sqrt{10 - \sqrt{10}} = 3 \sqrt{10} = 3 \left[\left(\sqrt{10 + \sqrt{10}}\right)^2 -10 \right] \in \Q[\sqrt{10 + \sqrt{10}}]
		\end{equation*}
		
		Por lo que $\quotient{\Q[\sqrt{10 + \sqrt{10}}]}{\Q}$ es Galois!
		
		Sea $\sigma \in Gal\left(\quotient{\Q[\sqrt{10 + \sqrt{10}}]}{\Q}\right)$ tal que $\sigma(\sqrt{10 + \sqrt{10}}) = \sqrt{10 - \sqrt{10}}$, luego:
		
		\begin{equation*}
			\begin{aligned}
				\sigma\left(\sqrt{10 - \sqrt{10}}\right) = & \dfrac{3 \left[\left(\sigma(\sqrt{10 + \sqrt{10}})\right)^2 -10 \right] }{\sigma(\sqrt{10+\sqrt{10}})} \\
				= &  \dfrac{3 \left[\left(\sqrt{10 - \sqrt{10}}\right)^2 -10 \right] }{\sqrt{10-\sqrt{10}}} \\
				= &  \dfrac{-3 \sqrt{10}}{\dfrac{3\sqrt{10}}{\sqrt{10 + \sqrt{10}}}} \\
				= & -\sqrt{10 + \sqrt{10}}
			\end{aligned}
		\end{equation*}
		
		Luego tenemos la siguiente acci\'on de $\sigma$ sobre $\sqrt{10+\sqrt{10}}$
		
		\begin{equation*}
			\gamma \rightarrow \sqrt{10 - \sqrt{10}} \rightarrow - \gamma \rightarrow - \sqrt{10 - \sqrt{10}} \rightarrow \gamma
		\end{equation*}
		
		Y $\sigma$ resulta un generador de order $4$ de $Gal\left(\quotient{\Q[\sqrt{10 + \sqrt{10}}]}{\Q}\right)$; luego $Gal\left(\quotient{\Q[\sqrt{10 + \sqrt{10}}]}{\Q}\right) \simeq \Z_4$
		
		\item \textcolor{red}{Notemos que si tomamos ahora $\sigma \circ \tau$ entonces $\Q[\beta]^{\ip{\sigma \circ \tau}} = \Q[\gamma]$. Como $\left(\sigma \circ \tau \right)^2 = \sigma^2 \circ \tau^2 = \sigma^4 = Id$ entonces sabemos que $\ip{\sigma \circ \tau} < \Z_4 \times \Z_2$ tiene \'indice 4; luego por el teorema de correspondencia de Galois sabemos que una subextensi\'on de $\quotient{\Q[\beta]}{\Q}$ de orden 4 es $\Q[\gamma]$}.\qed
		
	\end{itemize}
	
	\item[Ejercicio 3] 
	
	\begin{itemize}
		\item Primero veamos si podemos reducir a $f$ a una forma m\'as tratable. Para eso recordemos que si $p = ax^4  + bx^3 + cx^2 + dx + e$ entonces $\widetilde{p}(x) = \frac{p(x- \frac{b}{4a})}{a}$ cumple que no tiene t\'ermino c\'ubico y que si $\widetilde{p}$ es irreducible entonces $p$ lo es (esto lo vimos tanto en la pr\'actica como la te\'orica).
		
		Luego:
		
		\begin{equation*}
			\widetilde{f} = f\left(x +1\right) = x^4 - x^2 + 1 \qquad \text{ si hice bien las cuentas}
		\end{equation*}
		
		Con lo que llegamos a la hermosa conclusi\'on que $f$ es irreducible en $K$ si y s\'olo si $\Phi_{12}$ es irreducible en $K[X]$.
		
		Para el caso $K= \Q$ ya sabemos que todos los polinomios ciclot\'omicos son irreducibles asi que $f$ lo es.
		
		Ahora si $char(K) =2$ entonces notemos que $\Phi_{12} (x) = x^4 - x^2 + 1 = \left(x^2 - x + 1\right)^2$ por lo que $f$ no seria irreducible
		
		Si $char(K) = 3$ notemos que $\Phi_{12} (x) = x^4 - x^2 + 1 = \left(x^2 + 1\right)^2$ por lo que $f$ no ser\'ia irreducible.
		
		\textcolor{red}{Para finalizar, veamos el siguiente lema:
		\begin{lemma}
			Sobre $F$ finito con caracter\'istica $p$ con $p \not\vert n$ son equivalentes:
			\begin{enumerate}
				\item $\Phi_n$ es irreducible
				\item $[{F[\xi]}:{F}] = \phi(n)$
				\item $Gal\left(\quotient{F[\xi]}{F}\right) \simeq \left(\quotient{\Z}{n\Z}\right)^*$
				\item $p$ es un generador de $\left(\quotient{\Z}{n\Z}\right)^*$
			\end{enumerate}
		\end{lemma}
		\begin{proof}
			En efecto, como $\xi$ es una ra\'iz de $\Phi_n$ entonces $f(\xi, F) = \Phi_n$ pues es irreducible, m\'onico y anula; luego $[F[\xi]:F] = \varphi(n) = gr(\Phi_n)$. Como $\quotient{F[\xi]}{F}$ es Galois ($p \not\vert n$), sabemos que $\varphi(n) = \abs{Gal\left(\quotient{F[\xi]}{F}\right)} = \abs{(\quotient{\Z}{n\Z}} = \varphi(n)$, luego la inyecci\'on $Gal\left(\quotient{F[\xi]}{F}\right) \inc \left(\quotient{\Z}{n\Z}\right)^*$ es un isomorfismo de grupos pues de \'Algebra 2 sabemos que un homomorfismo inyectivo (esto lo sabemos de la te\'orica) entre grupos de igual orden es isomorfismo. Finalmente de la teor\'ia de cuerpos finitos sabemos que siempre:
			\begin{equation*}
				p = f\left(\xi, F\right) = \prod\limits_{i=1}^{k}{x-\xi^{p^i}}
			\end{equation*}
			donde $k = [{F[\xi]}:{F}]$. Luego sabemos que $ p \vert \Phi_n$ y van a ser iguales si y s\'olo si sus ra\'ices son iguales, es decir cuando $\xi^{a} \cong \xi^{p^i}$ para todo $a \leq n$ tal que $(a, n) = 1$. Esto vimos que pasa si y s\'olo si $a \cong p^i \ \mod(n)$ para alg\'un $i$, lo que pasa si y s\'olo si $p$ es generador de $\left(\quotient{\Z}{n\Z}\right)^*$. \qed
		\end{proof}
		}
		Luego, $\Phi_n$ es irreducible en $\mathbb{F}_p$ con $p \not\vert n$ si y s\'olo si $\left(\quotient{\Z}{n\Z}\right)^*$ es c\'iclico, que de \'Algebra 2 sabemos que pasa si y s\'olo si $n \in \sett{2, 4, l^s, 2l^s}$ con $l$ primo impar o $s \geq 1$; como $12$ no entra en ninguna de esas posibilidades sabemos que $\Phi_{12}$ es reducible en $\mathbb{F}_p$ para $p \neq 2,3$.
		
		Resumiendo vimos que $f$ es irreducible solo si $K = \Q$. \qed
		
		\item Notemos que podemos representar este problema con el siguiente diamante:
		
		\[
		\begin{tikzcd}
		& \Q[\sqrt[p]{5}, \xi_n] & \\
		\Q[\sqrt[p]{5}] \arrow{ru}{\varphi(n)?}  & & \Q[\xi_n] \arrow{lu} \\ \quad
		& \Q \arrow{ru}{\varphi(n)} \arrow{lu}{p}&
		\end{tikzcd}
		\]
		
		Pues $\Phi_n$ es irreducible en $\Q[\sqrt[p]{5}]$ si y s\'olo si el polinomio minimal de $\xi_n$ en $\Q[\sqrt[p]{5}]$ es $\Phi_n$ si y s\'olo si $[\Q[\sqrt[p]{5}, \xi_n]:\Q[\sqrt[p]{5}]] = \varphi(n)$. 
		
		Luego, sabemos que si $p \not\vert \varphi(n)$ entonces como los grados inferiores del diamante son coprimos $[\Q[\sqrt[p]{5}, \xi_n]:\Q[\sqrt[p]{5}]] = \varphi(n)$ y $\Phi_n$ resulta irreducible en $\Q[\sqrt[p]{5}]$.
		
		\textcolor{red}{Supongamos ahora que $p \vert \varphi(n)$,  veamos que $\Q[\sqrt[p]{5}] \cap \Q[\xi] = \Q$.}
		
		Supongamos que $\sqrt[p]{5} \in \Q[\xi_n]$, luego existe la torre de extensiones $\Q \subsetneq \Q[\sqrt[p]{5}] \subsetneq \Q[\xi_n]$ y por el teorema de correspondencia de Galois $Gal\left(\quotient{ \Q[\sqrt[p]{5}] }{\Q}\right)$ resulta un subgrupo de $Gal(\quotient{\Q[\xi_n]}{\Q})$. Recordemos que vale:
		
		\begin{equation*}
			Gal(\quotient{\Q[\xi_n]}{\Q}) \simeq \left(\quotient{\Z}{n\Z}\right)^*
		\end{equation*}
		
		Y este grupo resulta abeliano, luego todo subgrupo de un grupo abeliano resulta normal al menos. Por lo tanto, eso implicar\'ia que $Gal\left(\quotient{ \Q[\sqrt[p]{5}] }{\Q}\right)$ es un subgrupo normal, lo que implicar\'ia, por el teorema de correspondencia de Galois nuevamente, que la extensi\'on $(\quotient{ \Q[\sqrt[p]{5}] }{\Q}$ es normal. Esto es absurdo pues $p > 2$ y ya lo vimos varias veces en la materia. 
		
		\textcolor{red}{Luego, como claramente adem\'as $\xi \not\in \Q[\sqrt[p]{5}] \subset \R$ entonces concluimos que $\Q[\sqrt[p]{5}] \cap \Q[\xi] = \Q$. Recordemos el siguiente teorema:
		\begin{theorem}
				Sean $E,L$ extensiones de $F = E \cap L$ tal que $\quotient{E}{F}$ es Galois, luego $\quotient{EL}{L}$ es Galois y adem\'as $Gal(\quotient{EL}{L}) \simeq Gal(\quotient{E}{F})$
		\end{theorem}
		Luego entonces como $\quotient{Q[\xi]}{\Q}$ es Galois y $\Q[\sqrt[p]{5}] \cap \Q[\xi] = \Q$, usando el teorema sabemos que $\quotient{\Q[\xi, \sqrt[p]{5}]}{\Q[\sqrt[p]{5}]}$ es Galois y que $\varphi(n) = \abs{Gal(\quotient{Q[\xi]}{\Q})} = \abs{Gal(\quotient{\Q[\xi, \sqrt[p]{5}]}{\Q[\sqrt[p]{5}]})} = [\Q[\sqrt[p]{5}, \xi_n]:\Q[\sqrt[p]{5}]]$ pues adem\'as es Galois.
	}
		
		Luego, $[\Q[\sqrt[p]{5}, \xi_n]:\Q[\sqrt[p]{5}]] = \varphi(n)$ para este caso tambi\'en y entonces conclu\'imos que $\Phi_n$ resulta irreducible en $Q[\sqrt[p]{5}]$ para todo $p>2$.\qed 
		
	\end{itemize}
	
	\item[Ejercicio 4]
	
	\begin{itemize}
		\item Sea $S = \sett{K \text{ subextensiones de } \quotient{\C}{\Q} \tq \alpha \not \in K}$ y notemos que es no vac\'io pues $\Q \in S$; sea entonces $\F$ una cadena totalmente ordenada ee $S$ y tomemos $L = \Bigcup{K \in \F}{K}$, afirmo que $L$ es cota superior de $S$.
		
		En efecto, si $K \in \F$ entonces $K \subset L$ por definici\'on, $L$ es subextensi\'on de $\quotient{\C}{\Q}$ pues todos los $K \in \F$ lo son y $\alpha \not \in L$ pues si lo estuviese, entonces $\alpha \in K_{\F}$ para alg\'un $K_{\F} \in \F$ lo que es absurdo pues $K_{\F} \in S$.
	
		Luego, por el lema de Zorn, existe un elemento maximal $K \in S$ que resulta la subextensi\'on buscada.
		
		\item Sea $x \in \C - K$ y supongamos que es trascendente sobre $K$, lo esto implica que tenemos la torre de extesiones $\Q \subsetneq K \subsetneq K(x) \subset \C$ y $K(x)$ es una subextensi\'on de $\quotient{\C}{\Q}$. Afirmo que no tiene a $\alpha$.
		
		\textcolor{red}{En efecto, supongamos que $\alpha = \dfrac{f(x)}{g(x)}$ con $f,g \in K[X]$, luego por ser algebraico sobre $\Q$ sabemos que existe $h \in \Q[X]$ tal que $h(\alpha) = 0$. En particular, si notamos $h_0, \dots, h_n \in \Q$ a los coeficientes de $h$ esto es equivalente a:
		\begin{equation*}
			h_0 + h_1\dfrac{f(x)}{g(x)} + \dots + h_n\dfrac{f^n(x)}{g^n(x)} = 0
		\end{equation*}
		Luego:
		\begin{equation*}
		\begin{aligned}
			0 = & h_0g^n(x) + h_1g^{n-1}(x)f(x) + \dots + h_nf^n(x) \\
			=& \left(h_0g^n + h_1g^{n-1}f + \dots + h_n f^n\right) (x)
		\end{aligned}
		\end{equation*}
		Y como $h_0g^n + h_1g^{x-1}f + \dots + h_nf^n \in K[X]$ concluimos que $x$ es algebraico sobre $K$, lo que es absurdo.
	}
		Como $K$ era maximal uno concluir\'ia que $K = K(x)$ lo que es absurdo, luego $\quotient{\C}{K}$ es algebraico.
	
		\item Sea $\quotient{M}{K}$ una subextension finita de $\quotient{\C}{K}$ luego, por un lado, sabemos que $\alpha \in M$ por la maximalidad de $K$. 
		
		\textcolor{red}{Consideremos $\quotient{\widetilde{M}}{K}$ la clasura normal de $\quotient{M}{K}$, como $\quotient{\C}{K}$ es separable y $\C$ es algebraicamente cerrado sabemos que $\quotient{\widetilde{M}}{K}$ es una subextensi\'on normal y separable de $\quotient{\C}{K}$, luego Galois. Por ser Galois y $\alpha \not\in K$ existe $\sigma \in Gal\left(\quotient{\widetilde{M}}{K}\right)$ tal que $\sigma(\alpha) \neq \alpha$. Luego $\quotient{\widetilde{M}^{\ip{\sigma}}}{K}$ es una extensi\'on tal que $\alpha \not \in \widetilde{M}^{\ip{\sigma}}$. Por la maximalidad de $K$ concluimos que $K = \widetilde{M}^{\ip{\sigma}}$ y entonces por el teorema de correspondencia de Galois esto es si y s\'olo si $\ip{\sigma} = Gal\left(\quotient{\widetilde{M}}{K}\right)$. Concluimos que $\quotient{M}{K}$ es c\'iclico al ser subextensi\'on de $\quotient{\widetilde{M}}{K}$ que lo es.}
	
		\item \textcolor{red}{Dijimos que no valia} \qed
	
	\end{itemize}
	
	\item[Ejercicio 5]
	
	\begin{itemize}
		\item No se aun
		
		\item Veamos que es falso! Sea $f = x^4 -4x +2$ y su resolvente $g(x) = x^3 -8x -16$. Ambos resultan irreducibles pues $f$ es Einsenstein con $p=2$ y su resolvente pues $\widehat{g} = x^3 +2x +4$ es irreducible en $\mathbb{F}_5$. 
		
		Como el discriminante de $g$ es $-4864 \not \in \Q^2$ entonces concluimos que $Gal(g) \simeq Gal(\quotient{L}{\Q[\alpha]}) \simeq S_3$ con $\alpha$ ra\'iz de $f$ y $Gal(f) = Gal(\quotient{L}{\Q}) \simeq S_4$.
		
		Supongamos que existe $M$ subextensi\'on de grado $2$, luego tendr\'iamos la torre $\Q \subseteq M \subseteq \Q[\alpha] \subseteq L$ (pues $[\Q[\alpha] : \Q] =4$ pues $f$ irreducible) lo que implica que $S_3 \simeq Gal(\quotient{L}{\Q[\alpha]}) \leq Gal(\quotient{L}{M}) \leq Gal(\quotient{L}{\Q}) \simeq S_4$ (donde $\leq$ implica ser subgrupo) y veamos que eso no puede ser.
		
		En efecto, sea $H$ un subgrupo entre $S_3$ y $S_4$ con esas caracter\'isticas, entonces $H$ es transitivo (por ser un grupo de Galois), tiene un $n-1$ ciclo y una transposici\'on lo que implica que $H \simeq S_4$.
		
		Conclu\'imos que tal subextensi\'on cuadr\'atica $M$ no puede existir. \qed
		
		 \qed
	\end{itemize}
	
\end{enumerate}
	
\end{document}