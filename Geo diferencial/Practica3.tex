\documentclass[12pt, a4paper]{amsart}

\usepackage{picture}
\usepackage{amsthm}
\usepackage{amssymb}
\usepackage{amsmath, exercise}
\usepackage[activeacute, spanish]{babel}
\usepackage[utf8]{inputenc}
\usepackage{mathpazo}
\usepackage[pdftex]{color, graphicx}
\usepackage{fancyhdr}
\usepackage{multicol}
\usepackage{tikz-cd}
\usepackage{mathrsfs}
\usepackage{pstricks}
\usepackage[colorlinks=true]{hyperref}
\usetikzlibrary{calc}
\usetikzlibrary{matrix}

\usepackage{enumitem}

\usepackage[margin=0.8in]{geometry}

\usepackage{pgfplots}
\usepgfplotslibrary{colormaps}
\pgfplotsset{compat=1.13}
\usepackage{tikz}
\usetikzlibrary{arrows}
\usetikzlibrary{patterns}

\usepackage{booktabs}
\usepackage{float}
\usepackage{multirow}
\usepackage{subfig}

\newtheorem{teo}{Teorema}[]
\newtheorem{lem}[teo]{Lema}
\newtheorem{prop}[teo]{Proposición}
\newtheorem{cor}[teo]{Corolario}

\newtheorem*{propnonumber}{Proposición}

\theoremstyle{definition}
\newtheorem{prob}[teo]{Problema}
\newtheorem{conj}[teo]{Conjectura}
\newtheorem{defn}[teo]{Definición}
\newtheorem{ax}[teo]{Axioma}
\newtheorem{ex}[teo]{Ejemplo}
\newtheorem{exer}[teo]{Ejercicio}

\newtheorem*{obs}{Observación}
\newtheorem*{com}{Comentario}
\newtheorem*{defnonumber}{Definición}

\newcommand{\bd}[1]{\mathbf{#1}}  % for bolding symbols
\newcommand{\CC}{\mathbb{C}}
\newcommand{\RR}{\mathbb{R}}      % for Real numbers
\newcommand{\ZZ}{\mathbb{Z}}      % for Integers
\newcommand{\NN}{\mathbb{N}}
\newcommand{\QQ}{\mathbb{Q}}
\newcommand{\FF}{\mathbb{F}}
\newcommand{\mm}{\mathfrak{m}}
\newcommand{\col}[1]{\left[\begin{matrix} #1 \end{matrix} \right]}
\newcommand{\comb}[2]{\binom{#1^2 + #2^2}{#1+#2}}
\newcommand{\eps}{\varepsilon}
\newcommand{\norm}[1]{\left\| #1 \right\|}
\newcommand{\abs}[1]{\left\lvert #1 \right\rvert}
\newcommand{\pint}[1]{\left\langle #1 \right\rangle}
\newcommand{\tendsto}[1]{\xrightarrow{\smash{\raisebox{-2ex}{$\scriptstyle#1$}}}}
\newcommand*\diff{\mathop{}\!\mathrm{d}}
\renewcommand{\hom}{\mathrm{Hom}}

\renewcommand{\hom}{\mathrm{Hom}}
\let\oldemptyset\emptyset
\let\emptyset\varnothing
\DeclareMathOperator{\id}{id}
\DeclareMathOperator{\mcm}{mcm}
\DeclareMathOperator{\mcd}{mcd}
\DeclareMathOperator{\ord}{ord}
\DeclareMathOperator{\nil}{nil}
\DeclareMathOperator{\im}{im}
\DeclareMathOperator{\End}{End}
\DeclareMathOperator{\Aut}{Aut}
\DeclareMathOperator{\sg}{sg}
\DeclareMathOperator{\coker}{coker}
\DeclareMathOperator{\Obj}{Obj}
\DeclareMathOperator{\rank}{rk}
\DeclareMathOperator{\gr}{gr}
\DeclareMathOperator{\car}{car}
\DeclareMathOperator{\Nil}{Nil}
\DeclareMathOperator{\Spec}{Spec}
\DeclareMathOperator{\ev}{ev}
\DeclareMathOperator{\ann}{Ann}
\DeclareMathOperator{\Gal}{Gal}
\DeclareMathOperator{\HH}{H}
\DeclareMathOperator{\rg}{rg}
\def\acts{\curvearrowright}
\def\stca{\curvearrowleft}

\def\noteson{%
\gdef\note##1{\marginpar[##1]{##1}}}
\gdef\notesoff{\gdef\note##1{}}
\noteson

\renewcommand{\qed}{\hfill \mbox{\raggedright \rule{0.075in}{0.075in}}}
\renewcommand{\thefootnote}{[\arabic{footnote}]}

\usepackage{scrextend}% not needed with a KOMA-Script class, provides the
                      % `addmargin' environment

\usepackage[load-headings]{exsheets}
\DeclareInstance{exsheets-heading}{mylist}{default}{
  runin = true ,
  attach = {
    main[l,vc]number[l,vc](-2em,0pt) ; % 3em = indent of question body
    main[r,vc]points[l,vc](\linewidth+\marginparsep,0pt)
  }
}

\SetupExSheets{
  headings = mylist , % use the new headings instance
  headings-format = \textbf ,
  counter-format = qu ,
  counter-within = section
}


\usepackage{etoolbox}
% 3em = indent of question body :
\AtBeginEnvironment{question}{\addmargin[2em]{0em}}
\AtEndEnvironment{question}{\endaddmargin}

\usepackage{lipsum}

\begin{document}

\title{Geometría Diferencial -- 1er cuatrimestre 2017}
\author{}
% Remove command to get current date 
\date{}
\nocite{*}
%\begin{abstract}
%\end{abstract}
\maketitle
\begin{center}
\section*{Práctica 3: Subvariedades, inmersiones y submersiones}
\end{center}

\begin{question}
Sea $M$ una variedad de dimensión $d$ y sea $S\subseteq M$ un espacio con la topología subespacio. Probar que son equivalentes:
\begin{enumerate}[label=\textbf{\alph*.}]
\item Para cada $p\in S$ existe una carta local $(U,\phi)$ tal que $\phi(U)=V\times W$ donde $V\subseteq\RR^k$, $W\subseteq\RR^{d-k}$ son abiertos que contienen al origen y además se verifica que $\phi(U\cap S) = V\times\{0\}$.
\item $S$ es una variedad diferenciable de dimensión $k$ y la inclusión $S\hookrightarrow M$ es un embedding.
\end{enumerate}
En esta situación, diremos que $S$ es una \textit{subvariedad} de $M$ de dimensión $k$.
\end{question}

\begin{question}Sean $M,N$ variedades diferenciables de dimensión $m$ y $n$ respectivamente. Supongamos que $f:M\to N$ una función diferenciable.
\begin{enumerate}[label=\textbf{\alph*.}]
\item \textbf{Teorema del rango constante:}  Si $f$ tiene rango constante $k$, para cada $p\in M$ existen cartas $(U,\phi)$ de un entorno de $p$ y $(V,\psi)$ de un entorno de $f(p)$ de modo que $f(U)\subseteq V$ y se verifica que $$\psi\, f\, \phi^{-1} (x_1,\ldots,x_m) = (x_1,\ldots,x_k,0,\ldots,0).$$

\item Concluir que si $f:M\to N$ es una función diferenciable de rango constante $k$ entonces para todo $c\in N$ los conjuntos de nivel $f^{-1}(c)$ son subvariedades de $M$ de dimensión $m-k$.

\item Más aún, si $f:M\to N$ es una función diferenciable y $c\in N$ es un \textit{valor regular} (es decir, para todo punto $p\in f^{-1}(c)$ tenemos que $\diff_p f:T_pM\to T_{f(p)}N$ es sobreyectiva) entonces $f^{-1}(c)$ es una subvariedad de $M$ de dimensión $m-n$. Además, probar que $T_p(f^{-1}(c)) \simeq \ker(\diff_p f:T_pM\to T_cN)$ para todo $p\in f^{-1}(c)$.
\end{enumerate}
\end{question}

\begin{question}\vspace{-1.5em}\begin{enumerate}[label=\textbf{\alph*.}]
\item Probar que $S^n$ es una subvariedad de $\RR^{n+1}$.

\item Probar que el cono sin el vértice $$\{(x,y,z)\in\RR^3:x^2+y^2=z^2\}\smallsetminus\{(0,0,0)\}$$ es una subvariedad de $\RR^3$. ¿Qué pasa si consideramos todo el cono?

\item Probar que $\alpha:(0,2\pi)\to\RR^2$ definida por $\alpha(t)=(\cos(t),\sin(2t))$ es una inmersión pero no un embedding. Probar que la imagen de $\alpha$ no es una subvariedad de $\RR^2$.

\item Probar que la superficie de revolución generada por cualquier curva regular en el plano $xz$ alrededor del eje $z$ es una subvariedad de $\RR^3$.
\end{enumerate}
\end{question}

\begin{question}
Sean $M$ y $N$ dos variedades diferenciables y sean $TM$ y $TN$ sus variedades tangentes asociadas. Probar que si $f:M\to N$ es un embedding
entonces $\diff f:TM\to TN$ es un embedding. ¿Vale la recíproca?
\end{question}

\begin{question}
Sean $M,N$ variedades diferenciables y $f:M\to N$ una inmersión inyectiva. Probar que si $f$ es cerrada entonces $f$ es un embedding. Concluir, bajo estas hipótesis, que si adicionalmente $M$ es compacta o $f$ es propia, entonces $f$ es un embedding.
\end{question}

\begin{question}
Sea $M$ una variedad compacta de dimensión $d$ y sea $f:M\to\RR^d$ una función diferenciable. Probar que $f$ no es regular. Concluir que no hay un embedding de una variedad compacta de dimensión $d$ en $\RR^d$.
\end{question}

\begin{question}El objetivo de este ejercicio es probar que aunque no es cierto que toda subvariedad se consigue como el conjunto de nivel de un valor regular, localmente sí es cierto.
\begin{enumerate}[label=\textbf{\alph*.}]
\item La función $\mathbb{P}^1(\RR) \to \mathbb{P}^2(\RR)$ definida por $(x:y)\mapsto(x:y:0)$ es un embedding y así su imagen $S$ es una subvariedad de $\mathbb{P}^2(\RR)$. Probar que no existe ninguna función diferenciable $g:\mathbb{P}^2(\RR)\to \RR$ que tenga al $0$ como valor regular y tal que $g^{-1}(0)=S$.
\item Sea $S\subseteq M$ un subconjunto con la topología subespacio. Probar que $S$ es una subvariedad de dimensión $k$ de $M$ si y sólo si para cada punto $p\in S$ existe un entorno abierto $U\subseteq M$ tal que $U\cap S$ es el conjunto de nivel de una submersión $f:U\to\RR^{n-k}$.
\end{enumerate}
\end{question}

\begin{question}
Sean $M,N$ variedades diferenciables y $\pi:M\to N$ una submersión. Probar que $\pi$ es abierta. Más aún, probar que todo punto de $M$ está en la imagen de una sección local de $\pi$.
\end{question}

\begin{question}
Sean $X,Y$ variedades diferenciables y $f:X\to Y$ una submersión sobreyectiva. Supongamos que $Y$ es conexo y $f^{-1}(y)$ es conexo para todo $y\in Y$. Probar que $X$ es conexo.
\end{question}

\begin{question}
Sean $M$ y $N$ variedades de dimensión $d$, con $M$ compacta y $f:M\to N$ diferenciable.
\begin{enumerate}[label=\textbf{\alph*.}]
\item Probar que si $p\in N$ es valor regular, entonces $f^{-1}(p)$ es un conjunto finito.
\item Probar que la asignación $p\mapsto \#f^{-1}(p)$ es localmente constante (donde $p$ recorre valores regulares de $f$).
\end{enumerate}
\end{question}

\begin{question}
\vspace{-1.5em}
\begin{enumerate}[label=\textbf{\alph*.}]
\item Sea $(U, \phi_N)$ la carta de $S^2$ dada por la proyección estereográfica. Probar que un polinomio $P\in\CC[x]$ define una función diferenciable $f_P:S^2\to S^2$ cuya expresión local
$\phi_N\circ f_P \circ \phi_N^{-1}\colon \RR^2 \to \RR^2 $ se identifica con $P$.
\item Probar el Teorema Fundamental del Álgebra.
\end{enumerate}
\end{question}

\begin{question}
Sea $M$ una variedad de dimensión $d$, $U$ abierto en $M$ y $p\in U$ un punto del abierto. Sean $\phi_1,\dots, \phi_d :U\to \RR $ funciones diferenciables
tales que $\{\diff_p\phi_1,\dots, \diff_p\phi_d\}$ son linealmente independientes en $(T_pM)^*$. Probar que las funciones $\{\phi_i\}$ determinan una
carta de $M$ en un entorno de $p$.
\end{question}

\begin{question}
\vspace{-1.5em}
\begin{enumerate}[label=\textbf{\alph*.}]
\item
Sea $f:M\to N$  diferenciable, $\dim M\leq\dim N$. Sea $p\in M$ punto regular, y sea $(U,\phi)$ carta de $N$ con $f(p)\in U$.
Probar que un subconjunto de las funciones $\{\phi_i \circ f\}$ determina una carta de $M$ en un entorno de $p$.
\item
Sea $S\subset N$ una subvariedad y sea $f:M\to N$ diferenciable tal que $f(M)\subset S$. Probar que $f:M\to S$ es diferenciable.
\end{enumerate}
\end{question}
\begin{question}
Sea $f:M\to N$ una función diferenciable, y sea $X$ otra variedad diferencial.

\begin{enumerate}[label=\textbf{\alph*.}]

\item Probar que si $f$ es una inmersión entonces para toda $g:X\to M$ función continua tal que $f\circ g$ es diferenciable se puede deducir que
$g$ es diferenciable.


\item Probar que si $f$ es un embedding entonces para toda $g:X\to M$ función tal que $f\circ g$ es diferenciable se puede deducir que $g$ es continua
y diferenciable.

\item Probar que si $f$ es una submersión sobreyectiva entonces para toda $g:N\to X$ función tal que $g\circ f$ es diferenciable se puede deducir
que $g$ es diferenciable.
\end{enumerate}
\end{question}

\begin{question}
Sean $f:M \to N$ una función diferencial y $S$ una subvariedad de $N$ tales que para todo
$p \in f^{-1}(S)$ vale que $\diff_pf(T_pM)+T_{f(p)}S=T_{f(p)}N$. Probar que $f^{-1}(S)$ es una subvariedad de $M$
de dimensión $\dim M -\dim N +\dim S$.
\end{question}

\begin{question}
Decimos que dos subvariedades $S_1, S_2$ de $M$ son \textit{transversales}, y lo notamos $S_1 \pitchfork S_2$, si para
todo $p\in S_1\cap S_2$ tenemos que $T_p S_1 + T_p S_2= T_p M$. 
\begin{enumerate}[label=\textbf{\alph*.}]
\item Probar que si $S_1\pitchfork S_2$, entonces $S_1 \cap S_2$ es una subvariedad de $M$. 
\item Probar además que si $S_1\cap S_2\neq\emptyset$ entonces $$\dim S_1 \cap S_2=\max\{\dim S_1 +\dim S_2 - \dim M,0\}.$$
\end{enumerate}
\end{question}


\end{document}