\documentclass[12pt, a4paper]{amsart}

\usepackage{picture}
\usepackage{amsthm}
\usepackage{amssymb}
\usepackage{amsmath, exercise}
\usepackage[activeacute, spanish]{babel}
\usepackage[utf8]{inputenc}
\usepackage{mathpazo}
\usepackage[pdftex]{color, graphicx}
\usepackage{fancyhdr}
\usepackage{multicol}
\usepackage{tikz-cd}
\usepackage{mathrsfs}
\usepackage{pstricks}
\usepackage[colorlinks=true]{hyperref}
\usetikzlibrary{calc}
\usetikzlibrary{matrix}

\usepackage{enumitem}

\usepackage[margin=0.8in]{geometry}

\usepackage{pgfplots}
\usepgfplotslibrary{colormaps}
\pgfplotsset{compat=1.13}
\usepackage{tikz}
\usetikzlibrary{arrows}
\usetikzlibrary{patterns}

\usepackage{booktabs}
\usepackage{float}
\usepackage{multirow}
\usepackage{subfig}

\newtheorem{teo}{Teorema}[]
\newtheorem{lem}[teo]{Lema}
\newtheorem{prop}[teo]{Proposición}
\newtheorem{cor}[teo]{Corolario}

\newtheorem*{propnonumber}{Proposición}

\theoremstyle{definition}
\newtheorem{prob}[teo]{Problema}
\newtheorem{conj}[teo]{Conjectura}
\newtheorem{defn}[teo]{Definición}
\newtheorem{ax}[teo]{Axioma}
\newtheorem{ex}[teo]{Ejemplo}
\newtheorem{exer}[teo]{Ejercicio}

\newtheorem*{obs}{Observación}
\newtheorem*{com}{Comentario}
\newtheorem*{defnonumber}{Definición}

\newcommand{\bd}[1]{\mathbf{#1}}  % for bolding symbols
\newcommand{\CC}{\mathbb{C}}
\newcommand{\RR}{\mathbb{R}}      % for Real numbers
\newcommand{\ZZ}{\mathbb{Z}}      % for Integers
\newcommand{\NN}{\mathbb{N}}
\newcommand{\QQ}{\mathbb{Q}}
\newcommand{\FF}{\mathbb{F}}
\newcommand{\mm}{\mathfrak{m}}
\newcommand{\col}[1]{\left[\begin{matrix} #1 \end{matrix} \right]}
\newcommand{\comb}[2]{\binom{#1^2 + #2^2}{#1+#2}}
\newcommand{\eps}{\varepsilon}
\newcommand{\norm}[1]{\left\| #1 \right\|}
\newcommand{\abs}[1]{\left\lvert #1 \right\rvert}
\newcommand{\pint}[1]{\left\langle #1 \right\rangle}
\newcommand{\tendsto}[1]{\xrightarrow{\smash{\raisebox{-2ex}{$\scriptstyle#1$}}}}
\newcommand*\diff{\mathop{}\!\mathrm{d}}
\renewcommand{\hom}{\mathrm{Hom}}

\renewcommand{\hom}{\mathrm{Hom}}
\let\oldemptyset\emptyset
\let\emptyset\varnothing
\DeclareMathOperator{\id}{id}
\DeclareMathOperator{\mcm}{mcm}
\DeclareMathOperator{\mcd}{mcd}
\DeclareMathOperator{\ord}{ord}
\DeclareMathOperator{\nil}{nil}
\DeclareMathOperator{\im}{im}
\DeclareMathOperator{\End}{End}
\DeclareMathOperator{\Aut}{Aut}
\DeclareMathOperator{\sg}{sg}
\DeclareMathOperator{\coker}{coker}
\DeclareMathOperator{\Obj}{Obj}
\DeclareMathOperator{\rank}{rk}
\DeclareMathOperator{\gr}{gr}
\DeclareMathOperator{\car}{car}
\DeclareMathOperator{\Nil}{Nil}
\DeclareMathOperator{\Spec}{Spec}
\DeclareMathOperator{\ev}{ev}
\DeclareMathOperator{\ann}{Ann}
\DeclareMathOperator{\Gal}{Gal}
\DeclareMathOperator{\HH}{H}
\DeclareMathOperator{\rg}{rg}
\def\acts{\curvearrowright}
\def\stca{\curvearrowleft}

\def\noteson{%
\gdef\note##1{\marginpar[##1]{##1}}}
\gdef\notesoff{\gdef\note##1{}}
\noteson

\renewcommand{\qed}{\hfill \mbox{\raggedright \rule{0.075in}{0.075in}}}
\renewcommand{\thefootnote}{[\arabic{footnote}]}

\usepackage{scrextend}% not needed with a KOMA-Script class, provides the
                      % `addmargin' environment

\usepackage[load-headings]{exsheets}
\DeclareInstance{exsheets-heading}{mylist}{default}{
  runin = true ,
  attach = {
    main[l,vc]number[l,vc](-2em,0pt) ; % 3em = indent of question body
    main[r,vc]points[l,vc](\linewidth+\marginparsep,0pt)
  }
}

\SetupExSheets{
  headings = mylist , % use the new headings instance
  headings-format = \textbf ,
  counter-format = qu ,
  counter-within = section
}


\usepackage{etoolbox}
% 3em = indent of question body :
\AtBeginEnvironment{question}{\addmargin[2em]{0em}}
\AtEndEnvironment{question}{\endaddmargin}

\usepackage{lipsum}

\begin{document}

\title{Geometría Diferencial -- 1er cuatrimestre 2017}
\author{}
% Remove command to get current date 
\date{}
\nocite{*}
%\begin{abstract}
%\end{abstract}
\maketitle
\begin{center}
\section*{Práctica 4: Campos tangentes, flujos y grupos de Lie}
\end{center}
\vspace{1em}
\textsl{\textbf{Campos tangentes y flujos}}
\vspace{1em}

\begin{question}
Sea $M$ una variedad, $p\in M$ y $v\in T_pM$ un vector tangente. Probar que existe un campo $X\in\mathfrak{X}(M)$ tal que $X(p)=v$.
\end{question}

\begin{question}
Sea $M$ una variedad, $p\in M$ un punto y $X$ un campo definido en un entorno de $p$ tal que $X(p)\neq 0$. Probar que existe una carta $(U,\phi)$ de $p$ tal que $\left.X\right|_U = \dfrac{\partial}{\partial\phi_1}$.
\end{question}

\begin{question}
Sea $M$ una variedad, $p\in M$ un punto y $X_1,\cdots,X_k$ campos definidos en un entorno de $p$ tales que $\{X_1(p),\cdots,X_k(p)\}$ son linealmente independientes en $T_pM$. Probar que existe un entorno $U$ de $p$ tal que $\{X_1(q),\cdots,X_k(q)\}$ son linealmente independientes para todo $q\in U$.
\end{question}

\begin{question}
Sea $M$ una variedad y sean $X,Y\in\mathfrak{X}(M)$ dos campos.	El \textit{corchete de Lie} se define como $$[X,Y](f) = X(Y(f)) - Y(X(f))$$ para cada $f\in\mathscr{C}^\infty(M,\RR)$ pensando a los campos como derivaciones. Probar que $[X,Y]$ es un campo tangente. Probar que si $(U,\phi)$ es una carta de $M$ en la que $$X=\displaystyle\sum_{i=1}^n X^i \,\dfrac{\partial}{\partial\phi_i},\;\;\; Y = \displaystyle\sum_{i=1}^n Y^i\, \dfrac{\partial}{\partial\phi_i}$$ entonces $$[X,Y] = \displaystyle\sum_{i=1}^n\left(\displaystyle\sum_{j=1}^n X^j\,\dfrac{\partial Y^i}{\partial\phi_j} - Y^j\,\dfrac{\partial X^i}{\partial\phi_j}\right)\dfrac{\partial}{\partial\phi_i}.$$
\end{question}

\begin{question}
Dados campos $X,Y$ en una variedad $M$ y funciones diferenciables $f,g:M\to\RR$, probar la fórmula de Lie-Rinehart: $$[fX,gY]=fg[X,Y]+f X(g)Y  - gY(f)X.$$
\end{question}

\begin{question}
Probar que, bajo la identificación $T_I\mathrm{GL}_n(\RR)\simeq \mathrm{M}_n(\RR)$ se tiene que $$[X_A,X_B]=X_{[A,B]} = X_{AB-BA}$$ donde $X_A(M)=AM$ es el campo tangente asociado a $A$.
\end{question}

\begin{question}
Calcular las curvas integrales y el grupo uniparamétrico definidos por el campo $X$ en cada uno de los siguientes casos
\begin{enumerate}[label=\textbf{\alph*.}]
\item $M=\RR^2$ y $X(a,b)=b\dfrac{\partial}{\partial x} + a\dfrac{\partial}{\partial y}$.
\item $M=\RR^2$ y $X(a,b)=a\dfrac{\partial}{\partial x} - b\dfrac{\partial}{\partial y}$.
\item $M=S^2$ y $X(a,b,c) = (c-b)\dfrac{\partial}{\partial x} + a\dfrac{\partial}{\partial y} - a\dfrac{\partial}{\partial z}$.
\item $M=\mathrm{GL}_n(\RR)$ y $X(A)=AB\in T_A\mathrm{GL}_n(\RR)\simeq\RR^{n\times n}$ con $B\in\mathrm{M}_n(\RR)$.
\end{enumerate}
\end{question}

\begin{question}
Sea $\gamma$ una curva integral de un campo $X$ definido sobre $M$. Probar que si $\dot{\gamma}(t)=0$ para algún $t$, entonces $\gamma$ es constante.
\end{question}

\begin{question}
Sea $M$ una variedad, $\eps>0$ fijo y $X\in\mathfrak{X}(M)$ un campo tal que para todo $p\in M$ la curva integral de $X$ que pasa por $p$ está definida en $(-\eps,\eps)$. Probar que $X$ es completo.
\end{question}

\begin{question}\vspace{-1.5em}
\begin{enumerate}[label=\textbf{\alph*.}]
\item Sea $M$ una variedad compacta y $X$ es un campo tangente sobre $M$. Probar que $X$ es completo.
\item Probar que si $\mathrm{Sop}(X)=\overline{\{p\in M : X(p)\neq 0\}}$ es compacto, $X$ debe ser completo.
\end{enumerate}
\end{question}

\begin{question}
Sea $M$ una variedad y sea $\theta:\RR\times M\to M$ una acción suave. Para cada $t\in\RR$ notamos $\theta_t(p)=\theta(t,p)$ (con esta notación, el hecho de que $\theta$ sea una acción suave nos dice que $\theta_t$ es un difeomorfismo para cada $t$ y que $\theta_t\circ\theta_s=\theta_{t+s}$). Para cada $f:M\to\RR$ diferenciable definimos $$X(f)(p) = \displaystyle\lim_{t\to 0}\frac{f(\theta_t(p))-f(p)}{t}.$$ Probar que $X$ es un campo diferenciable y que su flujo es exactamente $\theta$.
\end{question}

\begin{question}
Sea $f:M\to N$ diferenciable. Un par de campos $X\in\mathfrak{X}(M)$, $Y\in\mathfrak{X}(N)$ se dicen \textit{$f$-relacionados} si $Y(f(p))=\diff_pf(X(p))$ para todo $p\in M$. Lo notaremos $X\sim_f Y$.
\begin{enumerate}[label=\textbf{\alph*.}]
\item Probar que si $X\sim_f Y$ y $Y\sim_g Z$ entonces $X\sim_{g\circ f} Z$.
\item Probar que si $X_1\sim_f Y_1$	 y $X_2\sim_f Y_2$ entonces $[X_1,X_2]\sim_f [Y_1,Y_2]$.
\end{enumerate}
\end{question}

\begin{question}
Sean $f:M\to N$ diferenciable, y sean $X\in\mathfrak{X}(M)$, $Y\in\mathfrak{X}(N)$ dos campos $f$-relacionados. Probar que si $\gamma:(a,b)\to M$ es una curva integral de $X$ entonces $f\circ\gamma$ es una curva integral de $Y$.
\end{question}


\begin{question}
(Teorema de Ehresmann) Sea $f:M\to U\subseteq \RR$ diferenciable, propia, sobreyectiva y tal que todo $t\in U$ es valor regular.
\begin{enumerate}[label=\textbf{\alph*.}]
\item Probar que existe $X\in\mathfrak{X}(M)$ tal que $X\sim_f\dfrac{\diff}{\diff t}$. ¿Es único?
\item Sea $X$ como en el ítem anterior. Probar que $X$ es completo.
\item Probar que $f^{-1}(u)$ es difeomorfo a $f^{-1}(u')$ para todos $u,u'\in U$.
\end{enumerate}
\end{question}

\textsl{\textbf{Grupos de Lie}}
\vspace{1em}


\begin{question}
Sea $G$ un grupo de Lie y notemos $L_g:G\to G$ el difeomorfismo $L_g(h)=gh$. Decimos que un campo $X\in\mathfrak{X}(G)$ es \textit{invariante a izquierda} si para todos $g,h\in G$ se tiene $(\diff_h L_g)(X(h)) = X(gh) = X(L_g(h))$.
\begin{enumerate}[label=\textbf{\alph*.}]
\item Probar que si dos campos invariantes a izquierda coinciden en un punto entonces coinciden en todo $G$.
\item Probar que si $v\in T_gG$, existe un único campo invariante a izquierda $X$ tal que $X(g)=v$.
\item Deducir que hay un isomorfismo entre $T_eG$ y el espacio de campos invariantes a izquierda.
\item Probar que todo grupo de Lie es paralelizable.
\end{enumerate}
\end{question}

\begin{question}
Un \textit{álgebra de Lie} sobre un cuerpo $k$ es un $k$-espacio vectorial $V$ provisto de una aplicación bilineal $[\cdot,\cdot]:V\times V\to V$ que satisface las siguientes dos condiciones:
\begin{itemize}
\item \textbf{Antisimetría:} $[X,Y] = -[Y,X]\;\; \forall\;X,Y\in V$.
\item \textbf{Identidad de Jacobi:} $[X,[Y,Z]]+[Y,[Z,X]] + [Z,[X,Y]]=0 \;\;\forall\;X,Y,Z\in V$.
\end{itemize}
Probar que si $M$ es una variedad, $\mathfrak{X}(M)$ es un álgebra de Lie sobre $\RR$.
\end{question}

\begin{question}
Probar que si $G$ es un grupo de Lie y $X,Y$ son campos invariantes a izquierda, entonces $[X,Y]$ es invariante a izquierda. Deducir que $\mathfrak{g}=T_eG$ hereda una estructura de álgebra de Lie.	Diremos que $\mathfrak{g}$ es el álgebra de Lie de $G$.
\end{question}

\begin{question}
Sea $G$ un grupo de Lie y tomemos $v\in T_eG$. Sea $X_v$ el único campo invariante a izquierda con $X(e)=v$ y denotemos por $\Phi_v^t$ el grupo uniparamétrico definido por $X_v$. Sea $\exp:T_eG\to G$ dada por $\exp(v)=\Phi_v^1(e)$. Probar que si $f:G\to H$ es un morfismo de grupos de Lie entonces $f\circ\exp = \exp\circ \diff_e f$.
\end{question}

\begin{question}
Probar que $\mathrm{GL}_n(\RR)$ es un grupo de Lie y que, identificando $T_I\mathrm{GL}_n(\RR)\simeq M_n(\RR)$, la exponencial $\exp:M_n(\RR)\to\mathrm{GL}_n(\RR)$ definida como en el ejercicio anterior coincide con la exponencial matricial $$\exp(A) = I+A + \dfrac{A^2}{2!}+\dfrac{A^3}{3!} + \cdots.$$ ¿Hay alguna relación entre $\exp(A+B)$ y $\exp(A)\exp(B)$?
\end{question}

\begin{question}
Sean $G,H$ grupos de Lie conexos y sean $\mathfrak{g},\mathfrak{h}$ sus álgebras de Lie respectivamente. Sea $f:G\to H$ un morfismo de grupos de Lie. Son equivalentes:
\begin{itemize}
\item $f$ es sobreyectiva y $\ker\, f$ es discreto.
\item El morfismo inducido $f_*:\mathfrak{g}\to\mathfrak{h}$ es un isomorfismo.
\end{itemize}
\end{question}

\end{document}