\documentclass[11pt]{article}

\usepackage{amsfonts}
\usepackage{amsmath,accents,amsfonts, amssymb, mathrsfs }
\usepackage{tikz-cd}
\usepackage{graphicx}
\usepackage{syntonly}
\usepackage{color}
\usepackage{mathrsfs}
\usepackage[spanish]{babel}
\usepackage[latin1]{inputenc}
\usepackage{fancyhdr}
\usepackage[all]{xy}
\usepackage[at]{easylist}
\usepackage[colorlinks=true,linkcolor=blue,urlcolor=black,bookmarksopen=true]{hyperref}

\usepackage{bookmark}

\topmargin-2cm \oddsidemargin-1cm \evensidemargin-1cm \textwidth18cm
\textheight25cm


\newcommand{\B}{\mathcal{B}}
\newcommand{\Cont}{\mathcal{C}}
\newcommand{\F}{\mathcal{F}}
\newcommand{\inte}{\mathrm{int}}
\newcommand{\A}{\mathcal{A}}
\newcommand{\C}{\mathbb{C}}
\newcommand{\Q}{\mathbb{Q}}
\newcommand{\Z}{\mathbb{Z}}
\newcommand{\inc}{\hookrightarrow}
\renewcommand{\P}{\mathcal{P}}
\newcommand{\R}{{\mathbb{R}}}
\newcommand{\N}{{\mathbb{N}}}
\newcommand\tq{~:~}
\newcommand{\dual}[1]{\left(#1\right)^{\ast}}
\newcommand{\ortogonal}[1]{\left(#1\right)^{\perp}}
\newcommand{\ddual}[1]{\left(#1^{\ast}\right)^{\ast}}
\newcommand{\x}[3]{#1_#2^#3}
\newcommand{\xx}[4]{#1_#3#2_#4}
\newcommand\dd{\,\mathrm{d}}
\newcommand{\norm}[1]{\left\lVert#1\right\rVert}
\newcommand{\abs}[1]{\left\lvert#1\right\rvert}
\newcommand{\ip}[1]{\left\langle#1\right\rangle}
\renewcommand\tt{\mathbf{t}}
\newcommand\nn{\mathbf{n}}
\newcommand\bb{\mathbf{b}}                      % binormal
\newcommand\kk{\kappa}
\newcommand{\sett}[1]{\left\lbrace#1\right\rbrace}
\newcommand{\interior}[1]{\accentset{\smash{\raisebox{-0.12ex}{$\scriptstyle\circ$}}}{#1}\rule{0pt}{2.3ex}}
\fboxrule0.0001pt \fboxsep0pt
\newcommand{\Bigcup}[2]{\bigcup\limits_{#1}{#2}}
\newcommand{\Bigcap}[2]{\bigcap\limits_{#1}{#2}}
\newcommand{\Bigprod}[2]{\prod\limits_{#1}{#2}}
\newcommand{\Bigcoprod}[2]{\coprod\limits_{#1}{#2}}
\newcommand{\Bigsum}[2]{\sum\limits_{#1}{#2}}
\newcommand{\BigsumA}[3]{ \sideset{}{^#2}\sum\limits_{#1}{#3}}
\newcommand{\Biglim}[2]{\lim\limits_{#1}{#2}}
\newcommand{\quotient}[2]{{\raisebox{.2em}{$#1$}\left/\raisebox{-.2em}{$#2$}\right.}}
\DeclareMathOperator{\rank}{ran}
\DeclareMathOperator{\graf}{Gr}
\DeclareMathOperator{\ball}{ball}

\def \le{\leqslant}	
\def \ge{\geqslant}
\def\noi{\noindent}
\def\sm{\smallskip}
\def\ms{\medskip}
\def\bs{\bigskip}
\def \be{\begin{enumerate}}
	\def \en{\end{enumerate}}
\def\deck{{\rm Deck}}
\def\Tau{{\rm T}}

\newtheorem{mytheorem}{Theorem}
%

\newtheorem{theorem}{Teorema}
\numberwithin{theorem}{subsection}
\newtheorem{lemma}[theorem]{Lema}

\newtheorem{proposition}[theorem]{Proposici\'on}

\newtheorem{corollary}[theorem]{Corolario}


\newenvironment{proof}[1][Demostraci\'on]{\begin{trivlist}
		\item[\hskip \labelsep {\bfseries #1}]}{\end{trivlist}}
\newenvironment{definition}[1][Definici\'on]{\begin{trivlist}
		\item[\hskip \labelsep {\bfseries #1}]}{\end{trivlist}}
\newenvironment{example}[1][Ejemplo]{\begin{trivlist}
		\item[\hskip \labelsep {\bfseries #1 }]}{\end{trivlist}}
\newenvironment{remark}[1][Observaci\'on]{\begin{trivlist}
		\item[\hskip \labelsep {\bfseries #1}]}{\end{trivlist}}
\newenvironment{declaration}[1][Afirmaci\'on]{\begin{trivlist}
		\item[\hskip \labelsep {\bfseries #1}]}{\end{trivlist}}


\newcommand{\qed}{\nobreak \ifvmode \relax \else
	\ifdim\lastskip<1.5em \hskip-\lastskip
	\hskip1.5em plus0em minus0.5em \fi \nobreak
	\vrule height0.75em width0.5em depth0.25em\fi}

\newcommand{\twopartdef}[4]
{
	\left\{
	\begin{array}{ll}
		#1 & \mbox{ } #2 \\
		#3 & \mbox{ } #4
	\end{array}
	\right.
}

\newcommand{\threepartdef}[6]
{
	\left\{
	\begin{array}{lll}
		#1 & \mbox{ } #2 \\
		#3 & \mbox{ } #4 \\
		#5 & \mbox{ } #6
	\end{array}
	\right.
}

\tikzset{commutative diagrams/.cd,
	mysymbol/.style={start anchor=center,end anchor=center,draw=none}
}
\newcommand\Center[2]{%
	\arrow[mysymbol]{#2}[description]{#1}}

\newcommand*\circled[1]{\tikz[baseline=(char.base)]{
		\node[shape=circle,draw,inner sep=2pt] (char) {#1};}}


\makeatletter
\newcommand{\xRightarrow}[2][]{\ext@arrow 0359\Rightarrowfill@{#1}{#2}}
\makeatother

\usepackage[auto-label]{exsheets}[2015/07/04]

\DeclareInstance{exsheets-heading}{myblock}{default}{
	attach = {
		main[l,vc]title[l,vc](0pt,0pt) ;
		main[r,vc]points[l,vc](\marginparsep,0pt)
	} ,
	title-post-code = \bfseries\space
	a la pregunta \GetQuestionProperty{ref}{\CurrentQuestionID}
}

\usepackage{hyperref}

\def\questionName{Ejercicio}
\def\solutionName{Soluci\'on}

\SetupExSheets{
    headings=block-subtitle,
    question/name=\questionName,
    question/pre-hook = \addcontentsline{toc}{subsection}{
        \questionName\space\thequestion.
        },
    solution/name=\solutionName,
    solution/pre-hook = \addcontentsline{toc}{subsection}{
    	\solutionName\space\thequestion.
		}
    }



\begin{document}

	\pagestyle{empty}
	\pagestyle{fancy}
	\fancyfoot[CO]{\slshape \thepage}
	\renewcommand{\headrulewidth}{0pt}
	
	
	
	\centerline{\bf An\'alisis Funcional}
	\centerline{\sc Final}
	\centerline{\sc Axel Sirota}
	
	\tableofcontents
	\newpage
	
\section{Problemas}

\begin{question}
	Probar que los siguientes conjuntos tienen una estructura de variedad diferencial, exhibir un atlas y hallar la dimensi\'on en cada caso.
	\begin{enumerate}
		\item Un espacio vectorial $V$ sobre $\R$.
		\item La esfera $S^n\subseteq\R^{n+1}$.
		\item El espacio proyectivo $\mathbb{P}^n(\R) = S^n / \sim$, donde $x\sim y$ si $x=- y$.
		\item El toro $T_n = S^1\times\cdots\times S^1$.
		\item El cilindro $\{(x,y,z)\in\R^3:x^2+y^2=1\}$.
		\item El grupo general lineal $\mathrm{GL}_n(\R)=\{A\in\mathrm{M}_n(\R):\det(A)\neq 0\}$.
		\item El grupo especial lineal $\mathrm{SL}_n(\R)=\{A\in\mathrm{M}_n(\R):\det(A)=1\}$.
		\item El grupo ortogonal $\mathrm{O}_n(\R) = \{A\in\mathrm{M}_n(\R):A\cdot A^\intercal=1\}$.
		\item El grupo especial ortogonal $\mathrm{SO}_n(\R)=\{A\in\mathrm{O}_n(\R) : \det(A)=1\}$.
	\end{enumerate}
\end{question}

\begin{solution}
	Vayamos por partes:
	
	\begin{enumerate}
		\item Sea $d$ la dimensi\'on de $V$, luego fijada una base $\B=\sett{v_1, \dots , v_d} \subset V$ existe un isomorfismo $T:V \rightarrow \R^d$ dado por $T(x^1v_1 + \dots + x^dv_d) = (x^1, \dots , x^d)$, por lo tanto notemos que $\A = \sett{(V, T)}$ es una atlas sobre $V$ por \'algebra lineal. Finalmente como $V \simeq \R^d$ como espacios topol\'ogicos entonces $V$ es Haussdorf y tiene base numerable. Conclu\'imos que $V$ es una variedad de dimensi\'on $d$
		
		\item Por ser subespacio de $\R^{n+1}$ sabemos que $S^{n}$ es Haussdorf y tiene base numerable por lo que debemos exhibir un conjunto de cartas diferencialmente compatibles. 
		
		Pensando en esto, consideremos $f_i^+ = (x^1, \dots, x^{i-1}, \sqrt{1-(x^1)^2 - \dots - (x^n)^2} ,x^{i+1}, \dots , x^{n})$ y $f_i^- = -f_i^+$, $U_i^+ = \sett{x \in B_1(0) \tq x_i > 0}$, $U_i^- = \sett{x \in B_1(0) \tq x_i < 0}$, $\pi_i(x^1, \dots, x^{n+1}) = \left(x^1, \dots , x^{i-1}, x^{i+1}, \dots, x^{n+1}\right)$ y finalmente $V_i^+ = f_i^+(U_i^+)$ y $V_i^- = f_i^- (U_i^-)$. Afirmo que $\A = \sett{(V_i^+, \pi_i), (V_i^-, \pi_i) \tq 1 \leq i \leq n+1}$ es un atlas para $S^n$.
		
		En efecto, fijado $V^j_i$ ($j \in \sett{+,-}$) entonces $\pi_i: V^j_i \rightarrow U^j_i$ es homeomorfismo con inversa $f^j_i$ y finalmente donde tenga sentido como todas las funciones son suaves la composicion $\pi_k \circ f^j_i$ es suave.
		  
		\item Notemos que si $\B$ es la base contable de bolas de centro y radio racional en $\R^{n+1}$ entonces sabemos que $\B \cap S^{n}$ es una base contable de $S^n$. Finalmente es un ejrcicio de topolog\'ia ver que $q(\B \cap S^n)$ es una base contable de $\mathbb{P}^n$.
		
		Por otro lado, si $[x] \neq [y] \in \mathbb{P}^n$ entonces notemos que $q^{-1}([x]) = \sett{x,-x}, q^{-1}([y]) = \sett{y,-y}$, como $S^n$ es Haussdorf entonces existe $U \ni x, V \ni y$ tal que $U \cap V = \emptyset$. Notemos que como $y \neq \pm x$ entonces existe $\tilde{V} \subset V, \tilde{U} \subset U$ tal que $\sett{x,-x} \subset \tilde{U} \cup -\tilde{U}, \sett{y,-y} \subset \tilde{V} \cup -\tilde{V}$ pero que ambos son entornos disjuntos. Conclu\'imos que  $[U] \ni [x], [V] \ni [y]$ son entornos abiertos disjuntos pues su preimagen es la uni\'on de abiertos.
		
		Para encontrar un atlas, sea $U_i = \sett{x \in \mathbb{P}^{n} \tq x^i \neq 0}$ que es un entorno abierto de $\mathbb{P}^n$ y consideremos $\phi_i : U_i \rightarrow \R^{n}$ dado por $\phi_i([x^1:\dots:x^i\dots:x^{n+1}]) = (\frac{x^1}{x^i}, \dots, \frac{x^{i-1}}{x^i}, \frac{x^{i+1}}{x^i}, \dots, \frac{x^{n+1}}{x^i})$ con inversa $\psi_i : \R^n \rightarrow U_i$ dada por $\psi_i(x^1, \dots, x^{i-1}, x^{i+1}, \dots, x^n) = [x^1 : \dots : x^{i-1}:1:x^{i+1}: \dots : x^n]$. Es muy simple ver usando la propiedad universal del cociente y a mano para el otro lado que estas son inversas y continuas, por lo cual son homeomorfismos y falta ver la compatibilidad suave.
		
		En pos de esto, sea $i < j$ y consideremos $\phi_i \circ \psi_j : \phi_j(U_i \cap U_j) \rightarrow \phi_i(U_i \cap U_j)$:
		\small
		\begin{equation*}
			\begin{aligned}
			\phi_i \circ \psi_j (x^1, \dots, x^{i-1}, x^{i}, x^{i+1}, \dots, x^{j-1}, x^{j+1}, \dots , x^n) = & \phi_i([x^1:\dots:x^{i-1}:x^{i}:x^{i+1}:\dots:x^{j-1}:1:x^{j+1}:\dots:x^n]) \\
			= &  (\frac{x^1}{x^i}, \dots, \frac{x^{i-1}}{x^i}, \frac{x^{i+1}}{x^i}, \dots, \frac{x^j-1}{x^i}, \frac{1}{x^i}, \frac{x^{j+1}}{x^i}, \dots, \frac{x^n}{x^i})
			\end{aligned}
		\end{equation*}
		\normalsize
		Que es trivialmente diferenciable.
		
		\item  Va a ser trivial consecuencia que producto de variedades es variedad 
		
		\item Idem antes
		
		\item Es trivial que $\mathrm{GL}_n(\R) \subset \mathrm{M}_n(\R)$ pues $det: \mathrm{M}_n(\R) \rightarrow \R$ es continua y $\mathrm{GL}_n(\R) = det^{-1}(\R^{\ast})$, por lo tanto esto quedara probado cuando en el proximo ejercicio veamos que los abiertos son variedades de la misma dimensi\'on.
		
		\item Asumamos el resultado del problema 12 por ahora, entonces afirmo que si consideramos $\det: \mathrm{GL}_n(\R) \rightarrow \R$ entonces afirmo que $1$ es un valor regular de $\det$, con lo que conclu\'imos que $\mathrm{SL}_n(\R)$ es variedad de dimensi\'on $n^2 -1$
		
		En efecto, si $A \in \mathrm{GL}_n(\R) $ notemos que:
		
		\begin{equation*}
			\begin{aligned}
				\det (A) = & \sum_{i=1}^{n}{(-1)^{i+1}a_{i,1}M_{i,1}} \\
				\Longrightarrow \quad \dfrac{\partial \det (A)}{\partial a_{i,j}} = & {(-1)^{i+1}M_{i,1}}
			\end{aligned}
		\end{equation*}
		
		Por lo tanto conclu\'imos que $D(\det)(A) = 0$ si y s\'olo si $M_{i,1} = 0$ para todo $i$ si y s\'olo si $\det (A) = 0$
		
		\item Sea $f:  \mathrm{M}_n(\R) :  \mathrm{S}_n(\R)$ dada por $f(A) = AA^t$ que es trivialmente una aplicacion diferenciable entre espacios vectoriales; nuevamente si asumimos el ejercicio 12 para probar que $ \mathrm{O}_n(\R)$ es una variedad de dimensi\'on $n^2 - \dfrac{n(n+1)}{2} = \dfrac{n(n-1)}{2}$ tenemos que ver que $Id_n$ es una valor regular de $f$.
		
		En efecto, sea $A \in f^{-1}(Id_n)$ y $B \in  \mathrm{M}_n(\R)$, luego:
		
		\begin{equation*}
			\begin{aligned}
				d_A(f)(B) = & \lim\limits_{h \rightarrow 0 }{\dfrac{f(A+hB) - f(A)}{h}} \\
				= & \lim\limits_{h \rightarrow 0 }{\dfrac{(A+hB)(A+hB)^t - AA^T}{h}} \\
				= & \lim\limits_{h \rightarrow 0 }{\dfrac{(A+hB)(A^t+hB^t) - AA^T}{h}} \\
				= & \lim\limits_{h \rightarrow 0 }{\dfrac{AA^t+hBA^t + hAB^t + h^2BB^t - AA^T}{h}} \\
				= & AB^t + BA^t
			\end{aligned}
		\end{equation*}
		
		Es claro que esta aplicaci\'on es suryectiva pues dado $C \in \mathrm{S}_n(\R)$ entonces si $B = \dfrac{CA}{2}$ entonces $d_A(f)(B) = AB^t + BA^t = \frac{AA^tC^t}{2} + \frac{CAA^t}{2} = \frac{C+C^t}{2} = \frac{2C}{2} = C$; luego $Id_n$ es un valor regular de $f$.
		
		\item Es claro que $\mathrm{SL}_n(\R) \subset \mathrm{O}_n(\R)$ es un conjunto abierto pues es $\det^{-1}(\R^{\ast}) \cap \mathrm{O}_n(\R)$, luego si asumimos el ejercicio siguiente conclu\'imos que $\mathrm{SL}_n(\R)$ es una variedad de dimensi\'on $\dfrac{n(n-1)}{2}$
		
	\end{enumerate}
	.\qed
	
\end{solution}

\begin{question}
	Sea $M$ una variedad diferencial de dimensi\'on $d$ y sea $U\subseteq M$ abierto. 
	\begin{enumerate}
		\item Probar que $U$ hereda una estructura de variedad con $\dim(U)=\dim(M)$ y que la inclusi\'on $U\hookrightarrow M$ es diferenciable para esa estructura.
		\item Probar que un subconjunto $S\subseteq M$ (con la topolog\'ia subespacio) es una variedad de dimensi\'on $d$ si y s\'olo si $S$ es abierto en $M$.
	\end{enumerate}
\end{question}

\begin{solution}
	Vayamos de a uno:
	
	\begin{enumerate}
		\item Dado que $U \subset M$ es subespacio entonces de topolog\'ia sabemos que $U$ es Haussdorf y admite base numerable; es m\'as notemos que si $\A = \sett{(U_i, \phi_i)}_{i \in I}$ es un atlas para $M$ entonces $\sett{(U_i \cap U, \phi_i \vert_{U})}_{i \in I}$ es un atlas para $U$.
		
		En efecto, como $\Bigcup{i \in I}{U_i} = M$ entonces $\Bigcup{i \in I}{U_i \cap U} = U$ y ademas como $\phi_i$ son homeomorfismos entonces $\phi_i\vert_{U}$ tambi\'en lo son. Finalmente como $\phi_j \circ \phi_i^{-1} : \phi_i(U_i \cap U_j) \rightarrow \phi_j(U_i \cap U_j)$ es suave entonces $\phi_j \vert_{U} \circ \phi_i^{-1} \vert_U = \phi_j \circ \phi_i^{-1} : \phi_i(U_i \cap U_j \cap U) \rightarrow \phi_j(U_i \cap U_j \cap U) $ tambi\'en lo es.
		
		Finalmente, para ver que $i : U \inc M$ es diferenciable tenemos que ver que $\phi_j \circ i \circ \left(\phi_i \vert_{U}\right)^{-1}$ es diferenciable para todos $i,j$, pero:
		
		\begin{equation*}
			\phi_j \circ i \circ \left(\phi_i \vert_{U}\right)^{-1} = \phi_j \circ i \circ \left(\phi_i \right)^{-1} = \phi_j \circ \left(\phi_i \right)^{-1}
		\end{equation*}
		
		Que ya era diferenciable por ser $\A$ atlas.
		
		\item Rec\'iprocamente, supongamos que $S$ es una variedad de $\dim = d$, luego si $\mathcal{S} = \sett{s_i}$ es un atlas de $S$ y $\A = \sett{\phi_i}$ es un atlas de $M$ entonces $\phi_j \circ i \circ s_i^{-1}: U \subset \R^n \rightarrow V \subset \R^n$ es una funci\'on continua e inyectiva, luego por invariance de dominio es abierta por lo que $i(S) = S$ es abierto.(Notemos que lo demostramos para una variedad topol\'ogica arbitraria, en el caso suave podemos recurrir a que $rk (i)$ es completo y por teorema de la funci\\on inversa tenemos un difeomorfismo local) \qed
		
	\end{enumerate}
	
\end{solution}

\begin{question}
	Sea $M$ una variedad diferencial conexa. Probar que para cada par de puntos $p,q\in M$ existe un camino suave $c:[0,1]\to M$ que los une (es decir, $c$ es una funci\'on continua en $[0,1]$, diferenciable en $(0,1)$, y $c(0)=p$, $c(1)=q$).
\end{question}

\begin{solution}
	Veamos primero el siguiente resultado \'util:
	
	\begin{proposition}
		
		\label{Existe una base de bolas regulares}
		
		Sea $M$ una variedad diferenciable de dimensi\'on $d$, luego existe una base $\B = \sett{B_n}_{n \in \N}$ con la propiedad que para cada $B_n$ existe una carta $\phi$ tal que $\phi(B_n) = B_r(0)$ para alg\'un $r$. Es m\'as dicha base se la puede tomar de modo que dado $B_n$ existe una carta $(B', \phi)$ con $B' \supset \overline{B_n}$ y $r <r'$ tal que:
		
		\begin{equation*}
			\phi(B) = B_r(0) \quad \phi(\overline{B}) = \overline{B_r(0)} \quad \phi(B') = B_{r'}(0)
		\end{equation*} 
				
	\end{proposition}
	
	\begin{proof}
		Supongamos primero que existe una \'unica carta $\phi:M \rightarrow \tilde{U}$ y consideremos:
		
		\begin{equation*}
		\B = \sett{B_r(x) \tq r \in \Q, \ x \in \Q^{d} , \ B_{r'}(x) \subset \tilde{U}}
		\end{equation*}
		
		Notemos entonces que $\phi^{-1}(\B)$ cumple que es una base contable que cumple lo pedido pues $\phi$ es un homeomorfismo.
		
		Ahora sea $M$ una variedad arbitraria y consideremos $\A = \sett{(U_i, \phi_i)}_{i \in I}$ su atlas, luego como $M$ admite base numerable existe un subcubrimiento abierto $\A' = \sett{U_n, \phi_n}_{n \in \N}$ de $M$ y por lo anteriormente demostrado cada $U_i$ (que ya probamos que era variedad) admite una base numerable $\B_n^{k}$ con las caracter\'isticas pedidas. Consideremos $\B = \Bigcup{n,k \in \N}{B_{n}^{k}}$ que es base numerable luego si $V \in \B$ entonces $V \subset U_n$ para alg\'un $n$ y conclu\'imos que existe $(B,\phi)$ carta con los requerimientos; conclu\'imos que $\B$ es la base pedida.\qed
		
	\end{proof}
	
	Ahora si notemos que como cada $V \in \B$ es localmente conexa (por ser homeomorfa a una bola de $\R^n$) entonces es conexa por el arco si y s\'olo si es conexa. 
	
	A continuaci\'on, dados $x \neq y \in M$ entonces existe $\gamma$ camino continuo y como $[0,1]$ es compacto existe finitos $V_1, \dots , V_n \in \B$ tal que $\gamma([0,1]) \subset \Bigcup{1 \leq i \leq n}{V_i}$; como $V_i \in \B$ podemos tomar $\tilde{V}_i \subset V_i$ para todo $i$ tal que $\gamma([0,1]) \subset \Bigcup{1 \leq i \leq n}{\tilde{V}_i}$ y $\tilde{V}_i \cap \tilde{V}_{i+1} = \sett{x_{i}}$ pues los achico en el sentido de tomar los que en su imagen den bolas de radio menor. 
	
	Finalmente notemos que dados $x_i, x_{i+1}$ existe un camino suave $\alpha_i$ entre $\phi_{i+1}(x_i), \phi_{i+1}(x_{i+1}) \in \R^d$ por teorema de existencia y unicidad de curvas, por lo que $\phi_{i+1}^{-1}(\alpha)$ es un camino trivialmente suave (componiendo con la carta queda la misma $\alpha$) entre $x_i, x_{i+1}$. Concatenando las curvas obtenidasd obtenemos el camino suave entre $x = x_{1}$ e $y = x_{n+1}$. \qed
	
\end{solution}

\begin{question}
	Sean $M,N$ variedades diferenciales. Probar que una funci\'on $f:M\to N$ es diferenciable si y s\'olo si $g\circ f:M\to\R$ es diferenciable para toda $g:N\to\R$ diferenciable.
\end{question}

\begin{solution}
	Supongamos que $f$ es diferenciable, entonces dado $p \in M$ existe $(U, \phi)$ carta en $M$ y $(V,\psi)$ carta en $N$ con $f(U) \subset V$ tal que $\psi \circ f \circ \phi{-1}: \phi(U) \rightarrow \psi(V)$ es diferenciable. Asimismo por ser $g$ diferenciable entonces dada $(V,\psi)$ sabemos que $g \circ \psi^{-1} : \psi(V) \rightarrow \R$ es diferenciable, por lo tanto $g \circ \psi^{-1} \circ \psi \circ f \circ \phi{-1} = g \circ f \circ \phi{-1}$ es diferenciable y conclu\'imos que $g \circ f$ es diferenciable. Notemos que usamos el resultado de la teorica que es equivalente pedir que existan un par de cartas conla condici\'on de diferenciabilidad a que para todos las cartas valga.
	
	Rec\'iprocamente, como $g \circ f$ para toda $g: N \rightarrow \R$ diferenciable, podemos tomar $\pi_i :N \rightarrow \R$ dada por $\pi_i(x^1, \dots, x^d) = x^i$ y como esta es diferenciable sabemos que $\pi_i \circ f = f^i$ es diferenciable. Como $f^i$ es diferenciable para cada $1 \leq i \leq d$ entonces $f$ es diferenciable. \qed
	
\end{solution}

\begin{question}
	Sea $M$ una variedad diferencial y $\pi:S^n\to\mathbb{P}^n(\R)$ la proyecci\'on can\'onica. Probar que $f:\mathbb{P}^n(\R)\to M$ es diferenciable si y s\'olo si $f\circ p:S^n\to M$ es diferenciable. Comparar el rango de $f$ con el de $f\circ p$.
\end{question}

\begin{solution}
	Veamoslo en dos partes, primero veamos que $\pi: S^n\to\mathbb{P}^n(\R)$ es una submersi\'on suryectiva.
	
	\begin{lemma}
		$\pi:S^n\to\mathbb{P}^n(\R)$ es una submersi\'on suryectiva
	\end{lemma}
	
	\begin{proof}
		Es trivial que es suryectiva y si consideramos una carta $(V^{i}_{j}, \pi_i)$ de $S^n$ y $(U_k, \phi_k)$  donde $k$ es tal que $\pi \circ \pi_i^{-1} (U_{j}^{i}) \subset U_k$, entonces $\phi_k \circ \pi \circ \pi_i^{-1} : U_{j}^{i} \rightarrow \phi_k(U_k)$ esta dada por (supongamos $j = +$):
		
		\begin{equation*}
			\begin{aligned}
				\phi_k \circ \pi \circ \pi_i^{-1}(x^1, \dots , x^n) = & g \circ \pi \circ \left(f^i_+\right) (x^1, \dots, x^{i-1}, x^{i+1} , \dots, x^n) \\ 
				= & \phi_k \circ \pi \left( x^1 , \dots , x^{i-1}, \sqrt{1-(x^{1})^2 - \dots - (x^n)^2}, x^{i+1}, \dots, x^n\right) \\
				= & \phi_k([x^1 : \dots : x^{i-1}: \sqrt{1-(x^{1})^2 - \dots - (x^n)^2}: x^{i+1}: \dots: x^n]) \\
				= & \left( \frac{x^1}{x^k} , \dots , \frac{x^{i-1}}{x^k}, \frac{\sqrt{1-(x^{1})^2 - \dots - (x^n)^2}, x^{i+1}}{x^k}, \dots, \frac{x^{k-1}}{x^k}, \frac{x^{k+1}}{x^k}, \dots , \frac{x^n}{x^k}\right) \in \mathscr{C}^\infty(\R^n)
			\end{aligned}
		\end{equation*}
		
		Finalmente de lo mismo notemos que:
		
		\begin{equation*}
			D(\pi) (x) = \left[\begin{array}{cccccc} \frac{1}{x_k} & \dots & \dots & -\frac{x_1}{x_k^2} & \dots & 0\\
			 0 & \frac{1}{x_k} & \dots & -\frac{x_2}{x_k^2} & \dots & 0\\ \vdots & \vdots & \vdots & \vdots & \vdots & \vdots \\  0 & \dots &  \dots & -\frac{x_n}{x_k^2} & \dots & \frac{1}{x_k}\end{array}\right].
		\end{equation*}
		
		Que tiene rango $n$, por lo tanto resulta que $\pi:S^n\to\mathbb{P}^n(\R)$ es una submersi\'on suryectiva.\qed
		
	\end{proof}
	
	Recordemos el siguiente resultado:
	
	\begin{theorem}[Teorema del Rango constante]
		\label{Teorema del rango constante}
		Sean $M,N$ de dimensiones $m,n$ respectivamente y sea $F : M \rightarrow N$ una funci\'on suave de rango constante $r$. Entonces para cada $p \in M$ existen cartas $(U,\phi)$ de $M$ centrada en $p$ y $(V, \psi)$ de $N$ centrada en $F(p)$ tal que $F(U) \subset V$ y:
		
		\begin{equation*}
			\psi \circ F \circ \phi^{-1} (x^1, \dots , x^r, x^{r+1}, \dots, x^m) = (x^1,  \dots , x^r, 0 \dots , 0)
		\end{equation*}
		
	\end{theorem}
	
	Ahora veamos la existencia local de secciones:
	
	\begin{proposition}
		Sean $M,N$ variedades  y $\pi : M \rightarrow N$ suave, luego $\pi$ es una submersi\'on si y s\'olo si para todo $p \in M$ existe una secci\'on local
	\end{proposition}
	
	\begin{proof}
		Si $\pi$ es una submersi\'on dado $p \in M$ sea $q = \pi(p)$, luego por \ref{Teorema del rango constante} existen cartas $(U,\phi)$ de $M$ centrada en $p$ y $(V, \psi)$ de $N$ centrada en $q$ tal que:
		
		\begin{equation*}
			\psi \circ \pi \circ \phi^{-1}\left(x^1, \dots, x^n, x^{n+1}, \dots , x^m\right) = (x^1 , \dots , x^n)
		\end{equation*}
		
		Luego tomemos $\epsilon> 0$ tal que $C_{\epsilon} = \sett{x \tq \abs{x^i} < \epsilon \ 1 \leq  i \leq m} \subset U$ sea un entorno de $p$ que cumpla que $\pi(C_{\epsilon}) = C'_{\epsilon} = \sett{y \tq \abs{y^i} < \epsilon \ 1 \leq i \leq n} \subset V$ sea un entorno de $q$. Con estos entornos sea $\sigma : C'_{\epsilon} \rightarrow C_{\epsilon}$ dada por:
		
		\begin{equation*}
			\sigma(x^1, \dots , x^n) = (x^1, \dots , x^n, 0, \dots , 0)
		\end{equation*}
		
		y es claro que es suave, de rango constante y cumple que $\pi \circ \sigma = Id_{C_{\epsilon}}$.
		
		Rec\'iprocamente si $\pi \circ \sigma = Id_U$ entonces es claro que (veremos mas tarde) $d\pi_p \circ d\sigma_q = Id_q$ con lo que $d\pi_p$ es suryectiva. (Lo usaremos en el caso real donde basta usar $D(\pi)$ la diferencial total peor el resultado es abstracto tomando en cuesta la diferencial entre espacios tangentes.) \qed
		
	\end{proof}
	
	Finalmente probemos la siguiente proposici\'on
	
	\begin{proposition}[Propiedad universal de las submerciones suryectivas]
		Sean $M,N$ variedades diferenciables y $\pi_M \rightarrow N$ una submersi\'on suryectiva, entonces para toda variedad diferenciable $P$ una funci\'on $F:N \rightarrow P$ es suave si y s\'olo si $F \circ \pi$ es suave.
	\end{proposition}
	
	\begin{proof}
		Si $F$ es suave entonces $F \circ \pi$ es suave.
		
		Rec\'iprocamente, si $F \circ \pi$ es suave, sea $q \in N$ y sea $p \in \pi^{-1}(q)$, luego por el resultado previo existe $U \ni q$ entorno abierta y $\sigma : U \rightarrow M$ suave tal que $\sigma(q) = p$ y $\pi \circ \sigma = Id_U$; finalmente notemos que:
		
		\begin{equation*}
			F \vert_U = F \vert_U \circ Id_U = F \circ \pi \circ \sigma = (F \circ \pi) \circ \sigma
		\end{equation*}
		
		Que es composici\'on de suaves, por lo tanto $F$ es suave en todo entorno $U$, conclu\'imos que $F$ es suave.\qed
		
	\end{proof}
	
	Por todo lo visto queda resuelto el punto. Finalmente es trivial usando la regla de la cadena $F(f \circ p) (q) = D(f)(p) \circ \underbrace{D(p)}_{rg(p)=n}(q)$ por lo que el rango de $f$ es el mismo que el rango de $f \circ p$ pues $p$ es submersi\'on. \qed
	
\end{solution}

\begin{question}
	Sea $M$ una variedad diferencial de dimensi\'on $d$ y $(U,\phi)$ una carta de $M$.
	\begin{enumerate}
		\item Probar que si $V\subseteq U$ es un abierto, entonces $(V,\left.\phi\right|_V)$ es una carta compatible de $M$.
		\item Probar que si $f:\phi(U)\to V\subseteq\R^d$ es un difeomorfismo, $(U,f\circ\phi)$ es una carta compatible de $M$.
	\end{enumerate}
\end{question}

\begin{solution}
	Vayamos por partes:
	
	\begin{enumerate}
		\item Resuelto arriba
		\item Trivial \qed
	\end{enumerate}
	
\end{solution}

\begin{question}
	Sea $M$ una variedad diferencial de dimensi\'on $d$.
	\begin{enumerate}
		\item Probar que $M$ admite un atlas $\mathscr{A}=\{(U_i,\phi_i):i\in I\}$ tal que para todo $i\in I$ se tiene que $\phi_i(U_i)$ es un abierto acotado de $\R^d$.
		\item Probar que $M$ admite un atlas $\mathscr{B}=\{(V_j,\psi_j):j\in J\}$ tal que para todo $j\in J$ se tiene que $\psi_j(V_i)=\R^d$.
	\end{enumerate}
\end{question}

\begin{solution}
	Notemos que resolvimos arriba ambos en \ref{Existe una base de bolas regulares} si consideramos luego que existe el obvio difeomorfismo $f : B_r(x) \rightarrow \R^d$ para todos $r> 0 $ y $x \in \R^d$. \qed
\end{solution}

\begin{question}
	Considerar en $\R$ las cartas $(\R,id)$ y $(\R,\phi)$ donde $\phi(t)=t^3$. Probar que las dos cartas no son compatibles pero que las variedades definidas por el atlas formado por cada una de las cartas son difeomorfas.
\end{question}

\begin{question}
	Sea $M$ la imagen de la funci\'on $f:(0,2\pi)\to\R^2$ donde $f(t)=(\sin(t),\sin(2t))$ con la estructura inducida por la carta $(M,f^{-1})$. Probar que la funci\'on $F:M\to M$ definida por $F(x,y)=(x,-y)$ no es diferenciable.
	
\end{question}

\begin{question}
	Probar que $SO_3(\R)$ es difeomorfo al espacio proyectivo $\mathbb{P}^3(\R)$.
\end{question}

\begin{question}
	Probar que $\R$ y $S^1$ son las \'unicas variedades diferenciales conexas de dimensi\'on $1$ salvo difeomorfismo.
\end{question}



\begin{question}\textbf{Preimagen de valor regular:} Sean $U\subseteq\R^n$ un abierto y $F:U\to\R^m$ ($n\geq m$) una funci\'on diferenciable tal que $c\in\R^m$ es un valor regular de $F$ (es decir, para cada punto $x\in U$ con $F(x)=c$ el rango de $\mathrm{D}F(x)$ es $m$). Probar que $M=F^{-1}(c)$ es una variedad de dimensi\'on $n-m$ y la inclusi\'on $M\hookrightarrow U$ es diferenciable.
\end{question}

\begin{question}\textbf{Producto cartesiano:} Sean $M$ y $N$ variedades diferenciales.
	\begin{enumerate}
		\item Probar que el producto cartesiano $M\times N$ es naturalmente una variedad diferencial con $\dim(M\times N)=\dim(M)+\dim(N)$ y que las proyecciones can\'onicas $\pi_1:M\times N\to M$ y $\pi_2:M\times N\to N$ son diferenciables. 
		
		\item El producto de variedades diferenciales est\'a caracterizado por la siguiente \textit{propiedad universal}: Si $P$ es una variedad diferencial junto con funciones diferenciables $p_1:P\to M, p_2:P\to N$ entonces existe una \'unica funci\'on diferenciable $f:P\to M\times N$ tal que $\pi_1\circ f = p_1$ y $\pi_2\circ f = p_2$.
	\end{enumerate}
\end{question}

\begin{question}
	\textbf{Pegado de variedades:} Sea $(M_i)_{i\in I}$ una familia numerable de variedades diferenciales, todas de dimensi\'on $n$. Supongamos que para cada par $i\neq j$ est\'an dados: dos abiertos $U_{ij}\subseteq M_i$ y $U_{ji}\subseteq M_j$, y un difeomorfismo $f_{ij}:U_{ij}\to U_{ji}$ que no puede extenderse continuamente a ning\'un punto de $\partial U_{ij}$, tales que se satisfacen las siguientes propiedades:
	\begin{itemize}
		\item $f_{ji}=f_{ij}^{-1}$.
		\item $f_{ij}(U_{ij}\cap U_{ik}) = U_{ji}\cap U_{jk}$.
		\item $f_{ik} = f_{jk}\circ f_{ij}$ en $U_{ij}\cap U_{ik}$.
	\end{itemize}
	
	Mostrar que existe una variedad diferencial $M$ y morfismos $\psi_i:M_i\to M$ tales que $\psi_i$ es un difeomorfismo entre $M_i$ y un abierto de $M$ y
	\begin{enumerate}
		\item los abiertos $\psi_i(M_i)$ cubren $M$,
		\item $\psi_i(U_{ij})=\psi_i(M_i)\cap\psi_j(M_j)$,
		\item $\psi_i=\psi_j\circ f_{ij}$ en $U_{ij}$.
	\end{enumerate}
\end{question}

\begin{question}
	\textbf{Suma conexa de variedades:} Sean $M$ y $N$ dos variedades conexas de la misma dimensi\'on $d$. Se consideran cartas $(U,\phi)$ y $(V,\psi)$ de $M$ y $N$ respectivamente tales que $\phi(U)=\psi(V)=B(0,1)$ y pongamos $p=\phi^{-1}(0)$ y $q=\psi^{-1}(0)$. Definimos una nueva variedad $M\# N$ como el pegado de $M\smallsetminus\{p\}$ y $N\smallsetminus\{q\}$ por los abiertos $U$ y $V$ a trav\'es del difeomorfismo $f:U\to V$ determinado por la ecuaci\'on $$\psi f\phi^{-1}(x)=\dfrac{1-\norm{x}}{\norm{x}} x \;\;\forall x\in B(0,1)\smallsetminus\{0\}.$$ La variedad $M\# N$ se llama la suma conexa de $M$ y $N$. Convencerse de que esta construcci\'on no depende de las cartas utilizadas.
	
	\noindent Probar que $M\# S^d$ es difeomorfa a $M$ y que la operaci\'on $\#$ es conmutativa y asociativa.
	\vspace{1em}
	
	\noindent \textit{Observaci\'on:} Se puede probar que cualquier variedad compacta de dimensi\'on $2$ es difeomorfa a la esfera $S^2$, a la suma de $n$ toros $T\#\cdots\# T$ o a la suma de $n$ planos proyectivos $\mathbb{P}(\R)^2\#\cdots\#\mathbb{P}(\R)^2$. Es m\'as, estas variedades no son homeomorfas entre s\'{\i}.
	
	
	
\end{question}

\begin{question}
	\textbf{Cociente por la acci\'on de un grupo:} Sea $M$ una variedad diferencial y $G$ un grupo que act\'ua en $M$ por difeomorfismos: para cada $g\in G$ se tiene $\phi_g:M\to M$ difeomorfismo de modo que $\phi_{1_G}=1_M$ y $\phi_g\phi_h=\phi_{gh}$. Supongamos adem\'as que la acci\'on es propiamente discontinua (es decir, todo $p\in M$ est\'a contenido en un abierto $U$ tal que $\phi_g(U)\cap U=\emptyset$ para todo $g\neq 1_G$) y para todos $p,q\in M$ en distintas \'orbitas existen abiertos $U$ y $V$ que los contienen respectivamente tales que $\phi_g(U)\cap V = \emptyset$ para todo $g\in G$.
	\begin{enumerate}
		\item Probar que el conjunto de \'orbitas $M/G$ es una variedad diferencial con la estructura inducida por $M$, la proyecci\'on can\'onica $M\to M/G$ es diferenciable y $\dim(M)=\dim(M/G)$.
		\item Expresar el espacio proyectivo $\mathbb{P}^n(\R)$ y el toro $n$-dimensional $T_n$ como cocientes $S^n/G$ y $\R^n/H$ para grupos y acciones convenientes.
	\end{enumerate}
\end{question}

\textsl{\textbf{\'Algebras de funciones}}
\vspace{1em}


\begin{question}
	Probar que $\mathscr{C}^\infty(M,\R)=\{f:M\to\R:f\text{ es diferenciable}\}$ es un anillo con la suma y el producto punto a punto. Probar que si $g:M\to N$ es diferenciable, entonces $g^*:\mathscr{C}^\infty(N,\R)\to\mathscr{C}^\infty(M,\R)$ es un morfismo de anillos.
\end{question}

\begin{question}
	Dadas $M$ y $N$ variedades diferenciales compactas, probar que:
	\begin{enumerate}
		\item Los ideales maximales de $\mathscr{C}^\infty(M,\R)$ son de la forma $$\mathfrak{m}_p = \{f\in\mathscr{C}^\infty(M,\R): f(p)=0\}.$$
		\item Todo morfismo de $\R$-\'algebras $\mathscr{C}^\infty(N,\R)\to\mathscr{C}^\infty(M,\R)$ viene de una funci\'on diferenciable $M\to N$.
	\end{enumerate}
	\vspace{1em}
	
	
	\noindent \textit{Observaci\'on:} Por \textbf{a.} podemos recuperar la variedad $M$ como conjunto a partir de $\mathscr{C}^\infty(M,\R)$, por $\textbf{b.}$ tambi\'en recuperamos su estructura diferenciable. ¿Qu\'e pasa si $M$ y $N$ no son compactas? ¿Vale \textbf{b.} si s\'olo pedimos morfismo de anillos?
\end{question}

\begin{question}
	Probar que el conjunto $\mathscr{D}_p(M)$ de g\'ermenes de funciones diferenciables a valores reales alrededor de un punto $p\in M$ es un anillo y si $g:M\to N$ es diferenciable entonces $g^*:\mathscr{D}_{g(p)}(N)\to\mathscr{D}_p(M)$ es un morfismo de anillos.
\end{question}

\begin{question}
	Dado $p\in M$ probar que la aplicaci\'on cociente $f\mapsto\overline{f}$ da un isomorfismo de $\R$-\'algebras $$\mathscr{C}^\infty(M,\R)/\mathfrak{m}_p^0\to\mathscr{D}_p(M)$$ donde $\mathfrak{m}_p^0 = \{f\in\mathscr{C}^\infty(M,\R): f \text{ se anula en un entorno de } p\}$.
	\vspace{1em}
	
	\noindent\textit{Observaci\'on}: Las $\R$-\'algebras $\mathscr{D}_p(M)$ son anillos locales cuyo \'unico ideal maximal son los g\'ermenes de funciones que se anulan en $p$. M\'as a\'un, $\mathscr{D}_p(M)$ es la localizaci\'on de $\mathscr{C}^\infty(M,\R)$ en el complemento del ideal maximal $\mathfrak{m}_p$.
\end{question}

\section{Soluciones}

\SetupExSheets{headings=myblock}
\printsolutions



\end{document}