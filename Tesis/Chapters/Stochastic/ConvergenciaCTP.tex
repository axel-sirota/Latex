\chapter{Convergencia CTP }\label{ch:convergenciaCTP}

\epigraph{At a purely formal level, one could call probability theory the study of measure spaces with total measure one, but that would be like calling number theory the study of strings of digits which terminate.}{Terence Tao}

Ahora que ya analizamos la convergencia en L1 de \ref{algo: DE} vimos que hay una distinci\'on entre el caso convexo y no convexo; donde en el caso convexo usualmente podemos asegurar convergencia a una cercan\'ia del m\'inimo mientras que en el no convexo solo asegurabamos la convergencia a un punto cr\'itico. Nuevamente en el estudio de la convergencia \textit{casi todo punto} vamos a separar en esos dos casos y va a volver a aparecer \ref{eq: Condicion robbins monro}. Por simpleza en los c\'alculos vamos a asumir en este cap\'itulo que $g$ es un estimador insesgado de $\nabla F(w_k)$.

\section{Caso d\'ebilmente convexo}

\begin{hyp}[Acotaciones al segundo momento de $g$]
	\label{hyp: Acotaciones varianza de g}
	Supongamos que dada $F$ funci\'on objetivo cumple \ref{def: Debilmente convexo} y $g$ la estimaci\'on insesgada del gradiente en \ref{algo: DE} vale que existen $A,B \geq 0$ tales que:
	
	\begin{equation*}
		\label{eq: Segundo momento acotado de g insesgado}
		\expectation{g(w_k, \upxi_{k})^2} \leq A + B \left(w_k - w^*\right)^2
	\end{equation*}
	
\end{hyp}

\begin{remark}
	Notemos que si $F$ cumple \ref{hyp: F es L lipshitz} y $g$ cumple \ref{hyp: Acotaciones momentos de g} entonces autom\'aticamente cumplen \ref{hyp: Acotaciones varianza de g}.
\end{remark}

\begin{theorem}[Objetivo d\'ebilmente convexo, incrementos decrecientes]
	\label{theorem: DE en debilmente convexo y alfa decreciente converge ctp}
	Supongamos \ref{def: Debilmente convexo}, \ref{hyp: Acotaciones varianza de g}; adem\'as supongamos que dado \ref{algo: DE} $\alpha_k$ cumple \ref{eq: Condicion robbins monro}:
	
	Luego para todo $k \in \N$ vale que:
	
	\begin{subequations}
		\begin{equation}
			w_k \ \xrightarrow[\text{ctp}]{k \ \rightarrow \ \infty } \ w^*
		\end{equation}
		\begin{equation}
			\left(w_k - w^*\right) \nabla F(w_k) \ \xrightarrow[\text{ctp}]{k \ \rightarrow \ \infty } \ 0
		\end{equation}
	\end{subequations}
	
\end{theorem}

\begin{proof}
	Vayamos de a pasos como cuando demostramos \ref{theorem: convergencia puntual batch, objetivo debilmente convexo, incrementos decrecientes}.
	
	\begin{enumerate}
		\item [Paso 1] Definamos el proceso estoc\'astico de Lyapunov $h_k := \left(w_k - w^*\right)^2$
		
		\item [Paso 2] An\'alogamente a casos anteriores notemos que:
		
		\begin{equation}
			h_{k+1} - h_k = -2 \alpha_k \left(w_k - w^*\right) g(w_k, \upxi_{k}) + \alpha_k^2 g(w_k, \upxi_{k})^2
		\end{equation}
		
		\marginpar{\color{Red} Como definimos bien la filtracion? Es saber toda esa informacion a tiempo $k$, como lo notamos?}
		
		Definamos ahora la filtraci\'on $\mathcal{P}_k = \sett{\upxi_{0}, \dots, \upxi_{k-1}, w_0, \dots, w_k, \alpha_0, \dots, \alpha_k}$ que determina toda la informaci\'on a tiempo $k$ \textbf{antes} de tomar la muestra $\upxi_{k}$; luego si tomamos la esperanza condicional a esta filtraci\'on:
		
		\begin{equation*}
		\begin{aligned}
			\expectationfilt{h_{k+1} - h_k } & = & -2 \alpha_k \expectationfilt{\underbrace{\left(w_k - w^*\right)}_{\mathcal{P}_k \text{ medible}} g(w_k, \upxi_{k}) } + \alpha_k^2 \expectationfilt{g(w_k, \upxi_{k})^2} \\
			& = & -2 \alpha_k \left(w_k - w^*\right) \expectationfilt{g(w_k, \upxi_{k})} + \alpha_k^2 \expectationfilt{g(w_k, \upxi_{k})^2} 
		\end{aligned}
		\end{equation*}
		
		Pero como $\upxi_{k}$ es \textbf{independiente } de $\mathcal{P}_k$ entonces si recordamos que $g$ es insesgado tenemos:
		
		\begin{equation*}
		\expectationfilt{h_{k+1} - h_k } = -2 \alpha_k \left(w_k - w^*\right) \nabla F(w_k) + \alpha_k^2 \expectationfilt{g(w_k, \upxi_{k})^2} 
		\end{equation*}		
		
		Ahora si incorporamos \ref{hyp: Acotaciones varianza de g} obtenemos:
		
		\begin{equation}
			\label{eq: variaciones proceso lyapunov}
				\expectationfilt{h_{k+1} - \left(1 + \alpha_k^2 B\right)h_k } \leq  -2 \alpha_k \left(w_k - w^*\right) \nabla F(w_k) + \alpha_k^2 A
		\end{equation}
		
		Definamos ahora las sucesiones auxiliares:
		
		\begin{subequations}
			\begin{equation}
			\mu_k = \prod\limits_{j=1}^{k-1} {\dfrac{1}{1 + \alpha_j^2B}}
			\end{equation}
			\begin{equation}
			h_k' = \mu_k h_k
			\end{equation}
		\end{subequations}
	
		Por lo que replicando las operaciones en \ref{theorem: convergencia puntual batch, objetivo debilmente convexo, incrementos decrecientes} llegamos a:
		
		\begin{equation*}
			\expectationfilt{h_{k+1}' - h_k' } \leq \alpha_k^2 \mu_k A 
		\end{equation*}
		
		Luego, recordando \ref{def: Variaciones positivas de un proceso} notemos que:
		
		\begin{equation*}
			\expectation{\delta_k^{h'} \left(h_{k+1}' - h_k'\right)} \underbrace{=}_{\text{ por definici\'on de } \delta_k^{h'}} \expectation{\delta_k^{h'} \expectationfilt{h_{k+1}' - h_k'}} \leq  \alpha_k^2 \mu_k A
		\end{equation*}
		
		Por el mismo motivo que en \ref{theorem: convergencia puntual batch, objetivo debilmente convexo, incrementos decrecientes} conclu\'imos que:
		
		\begin{subequations}
			\begin{equation}
				h_k' \geq 0 
			\end{equation}
			\begin{equation}
				\sum\limits_{k=1}^{\infty} {\expectation{\delta_k^{h'} \left(h_{k+1}' - h_k'\right)}} < \infty
			\end{equation}
		\end{subequations}
		
		Por \ref{theorem: Convergencia de cuasi martingalas} conclu\'imos que $h_k'$ converge \textit{ctp}; como $\underbrace{\mu_k}_{\geq 0} \ \rightarrow \ \mu_{\infty} > 0$ entonces $\sett{h_k}$ converge \textit{ctp}.
		
		\item[Paso 3] Como $h_k$ converge ctp, de \ref{eq: variaciones proceso lyapunov} conclu\'imos que:
		
		\begin{equation*}
		\sum\limits_{k=1}^{\infty} {\alpha_k \left(w_k - w^* \right)\nabla F(w_k)} < \infty \qquad ctp
		\end{equation*}
		
		Supongamos que $\mathbb{P}\left(\sett{\lim\limits_{k} {h_k} > 0}\right) > \widetilde{\epsilon}$, luego $\mathbb{P}\left(\sett{\alpha_k \left(w_k - w^* \right)\nabla F(w_k) > C\alpha_k}\right) > \widetilde{\epsilon}$ lo que implicar\'ia por \ref{eq: Condicion robbins monro} que $\mathbb{P}\left(\sett{\sum\limits_{k=1}^{\infty} {\alpha_k \left(w_k - w^* \right)\nabla F(w_k)} = \infty}\right) > \widetilde{\epsilon}$; conclu\'imos entonces que:
		
			\begin{subequations}
			\begin{equation}
			w_k \ \xrightarrow[\text{ctp}]{k \ \rightarrow \ \infty } \ w^*
			\end{equation}
			\begin{equation}
			\left(w_k - w^*\right) \nabla F(w_k) \ \xrightarrow[\text{ctp}]{k \ \rightarrow \ \infty } \ 0
			\end{equation}
		\end{subequations}\qed
		
	\end{enumerate}
	
\end{proof}

\section{Caso no convexo}

Vamos a tomar las siguientes hip\'otesis para probar la convergencia ctp a un punto extremal.

\begin{hyp}[Hip\'otesis caso no convexo]
	\label{hyp: Condiciones caso no convexo convergencia ctp}
	Sea $F$ una funci\'on de costo objetivo y supongamos que el algoritmo \ref{algo: DE} cumple que $g$ es un estimador insesgado, ie: $\expectation{g(w_k, \upxi_{k})} = \nabla F(w_k)$, luego tomemos las siguientes hip\'otesis:
	
	\begin{enumerate}
		\item $F \in C^3$
		\item Existe $w^* \in \mani$ tal que $F_{inf} = F(w^*) \leq F(w)$, aunque notemos que no necesariamente es \'unico
		\item $F(w) \geq 0$ para todo $w \in \mani$ (Notemos que como hay un m\'inimo global, podemos redefinir $\widetilde{F} = F - F_{inf} \geq 0$)
		\item Sean $\sett{\alpha_k}$ los incrementos del algoritmo \ref{algo: DE}, entonces estos cumplen \ref{eq: Condicion robbins monro}
		\item Para $j = 2, 3, 4$ existen $A_j,B_j \geq 0$ tal que:
		
		\begin{equation}
			\label{eq: Acotaciones momentos de g}
			\expectation{\norm{g(w_k, \upxi_{k})}_2^j} \leq A_j + B_j \norm{w}_2^j
		\end{equation}
		
		\item  Existe $D > 0$ tal que:
		
		\begin{equation}
			\label{eq: Condicion de horizonte en g}
			\inf\limits_{(w)^2 > D} {w \nabla F(w)} >0
		\end{equation}
	\end{enumerate}
	
\end{hyp}

\subsection{Acotaci\'on global del algoritmo}

Usemos \ref{eq: Condicion de horizonte en g} para probar que existe un entorno $w_1 \in U \subset \mani$ tal que si $\sett{w_k}$ son las iteraciones del algoritmo \ref{algo: DE} con $F,g$ cumpliendo \ref{hyp: Condiciones caso no convexo convergencia ctp}, entonces $\sett{w_k} \subset U$.

\begin{theorem}[Acotaci\'on global del algoritmo estoc\'astico insesgado]
	\label{theorem: Acotaci\'on global del algoritmo estoc\'astico insesgado}
	Sea $F$ funci\'on de costo objetivo y $g$ un estimador insesgado de $\nabla F$ tal que ambos cumplen \ref{hyp: Condiciones caso no convexo convergencia ctp}, luego existe un entorno $w_1 \in U \subset \mani$ tal que si $\sett{w_k}$ son las iteraciones del algoritmo \ref{algo: DE} entonces $\sett{w_k} \subset U$.
\end{theorem}

\begin{proof}
	Nuevamente probemos esto en tres pasos:
	
	\begin{enumerate}
		\item [Paso 1] Sea $D$ el par\'ametro de horizonte dado por \ref{eq: Condicion de horizonte en g} y definamos $\phi: \mani \mapsto \R$ dada por:
		
		\begin{equation*}
		\phi(x) = \left\lbrace
		\begin{array}{cc}
		 0 & \text{ si } x < D \\
		 (x - D)^2 & \text{ si } x \geq D
		\end{array}\right.
		\end{equation*}
		
		Luego, sea:
		
		\begin{equation*}
			f_k = \phi\left(w_k^2\right)
		\end{equation*}
		
		\item [Paso 2] Notemos que por la definici\'on vale:
		
		\begin{equation*}
			\phi(y) - \phi(x) \leq \left(y - x\right)\phi\prime(x) + \left(y-x\right)^2
		\end{equation*}
		
		Y la igualdad se da si y s\'olo si $x,y \geq D$; con esto en forma an\'aloga a antes calculemos las variaciones del proceso de Lyapunov $\sett{f_k}$:
		
		\begin{equation}
		\label{eq: Condicion de proceso de lyapunov, confinamiento global}
		\begin{array}{rcl}
			f_{k+1} - f_k & \leq & \left(-2 \alpha_k w_k g(w_k, \upxi_{k}) + \alpha_k^2 g(w_k, \upxi_{k})^2\right) \phi'(w_k^2) \\ 
			& & + 4 \alpha_k^2 \left(w_kg(w_k, \upxi_{k})\right)^2 -4 \alpha_k^3 w_k g(w_k, \upxi_{k})^3 \\
			 & &  + 4 \alpha_k^4g(w_k, \upxi_{k})^4 \\
		\end{array}
		\end{equation}
		
		Por Cauchy-Schwartz sabemos que $w_k g(w_k, \upxi_{k}) = \ip{w_k, g(w_k, \upxi_{k})} \leq \norm{w_k}_2 \norm{g(w_k, \upxi_{k})}_2$ y que $g(w_k, \upxi_{k})^2 = \ip{g(w_k, \upxi_{k}), g(w_k, \upxi_{k})} \leq \norm{g(w_k, \upxi_{k})}_2^2$, si sumamos esto a la anterior ecuaci\'on tenemos:
		
		\begin{equation*}
		\begin{array}{rcl}
		f_{k+1} - f_k & \leq & -2 \alpha_k w_k g(w_k, \upxi_{k}) \phi'(w_k^2)  + \alpha_k^2 \phi'(w_k^2) \norm{g(w_k, \upxi_{k})}_2^2 \\ 
		& & + 4 \alpha_k^2 \norm{w_k}_2^2 \norm{g(w_k, \upxi_{k})}_2^2 -4 \alpha_k^3 \norm{w_k}_2 \norm{g(w_k, \upxi_{k})}_2^3  \\
		& & + 4 \alpha_k^4 \norm{g(w_k, \upxi_{k})}_2^4 \\
		\end{array}
		\end{equation*}
		
		Lo que implica si tomamos esperanza condicional a la filtraci\'on $\sett{\mathcal{P}_k}$, recordando que $\alpha_k, w_k, \phi'(w_k^2)$ son $\mathcal{P}_k$ medibles y $g(w_k, \upxi_{k})$ es independiente de $\mathcal{P}_k$:
		
		\begin{equation*}
		\begin{array}{rcl}
		\expectationfilt{f_{k+1} - f_k} & \leq & -2 \alpha_k w_k \nabla F (w_k)\phi'(w_k^2)  + \alpha_k^2 \phi'(w_k^2) \expectation{\norm{g(w_k, \upxi_{k})}_2^2} \\ 
		& & + 4 \alpha_k^2 \norm{w_k}_2^2 \expectation{\norm{g(w_k, \upxi_{k})}_2^2} -4 \alpha_k^3 \norm{w_k}_2 \expectation{\norm{g(w_k, \upxi_{k})}_2^3}  \\
		& & + 4 \alpha_k^4 \expectation{\norm{g(w_k, \upxi_{k})}_2^4} \\
		\end{array}
		\end{equation*}
		
		\marginpar{\color{Red} Porque pablo? Aca esto no me cierra que tantas acotaciones hace!}
		
		Si ahora inclu\'imos \ref{eq: Acotaciones momentos de g} entonces esto implica que existen $A,B \geq 0$ tal que:
		
		\begin{equation*}
		\expectationfilt{f_{k+1} - f_k} \leq -2 \alpha_k w_k \nabla F (w_k)\phi'(w_k^2)  + \alpha_k^2\left(A + B f_k\right)
		\end{equation*}
		
		Notemos ahora que si $w_k^2 < D$ entonces $\phi'(w_k^2) = 0$, y si $w_k^2 \geq D$ entonces $-2 \alpha_k w_k \nabla F (w_k)\phi'(w_k^2) < 0$ por \ref{eq: Condicion de horizonte en g} por lo que deducimos:
		
		\begin{equation}
		\expectationfilt{f_{k+1} - f_k} \leq \alpha_k^2\left(A + B f_k\right)
		\end{equation}	
		
		\marginpar{\color{Red}Dado que hicimos lo mismo 3 veces, valdr\'ia que sea un lema?}
		
		Ahora siguiendo los mismo pasos que al demostrar \ref{theorem: DE en debilmente convexo y alfa decreciente converge ctp} definiendo $\mu_k, f_k'$ y usando \ref{theorem: Convergencia de cuasi martingalas} conclu\'imos que $\sett{f_k}$ converge ctp.
		
		\item[Paso 3] Supongamos que $f_{inf} >0$, entonces existe $T \in \N$  tal que $w_k^2, w_{k+1}^2 > D$ para todo $k \geq T$ por lo que \ref{eq: Condicion de proceso de lyapunov, confinamiento global} es una igualdad y deducimos que:
		
		\begin{equation}
			\sum\limits_{k=1}^{\infty} {\alpha_k w_k \nabla F(w_k) \phi'(w_k^2)} < \infty \quad ctp
		\end{equation}
		
		Pero por otro lado como $f_{inf} >0$, $\sum\limits_{k=1}^{\infty} {\alpha_k} = \infty$, $0 < \lim \phi'(w_k^2) < \infty$ existe $\widetilde{\epsilon} > 0$ y $M > 0$ tal que:
		
		\begin{equation*}
			\begin{array}{rcl}
			\widetilde{\epsilon} & < & \mathbb{P} \left(\sett{\sum\limits_{k=1}^{\infty} {\alpha_k w_k \nabla F(w_k) \phi'(w_k^2)} = \infty }\right) \\
			& = & \mathbb{P} \left(\sett{\sum\limits_{k=1}^{\infty} {\alpha_k w_k \nabla F(w_k) \phi'(w_k^2)} > M \liminf \phi'(w_k^2)\sum\limits_{k=1}^{\infty} {\alpha_k}}\right) 
			\end{array}
		\end{equation*}
		
		Conclu\'imos que $f_{inf} = 0$ y entonces existe $K \in \N$ tal que $\sett{w_k}_{k \geq K} \subset \sett{x \in \mani \tq \norm{x} < D }$. \qed
		
	\end{enumerate}
	
\end{proof}

\subsection{Convergencia del algoritmo}

\begin{theorem}[Convergencia a puntos extremales,  Caso insesgado]
	Sea $F$ funci\'on de costo objetivo y $g$ un estimador insesgado de $\nabla F$ tal que ambos cumplen \ref{hyp: Condiciones caso no convexo convergencia ctp}, si $\sett{w_k}$ son las iteraciones del algoritmo \ref{algo: DE} entonces valen:
	
		\begin{subequations}
		\begin{equation}
		F(w_k) \ \xrightarrow[\text{ctp}]{k \ \rightarrow \ \infty } \ F_{\infty}
		\end{equation}
		\begin{equation}
		\nabla F(w_k) \ \xrightarrow[\text{ctp}]{k \ \rightarrow \ \infty } \ 0
		\end{equation}
	\end{subequations}
	
\end{theorem}

\begin{proof}
	Vayamos como siempre de a pasos definiendo el proceso de Lyapunov correspondiente:
	
	\begin{enumerate}
		\item [Paso 1] Definamos $h_k = F(w_k) \geq 0$ por hip\'otesis
		\item[Paso 2] Por \ref{theorem: Acotaci\'on global del algoritmo estoc\'astico insesgado} si suponemos que $\dim(\mani) < \infty$ entonces existe $K_1 > 0$ tal que $\nabla^2 F(w_k) \leq K_1$, luego si desarrollamos en Taylor en $w_k$ tenemos:
		
		\begin{equation*}
			\begin{aligned}
				h_{k+1} - h_k & = & \nabla F(w_k) \left(w_{k+1} - w_k\right) + \frac{1}{2} \nabla^2 F(w_k) \left(w_{k+1} - w_k\right) \\
				& = & - \alpha_k \nabla F(w_k) g(w_k, \upxi_{k})^2 + \frac{1}{2} \nabla^2 F(w_k)  \alpha_k^2 g(w_k, \upxi_{k})^2 \\
				& \leq & -2 \alpha_k \nabla F(w_k) g(w_k, \upxi_{k})^2 + K_1 \alpha_k^2 g(w_k, \upxi_{k})^2
			\end{aligned}
		\end{equation*}
		 
		Tomando esperanza condicional respecto a $\mathcal{P}_k$:
		
		\begin{equation}
			\label{eq: Acotacion de la sucesin en la iteracion}
			\expectationfilt{h_{k+1} - h_k} \leq \underbrace{-2 \alpha_k \left(\nabla F(w_k)\right)^2}_{\leq 0}  + \alpha_k^2 K_1 \underbrace{\expectation{g(w_k, \upxi_{k})}}_{\leq K_2 \text{ por  \ref{theorem: Acotaci\'on global del algoritmo estoc\'astico insesgado}}}
		\end{equation}
		
		Lo que implica:
		
		\begin{equation}
			\expectationfilt{h_{k+1} - h_k} \leq \alpha_k^2 K_1K_2
		\end{equation}
		 
		 Luego como tenemos de esto que:
		 
		 \begin{equation*}
		 \begin{aligned}
		 \sum\limits_{k=1}^{\infty} {\expectation{\delta_k^{h} \left(h_{k+1} - h_k\right)}} & = & \sum\limits_{k=1}^{\infty} {\expectation{\delta_k^{h} \expectationfilt{\left(h_{k+1} - h_k\right)}}} \\
		 & \leq & K_1K_2\sum\limits_{k=1}^{\infty} {\alpha_k^2} < \infty
		 \end{aligned}
		 \end{equation*}
		 
		  Por lo que por \ref{theorem: Convergencia de cuasi martingalas} obtenemos:
		 
		 \begin{equation}
		 F(w_k) \ \xrightarrow[\text{ctp}]{k \ \rightarrow \ \infty } \ F_{\infty}
		 \end{equation}
		 
		 \item[Paso 3] Si retomamos \ref{eq: Acotacion de la sucesin en la iteracion}, reordenamos, sumamos hasta $k$ y tomamos esperanza tenemos: 
		 
		 \begin{equation}
		 \label{eq: los gradientes ponderados son sumables en caso no convexo ctp}
		 	\begin{aligned}
		 	2\sum\limits_{k=1}^{K} {\alpha_k \left(\nabla F(w_k)\right)^2} & \leq & \sum\limits_{k=1}^{K} {\expectation{h_{k+1} - h_k}} + K_2K_1 \sum\limits_{k=1}^{K} {\alpha_k^2} \\
		 	& = & \expectation{\sum\limits_{k=1}^{K} {h_{k+1} - h_k}} + K_2K_1 \sum\limits_{k=1}^{K} {\alpha_k^2} \\
		 	& = & \expectation{h_{k+1}} + K_2K_1 \sum\limits_{k=1}^{K} {\alpha_k^2} \\
		 	& \xrightarrow[\ref{theorem: Propiedades de esperanza condicional}]{\text{ ctp }} & F_{\infty} + K_2K_1 \sum\limits_{k=1}^{\infty} {\alpha_k^2}
		 	\end{aligned}
		 \end{equation}
		 
		 Sea ahora $g_k = \left(\nabla F (w_k)\right)^2$ y volvamos a expandir Taylor en $w_k$:
		 
		 \begin{equation*}
		 \begin{aligned}
		 g_{k+1} - g_k & = & \nabla g(w_k) \left(w_{k+1} - w_k\right) + \frac{1}{2} \nabla^2 g(w_k) \left(w_{k+1} - w_k\right)^2 \\
		 &\underbrace{\leq}_{\text{ por \ref{theorem: Acotaci\'on global del algoritmo estoc\'astico insesgado}}} & - 2 \alpha_k \nabla F(w_k) K_4 g(w_k, \upxi_{k}) + K_3 \alpha_k^2 g(w_k, \upxi_{k})^2 \\
		 \end{aligned}
		 \end{equation*}
		 
		 Tomando esperanza condicional respecto a $\mathcal{P}_k$:
		 
		 \begin{equation}
		 	\expectationfilt{g_{k+1} - g_k } \leq 2 \alpha_k K_4 \left(\nabla F(w_k)\right)^2 + K_2K_3\alpha_k^2
		 \end{equation}
		 
		 Por lo tanto si sumamos las variaciones asociadas al proceso estoc\'astico $\sett{g_k}$:
		 
		 \begin{equation}
		 \begin{array}{rcl}
		 \sum\limits_{k=1}^{\infty} {\expectation{\delta_k^{g} \left(g_{k+1} - g_k\right)}} & = & \sum\limits_{k=1}^{\infty} {\expectation{\delta_k^{g} \expectationfilt{\left(g_{k+1} - g_k\right)}}} \\
		 & \leq & K_4 \underbrace{\sum\limits_{k=1}^{\infty} {\alpha_k \left(\nabla F(w_k)\right)^2}}_{\ref{eq: los gradientes ponderados son sumables en caso no convexo ctp}} + K_2K_3 \underbrace{\sum\limits_{k=1}^{\infty} {\alpha_k^2}}_{\ref{eq: Condicion robbins monro}} \\
		 & < &  \infty
		 \end{array}
		 \end{equation}
		 
		 Nuevamente por \ref{theorem: Convergencia de cuasi martingalas}conclu\'imos que $\sett{g_k}$ converge \textit{ctp}, y por \ref{eq: los gradientes ponderados son sumables en caso no convexo ctp} este l\'imite es $0$; luego:
		 
		 	\begin{subequations}
		 	\begin{equation}
		 	g_k \ \xrightarrow[\text{ctp}]{k \ \rightarrow \ \infty } \ 0
		 	\end{equation}
		 	\begin{equation}
		 	\nabla F(w_k) \ \xrightarrow[\text{ctp}]{k \ \rightarrow \ \infty } \ 0
		 	\end{equation}
		 \end{subequations}
		 \qed
	\end{enumerate}
	
\end{proof}