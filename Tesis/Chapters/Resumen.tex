\chapter{Resumen de Resultados}\label{ch:resumen}

\epigraph{Si de verdad amas a alguien, reg\'alale un teorema; eso s\'i que es para siempre}{Eduardo Sanchez de Cabez\'on}

A modo de s\'intesis, presentamos un resumen de los resultados vistos en este trabajo.

\section{Algoritmos de tipo Batch}

\subsection{Objetivos convexos}

\begin{theorem}
	Sea $F: \R^d \rightarrow \R$ tal que $F \in C^1$ la funci\'on objetivo, asumamos que $F$ es \textit{d\'ebilmente convexo}, $w^*$ su m\'inimo y que existen $A,B \geq 0$ tal que para todo $w \in \R^d$ vale que:
	
	\begin{equation*}
	\norm{\nabla F(w)}^2 \leq A + B \norm{w - w^*}^2
	\end{equation*}
	
	Luego si consideramos el algoritmo de \textit{descenso de gradiente por batch} tal que los incrementos $\sett{\alpha_k}$  cumplen la \textit{condici\'on Robbins - Monro} entonces:
	
	\begin{equation*}
	w_k \underlimitinf{k} w^*
	\end{equation*}
	
\end{theorem}

\begin{theorem}
	Sea $F: \R^d \rightarrow \R$ la funci\'on objetivo tal que $F \in C^1$, $F$ es \textit{L-Lipshitz} y \textit{PL-convexa}; entonces el algortimo \textit{descenso de gradiente por batch} con incremento fijo $\alpha_k = \frac{1}{L}$ cumple:
	\begin{equation*}
	F(w_k) - F_{inf} \leq \left(1 - \frac{\mu}{L}\right)^k \left(F(w_1) - F_{inf}\right)
	\end{equation*}
\end{theorem}

\subsection{Objetivos no convexos}

\begin{theorem}
	Sea $F: \R^d \rightarrow \R$ tal que $F \in C^2$ la funci\'on objetivo con Hessiano acotado con constante $L$, $w^*$ alg\'un m\'inimo local de $F$; entonces el algoritmo \textit{descenso de gradiente por batch} con incremento fijo $\alpha < \frac{1}{L}$ cumple:
	
	\begin{equation*}
		w_k \xrightarrow[k \rightarrow \infty]{c.t.p.} w^*
	\end{equation*}
	
\end{theorem}

\begin{theorem}
	Sea $F: \R^d \rightarrow \R$ tal que $F \in C^2$ la funci\'on objetivo con Hessiano acotado con constante $L$, $w^*$ alg\'un m\'inimo local de $F$; entonces el algoritmo \textit{punto pr\'oximo } con incremento fijo $\alpha < \frac{1}{L}$ cumple:
	
	\begin{equation*}
	w_k \xrightarrow[k \rightarrow \infty]{c.t.p.} w^*
	\end{equation*}
	
\end{theorem}

\begin{theorem}
	Sea $F: \R^d \rightarrow \R$ tal que $F \in C^2$  la funci\'on objetivo con Hessiano acotado por bloques con constante $L_b$, $w^*$ alg\'un m\'inimo local de $F$; entonces el algoritmo \textit{descenso de gradiente por coordenadas} con incremento fijo $\alpha < \frac{1}{L_b}$ cumple:
	
\begin{equation*}
w_k \xrightarrow[k \rightarrow \infty]{c.t.p.} w^*
\end{equation*}
	
\end{theorem}

\begin{theorem}
	Consideremos el algoritmo \textit{descenso de gradiente por batch} con $w_0$ elegido uniformemente en $[-1,1]^d$; luego existe $F : \R^d \mapsto \R$ funci\'on objetivo $B-$acotada, $l-$Lipshitz, $\mu-$Lipshitz en el Hessiano con $B,l,\mu \in \text{poly}(d)$ tal que si $\alpha_k = \alpha \leq \frac{1}{l}$ entonces $w_k$ va a estar a $\Omega(1)$ de cualquier m\'inimo para todo $k \leq e^{\Omega(d)}$
\end{theorem}

\section{Algoritmos estoc\'asticos}

\subsection{Objetivos convexos}

\begin{theorem}
	Sea $F: \R^d \rightarrow \R$ tal que $F \in C^1$ la funci\'on objetivo tal que existe $F_{inf}$ valor m\'inimo, $F$ es \textit{L-Lipshitz}, $F$ es \textit{fuertemente convexa} y supongamos adem\'as que $g$ tiene \textit{varianza acotada}; entonces el algoritmo \textit{descenso estoc\'astico de gradiente generalizado} con incremento fijo $0  <  \alpha_k = \alpha \leq \dfrac{\mu}{LM_G}$ cumple:

	\begin{equation*}
	\expectation{F(w_k) - F_{inf}} \underlimitinf{k} \dfrac{\alpha LM}{2 c \mu}
	\end{equation*}
	
\end{theorem}

\begin{theorem}
	Sea $F: \R^d \rightarrow \R$ tal que $F \in C^1$ la funci\'on objetivo tal que existe $F_{inf}$ valor m\'inimo, $F$ es \textit{L-Lipshitz}, $F$ es \textit{fuertemente convexa}; supongamos adem\'as que $g$ tiene \textit{varianza acotada} y que los incrementos $\alpha_k$ cumplen:
	
	\begin{equation}
	\alpha_k =  \dfrac{\beta}{\gamma + k} \quad \ \text{ para alg\'un } \ \beta > \frac{1}{c \mu} \  \text{ y } \ \gamma > 0 \  \text{ tal que } \ \alpha_1 \leq \dfrac{\mu}{L M_G} 
	\end{equation}
	
	Luego el algoritmo \textit{descenso estoc\'astico de gradiente generalizado} cumple::
	
	\begin{equation*}
		\expectation{R(w_k) - R^*} = \mathcal{O} \left(\frac{1}{k}\right)
	\end{equation*}
	
\end{theorem}

\begin{theorem}
	Sea $F: \R^d \rightarrow \R$ tal que $F \in C^1$ la funci\'on objetivo tal que existe $F_{inf}$ valor m\'inimo, $F$ es \textit{L-Lipshitz}, $F$ es \textit{fuertemente convexa}; supongamos adem\'as que $g$ tiene \textit{varianza acotada geom\'etricamente}. Luego el algoritmo \textit{descenso estoc\'astico de gradiente generalizado} con incremento fijo $0  < \alpha_k = \alpha \leq \min\sett{\dfrac{\mu}{L \mu_G^2}, \frac{1}{ \mu}} $ cumple:
	
	\begin{equation*}
	\expectation{R(w_k) - R^*} = \mathcal{O} \left(\rho^{k}\right)
	\end{equation*}
	
\end{theorem}

\begin{theorem}
	Sea $F: \R^d \rightarrow \R$ tal que $F \in C^1$ la funci\'on objetivo tal que existe $F_{inf}$ valor m\'inimo, $F$ es \textit{d\'ebilmente convexo}; supongamos adem\'as que $g$ tiene \textit{varianza acotada} y que los incrementos cumplen la \textit{condici\'on de Robbins Monro}. Luego el algoritmo \textit{descenso estoc\'astico de gradiente generalizado} con incremento decrecientes cumple:
	
	\begin{subequations}
		\begin{equation*}
		w_k \xrightarrow[k \rightarrow \infty]{c.t.p.} w^*
		\end{equation*}
		\begin{equation*}
		\left(w_k - w^*\right)^T \nabla F(w_k) \xrightarrow[k \rightarrow \infty]{c.t.p.}  0
		\end{equation*}
	\end{subequations}
	
\end{theorem}

\subsection{Objetivos no convexos}

\begin{theorem}
	Sea $F: \R^d \rightarrow \R$ tal que $F \in C^1$ la funci\'on objetivo , $F$ es \textit{L-Lipshitz} y supongamos adem\'as que $g$ tiene \textit{varianza acotada}. Luego el algoritmo \textit{descenso estoc\'astico de gradiente generalizado} con incremento fijo $0  < \alpha_k = \alpha \leq \dfrac{\mu}{LM_G} $ cumple:
	
	\begin{equation*}
		\expectation{\dfrac{1}{K}\sum\limits_{k=1}^{K} {\norm{\nabla F (w)_k}^2}} \underlimitinf{k}  \dfrac{\alpha LM}{\mu}
	\end{equation*}
	
\end{theorem}

\begin{theorem}
	Sea $F: \R^d \rightarrow \R$ tal que $F \in C^1$ la funci\'on objetivo, $F$ es \textit{L-Lipshitz}, supongamos adem\'as que $g$ tiene \textit{varianza acotada} y que los incrementos cumplen la \textit{condici\'on de Robbins Monro}. Luego si notamos $A_K := \sum\limits_{k=1}^{K} {\alpha_k}$, el algoritmo \textit{descenso estoc\'astico de gradiente generalizado} con incrementos decrecientes cumple:
	
	\begin{equation*}
		\expectation{\dfrac{1}{A_K}\sum\limits_{k=1}^{K} {\alpha_k \norm{\nabla F (w)_k)}^2}} \underlimitinf{k} 0
	\end{equation*}
	
\end{theorem}

\begin{theorem}
	Sea $F: \R^d \rightarrow \R$ tal que $F \in C^2$ la funci\'on objetivo, $F$ es \textit{L-Lipshitz} y $w \mapsto \norm{\nabla F (w)}_2^2$ sea \textit{L-Lipshitz}; supongamos adem\'as que $g$ tiene \textit{varianza acotada} y que los incrementos cumplen la \textit{condici\'on de Robbins Monro}. Luego el algoritmo \textit{descenso estoc\'astico de gradiente generalizado} con incrementos decrecientes cumple:
	
	\begin{equation*}
	\lim\limits_{k \rightarrow \infty} \expectation{\norm{\nabla F (w_k)}^2} = 0
	\end{equation*}
	
\end{theorem}

\begin{theorem}
	Sea $F: \R^d \rightarrow \R$  funci\'on objetivo y  $g$ un estimador insesgado de $\nabla F$ tal que ambos cumplen:
	
		\begin{enumerate}
		\item $F \in C^3$
		\item Existe $w^* \in \R^d$ tal que $F_{inf} = F(w^*) \leq F(w)$ para todo $w \in U$ entorno.
		\item $F(w) \geq 0$ para todo $w \in \R^d$
		\item Los incrementos cumplen la \textit{condici\'on de Robbins Monro}
		\item Existe $B_2 \geq 0$ tal que:
		
		\begin{equation*}
		\expectation{\norm{g(w_k, \upxi_{k})}^2} \leq B_2 \norm{w}^2
		\end{equation*}		
		
		\item Para $j=3, 4$ existen $A_j,B_j \geq 0$ tal que:
		
		\begin{equation*}
		\expectation{\norm{g(w_k, \upxi_{k})}^j} \leq A_j + B_j \norm{w}^j
		\end{equation*}
		
		\item  Existe $D > 0$ tal que:
		
		\begin{equation*}
		\inf\limits_{(w)^2 > D} {w^T \nabla F(w)} >0
		\end{equation*}
	\end{enumerate}
	
	
	Luego el algoritmo \textit{descenso estoc\'astico de gradiente generalizado} con incrementos decrecientes cumple:
	
	\begin{subequations}
		\begin{equation*}
		F(w_k) \xrightarrow[k \rightarrow \infty]{c.t.p.} \ F_{\infty}
		\end{equation*}
		\begin{equation*}
		\nabla F(w_k) \xrightarrow[k \rightarrow \infty]{c.t.p.} 0
		\end{equation*}
	\end{subequations}
	
\end{theorem}