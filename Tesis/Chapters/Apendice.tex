%********************************************************************
% Appendix
%*******************************************************
% If problems with the headers: get headings in appendix etc. right
%\markboth{\spacedlowsmallcaps{Appendix}}{\spacedlowsmallcaps{Appendix}}
\chapter{Ap\'endice}\label{ch: apendice}



\section{Proposiciones enunciadas}

\begin{theorem}
	\label{theorem: splines 1}
	Dados $y_0 < y_1$, valores $f(y_0), f(y_1)$ y sus derivadas $f'(y_0), f'(y_1)$ con $f'(y_0)<0$ el \textit{polinomio c\'ubico interpolante de Hermite} se define por:
	
	\begin{equation}
	p(y) = c_0 + c_1 \delta_y + c_2 \delta_y^2 + c_3 \delta_y^3
	\end{equation}
	
	Donde:
	
	\begin{equation*}
		\begin{array}{rcl}
		y & \in & [y_0, y_1] \\
		c_0 & = & f(y_0) \\
		c_1 & = & f'(y_0) \\
		c_2 & = & \dfrac{3S - f'(y_0) - 2f'(y_0)}{y_1 - y_0} \\
		c_3 & = & - \dfrac{2S - f'(y_1) - f'(y_0)}{\left(y_1 - y_0\right)^2} \\
		\delta_y & = & y - y_0 \\
		S & = & \dfrac{f(y_1) - f(y_0)}{y_1 - y_0}
		\end{array}
	\end{equation*}
	
	Y $p(y)$ satisface $p(y_0) = f(y_0), \ p(y_1) = f(y_1), \ p'(y_0) = f'(y_0)$ y $p'(y_1)=f'(y_1)$; adem\'as si $f(y_1) < f(y_0) < 0$ y:
	
	\begin{equation*}
		f'(y_1) \geq \dfrac{3 \left(f(y_1) - f(y_0)\right)}{y_1 - y_0}
	\end{equation*}
	
	Entonces para $y \in [y_0, y_1]$ vale que $p(y) \in \left[f(y_1), f(y_0)\right]$
	
\end{theorem}

\begin{proof}
	Ver \cite{dougherty:1989}
\end{proof}

\begin{theorem}
	\label{theorem: Extension de Whitney mejorado}
	Sea $E \subset \R^d$ y dotemos a $C^m(E)$ de la norma $\norm{f}_{C^m} = \sup\sett{\norm{\partial^{\alpha}f\vert_E }_{\infty} \tq \abs{\alpha} < m}$, si $E$ es cerrado en $\R^d$ entonces existe $T \in L\left(C^m(E), C^m\left(\R^d \right) \right)$ tal que $T(f) \vert_E = f$ y $T(f) \in C^{\infty}(E^c)$. Es m\'as, $\norm{T} \leq C(m) d^{\frac{5m}{2}}$.
\end{theorem}

\begin{proof}
	Ver \cite{cheng:2015}
\end{proof}

\section{Demostraciones}

\begin{proof}{[De \ref{prop: equivalencias convexidad fuerte}]}
	\color{Red} TODO
\end{proof}

\begin{proof}{[De \ref{prop: implicancias de convexidad fuerte}]}
	\color{Red} TODO
\end{proof}

\begin{proof}{[De \ref{prop: Implicancias L lipshitz}]}
	\color{Red} TODO
\end{proof}

\begin{proof}{[De \ref{lemma: Convergencia de sucesiones positivas acotadas sumables}]}
	Definamos:
	
	\begin{subequations}
		\begin{equation}
			S^+_t := \sum\limits_{k=1}^{t-1} {\left(u_{k+1} - u_k\right)_+}
		\end{equation}
		\begin{equation}
			S^-_t := \sum\limits_{k=1}^{t-1} {\left(u_{k+1} - u_k\right)_-}
		\end{equation}
	\end{subequations}
	
	Donde recordemos que $(x)_{\pm} = x1_{\sett{\R_{\pm}}}$. Como sabemos que $ {\left(u_{k+1} - u_k\right)_+} \geq 0$ para todo $k \in \N$ entonces $S_t^+ \nearrow S_{\infty}^+$; asimismo,  $ {\left(u_{k+1} - u_k\right)_-} \leq 0$ para todo $k \in \N$ entonces $S_t^- \leq 0$. Por lo tanto:
	
	\begin{subequations}
		\begin{equation}
			0 \leq u_k = u_0 + S_k^+ + S_k^- \leq u_0 + S_{\infty}^+
		\end{equation}
		\begin{equation}
			-u_0 - S_{\infty}^+ \leq S_k^- \leq 0
		\end{equation}
	\end{subequations}
	
	Luego como $S_{k+1}^- \leq S_k^-$ conclu\'imos que $S_{k}^- \searrow S_{\infty}^-$. Por lo tanto como $S_k^+, \ S_k^-$ convergen entonces $u_k = u_0 + S_k^+ + S_k^- $ converge. \qed
	
\end{proof}

