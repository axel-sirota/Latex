\chapter{Convergencia Puntual }\label{ch:convergenciaPuntual}

\epigraph{“El libro de la naturaleza esta escrito en el lenguaje de las Matem\'aticas"}{Galileo}

Si existe un algoritmo que todo estudiante o practicante del Machine Learning conoce, es el descenso de gradiente cl\'asico (o \textit{Descenso de gradiente en batch}) [Ver algoritmo \ref{algo: GD - intro}].

Un buen inicio es analizar la convergencia puntual del descenso de gradiente y bajo qu\'e condiciones se da. Para esto recordemos que dado una funci\'on de p\'erdida objetivo uno estima el \textit{riesgo esperado} por el \textit{riesgo emp\'irico} (Ver \ref{eq: Riesgo empirico}), luego considerando esto el algoritmo a analizar resulta:

\RestyleAlgo{boxruled}
\LinesNumbered
\begin{algorithm}[H]
	\caption{Descenso de gradiente en batch \label{algo: GD}}
	\textbf{Input:} $F: \R^d \rightarrow \R$, $f:\R^d \rightarrow \R$ tal que $f \in C^1$, $\alpha_k >0$, $N \in \N$, $w_1 \in \R^d$, $X = \sett{\upxi_j}_{j \leq N}$ \text{ muestra aleatoria i.i.d.} \\
	\For{$k \in \N$}{
		$w_{k+1} \gets w_{k} - \alpha_k \sum\limits_{j=1}^{N} \nabla F(\upxi_j) = w_{k} - \alpha_k \nabla R_N(w_k)$
	}
\end{algorithm}

Asumamos por esta secci\'on la siguiente condici\'on, que llamaremos \textit{convexidad d\'ebil}:

\begin{definition}
	\label{def: Debilmente convexo}
	Decimos que $F: \R^d \rightarrow \R$, $f:\R^d \rightarrow \R$ tal que $f \in C^1$ es \textit{d\'ebilmente convexo} si cumple las siguientes dos propiedades:
	
	\begin{itemize}
		\item Existe un \'unico $w^*$ tal que $F_{inf} := F(w^*) \leq F(w)$ para todo $w \in \R^n$.
		\item Para todo $\epsilon > 0$ vale que $\inf\limits_{\norm{w-w^*}^2 > \epsilon} {\left(w - w^*\right) \nabla F(w) > 0}$
	\end{itemize}
	
\end{definition}

\begin{remark}
	Si  $F: \R^d \rightarrow \R$, tal que $F \in C^1$ es convexa, entonces es d\'ebilmente convexa.
	
	En efecto, como $F$ es convexa existe $w* \in \R^d$ m\'inimo global y como $F \in C^1$ entonces $\nabla F(w^*) = 0$. Entonces $F$ cumple:
	
	\begin{equation*}
		\left(\nabla F(x) - \nabla F(y)\right)^T \left(x-y\right) >0
	\end{equation*}
	
	Para todos $x,y \in \R^d$; en particular si $y = w^*$ vemos que $F$ es d\'ebilmente convexa.
	
\end{remark}

\begin{remark}
	Notemos que existen funciones no convexas tal que cumplen \ref{def: Debilmente convexo}. En efecto, sea $M > \epsilon >0$ y basta con tomar:
	
	\begin{equation*}
	f(x) = \left\lbrace \begin{array}{cc}
	x^2 & \text{ si } \abs{x}< M - \epsilon \\
	g_+(x) & \text{ si } M - \epsilon <  x < M + \epsilon \\
	g_-(x) & \text{ si } -M - \epsilon < x < -M + \epsilon \\
	\sqrt{x} & \text{ si } \abs{x} > M+ \epsilon
	\end{array} \right.
	\end{equation*}
	
	Con $g_+,g_-$ funciones tal que $f$ resulte $C^1$ [Para ver la existencia basta ver \ref{ch: apendice}]. Luego $f$ es d\'ebilmente convexa pero no convexa pues $\sqrt{x}$ no resulta convexa en $\left(- \infty, -M - \epsilon \right) \cup \left(M + \epsilon , \infty\right)$
	
\end{remark}

\section{Intuici\'on}\label{section: Intuicion convergencia puntual batch}

A fin de ganar intuici\'on acerca del proceso de c\'omo probar la convergencia del algoritmo \ref{algo: GD} vamos a analizar el caso continuo. En el caso continuo, tenemos que demostrar que la soluci\'on $w(t)$ de la ecuaci\'on diferencial \ref{eq: Ecuacion diferencial de descenso de gradiente} tiene l\'imite $w^*$ y adem\'as que $w^*$ es m\'inimo de $F$.

\begin{equation}
\label{eq: Ecuacion diferencial de descenso de gradiente}
\dfrac{d w}{dt} = - \nabla F(w)
\end{equation}

Para eso, vamos a dividir la demostraci\'on en tres pasos:

\begin{enumerate}
	\item Vamos a definir $h \in C^1(\R)$ una \textit{funci\'on de Lyapunov}
	\item Vamos a verificar computando $\dfrac{dh}{dt}$ que es una funci\'on mon\'otona decreciente y acotada, por lo que converge
	\item Vamos a probar que $h(t) \underlimitinf{t} 0$.
\end{enumerate}

\begin{proposition}[Objetivo d\'ebilmente convexo, Versi\'on continua]
	\label{Objetivo debilmente convexo, GD continue converge}
	Sea $f:\R^d \rightarrow \R$ tal que $f \in C^1$ que cumple \ref{def: Debilmente convexo} y supongamos que el algoritmo \ref{algo: GD} cumple $\alpha_k=\alpha>0$ para $w(t)$ continua. Luego si notamos al m\'inimo de $F$ como $w^*$, vale:
	
	\begin{equation*}
		\lim\limits_{t \rightarrow \infty} w(t) =  w_*
	\end{equation*}

\end{proposition}

\begin{proof} Sigamos la estrategia definida anteriormente:
	
	
	\begin{enumerate}
		
		
		\item[Paso 1] Definamos la funci\'on de Lyapunov:
		
		\begin{equation*}
			h(t) = \left(w(t) - w^*\right)^2 \geq 0
		\end{equation*}
		
		\item[Paso 2] Notemos que:
		
		\begin{equation}
		\label{eq: Intuicion batch ctp, ecuacion derivada}
			\dfrac{dh}{dt} = 2 \left(w(t) - w^*\right) \dfrac{dw}{dt} = -2 \left(w(t) - w^*\right) \nabla F(w) \underbrace{<}_{\substack{\ref{def: Debilmente convexo} \\ w \neq w^*}} 0
		\end{equation}
		
		Luego como $h(t) \geq 0$ y $\dfrac{dh}{dt} \leq C < 0$ para $C \in \R$ contante, existe $h_{inf} \geq 0$ tal que $h(t) \searrow h_{inf}$
		
		\item[Paso 3] Como $h(t) \searrow h_{inf}$ entonces existe una sucesi\'on $\sett{t_k}_{k \in \N} \subset \R_+$ tal que $\dfrac{dh}{dt}(t_k) \underlimitinf{k} 0$. 
		
		Supongamos por el absurdo que $h_{inf} > 0$, luego existe $\widetilde{\epsilon} >0$ y $K \in \N$ tal que para todo $k \geq K$ vale que $h(t_k) = \left(w(t_k) - w^*\right)^2 > \widetilde{\epsilon}$. Si juntamos entonces el paso anterior con la definici\'on \ref{def: Debilmente convexo} llegamos a un absurdo; luego concluimos que $h_{inf} = 0$ por lo que:
		
		\begin{equation*}
			w(t) \rightarrow w_*
		\end{equation*}
		\qed
	\end{enumerate}
\end{proof}

\section{Caso discreto}

Ahora s\'i, analicemos la convergencia del algoritmo \ref{algo: GD} para el caso discreto. Para esto enunciamos un lema \'util cuya demostraci\'on referimos al lector al Ap\'endice:

\begin{lemma}
	\label{lemma: Convergencia de sucesiones positivas acotadas sumables}
	Sea $\sett{u_k}_{k \in \N} \subset \R$ una sucesi\'on tal que $u_k \geq 0$ para todo $k$. Luego si:
	
	\begin{equation*}
		\sum\limits_{k=1}^{\infty} {\left(u_{k+1} - u_k\right)_{+}} < \infty
	\end{equation*}
	
	Con:
	
	\begin{equation*}
		(x)_+ = \left\lbrace \begin{array}{cc}
		x & \text{ si } x >0 \\
		0 & \text{ si no}
		\end{array}\right.
	\end{equation*}
	
	Y $(x)_-$ definida an\'alogamente. Luego:
	
	\begin{equation*}
		\sum\limits_{k=1}^{\infty} {\left(u_{t+1} - u_t\right)_{-}} < \infty
	\end{equation*}
	
	 y $\left(u_k\right)$ converge. 
	 
	 Es m\'as, si notamos $S^{\pm}_{\infty} = \sum\limits_{k=1}^{\infty}  {\left(u_{t+1} - u_t\right)_{\pm}} $ entonces $u_{\infty} = \lim\limits_{k \rightarrow \infty} {u_k} = u_0 + S_{\infty}^+ + S_{\infty}^-$
\end{lemma}

\begin{proof}
	Ver \ref{ch: apendice}
\end{proof}

Consideremos ahora el algoritmo \ref{algo: GD}, decimos que los incrementos $\sett{\alpha_k}$  cumplen la condici\'on de \textit{Robbins - Monro} (ver \cite{robbins:1951}) si:

\begin{equation}
\label{eq: Condicion robbins monro}
\sum\limits_{k=1}^{\infty} {\alpha_k} = \infty \quad \text{ y } \quad \sum\limits_{k=1}^{\infty} {\alpha_k^2} < \infty
\end{equation}

\begin{theorem}[Objetivo d\'ebilmente convexo, incrementos decrecientes]
	\label{theorem: convergencia puntual batch, objetivo debilmente convexo, incrementos decrecientes}
	Sea $F: \R^d \rightarrow \R$ tal que  $F\in C^1$, cumple la definici\'on \ref{def: Debilmente convexo} y existen $A,B \geq 0$ tal que para todo $w \in \R^d$ vale que:
	
	\begin{equation}
	\label{eq: Condicion gradiente acotado, ctp, batch}
		\norm{\nabla F(w) }^2 \leq A + B \norm{w - w^*}^2
	\end{equation}
	
	Luego si consideramos el algoritmo \ref{algo: GD} con incrementos $\sett{\alpha_k}$ que cumplen la condici\'on \ref{eq: Condicion robbins monro} entonces:
	
	\begin{equation}
		w_k \underlimitinf{k} w^*
	\end{equation}
	
\end{theorem}

\begin{proof}
	Hagamos los 3 pasos an\'alogos a la demostraci\'on de la proposici\'on \ref{Objetivo debilmente convexo, GD continue converge}:
	
	\begin{enumerate}
		\item [Paso 1] Sea $h_k = \norm{w_k - w^*}^2$.
		\item [Paso 2] An\'alogo a la demostraci\'on de la proposici\'on \ref{Objetivo debilmente convexo, GD continue converge} notemos que:
		
		\begin{equation*}
			\begin{aligned}
			h_{k+1} = & \norm{w_{k+1} - w^*}^2 \\
			 = & \ip{\left(w_k - w^*\right) - \alpha_k \nabla F(w_k), \left(w_k - w^*\right) - \alpha_k \nabla F(w_k)} \\
			 = & \norm{w_k - w^*}^2 - 2  \alpha_k  \left(w_k - w^*\right)^T\nabla F(w_k) + \alpha_k^2 \norm{\nabla F(w_k)}^2
			\end{aligned}
		\end{equation*}
		
		Luego:
		
		\begin{equation*}
			h_{k+1} - h_k = -2 \alpha_k \left(w_k - w^* \right)^T\nabla F(w_k) + \alpha_k^2 \norm{\nabla F(w_k)}^2
		\end{equation*}
		
		A diferencia de antes la naturaleza discreta del algoritmo lleva a un t\'ermino positivo en las variaciones, pero si usamos la condici\'on \ref{eq: Condicion robbins monro} y la hip\'otesis \ref{eq: Condicion gradiente acotado, ctp, batch} entonces:
		
		\begin{equation*}
		\begin{aligned}
			h_{k+1} - h_k = & \underbrace{-2 \alpha_k \left(w_k - w^* \right)^T\nabla F(w_k)}_{\substack{\leq 0 \\ \text{ por definici\'on \ref{def: Debilmente convexo}}}}  + \alpha_k^2 \norm{\nabla F(w_k)}^2 \\
			\leq & \alpha_k^2 \norm{\nabla F(w_k)}^2 \\
			\leq & \alpha_k^2 \left(A + B h_k\right)
		\end{aligned}
		\end{equation*}
		
		Por lo que:
		
		\begin{equation}
		\label{eq: Variaciones sucesion lyapunov, objetivo debilmente convexo, incrmentes decrecientes}
			h_{k+1} - \left(1 + \alpha_k^2 B\right)h_k \leq \alpha_k^2A 
		\end{equation}
		
		Definamos ahora las sucesiones auxiliares:
		
		\begin{subequations}
			\begin{equation}
				\mu_k = \prod\limits_{j=1}^{k-1} {\dfrac{1}{1 + \alpha_j^2B}}
			\end{equation}
			\begin{equation}
				h_k' = \mu_k h_k
			\end{equation}
		\end{subequations}
		
		Notemos que $\log (\mu_k) = - \sum\limits_{j=1}^{k-1} {\log \left(1 + \underbrace{\alpha_j^2 B}_{\geq 0}\right)} \geq - B\sum\limits_{j=1}^{k-1} {\alpha_j^2} \geq -B \sum\limits_{j=1}^{\infty} {\alpha_j^2}$, por lo que $\mu_k$ es una sucesi\'on decreciente acotada inferiormente por $e^{-B \sum\limits_{j=1}^{\infty} {\alpha_j^2}}$, luego $\mu_k \searrow \mu_{\infty}$ >0. Ahora si volvemos  a la ecuaci\'on \ref{eq: Variaciones sucesion lyapunov, objetivo debilmente convexo, incrmentes decrecientes} del paso anterior y observamos que $\mu_{k+1} = \dfrac{\mu_k}{1 + B \alpha_k^2}$ tenemos que:
		
		\begin{equation*}
			h_{k+1}' - h_{k}' \leq \alpha_k^2 A \mu_k \leq \alpha_k^2 A
		\end{equation*}
		
		Como $\sum\limits_{k=1}^{\infty} {\alpha_k^2 A} < \infty$ entonces $\sum\limits_{k=1}^{\infty}{h_{k+1}' - h_{k}'} < \infty$ y por \ref{lemma: Convergencia de sucesiones positivas acotadas sumables} concluimos que $\sett{h_k'}$ converge; como $\underbrace{\mu_k}_{\geq 0} \ \rightarrow \ \mu_{\infty} > 0$ entonces $\sett{h_k}$ converge.
		
		\item[Paso 3] De la ecuaci\'on \ref{eq: Variaciones sucesion lyapunov, objetivo debilmente convexo, incrmentes decrecientes} del paso anterior vimos que $	h_{k+1} - \left(1 + \alpha_k^2 B\right)h_k$ es sumable concluimos que:
		
		\begin{equation*}
			\sum\limits_{k=1}^{\infty} {\alpha_k \left(w_k - w^* \right)\nabla F(w_k)} < \infty
		\end{equation*}
		
		Supongamos que $h_k \rightarrow h_{inf} \neq 0$, luego existir\'ia $K\in \N$ y $\widetilde{\epsilon} >0$ tal que $h_k = \left(w_k - w^*\right)^2 > \widetilde{\epsilon}$ para todo $k \geq K$; luego de la definici\'on \ref{def: Debilmente convexo} concluimos que existe $M > 0$ tal que $M \leq \left(w_k - w^* \right)\nabla F(w_k) $ para todo $k \geq K$. Por la condici\'on \ref{eq: Condicion robbins monro} eso implica que $\sum\limits_{k=1}^{\infty} {\alpha_k \left(w_k - w^* \right)\nabla F(w_k)} = \infty$, concluimos que $w_k \underlimitinf{k} w^*$. \qed
		
	\end{enumerate}
	
\end{proof}

\begin{corollary}
	Sea $F \in C^2$, asumamos \ref{def: Debilmente convexo} y que $\norm{\nabla^2 F}_2^2 \leq L$; si consideramos el algoritmo \ref{algo: GD} tal que incrementos $\sett{\alpha_k}$  cumplen \ref{eq: Condicion robbins monro} entonces:
	
	\begin{equation}
	w_k \underlimitinf{k} w^*
	\end{equation}
\end{corollary}

\section{Acerca de convexidad fuerte y funciones L-Lipschitz}

Como ya notamos previamente, la condici\'on de convexidad (en alguna medida) es central para analizar la convergencia de los algoritmos comunes en Machine Learning. Por lo tanto es una buena inversi\'on dedicar esta secci\'on a repasar las equivalencias de dos condiciones que van a aparecer repetidamente: \textbf{L-Lipschitz} y \textbf{Convexidad fuerte}.

\begin{definition}
	\label{def: Fuertemente convexa}
	Sea $f:\R^d \rightarrow \R$ tal que $f \in C^1$ y $\mu >0$, decimos que es \textit{fuertemente convexa} o $\mu$\textit{-convexa} si para todos $x,y \in \R^d$ vale:
	
	\begin{equation}
		f(y) \geq f(x) + \nabla f(x)^T\left(y-x\right) + \frac{\mu}{2} \norm{y-x}_2^2
	\end{equation}
\end{definition}

\begin{proposition}
	\label{prop: equivalencias convexidad fuerte}
	Sea $f:\R^d \rightarrow \R$ tal que $f \in C^1$ y $\mu >0$, entonces son equivalentes:
	\begin{enumerate}
		\item $f(y) \geq f(x) + \nabla f(x)^T\left(y-x\right) + \frac{\mu}{2} \norm{y-x}_2^2$ para todos $x,y \in \R^d$
		\item $g(x) = f(x) - \frac{\mu}{2} \norm{x}_2^2$ es convexa para todo $x \in \R^d$
		\item $\left(\nabla f(y) - \nabla f(x)\right)^T \left(y-x\right) \leq \mu \norm{y-x}_2^2$ para todos $x,y \in \R^d$
		\item $f(\alpha x + (1-\alpha)y) \leq \alpha f(x) + (1-\alpha) f(y) - \dfrac{\alpha \left(1 - \alpha \right) \mu}{2}\norm{y-x}_2^2$ para  todos $x,y \in \R^d$, $\alpha \in [0,1]$
	\end{enumerate}
\end{proposition}

\begin{proof}
	Ver \ref{ch: apendice}
\end{proof}

\begin{definition}
	\label{def: Condicion PL}
	Decimos que una funci\'on $f:\R^d \rightarrow \R$ tal que $f \in C^1$ es $PL-$convexa, o cumple la condici\'on de \textit{Polyak- Lojasiewicz} (ver \cite{polyak:1963}, \cite{lojasiewicz:1963}) si existe $\mu >0$ tal que para todo $x \in \R^d$ vale:
	\begin{equation}
	\frac{1}{2} \norm{\nabla f(x)}^2_2 \geq \mu \left(f(x) - f_{inf}\right)
	\end{equation}
\end{definition}

\begin{proposition}
	\label{prop: implicancias de convexidad fuerte}
	Sea $f:\R^d \rightarrow \R$ tal que $f \in C^1$ una funci\'on $\mu-$convexa, entonces valen:
	\begin{enumerate}
		\item $f$ es $PL$-convexa
		\item $\norm{\nabla f(x) - \nabla f(y)}_2 \geq \mu \norm{x-y}_2$
		\item $f(y) \leq f(x) + \nabla f(x)^T (y-x) + \frac{1}{2 \mu} \norm{\nabla f(x) - \nabla f(y)}_2^2$
		\item $ \left(\nabla f(x) - \nabla f(y)\right)^T (x-y) \leq \frac{1}{\mu} \norm{\nabla f(x) - \nabla f(y)}_2^2$
	\end{enumerate}
\end{proposition}

\begin{proof}
	Ver \ref{ch: apendice}
\end{proof}

\begin{definition}
	Dada una funci\'on objetivo $f:\R^d \rightarrow \R$ tal que $f \in C^1$ definimos el \textit{problema de optimizaci\'on regularizado en $L2$} cuando se minimiza $\tilde{f} = f + \lambda \norm{x}_2^2$ para $\lambda >0$.
\end{definition}

En Machine Learning es usual resolver el problema de optimizaci\'on regularizado en vez de el problema original. Esto se debe mayormente al siguiente resultado, que aunque no lo usemos \textit{per-se} en esta tesis, es de sumo inter\'es:

\begin{corollary}
	\label{coro: convexa y fuertemente convexa es fuertemente convexa}
	Sea $h = f+g$ donde $g$ es fuertemente convexa y $f$ es convexa, entonces $h$ es fuertemente convexa. En particular, si $f$ es convexa entonces $\tilde{f} = f + \lambda \norm{x}_2^2$ es fuertemente convexo.
\end{corollary}

\begin{proof}
	Sean $x,y \in \R^d$ y $\alpha \in [0,1]$, luego:
	\begin{equation*}
	\begin{array}{rcl}
	h\left(\alpha x + (1-\alpha)y\right) & = & f\left(\alpha x + (1-\alpha)y\right)  + g\left(\alpha x + (1-\alpha)y\right)  \\
	& \underbrace{\leq}_{\text{Por proposici\'on \ref{prop: equivalencias convexidad fuerte}}} & \alpha(f + g)(x) + (1-\alpha)(f+g)(y) - \dfrac{\mu \alpha \left(1-\alpha\right)}{2} \norm{x-y}^2_2 \\
	& = & \alpha h(x) + (1-\alpha)h(y) - \dfrac{\mu \alpha \left(1-\alpha\right)}{2} \norm{x-y}^2_2 
	\end{array}
	\end{equation*}\qed
\end{proof}

Una condici\'on dual a la de convexidad fuerte es la de $L-$ Lipschitz.

\begin{definition}
	\label{def: L Lipschitz}
	Sea $f:\R^d \rightarrow \R$ tal que $f \in C^1$, decimos que es $L$\textit{-Lipschitz} global si existe $L > 0$ tal que para todos $x,y \in \R^d$ vale:
	\begin{equation}
	\norm{\nabla f(x) - \nabla f (y)}_2 \leq  L\norm{y-x}_2
	\end{equation}
\end{definition}

\begin{proposition}
	\label{prop: Implicancias L Lipschitz}
	Sea $f:\R^d \rightarrow \R$ tal que $f \in C^1$, entonces para las siguientes propiedades:
	\begin{enumerate}
		\item $\norm{\nabla f(x) - \nabla f (y)}_2 \leq  L\norm{y-x}_2$
		\item $g(x) = \frac{L}{2} x^Tx - f(x)$ es convexa
		\item $f(y) \leq f(x) + \nabla f(x)^T (y-x) + \frac{L}{2}\norm{y-x}_2^2$
		\item $\left(\nabla f(x) - \nabla f(y)\right)^T (x-y) \leq L\norm{x-y}^2_2$
		\item $f\left(\alpha x + (1-\alpha)y\right) \geq \alpha f(x) + (1-\alpha) f(y) - \dfrac{\alpha \left(1 - \alpha \right)}{2L}\norm{y-x}_2^2$
		\item $f(y) \geq f(x) + \nabla f(x)^T (y-x) +\frac{1}{2L} \norm{y-x}^2_2$
		\item $\left(\nabla f(x) - \nabla f(y)\right)^T (x-y) \geq \frac{1}{L}\norm{\nabla f(x)- \nabla f(y)}^2_2$
		\item $f\left(\alpha x + (1-\alpha)y\right) \leq \alpha f(x) + (1-\alpha) f(y) - \dfrac{\alpha \left(1 - \alpha \right)}{2L}\norm{\nabla f(y)- \nabla f (x)}_2^2$
	\end{enumerate}

	Valen las siguientes cadenas de equivalencias:
	
	\begin{equation*}
		6 \ \Longleftrightarrow \ 8 \Longrightarrow \ 7 \Longrightarrow \ 1 \Longrightarrow \ 2 \ \Longleftrightarrow \ 3 \ \Longleftrightarrow \ 4 \ \Longleftrightarrow \ 5
	\end{equation*}
	
	Es m\'as, si $f$ adem\'as es $\mu-$convexa entonces las 8 propiedades son equivalentes
\end{proposition}

\begin{proof}
	Ver \ref{ch: apendice}
\end{proof}

Con todas estas propiedades, probemos el resultado hist\'orico de  \cite{polyak:1963}

\begin{theorem}[Convergencia lineal , Objetivos L-Lipschitz y PL-convexos]
	\label{theorem: convergencia lineal GD}
	Sea $F:\R^d \rightarrow \R$ tal que $F \in C^1$, existe $F_{inf}$ valor m\'inimo, $F$ es $L$-Lipschitz y $PL$-convexa; entonces el algortimo \ref{algo: GD} con incremento fijo $\alpha_k = \frac{1}{L}$ cumple:
	\begin{equation}
	F(w_k) - F_{inf} \leq \left(1 - \frac{\mu}{L}\right)^k \left(F(w_1) - F_{inf}\right)
	\end{equation}
\end{theorem}

\begin{proof}
	Notemos que si usamos \ref{prop: Implicancias L Lipschitz} entonces tenemos:
	
	\begin{equation*}
		F(w_{k+1}) - F(w_k) \leq \nabla F(w_k)^T \left(-\frac{1}{L} \nabla F(w_k)\right) + \frac{L}{2} \norm{\frac{\nabla F(w_k)}{L}}_2^2 \leq -\frac{1}{2L} \norm{\nabla F(w_k)}_2^2
	\end{equation*}
	
	Luego por la definici\'on \ref{def: Condicion PL} tenemos:
	
	\begin{equation*}
		F(w_{k+1}) - F(w_k) \leq - \frac{\mu}{L} \left(F(w_k) - F_{inf}\right)
	\end{equation*}
	
	Luego obtenemos:
	
	\begin{equation*}
	F(w_{k+1}) - F_{inf} \leq \left(1 - \frac{\mu}{L} \right) \left(F(w_k) - F_{inf}\right) \leq \left(1 - \frac{\mu}{L}\right)^k \left(F(w_1) - F_{inf}\right)
	\end{equation*}\qed
	
	
\end{proof}
