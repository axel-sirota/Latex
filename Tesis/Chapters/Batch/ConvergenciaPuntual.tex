\chapter{Convergencia Puntual }\label{ch:convergenciaPuntual}

\epigraph{“The book of nature is written in the language of Mathematic”}{Galileo}

Si existe un algoritmo que todo estudiante o practicante del Machine Learning conoce, es el descenso de gradiente cl\'asico (o \textit{Descenso de gradiente en batch}) [Ver algoritmo \ref{algo: GD}].
Un buen inicio es analizar la convergencia puntual del descenso de gradiente y bajo que condiciones se da.

\RestyleAlgo{boxruled}
\LinesNumbered
\begin{algorithm}[H]
	\caption{Descenso de gradiente en batch \label{algo: GD}}
	\textbf{Input:} $F \in C^1$, $\alpha_k >0$, $w_1 \in \R^d$, $X = \sett{\upxi_j}_{j \leq N}$ \text{ muestra} \\
	\For{$k \in \N$}{
		$w_{k+1} \gets w_{k} - \alpha_k \sum\limits_{j=1}^{N} \nabla F(\upxi_j)$
	}
\end{algorithm}

Asumamos por esta secci\'on la siguiente condici\'on, que llamaremos \textit{convexidad d\'ebil}:

\begin{definition}
	\label{def: Debilmente convexo}
	Decimos que $F \in C^1$ es \textit{d\'ebilmente convexo} si cumple las siguientes dos propiedades:
	
	\begin{itemize}
		\item Existe un \'unico $w^*$ tal que $F_{inf} := F(w^*) \leq F(w)$ para todo $w \in \R^n$.
		\item Para todo $\epsilon > 0$ vale que $\inf\limits_{(w-w^*)^2 > \epsilon} {\left(w - w^*\right) \nabla F(w) > 0}$
	\end{itemize}
	
\end{definition}

\begin{remark}
	Notemos que existen funciones no convexas tal que cumplen \ref{def: Debilmente convexo}.
\end{remark}

\section{Intuici\'on}\label{section: Intuicion convergencia puntual batch}

Ganemos intuici\'on acerca del proceso como probar la convergencia del algoritmo \ref{algo: GD} en el caso continuo. En el caso continuo, tenemos que demostrar que la soluci\'on $w(t)$ de la ecuaci\'on diferencial \ref{eq: Ecuacion diferencial de descenso de gradiente} tiene l\'imite $w^*$ y adem\'as que $w^*$ es m\'inimo de $F$.

\begin{equation}
\label{eq: Ecuacion diferencial de descenso de gradiente}
\dfrac{d w}{dt} = - \nabla F(w)
\end{equation}

Para eso, vamos a dividir la demostraci\'on en tres pasos:

\begin{enumerate}
	\item Vamos a definir una \textit{funci\'on de Lyapunov}
	\item Vamos a verificar computando su derivada temporal que es una funci\'on mon\'otona decreciente y acotada, por lo que converge
	\item Vamos a probar que converge a 0.
\end{enumerate}

\begin{proposition}[Objetivo d\'ebilmente convexo, Versi\'on continua]
	\label{Objetivo debilmente convexo, GD continue converge}
	Sea $F \in C^1$ que cumple \ref{def: Debilmente convexo} y supongamos que el algoritmo \ref{algo: GD} cumple $\alpha_k=\alpha>0$ para $w(t)$ continua. Luego si notamos al m\'inimo de $F$ como $w^*$, vale:
	
	\begin{equation*}
		\lim\limits_{t \rightarrow \infty} w(t) =  w_*
	\end{equation*}

\end{proposition}

\begin{proof} Vayamos con los pasos que definimos:
	
	
	\begin{enumerate}
		
		
		\item[Paso 1] Definamos la funci\'on de Lyapunov:
		
		\begin{equation*}
			h(t) = \left(w(t) - w^*\right)^2 \geq 0
		\end{equation*}
		
		\item[Paso 2] Notemos que:
		
		\begin{equation}
		\label{eq: Intuicion batch ctp, ecuacion derivada}
			\dfrac{dh}{dt} = 2 \left(w(t) - w^*\right) \dfrac{dw}{dt} = -2 \left(w(t) - w^*\right) \nabla F(w) \underbrace{\leq}_{\ref{def: Debilmente convexo}} 0
		\end{equation}
		
		Luego como $h(t) \geq 0$ y $\dfrac{dh}{dt} \leq 0$ existe $h_{inf}$ tal que $h(t) \searrow h_{inf}$
		
		\item[Paso 3] Como $h(t) \searrow h_{inf}$ entonces $\dfrac{dh}{dt} \rightarrow 0$, supongamos por el absurdo que $h_{inf} > 0$, luego existe $\widetilde{\epsilon} >0$ y $T \in \R$ tal que para todo $t \geq T$ vale que $h(t) = \left(w(t) - w^*\right)^2 > \widetilde{\epsilon}$. Si juntamos entonces \ref{eq: Intuicion batch ctp, ecuacion derivada} y \ref{def: Debilmente convexo} llegamos a un absurdo, conclu\'imos que $h_{inf} = 0$ por lo que:
		
		\begin{equation*}
			w(t) \rightarrow w_*
		\end{equation*}
		\qed
	\end{enumerate}
\end{proof}

\section{Caso discreto}

Ahora s\'i, analicemos la convergencia del algoritmo \ref{algo: GD} para el caso discreto. Para esto enunciamos un lema \'util cuya demostraci\'on referimos al lector al Ap\'endice:

\begin{lemma}
	\label{lemma: Convergencia de sucesiones positivas acotadas sumables}
	Sea $\sett{u_k}_{k \in \N} \subset \R$ una sucesi\'on tal que $u_k \geq 0$ para todo $k$. Luego si:
	
	\begin{equation*}
		\sum\limits_{k=1}^{\infty} {\left(u_{t+1} - u_t\right)_{+}} < \infty
	\end{equation*}
	
	Donde $(x)_{\pm} = x * 1_{\sett{\R_{\pm}}}$, entonces:
	
	\begin{equation*}
		\sum\limits_{k=1}^{\infty} {\left(u_{t+1} - u_t\right)_{-}} < \infty
	\end{equation*}
	
	 y $\left(u_k\right)$ converge. 
	 
	 Es m\'as, si notamos $S^{\pm}_{\infty} = \sum\limits_{k=1}^{\infty}  {\left(u_{t+1} - u_t\right)_{\pm}} $ entonces $u_{\infty} = \lim\limits_{k \rightarrow \infty} {u_k} = u_0 + S_{\infty}^+ + S_{\infty}^-$
\end{lemma}

\begin{proof}
	Ver \ref{ch: apendice}
\end{proof}

Consideremos ahora el algoritmo \ref{algo: GD}, decimos que los incrementos $\sett{\alpha_k}$  cumplen la condici\'on de \textit{Robbins - Monro} (ver \cite{robbins:1951}) si:

\begin{equation}
\label{eq: Condicion robbins monro}
\sum\limits_{k=1}^{\infty} {\alpha_k} = \infty \quad \text{ y } \quad \sum\limits_{k=1}^{\infty} {\alpha_k^2} < \infty
\end{equation}

\begin{theorem}[Objetivo d\'ebilmente convexo, incrementos decrecientes]
	\label{theorem: convergencia puntual batch, objetivo debilmente convexo, incrementos decrecientes}
	Sea $F \in C^1$, asumamos \ref{def: Debilmente convexo} y que existen $A,B \geq 0$ tal que para todo $w \in \R^d$ vale que:
	
	\begin{equation}
	\label{eq: Condicion gradiente acotado, ctp, batch}
		\left(\nabla F(w)\right)^2 \leq A + B \left(w - w^*\right)^2
	\end{equation}
	
	Luego si consideramos el algoritmo \ref{algo: GD} tal que incrementos $\sett{\alpha_k}$  cumplen \ref{eq: Condicion robbins monro} entonces:
	
	\begin{equation}
		w_k \rightarrow w^*
	\end{equation}
	
\end{theorem}

\begin{proof}
	Hagamos los 3 pasos an\'alogos a \ref{Objetivo debilmente convexo, GD continue converge}:
	
	\begin{enumerate}
		\item [Paso 1] Sea $h_k = \left(w_k - w^*\right)$ una sucesi\'on de Lyapunov
		\item [Paso 2] An\'alogo a \ref{Objetivo debilmente convexo, GD continue converge} notemos que:
		
		\begin{equation*}
			h_{k+1} - h_k = -2 \alpha_k \left(w_k - w^* \right)\nabla F(w_k) + \alpha_k^2 (\nabla F(w_k))^2
		\end{equation*}
		
		Notemos que a diferencia de antes la naturaleza discreta del algoritmo lleva a un t\'ermino positivo de ruido en las variaciones. Notemos que si usamos \ref{eq: Condicion robbins monro} y \ref{eq: Condicion gradiente acotado, ctp, batch} entonces:
		
		\begin{equation*}
		\label{eq: Variaciones sucesion lyapunov, objetivo debilmente convexo, incrmentes decrecientes}
			h_{k+1} - \left(1 + \alpha_k^2 B\right)h_k \leq \underbrace{-2 \alpha_k \left(w_k - w^* \right)\nabla F(w_k)}_{\leq 0 \text{ por \ref{def: Debilmente convexo}}} + \alpha_k^2A \leq \alpha_k^2A 
		\end{equation*}
		
		Definamos ahora las sucesiones auxiliares:
		
		\begin{subequations}
			\begin{equation}
				\mu_k = \prod\limits_{j=1}^{k-1} {\dfrac{1}{1 + \alpha_j^2B}}
			\end{equation}
			\begin{equation}
				h_k' = \mu_k h_k
			\end{equation}
		\end{subequations}
		
		Notemos que $\log (\mu_k) = - \sum\limits_{j=1}^{k-1} {\log \left(1 + \underbrace{\alpha_j^2 B}_{\geq 0}\right)} \geq - B\sum\limits_{j=1}^{k-1} {\alpha_j^2} \geq -B \sum\limits_{j=1}^{\infty} {\alpha_j}$, por lo que $\mu_k$ es una sucesi\'on decreciente acotada inferiormente por $e^{-B \sum\limits_{j=1}^{\infty} {\alpha_j}}$, luego $\mu_k \searrow \mu_{\infty}$ >0. Ahora si volvemos  a \ref{eq: Variaciones sucesion lyapunov, objetivo debilmente convexo, incrmentes decrecientes} tenemos que:
		
		\begin{equation*}
			h_{k+1}' - h_{k}' \leq \alpha_k^2 A \mu_k \leq \alpha_k^2 A
		\end{equation*}
		
		Como $\sum\limits_{k=1}^{\infty} {\alpha_k^2 A} < \infty$ entonces $\sum\limits_{k=1}^{\infty}{h_{k+1}' - h_{k}'} < \infty$ y por \ref{lemma: Convergencia de sucesiones positivas acotadas sumables} conclu\'imos que $\sett{h_k'}$ converge; como $\underbrace{\mu_k}_{\geq 0} \ \rightarrow \ \mu_{\infty} > 0$ entonces $\sett{h_k}$ converge.
		
		\item[Paso 3] De \ref{eq: Variaciones sucesion lyapunov, objetivo debilmente convexo, incrmentes decrecientes} como ya vimos que $	h_{k+1} - \left(1 + \alpha_k^2 B\right)h_k$ es sumable conclu\'imos que:
		
		\begin{equation*}
			\sum\limits_{k=1}^{\infty} {\alpha_k \left(w_k - w^* \right)\nabla F(w_k)} < \infty
		\end{equation*}
		
		Supongamos que $h_k \rightarrow h_{inf} \neq 0$, luego existir\'ia $K\in \N$ y $\widetilde{\epsilon} >0$ tal que $h_k = \left(w_k - w^*\right)^2 > \widetilde{\epsilon}$ para todo $k \geq K$; luego de \ref{def: Debilmente convexo} conclu\'imos que existe $M > 0$ tal que $M \leq \left(w_k - w^* \right)\nabla F(w_k) $ para todo $k \geq K$. Por \ref{eq: Condicion robbins monro} eso implica que $\sum\limits_{k=1}^{\infty} {\alpha_k \left(w_k - w^* \right)\nabla F(w_k)} = \infty$, conclu\'imos que $w_k \rightarrow w^*$. \qed
		
	\end{enumerate}
	
\end{proof}

\begin{corollary}
	Sea $F \in C^2$, asumamos \ref{def: Debilmente convexo} y que $\norm{\nabla^2 F}_2^2 \leq L$; si consideramos el algoritmo \ref{algo: GD} tal que incrementos $\sett{\alpha_k}$  cumplen \ref{eq: Condicion robbins monro} entonces:
	
	\begin{equation}
	w_k \rightarrow w^*
	\end{equation}
\end{corollary}

\section{Acerca de convexidad fuerte y funciones L-Lipshitz}

Como ya notamos previamente, la condici\'on de convexidad (en alguna medida) es central para analizar la convergencia de los algoritmos comunes en Machine Learning. Por lo tanto es una buena inversi\'on dedicar esta secci\'on a repasar las equivalencias de dos condiciones que van a aparecer repetidamente: \textbf{L-Lipshitz} y \textbf{Convexidad fuerte}.

\begin{definition}
	\label{def: Fuertemente convexa}
	Sea $f \in C^1$, decimos que es \textit{fuertemente convexa} o $\mu$\textit{-convexa} si existe $\mu > 0$ tal que para todos $x,y \in \R^d$ vale:
	
	\begin{equation}
		f(y) \geq f(x) + \nabla F(x)^T\left(y-x\right) + \frac{\mu}{2} \norm{y-x}_2^2
	\end{equation}
\end{definition}

\begin{proposition}
	\label{prop: equivalencias convexidad fuerte}
	Sea $f \in C^1$ una funci\'on $\mu-$convexa, entonces son equivalentes:
	\begin{enumerate}
		\item $f(y) \geq f(x) + \nabla F(x)^T\left(y-x\right) + \frac{\mu}{2} \norm{y-x}_2^2$ para todos $x,y \in \R^d$
		\item $g(x) = f(x) - \frac{\mu}{2} \norm{x}_2^2$ es convexa para todo $x \in \R^d$
		\item $\left(\nabla f(y) - \nabla f(x)\right)^T \left(y-x\right) \leq \mu \norm{y-x}_2^2$ para todos $x,y \in \R^d$
		\item $f(\alpha x + (1-\alpha)y) \leq \alpha f(x) + (1-\alpha) f(y) - \dfrac{\alpha \left(1 - \alpha \right) \mu}{2}\norm{y-x}_2^2$ para  todos $x,y \in \R^d$, $\alpha \in [0,1]$
	\end{enumerate}
\end{proposition}

\begin{proof}
	Ver \ref{ch: apendice}
\end{proof}

\begin{definition}
	\label{def: Condicion PL}
	Decimos que una funci\'on $f \in C^1$ es $PL-$convexa, o cumple la condici\'on de \textit{Polyak- Lojasiewicz} (ver \cite{polyak:1963}, \cite{lojasiewicz:1963}) si existe $\mu >0$ tal que para todo $x \in \R^d$ vale:
	\begin{equation}
	\frac{1}{2} \norm{\nabla f(x)}^2_2 \geq \mu \left(f(x) - f_{inf}\right)
	\end{equation}
\end{definition}

\begin{proposition}
	\label{prop: implicancias de convexidad fuerte}
	Sea $f \in C^1$ una funci\'on $\mu-$convexa, entonces valen:
	\begin{enumerate}
		\item $f$ es $PL$-convexa
		\item $\norm{\nabla f(x) - \nabla f(y)}_2 \geq \mu \norm{x-y}_2$
		\item $f(y) \leq f(x) + \nabla f(x)^T (y-x) + \frac{1}{2 \mu} \norm{\nabla f(x) - \nabla f(y)}_2^2$
		\item $ \left(\nabla f(x) - \nabla f(y)\right)^T (x-y) \leq \frac{1}{\mu} \norm{\nabla f(x) - \nabla f(y)}_2^2$
	\end{enumerate}
\end{proposition}

\begin{proof}
	Ver \ref{ch: apendice}
\end{proof}

De esto podemos deducir un resultado, que aunque no lo usemos \textit{per-se} en esta tesis, es de sumo interes:

\begin{corollary}
	Sea $h = f+g$ donde $f$ es fuertemente convexa y $g$ es convexa, entonces $h$ es fuertemente convexa. En particular, si $f$ es convexa entonces el \textit{problema regularizado en $L2$} de minimizar $h = f + \lambda \norm{x}_2^2$ es fuertemente convexo.
\end{corollary}

\begin{proof}
	Sean $x,y \in \R^d$ y $\alpha \in [0,1]$, luego:
	\begin{equation*}
	\begin{array}{rcl}
	h\left(\alpha x + (1-\alpha)y\right) & = & f\left(\alpha x + (1-\alpha)y\right)  + g\left(\alpha x + (1-\alpha)y\right)  \\
	& \underbrace{\leq}_{\ref{prop: equivalencias convexidad fuerte}} & \alpha(f + g)(x) + (1-\alpha)(f+g)(y) - \dfrac{\mu \alpha \left(1-\alpha\right)}{2} \norm{x-y}^2_2 \\
	& = & \alpha h(x) + (1-\alpha)h(y) - \dfrac{\mu \alpha \left(1-\alpha\right)}{2} \norm{x-y}^2_2 
	\end{array}
	\end{equation*}\qed
\end{proof}

Una condici\'on dual a la de convexidad fuerte es la de $L-$ Lipshitz.

\begin{definition}
	\label{def: L Lipshitz}
	Sea $f \in C^1$, decimos que es $L$\textit{-Lipshitz} si existe $L > 0$ tal que para todos $x,y \in \R^d$ vale:
	\begin{equation}
	\norm{\nabla f(x) - \nabla f (y)}_2 \leq  L\norm{y-x}_2
	\end{equation}
\end{definition}

\begin{proposition}
	\label{prop: Implicancias L lipshitz}
	Sea $f \in C^1$ una funci\'on $L-$Lipshitz, entonces para las siguientes propiedades:
	\begin{enumerate}
		\item $\norm{\nabla f(x) - \nabla f (y)}_2 \leq  L\norm{y-x}_2$
		\item $g(x) = \frac{L}{2} x^Tx - f(x)$ es convexa
		\item $f(y) \leq f(x) + \nabla f(x)^T (y-x) + \frac{L}{2}\norm{y-x}_2^2$
		\item $\left(\nabla f(x) - \nabla f(y)\right)^T (x-y) \leq L\norm{x-y}^2_2$
		\item $f\left(\alpha x + (1-\alpha)y\right) \geq \alpha f(x) + (1-\alpha) f(y) - \dfrac{\alpha \left(1 - \alpha \right)}{2L}\norm{y-x}_2^2$
		\item $f(y) \geq f(x) + \nabla f(x)^T (y-x) +\frac{1}{2L} \norm{y-x}^2_2$
		\item $\left(\nabla f(x) - \nabla f(y)\right)^T (x-y) \geq \frac{1}{L}\norm{\nabla f(x)- \nabla f(y)}^2_2$
		\item $f\left(\alpha x + (1-\alpha)y\right) \leq \alpha f(x) + (1-\alpha) f(y) - \dfrac{\alpha \left(1 - \alpha \right)}{2L}\norm{\nabla f(y)- \nabla f (x)}_2^2$
	\end{enumerate}

	Valen las siguientes cadenas de equivalencias:
	
	\begin{equation*}
		6 \ \Longleftrightarrow \ 8 \Longrightarrow \ 7 \Longrightarrow \ 1 \Longrightarrow \ 2 \ \Longleftrightarrow \ 3 \ \Longleftrightarrow \ 4 \ \Longleftrightarrow \ 5
	\end{equation*}
	
	Es m\'as, si $f$ adem\'as es $\mu-$convexa entonces las 8 propiedades son equivalentes
\end{proposition}

\begin{proof}
	Ver \ref{ch: apendice}
\end{proof}

Con todas estas propiedades, probemos el resultado hist\'orico de  \cite{polyak:1963}

\begin{theorem}[Convergencia lineal , Objetivos L-Lipshitz y PL-convexos]
	\label{theorem: convergencia lineal GD}
	Sea $F \in C^1$ tal que existe $F_{inf}$ valor m\'inimo, $F$ cumple \ref{def: L Lipshitz} y \ref{def: Condicion PL}; entonces el algortimo \ref{algo: GD} con incremento fijo $\alpha_k = \frac{1}{L}$ cumple:
	\begin{equation}
	F(w_k) - F_{inf} \leq \left(1 - \frac{\mu}{L}\right)^k \left(F(w_1) - F_{inf}\right)
	\end{equation}
\end{theorem}

\begin{proof}
	Notemos que si usamos \ref{prop: Implicancias L lipshitz} entonces tenemos:
	
	\begin{equation*}
		F(w_{k+1}) - F(w_k) \leq \nabla F(w_k)^T \left(-\frac{1}{L} \nabla F(w_k)\right) + \frac{L}{2} \norm{\frac{\nabla F(w_k)}{L}}_2^2 \leq -\frac{1}{2L} \norm{\nabla F(w_k)}_2^2
	\end{equation*}
	
	Luego por \ref{def: Condicion PL} tenemos:
	
	\begin{equation*}
		F(w_{k+1}) - F(w_k) \leq - \frac{\mu}{L} \left(F(w_k) - F_{inf}\right)
	\end{equation*}
	
	Luego obtenemos:
	
	\begin{equation*}
	F(w_{k+1}) - F_{inf} \leq \left(1 - \frac{\mu}{L} \right) \left(F(w_k) - F_{inf}\right) \leq \left(1 - \frac{\mu}{L}\right)^k \left(F(w_1) - F_{inf}\right)
	\end{equation*}\qed
	
	
\end{proof}
