\chapter{Resultados negativos}\label{ch:resultadosNegativos}

Ya vimos de \ref{Dg converge a minimos}, que el descenso de gradiente, con cualquier inicialización aleatoria razonable, siempre escapará de los puntos de silla estrictos \textit{eventualmente}, pero sin ninguna garantía sobre el número de pasos requeridos. Esto motiva a la siguiente
pregunta:

\textbf{¿El descenso de gradiente inicializado aleatoriamente generalmente escapa de los puntos de silla en tiempo polinomial?}

\section{Ejemplos \textit{patol\'ogicos}:}

\begin{example}[Inicializaci\'on uniforme en una banda exponencialmente chica]
	Consideremos $f \in C^2(\R^2)$ con un punto silla estricto en $(0,0)$. Supongamos que a orden chico en $U = [-1,1]^2$ un entorno del punto silla $f$ es localmente de la forma $f(x_1, x_2) = x_1^2 - x_2^2$, luego si utilizamos el algoritmo \ref{algo: GD} con $\alpha_k = \alpha = \frac{1}{4}$ nos queda:
	
	\begin{equation*}
		\left(x_1^{k+1}, x^{k+1}_2\right) = \left(x^k_1, x^k_2\right)- \frac{1}{4} \left(2x^k_1, -2x^k_2\right) = \left(\frac{x^k_1}{2}, \frac{3 x^k_2}{2}\right)
	\end{equation*}
	
	Luego si tomamos $\epsilon >0$ y $w_0 = (x_1^0, x_2^0)$ uniformemente en $w_0\in \widetilde{U} = [-1,1] \times \left[- \frac{3}{2}^{-e^{\frac{1}{\epsilon}}}, \frac{3}{2}^{-e^{\frac{1}{\epsilon}}}\right]$ entonces el algoritmo \ref{algo: GD} necesita $k \geq e^{\frac{1}{\epsilon}}$ pasos para que $w_k \not \in U$. Conclu\'imos que el algoritmo es exponencial en converger a cualquier m\'inimo si $w_0 \in \widetilde{U}$.\qed 
	
\end{example}

\begin{example}[Inicializaci\'on exponencialmente lejana]
	Consideremos nuevamente $f \in C^2(\R^2)$ dada por:
	
	\begin{equation*}
		f(x_1, x_2) = \left\lbrace \begin{array}{cc}
		x_1^2 - x_2^2 & \text{ si } x_1 \in (-1,1) \\
		-4x_1 + x_2^2 & \text{ si } x_1 < -2 \\
		h(x_1,x_2) & \text{ sino }
		\end{array}\right.
	\end{equation*}
	
	Con $h$ una funci\'on suave tal que $f \in C^2$ y $x_2$ no crezca demasiado en el intervalo donde es $h$ (Una forma de definir esto es con splines c\'ubicos).
	
	Luego si para el algoritmo \ref{algo: GD} tomamos $\alpha_k = \alpha = \frac{1}{4}$ tendr\'iamos la siguiente din\'amica:
	
	\begin{equation*}
	\begin{array}{rcl}
	\left(x_1^{k+1}, x^{k+1}_2\right) & = & \left\lbrace\begin{array}{cc}
	\left(x^k_1, x^k_2\right)- \frac{1}{4} \left(2x^k_1, -2x^k_2\right) & \text{ si } x_1 \in (-1,1) \\
	\left(x^k_1, x^k_2\right)- \frac{1}{4} \left(-4, 2x^k_2\right) & \text{ si } x_1 < -2 \\
	\end{array}\right. \\
	& = & \left\lbrace\begin{array}{cc}
	\left(\frac{x^k_1}{2}, \frac{3 x^k_2}{2}\right) & \text{ si } x_1 \in (-1,1) \\
	\left(x^k_1 + 1, \frac{x^k_2}{2}\right) & \text{ si } x_1 < -2 \\
	\end{array}\right.
	\end{array}
	\end{equation*}
	
	Luego si tomamos $R >0$ grande y $w_0 = (x_1^0, x_2^0)$ uniformemente en $w_0\in \widetilde{U} = \left[ -R-1, -R+1 \right]\times [-1,1]$ entonces notando $t$ como la primera vez que $x_1 > -1$ tenemos que $t \approx R$, con lo que $x_2^t = x_2^0 \left(\frac{1}{2}\right)^R$. Por ende, el algoritmo nuevamente  necesita $R \approx e^{\frac{1}{\epsilon}}$ iteraciones para poder salir de $U = [-1,1]^2$; conclu\'imos que el algoritmo es exponencial en converger a cualquier m\'inimo si $w_0 \in \widetilde{U}$.\qed 
	
\end{example}

\marginpar{Notacion $\Omega$}

\begin{theorem}[Convergencia exponencial, Inicializaci\'on uniforme en el cubo]
	Consideremos el algoritmo \ref{algo: GD} con $w_0$ elegido uniformemente en $[-1,1]^d$; luego existe $f : \R^d \mapsto \R$ $B-$acotada, $l-$Lipshitz, $\mu-$Lipshitz en el Hessiano con $B,l,\mu \in \text{poly}(d)$ tal que si $\alpha_k = \alpha \leq \frac{1}{l}$ entonces $w_k$ va a estar a $\Omega(1)$ de cualquier m\'inimo para todo $k \leq e^{\Omega(d)}$
\end{theorem}

Antes de pasa a la prueba veamos un ejemplo modelo para generar intuici\'on de la demostraci\'on:

\begin{remark}[Escapar de dos puntos silla consecutivos]
	Sean $L > \gamma > 0$ y $f \in [0,3] \times [0,3]$ dada por:
	
	\begin{equation}
		f(x_1, x_2) = \left\lbrace \begin{array}{cc}
		- \gamma x_1^2 + Lx_2^2 & \text{ si } (x_1,x_2) \in [0,1] \times [0,1] \\
		L \left(x_1 - 2\right)^2 - \gamma x_2^2 & \text{ si } (x_1,x_2) \in [1,3] \times [0,1] \\
		L \left(x_1 - 2\right)^2 + L \left(x_2 - 2\right)^2 & \text{ si } (x_1,x_2) \in [1,3] \times [1,3] \\
		\end{array} \right.
	\end{equation}
	
	Notemos que $f$ tiene dos puntos silla estrictos en $(0,0)$ y $(2,0)$, mientras que tiene un \'optimo en $(2,2)$. Sean $U = [0,1]^2$, $V= [1,3] \times [0,1]$ y $W = [1,3]^2$ entornos respectivos de los tres puntos cr\'iticos, supongamos que $w_0 = \left(x^0_1, x^0_2\right) \in U$ y definamos:
	
	\begin{subequations}
		\begin{equation*}
			k_1 = \inf\limits_{x^k_1 \geq 1}{k} = \min\limits_{x^k_1 \geq 1}{k}
		\end{equation*}
		\begin{equation*}
		k_2 = \inf\limits_{x^k_2 \geq 1}{k} = \min\limits_{x^k_2 \geq 1}{k}
		\end{equation*}
	\end{subequations}
	
	Notemos que como la direcci\'on de escape en $(0,0)$ es por $x_1$ y \textit{luego} por $x_2$ (por el cambio de comportamiento de $f$) podemos concluir que $k_1,k_2$ estan bien definidos y que $k_2 \geq k_1 \geq 0$; la observaci\'on clave va a ser que $k_2 = Ck_1$ con $C>1$, es decir que el tiempo en pasar el siguiente punto silla es exponencialmente mayor que los anteriores. En pos de esto, veamos como va a ser la iteraci\'on del algoritmo \ref{algo: GD}:
	
	\begin{equation*}
	\begin{array}{rcl}
	\left(x_1^{k+1}, x^{k+1}_2\right) & = & \left\lbrace\begin{array}{cc}
	\left(x^k_1, x^k_2\right)- \alpha\left(-2 \gamma x^k_1, 2Lx^k_2\right) & \text{ si } x_1 \leq 1\\
	\left(x^k_1, x^k_2\right)- \alpha\left(2L \left(x_1^k -2\right), -2 \gamma x^k_2\right) & \text{ si } x_1 \geq 1 \ , \ x_2 \leq 1 \\
	\left(x^k_1, x^k_2\right)- \alpha\left(2L \left(x_1^k -2\right), 2L \left(x_2^k -2\right)\right) & \text{ si } x_1 \geq 1 \ , \ x_2 \geq 1
	\end{array}\right. \\
	&&\\
	& = &\left\lbrace\begin{array}{cc}
	\left(\left(1 + 2\alpha \gamma \right)x^k_1, \left(1 - \alpha 2L\right)x^k_2\right) & \text{ si } x_1 \leq 1\\
	\left(\left(1 - 2L \alpha \right)x^k_1 + 4L \alpha, \left(1 + 2 \alpha \gamma \right)x^k_2\right) & \text{ si } x_1 \geq 1 \ , \ x_2 \leq 1 \\
	\left(\left(1 - 2L \alpha \right)x^k_1 + 4L \alpha, \left(1 - 2L \alpha \right)x^k_2 + 4L \alpha\right)  & \text{ si } x_1 \geq 1 \ , \ x_2 \geq 1
	\end{array}\right.
	\end{array}
	\end{equation*}
	
	Luego evaluando en $k_1$ y $k_2$:
	
	\begin{equation*}
	\begin{array}{rcl}
	x^{k_1}_1 & = & \left(1 + 2 \alpha \gamma\right)^{k_1} x^0_1 \\
	x^{k_1}_2 & = & \left(1 - 2 \alpha L\right)^{k_1} x^0_1 \\
	&&\\
	x^{k_2}_1 & = & \left(1 - 2L\alpha\right)^{k_2 - k_1}\left(1 + 2 \alpha \gamma\right)^{k_1} x^0_1 + K\geq 1 \quad K \text{ constante}\\
	x^{k_2}_2 & = & \left(1 + 2 \alpha \gamma\right)^{k_2-k_1}\left(1 - 2 \alpha L\right)^{k_1} x^0_2\geq 1 \\
	\end{array}
	\end{equation*}
	
	Conclu\'imos que:
	
	\begin{equation}
		k_2 \geq \dfrac{2 \alpha \left(L + \gamma\right)k_1 - \log\left(x_2^0\right)}{2 \alpha \gamma} \geq \dfrac{L + \gamma}{\gamma} k_1 
	\end{equation}
	
	Esta $f$ que presentamos tiene varios problemas:
	
	\begin{enumerate}
		\item $f$ no es continua, y mucho menos $C^2$
		\item $f$ no podemos asegurar que sea $l-$Lipshitz o $\mu-$Lipshitz en el hessiano
		\item Los puntos cr\'iticos estan en el borde del dominio, lo que no es ideal
		\item $f$ no est\'a definida en todo $\R^d$
		\item Estrictamente $f$ es a\'un resuelto en tiempo polinomial
	\end{enumerate}
	
	La clave va a ser usar splines para resolver los primeros puntos, espejar $f$ para hacer los puntos extremales interiores, asignar $d$ puntos cr\'iticos similares para generar el tiempo exponencial en $d$ y extender esa funci\'on $\tilde{f}$ a $\R^d$ con el Teorema de extensi\'on de Whitney. Aunque la demostraci\'on es larga y tediosa, la idea clave es la vista aqu\'i.
\end{remark}

\begin{proof} Vayamos de a pasos
		
	\medskip
	
	\textbf{Paso 1 - Definiciones}
	
	\medskip
	
	Fijemos 4 constantes: $L=e, \gamma = 1, \tau = e, \eta$ a definir proximamente; inspirados en el ejemplo anterior vamos a construir una $f$ definida en un cerrado $D_0$ tal que tenga $d-1$ puntos silla estrictos y la complejidas del algoritmo \ref{algo: GD} sea exponencial. Sea $D_0$ dado por:
	
	\begin{equation}
	\begin{array}{cl}
	D_ 0 = & \bigcup\limits_{i=1}^{d+1} \sett{x \in \R^d \tq 6\tau \geq x_1, \dots, x_{i-1} \geq 2\tau; 2\tau \geq x_i \geq 0 ; \tau \geq x_{i+1}, \dots, x_d \geq 0 }  \\
	:= & \bigcup\limits_{i=1}^{d+1}{D_i}
	\end{array}
	\end{equation}
	
	Y partamos $D_i = D_{i,1} \cup D_{i,2}$ donde $D_{i,1} = \sett{x \in D_i \tq 0 \leq x_i \leq \tau}$ y $D_{i,2} = \sett{x \in D_i, \tq \tau \leq x_i \leq 2\tau}$.
	
	Para un dado $1 \leq i \leq d-1$ definamos:
	
	\begin{equation}
	f\vert_{D_i}(x) = \left\lbrace \begin{array}{l}
	\sum\limits_{j=1}^{i-1}{L \left(x_j - 4\tau\right)^2}  - \gamma x_i^2 + \sum\limits_{j={i+1}}^{d} {L x_j^2 - \left(i-1\right)\eta } \\
	\triangleq f_{i,1}(x) \text{ si } x\in D_{i,1} \\
	\\
	\\
	\sum\limits_{j=1}^{i-1}{L \left(x_j - 4\tau\right)^2}  + g(x_i, x_{i+1})+ \sum\limits_{j={i+2}}^{d} {L x_j^2 - \left(i-1\right)\eta } \\ \triangleq f_{i,2}(x)  \text{ si } x\in D_{i,2}
	\end{array}\right.
	\end{equation}
	
	Donde nuevamente $\eta$ esta pendiente de definici\'on y $g: \R^2 \rightarrow \R$ tambi\'en la definiremos proximamente para que $f$ resulte $C^2$, $B-$acotada, $l-$Lipshitz, $\mu-$Lipshitz en el Hessiano con $B,l,\mu \in \text{poly}(d)$. 
	
	Para $i=d$ definamos:
	
	\begin{equation}
	f\vert_{D_d}(x) = \left\lbrace \begin{array}{l}
	\sum\limits_{j=1}^{d-1}{L \left(x_j - 4\tau\right)^2}  - \gamma x_d^2 - \left(d-1\right)\eta  \\ \triangleq f_{d,1}(x)  \text{ si } x\in D_{d,1} \\ \\ \\
	\sum\limits_{j=1}^{d-1}{L \left(x_j - 4\tau\right)^2}  + g_1(x_d)  - \left(d-1\right)\eta \\ \triangleq f_{d,2}(x)  \text{ si } x\in D_{d,2}
	\end{array}\right.
	\end{equation}
	
	Donde como antes, $g_1$ lo definiremos proximamente. Finalmente si $i = d+1$ entonces $6\tau \geq x_i \geq 2_tau$ para todo $1 \leq i \leq d$ y definimos:
	
	\begin{equation}
		f\vert_{D_{d+1}} = 	\sum\limits_{j=1}^{d}{L \left(x_j - 4\tau\right)^2}  - d\eta \triangleq f_{d+1,1}
	\end{equation}
	
	\begin{lemma}
		\label{lemma: GD toma tiempo exponencial}
		Sea $g(x_i,x_{i+1}) = g_1(x_i) + g_2(x_i)x_{i+1}^2$, existen $g_1,g_2$ polinomios y $\eta = - g_1(2\tau) + 4L\tau^2$ tal que para todo $1 \leq i \leq d$ si $x_i = \tau$ vale:
		
		\begin{equation*}
			\begin{aligned}
				f_{i,2}(x) &  = & f_{i,1}(x) \\
				\nabla f_{i,2}(x) &  = & \nabla f_{i,1}(x) \\
				\nabla ^2 f_{i,2}(x) &  = & \nabla ^2f_{i,1}(x)
			\end{aligned}
		\end{equation*}
		
		Y si $x_i = 2\tau$ entonces:
		
		\begin{equation*}
		\begin{aligned}
		f_{i,2}(x) &  = & f_{i+1,1}(x) \\
		\nabla f_{i,2}(x) &  = & \nabla f_{i+1,1}(x) \\
		\nabla ^2 f_{i,2}(x) &  = & \nabla ^2f_{i+1,1}(x)
		\end{aligned}
		\end{equation*}
		
		Es m\'as, si $x \in D_{i,2} \cap D_{i+1,1}$ entonces:
		
		\begin{equation*}
		\begin{aligned}
		-4L\tau & \leq & \dfrac{\partial g}{\partial x_i}(x_i, x_{i+1}) & \leq & -2 \gamma \tau \\
		-2\gamma x_{i+1} & \leq & \dfrac{\partial g}{\partial x_{i+1}}(x_i, x_{i+1}) & &
		\end{aligned}
		\end{equation*}
		
		Y finalmente si $x \in D_{i,2}$ entonces:
		
		\begin{equation*}
			-4L\tau \leq \dfrac{\partial g_1}{\partial x_i}(x_i)  \leq  -2 \gamma \tau 
		\end{equation*}
		
	\end{lemma}

	\begin{proof}
		Ver \ref{ch: apendice}
	\end{proof}

	\begin{remark}
		Del lema anterior podemos ver que $\deg(g_1), \deg(g) \leq 5$ por lo que estan acotados. Conclu\'imos que ambas son $B-$acotadas y $\mu-$Lipshitz con $B, \mu \in \text{poly}(L)$
	\end{remark}

	\begin{remark}
		Notemos que $\norm{g_1}, \norm{g} > \gamma \tau > 0$ por lo que ninguna de las dos aporta puntos cr\'iticos en $D_0$.
	\end{remark}

	\begin{remark}
		Notemos que $f$ queda $C^2$ y que sus $d+1$ puntos cr\'iticos son $z_i = \left(4\tau, \dots, \underbrace{4\tau}_{i}, 0, \dots, 0\right)$ donde todos son puntos silla estrictos menos $z_{d} = \left(4\tau, \dots, 4\tau\right)$ que es m\'inimo.
	\end{remark}
	
	\medskip
	
	\textbf{Paso 2- Cota superior a $T_k^{\tau}$}
	
	\medskip
	
	Supongamos ahora que $\tau > e$, $\alpha \leq \frac{1}{2L}$ y tomemos $w_0 \in [-1,1]^d \cap D_0$, veamos que para todo $T \leq \left(\frac{L+ \gamma}{\gamma}\right)^{d-1}$ vale que $x_{d}^T \leq 2\tau \not \in D_{d+1}$.
	
	Sea $T_0 = 0$ y definamos $T_k = \min\limits_{x^t_k \geq 2\tau}$ el tiempo de escape de $D_{k,2}$; notemos que como $x^0 \in D_{1,1}$ vale que $T_k \geq 0$ para todo $k$ y esta bien definido. Definamos adem\'as $T_k^{\tau}$ como la cantidad de iteraciones que $x^k$ esta en $D_{k,2}$ antes de escapar; como del lema $\dfrac{\partial g}{\partial x_k}\left(x_k, x_{k+1}\right) \leq -2\gamma \tau$ tenemos que $\abs{x^k - x^{k+1}} \geq 2 \alpha \gamma \tau $ por lo que:
	
	\begin{equation*}
		T_k^{\tau} \leq \frac{\tau}{2 \alpha \gamma \tau} = \frac{1}{2 \gamma \alpha} \qquad \forall k \in \sett{1, \dots, d+1}
	\end{equation*}
	
	\medskip
	
	\textbf{Paso 3 - Cota inferior para $T_1$}:
	
	\medskip
	
	Notemos que $T_1$ es el m\'inimo valor tal que $x^{T_1}_1 \geq 2\tau$ y entonces vale que $x_1^{T_1 - T_1^{\tau}} \geq \tau$, como del algoritmo \ref{algo: GD} sabemos que en $D_{1,2}$ vale la relaci\'on:
	
	\begin{equation*}
	x_1^{t} = \left(1 + 2 \alpha \gamma\right)^t x_1^0
	\end{equation*}
	
	Tenemos que:
	
	\begin{equation*}
	\begin{array}{crcl}
	& x_1^0 \left(1 + 2 \alpha \gamma \right)^{T_1 - T_1^{\tau}} & \geq & \tau \\
	\Longrightarrow & T_1 - T_1^{\tau} & \geq & \frac{1}{2 \alpha \gamma} \underbrace{\log\left(\frac{\tau}{x_1^0}\right)}_{\geq 1}  \geq T_1^{\tau}
	\end{array}
	\end{equation*}
	
	\medskip
	
	\textbf{Paso 4 - El algoritmo \ref{algo: GD} se queda confinado a $D_0$}:
	
	\medskip
	
	Si $x^t \in D_{k,1}$ luego las iteraciones del algortimo son:
	
	\begin{equation*}
		x_j^{t+1} = \left\lbrace \begin{array}{cr}
		\left(1 - \alpha L\right)x_j^t - 4\alpha L \tau \in [2\tau, 6\tau] & 1 \leq j \leq k-1 \\
		\left(1 + 2\alpha \gamma \right)x_j^t \tau \in [0, 2\tau] & j = k \\
		\left(1 - 2\alpha L\right)x_j^t \in [0, \tau] & j \geq k+1 
		\end{array}\right.
	\end{equation*}
	
	Mientras que si $x^t \in D_{k,2}$ entonces:
	
	\begin{equation*}
	x_j^{t+1} = \left\lbrace \begin{array}{cr}
	\left(1 - \alpha L\right)x_j^t - 4\alpha L \tau \in [2\tau, 6\tau] & 1 \leq j \leq k-1 \\
	x_j^t - \alpha \dfrac{\partial g}{\partial x_k} \left(x_k, x_{k+1}\right) \leq x_j^t  + 2\alpha \gamma \tau \in [0, 6\tau] & j = k \\
	\left(1 - 2\alpha L\right)x_j^t \in [0, \tau] & j \geq k+2
	\end{array}\right.
	\end{equation*}	
	
	Separemos el caso $j= k+1$, donde el lema \ref{lemma: GD toma tiempo exponencial} nos dice que:
	
	\begin{equation*}
		\dfrac{\partial f}{\partial x_{k+1}}\left(x\right) \geq -2 \gamma x_{k+1}
	\end{equation*}
	
	Luego para $t = T_k - T_k^{\tau} + 1, \dots, T_k$ vale:
	
	\begin{equation*}
		x_{k+1}^{t} \leq x_{k+1}^0 \left(1 - 2 \alpha L\right)^{T_k - T_k^{\tau}} \left(1 + 2 \alpha \gamma \right)^{t - \left(T_k - T_k^{\tau}\right)} \leq \tau
	\end{equation*}
	
	Y conclu\'imos que $x^t \in D_0$
	
	\medskip
	
	\textbf{Paso 5 - Relaci\'on entre $T_{k+1}$ y $T_k$}:
	
	\medskip
	
	Por un lado, por la definici\'on de $T_k$ y $T_k^{\tau}$:
	
	\begin{equation*}
		x_{k+1}^{T_k} \leq x_{k+1}^{0} \left(1 - 2 \alpha L\right)^{T_k - T_k^{\tau}}\left(1 + 2 \gamma \alpha\right)^{T_k^{\tau}}
	\end{equation*}
	
	Por el otro, usando el mismo arguemento que cuando acotamos por debajo a $T_1$:
	
	\begin{equation*}
	\begin{aligned}
		& x_{k+1}^{T_{k+1} - T_{k+1}^{\tau}} & \geq & \tau \\
		\Rightarrow & x_{k+1}^{T_k} \left(1 + 2 \alpha \gamma\right)^{T_{k+1} - T_{k+1}^{\tau} - T_k} & \geq & \tau \\
		\Rightarrow & x_{k+1}^{0} \left(1 - 2 \alpha L\right)^{T_k - T_k^{\tau}}\left(1 + 2 \gamma \alpha\right)^{T_k^{\tau}} \left(1 + 2 \alpha \gamma\right)^{T_{k+1} - T_{k+1}^{\tau} - T_k} & \geq & \tau \\
	\end{aligned}
	\end{equation*}
	
	Luego como $\alpha < \frac{1}{2L}$:
	
	\begin{equation*}
	2 \alpha \gamma \left(T_{k+1} - T_{k+1}^{\tau} - \left(T_k - T_k^{\tau}\right)\right) \geq \underbrace{\log \left(\frac{\tau}{x_{k+1}^0}\right)}_{\geq 1} + 2 \alpha L\left(T_k - T_k^{\tau}\right)
	\end{equation*}
	
	\begin{equation*}
		\Rightarrow \quad T_{k+1} - T_{k+1}^{\tau} \geq \dfrac{L + \gamma}{\gamma} \left(T_k - T_k^{\tau}\right)
	\end{equation*}
	
	Inductivamente:
	
	\begin{equation}
		T_d \geq T_d - T_d^{\tau} \geq \left(\dfrac{L+\gamma}{\gamma}\right)^{d-1} \left(T_1 - T_1^{\tau}\right) \geq \dfrac{1}{2 \alpha \gamma}  \left(\dfrac{L+\gamma}{\gamma}\right)^{d-1}  \geq \left(\dfrac{L+\gamma}{\gamma}\right)^{d-1} 
	\end{equation}
	
	\medskip
	
	\textbf{Paso 6 - Extender $D_0$ para que los puntos extremales sean interiores}
	
	\medskip
	
	Ya probamos que si $x^0 \in [-1,1]^d \cap D_0$ entonces el algoritmo \ref{algo: GD} necesita tiempo exponencial para converger al m\'inimo, ataquemos el caso $x^0 \in [-1,1]^d \cap D_0^c$
	
	Para $a=0, \dots, 2^d -1$ sea $a_2$ la representaci\'on binaria de $a$ y notemos $a_2(0)$ los \'indices donde $a_2$ tiene $0$ y an\'alogo con $a_2(1)$. Definamos:
	
	\begin{equation*}
	\begin{split}
		D_a = \bigcup\limits_{i=1}^{d} & \left\lbrace x \in \R^d \tq x_i \geq 0 \text{ si } i \in a_2(0), \ x_i \leq 0 \text{ sino}, \right. \\
			&  \left. 6\tau \geq \abs{x_1}, \dots, \abs{x_{i-1}} \geq 2\tau,\ \abs{x_i} \leq 2\tau,\ \abs{x_{i+1}}, \dots, \abs{x_d} \leq \tau \right\rbrace
	\end{split}
	\end{equation*}
	
	\begin{equation*}
		D = \bigcup\limits_{a = 0}^{2^d -1} {D_a}
	\end{equation*}
	
	Notemos que $D$ es cerrado y que $[-1,1]^d \subset D$. Ahora definamos la funci\'on $f$; sea $i = 0 , \dots, d$ y definamos los subdominios:
	
	\begin{equation*}
		\begin{array}{l}
		\widetilde{D}_{i,1} = \sett{x \in \R^d \tq 6\tau \geq \abs{x_1}, \dots, \abs{x_{i-1}} \geq 2\tau,\ \abs{x_i} \leq \tau,\ \abs{x_{i+1}}, \dots, \abs{x_d} \leq \tau} \\
		\widetilde{D}_{i,2} = \sett{x \in \R^d \tq 6\tau \geq \abs{x_1}, \dots, \abs{x_{i-1}} \geq 2\tau,\ \tau \leq \abs{x_i} \leq 2\tau,\ \abs{x_{i+1}}, \dots, \abs{x_d} \leq \tau} \\
		\widetilde{D}_{d+1} = \sett{x \in \R^d \tq 6\tau \geq \abs{x_1}, \dots, \abs{x_{d}} \geq 2\tau}
		\end{array}
	\end{equation*}
	
	Luego definimos:
	
	\begin{equation*}
	f(x) = \left\lbrace \begin{array}{cr}
	\Bigsum{j \leq i-1, j \in a_2(0)}{L \left(x_j - 4\tau\right)^2} + \Bigsum{j \leq i-1, j \in a_2(1)}{L \left(x_j + 4\tau\right)^2} & \\
	- \gamma x_i^2 + \Bigsum{j \geq i+1}{L x_j^2} - \left(i-1\right)\eta & \text{ si } x \in D_{i,1} ,\ i< d\\
	& \\
	\Bigsum{j \leq i-1, j \in a_2(0)}{L \left(x_j - 4\tau\right)^2} + \Bigsum{j \leq i-1, j \in a_2(1)}{L \left(x_j + 4\tau\right)^2} & \\
	+ G(x_i, x_{i+1}|) + \Bigsum{j \geq i+2}{L x_j^2} - \left(i-1\right)\eta & \text{ si } x \in D_{i,2} ,\ i< d \\
	& \\
	\Bigsum{j \leq d-1, j \in a_2(0)}{L \left(x_j - 4\tau\right)^2} + \Bigsum{j \leq d-1, j \in a_2(1)}{L \left(x_j + 4\tau\right)^2} & \\
	- \gamma x_d^2 - \left(d-1\right)\eta & \text{ si } x \in D_{d,1} \\
	& \\
	\Bigsum{j \leq d-1, j \in a_2(0)}{L \left(x_j - 4\tau\right)^2} + \Bigsum{j \leq d-1, j \in a_2(1)}{L \left(x_j + 4\tau\right)^2} & \\
	+ G_1(x_d) - \left(d-1\right)\eta & \text{ si } x \in D_{d,2} \\
	& \\
	\Bigsum{j \leq d, j \in a_2(0)}{L \left(x_j - 4\tau\right)^2} + \Bigsum{j \leq d, j \in a_2(1)}{L \left(x_j + 4\tau\right)^2} & \\
	- d\eta & \text{ si } x \in D_{d+1} \\
	\end{array} \right.
	\end{equation*}
	
	Donde:
	
	\begin{equation*}
		\begin{array}{l}
		G(x_i, x_{i+1}) = \left\lbrace \begin{array}{cc}
			g(x_i, x_{i+1}) & \text{ si } i \in a_2(0) \\
			g(-x_i, x_{i+1}) & \text{ si } i \in a_2(1) \\			
		\end{array}\right. \\
		\\
		G_1(x_i) = \left\lbrace \begin{array}{cc}
			g_1(x_i) & \text{ si } i \in a_2(0) \\
			g_1(-x_i) & \text{ si } i \in a_2(1) \\			
		\end{array}\right.
		\end{array}
	\end{equation*}
	
	Notemos que por simetr\'ia de la definici\'on, si espejamos la demostraci\'on del punto anterior es claro que si $\tau \geq e$ y $x^0 \in [-1,1]^d$ entonces el algoritmo \ref{algo: GD} con $\alpha < \frac{1}{2L}$ cumple $x_d^T \leq 2\tau$ para todo $T \leq \left(\frac{L + \gamma}{\gamma}\right)$ y por lo tanto necesita $e^{\Omega(d)}$ operaciones para llegar al \'unico m\'inimo $\left(4\tau, \dots, 4\tau\right)$.
	
	\medskip
	
	\textbf{Paso 7 - Extender de $D$ a $\R^d$}
	
	\medskip
	
	Por \ref{theorem: Extension de Whitney mejorado} sabemos que existe $F : \R^d \rightarrow \R$ que extiende a $f$ y que $\norm{F}_{\infty}, \norm{F}_{C^m} \leq \mathcal{O}\left(\text{poly}(d)\right)$; y aunque $F$ puede admitir nuevos puntos cr\'iticos, del paso 4 y 6 sabemos que si $x^0 \in [-1,1]^d$ entonces $\sett{x^k} \subset D$. \qed
	
\end{proof}