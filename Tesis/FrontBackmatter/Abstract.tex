%*******************************************************
% Abstract
%*******************************************************
%\renewcommand{\abstractname}{Abstract}
\pdfbookmark[1]{Abstract}{Abstract}
\begingroup
\let\clearpage\relax
\let\cleardoublepage\relax
\let\cleardoublepage\relax

\chapter*{Resumen}

En esta tesis intentamos responder a la simple pregunta: \textbf{Bajo que condiciones convergen los m\'etodos de primer orden usuales?} Para ello utilizamos herramientas de sistem\'as din\'amicos y procesos estoc\'asticos para analizar la convergencia tanto en algoritmos determin\'isticos como estoc\'asticos. 


Notamos que aunque los algoritmos determin\'isticos gozan de una velocidad excepcional en el caso convexo, esto no se generaliza al caso no convexo donde la convergencia puede ser hasta de orden exponencial; mientras que en el caso estoc\'astico la misma naturaleza de este garantiza una complejidad uniforme en ambos casos, aun en el rango del \textit{big data}. Sumado a esto dimos motivos tanto te\'oricos como pr\'acticos para la preferencia de los algoritmos estoc\'asticos referido a la velocidad de convergencia a entornos de la soluci\'on en el caso general.


Queda como posible l\'inea futura analizar que tan restrictivas son estas condiciones asi como profundiz\'ar el estudio de la complejidad y convergencia a m\'inimos en los algoritmos mixtos que surgen de intentar juntar caracter\'isticas de los algoritmos m\'as usuales.

\endgroup

\vfill
