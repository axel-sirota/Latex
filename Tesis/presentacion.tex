\documentclass[spanish]{beamer}
\usepackage{lmodern}
\usepackage[T1]{fontenc}
\usepackage[utf8]{inputenc}
\usepackage{babel}
\usepackage[ruled,vlined,linesnumbered,resetcount]{algorithm2e}
\usepackage{amsfonts}
\usepackage{xr}
\usepackage{amsmath,amsfonts, amssymb, mathrsfs }
\usepackage{tikz-cd}
\usepackage{syntonly}
\usepackage{mathrsfs}
\usepackage{lmodern}
\usepackage{amsmath,amsbsy,amscd,amssymb,graphicx, epsfig,color}
\usetheme{Boadilla}
\usecolortheme{beaver}
\setbeamercolor{block body alerted}{bg=normal text.bg!90!black}
\setbeamercolor{block body}{bg=normal text.bg!90!black}
\setbeamercolor{block body example}{bg=normal text.bg!90!black}
\setbeamercolor{block title alerted}{use={normal text,alerted text},fg=alerted text.fg!75!normal text.fg,bg=normal text.bg!75!black}
\setbeamercolor{block title}{bg=shadecolor, fg=black}
\setbeamercolor{block title example}{use={normal text,example text},fg=example text.fg!75!normal text.fg,bg=normal text.bg!75!black}
%\setbeamercolor{item projected}{fg=black, bg=darkred}
\definecolor{shadecolor}{rgb}{0.8,0.8,0.8}
\definecolor{mapblue}{HTML}{3333B3}

\uselanguage{Spanish}
\languagepath{Spanish}

\def \ee{\varepsilon}
\def \a{\alpha}
\def \b{\beta}

\newtheorem{proposition}[theorem]{Proposici\'on}
\newtheorem{claim}[theorem]{Afirmaci\'on}
\newtheorem{hyp}[theorem]{Hip\'otesis}

\def \K{\hbox{Ker}}

\newcommand{\B}{\mathcal{B}}
\newcommand{\Cont}{\mathcal{C}}
\newcommand{\F}{\mathcal{F}}
\newcommand{\G}{\mathcal{G}}
\newcommand{\inte}{\mathrm{int}}
\newcommand{\A}{\mathcal{A}}
\newcommand{\C}{\mathbb{C}}
\newcommand{\Q}{\mathbb{Q}}
\newcommand{\Z}{\mathbb{Z}}
\newcommand{\inc}{\hookrightarrow}
\renewcommand{\P}{\mathcal{P}}
\newcommand{\R}{{\mathbb{R}}}
\newcommand{\N}{{\mathbb{N}}}
\newcommand\tq{~:~}
\newcommand{\dual}[1]{\left(#1\right)^{\ast}}
\newcommand{\ortogonal}[1]{\left(#1\right)^{\perp}}
\newcommand{\ddual}[1]{\left(#1^{\ast}\right)^{\ast}}
\newcommand{\parenthesis}[1]{\left(#1\right)}
\newcommand{\x}[3]{#1_#2^#3}
\newcommand{\xx}[4]{#1_#3#2_#4}
\newcommand\dd{\,\mathrm{d}}
\newcommand{\norm}[1]{\left\lVert#1\right\rVert}
\newcommand{\abs}[1]{\left\lvert#1\right\rvert}
\newcommand{\ip}[1]{\left\langle#1\right\rangle}
\renewcommand\tt{\mathbf{t}}
\newcommand\nn{\mathbf{n}}
\newcommand\bb{\mathbf{b}}                      % binormal
\newcommand\kk{\kappa}
\newcommand{\sett}[1]{\left\lbrace#1\right\rbrace}
\newcommand{\interior}[1]{\accentset{\smash{\raisebox{-0.12ex}{$\scriptstyle\circ$}}}{#1}\rule{0pt}{2.3ex}}
\fboxrule0.0001pt \fboxsep0pt
\newcommand{\Bigcup}[2]{\bigcup\limits_{#1}{#2}}
\newcommand{\Bigcap}[2]{\bigcap\limits_{#1}{#2}}
\newcommand{\Bigprod}[2]{\prod\limits_{#1}{#2}}
\newcommand{\Bigcoprod}[2]{\coprod\limits_{#1}{#2}}
\newcommand{\Bigsum}[2]{\sum\limits_{#1}{#2}}
\newcommand{\BigsumA}[3]{ \sideset{}{^#2}\sum\limits_{#1}{#3}}
\newcommand{\Biglim}[2]{\lim\limits_{#1}{#2}}
\newcommand{\quotient}[2]{{\raisebox{.2em}{$#1$}\left/\raisebox{-.2em}{$#2$}\right.}}
\newcommand{\expectation}[1]{\mathbb{E} \left[#1\right]}
\newcommand{\conditionalExpectation}[2]{\mathbb{E} \left[#1 \vert #2\right]}
\newcommand{\expectationsub}[2]{\mathbb{E}_{#1} \left[#2\right]}
\newcommand{\variancesub}[2]{\mathbb{V}_{#1} \left[#2\right]}
\newcommand{\expectationchik}[1]{\expectationsub{\upxi_{k}}{#1}}
\newcommand{\expectationfilt}[1]{\mathbb{E} \left[{#1} \vert \mathcal{P}_{k}\right]}
\newcommand{\variancechik}[1]{\variancesub{\upxi_{k}}{#1}}
\newcommand{\mani}{\upchi}
\DeclareMathOperator{\rank}{ran}
\DeclareMathOperator{\graf}{Gr}
\DeclareMathOperator{\ball}{ball}

\def \le{\leqslant}	
\def \ge{\geqslant}
\def\noi{\noindent}
\def\sm{\smallskip}
\def\ms{\medskip}
\def\bs{\bigskip}
\def \be{\begin{enumerate}}
	\def \en{\end{enumerate}}
\def\deck{{\rm Deck}}
\def\Tau{{\rm T}}
\newcommand{\myTitle}{M\'etodos de primer orden\xspace}
\newcommand{\mySubtitle}{An\'alisis de convergencia\xspace}
\newcommand{\myDegree}{Tesis de Licenciatura\xspace}
\newcommand{\myName}{Axel Sirota\xspace}
\newcommand{\myDirector}{Director de Tesis: Dr. Pablo Amster\xspace}
\newcommand{\myFaculty}{Facultad de Ciencias Exactas y Naturales\xspace}
\newcommand{\myDepartment}{Departamento de Matem\'atica\xspace}
\newcommand{\myUni}{Universidad de Buenos Aires\xspace}
\newcommand{\myLocation}{Buenos Aires\xspace}
\newcommand{\myTime}{Septiembre 2018\xspace}
\newcommand{\myVersion}{version 1.0}



\title[\myTitle: \mySubtitle]{\bf {\myTitle}}
\author[\myName]{\myName}
\institute[\myFaculty]{}
\date[\myTime]{\myDepartment}

\AtBeginSection[] {
\begin{frame}{Panorama}
\tableofcontents[currentsection]
\end{frame}
}

\begin{document}

\begin{frame}
\titlepage
\end{frame}

\section{Introduction}
\begin{frame}
 \frametitle{Motivaci\'on}

\end{frame} 




\begin{frame} 
\frametitle{The whole truth about Nicholson}
\begin{enumerate}

    \item $p\le d \Longrightarrow$  no positive equilibrium points. 
    Furthermore,  $0$ is a global attractor of the solutions with $\varphi>0$. 
    \pause 
    \item $p>d \Longrightarrow$ unique equilibrium point, which is locally asymptotically stable: 
    
      \begin{itemize}
          \item  for all $\tau$ when $p<de^2$,
          \item  for $\tau<\tau^*(p)$ when $p\ge de^2$.
      \end{itemize}   Moreover if  $\varphi>0$  then $$\liminf_{t\to+\infty}x(t)\ge \min \{\ln \left(\frac pd\right), 
    e^{-\tau d}\}.$$
    \pause 
    \item $p, d \in C_T \Longrightarrow$   positive  $T$-periodic solutions if $p(t)>d(t)$ for all $t$ and no $T$-periodic solutions if $p(t)\le d(t)$ 
    for all $t$ (furthermore, $0$ is a global attractor). 
\end{enumerate}

\end{frame}

\begin{frame}
\frametitle{Generalisation: Nicholson system}

\begin{equation}
    \label{nich-sys}
    \left\{\begin{array}{c}
        x_1'(t)=-d_1 x_1(t) + b_1x_2(t)+ p_1 x_1(t-\tau)e^{-x_1(t-\tau)}    \\
               x_2'(t)=-d_2 x_2(t) + b_2x_1(t)+ p_2 x_2(t-\tau)e^{-x_2(t-\tau)}  
    \end{array}
    \right.
    \end{equation}

\begin{enumerate}
    \item $p_i+b_i\le d_i\Longrightarrow 0$ is a global attractor of positive solutions. 
        \item Uniform persistence if $p_i+b_i>d_i$.
        
        \item $T$-periodic solutions $b_i(t)<d_i(t) <p_i(t)+d_i(t)$ for all $t$. 
        
    
\end{enumerate}
\end{frame}




\begin{frame}

\frametitle{Persistence definitions}

Let $X\ne \emptyset$ and 
$\rho: X \to \R^+$. A semiflow $\Phi : J \times X \to  X$ is called 



\begin{itemize}
    \item weakly $\rho$-persistent, if
$$\limsup_{t\to+\infty}  
\rho(\Phi(t, x)) > 0\qquad \forall\, x \in  X, \rho(x) > 0.$$
    
    \item 
     strongly $\rho$-persistent, if
$$\liminf_{t\to+\infty}  
\rho(\Phi(t, x)) > 0\qquad \forall\, x \in  X, \rho(x) > 0.$$
\item 
uniformly weakly (strongly) $\rho$-persistent, if
there exists some  $\varepsilon > 0$ such that
$$\limsup_{t\to+\infty}  
\rho(\Phi(t, x)) > \varepsilon \qquad \forall\, x \in  X, \rho(x) > 0\qquad \hbox{(resp. liminf)}.$$
 

\medskip
In our case 
$\rho(x) = |x|$ (persistence of some  species).

\end{itemize}

\end{frame}

\begin{frame}{}



Clearly, 

\begin{center}
    (USP)$\Longrightarrow$ (UWP) $\Longrightarrow$ (WP) and
(SP) $ \Longrightarrow$ (WP).
\end{center}

\medskip {\bf Results for $X$ locally compact, $X\backslash\{0\}$ positively invariant:}   


\begin{center}
    
(UWP) $\Longrightarrow$ (USP), but 
(WP) $\not\Longrightarrow$ (SP)
and 
(SP) $\not\Longrightarrow$ (USP). 
\end{center}

Extra conditions $\Longrightarrow$  all definitions are equivalent (see \cite{FM}).

   \smallskip 
\pause  
Fonda \cite{F}:
   \smallskip 
    
    (USP) $\Longleftrightarrow$
        $\exists \, U\ni 0$ open and $V:U\to [0,+\infty)$ continuous such that: 
        \begin{enumerate}
            \item $V(x)=0 \iff x=0$.
            \item $\forall\, x\ne 0\, \exists t_x>0$ such that $V(\Phi(t_x,x)) > V(x)$. 
        \end{enumerate}
\end{frame}

\section{Delayed systems}

\begin{frame}
    
  

Our system reads
\begin{equation}
\label{eq}
x'(t)=f(t,x(t),x(t-\tau))
\end{equation}
with
$f:[0,+\infty)\times [0,+\infty)^{2N}\to \R^N$  continuous.

Initial condition:
\begin{equation}
\label{ic}
x(t) =\varphi(t)
\qquad {-\tau\le t\le 0}
\end{equation}
\begin{center}
    $\varphi\in X:=$ positive cone of $C([-\tau,0],\R^N)$,
\end{center}
which is \textbf{not} locally compact. 

\medskip 

{\bf Basic assumption}: 

\smallskip

{\bf (H1)} \label{posit}
If $x_j=0$ for some $j$ and $y\ne 0$ then $f_j(t,x,y)>0$ for all $t>0$.

\pause 
\bigskip
{\bf (H1)} $\Longrightarrow X^\circ$ pos.  invariant, i.e.: $\varphi>0\Longrightarrow x>0$, but {\bf (H1)} $\not\Longrightarrow$ (WP). 

\end{frame}

\begin{frame}{}

 
Consider 
$V:(0,+\infty)^N\to (0,+\infty)$ smooth such that 
$V(x)\to 0$ as $x\to 0$ in $(0,+\infty)^N$. 
\medskip 

Obvious choice: $V(x)= |x|^2$. Nicholson system $V=\min x_i$ (nonsmooth). 

\pause 
\smallskip{}

{\bf Idea}: find conditions that guarantee
 $\dot V>0$ for $x(t)$ close to $0$, where $\dot V(t)=v'(t):=(V\circ x)'(t)$. 


\pause 
\medskip 
{\bf Easy case}: 
\smallskip

{\bf (H2)} There exist $t_0,r>0$ such that
$$ \langle \nabla V(x),f(t,x,y)\rangle >0 \qquad \hbox{for } t>t_0,  V(x),V(y)<r.$$
  
\begin{center}
    {\bf (H1)} + {\bf (H2)} $\Longrightarrow$ (WP)
\end{center}

\smallskip{
}

{\bf (H2)} $\Longrightarrow$ (Fonda) when $\tau=0$. 
However,  {\bf (H2)} is not satisfied e.g. in (\ref{nich}).    
    
\end{frame}


\begin{frame}{}

More realistic: 
\smallskip 

{\bf (H2')} 
There exist $t_0,r>0$ such that
$$ \langle \nabla V(x),f(t,x,x)\rangle >0 \qquad \hbox{for } t>t_0\hbox{ and } V(x)<r.$$

\medskip 

{\bf (H1)}  + {\bf (H2')} do not suffice because 
$x(t)-x(t-\tau)$ may be large. 

\medskip 

{\bf Proposition}:
(SP) holds if we also assume monotonicity:

\medskip


{\bf (H3)} 
$$\langle \nabla V(x),f(t,x,y)\rangle \ge \langle \nabla V(x),f(t,x,x)\rangle$$
whenever $V(x)\le V(y)$. 



\end{frame}

\begin{frame}{}
\emph{Proof}: Suppose  $s_n\to +\infty$, $x(s_n)\to 0 \Longrightarrow v(s_n)\to 0$. 
\smallskip 

Set $t_n$ such that    $v(t_n) =\min_{t\le s_n} v(t)$. 
For $n\gg 0$, 
\begin{center}
    $v'(t_n)\le 0$ and  $v(t_n-\tau)\ge v(t_n)$.
\end{center}
Thus,
$$0\ge v'(t_n)=\langle \nabla V(x(t_n)),f(t_n,x(t_n),x(t_n-\tau)\rangle
$$
$$\ge \langle \nabla V(x(t_n)),f(t_n,x(t_n),x(t_n)\rangle >0,$$  a contradiction. 
\medskip 
\pause 


Nicholson does not satisfy {\bf (H3)}, then: why is it (USP)?

\end{frame}    
\begin{frame}{Local monotonicity}

{\bf (H3')} There exists $\eta >0$ such that
$$\langle \nabla V(x),f(t,x,y)\rangle \ge \langle \nabla V(x),f(t,x,x)\rangle$$
whenever $V(x)\le V(y)\le \eta$. 

    \bigskip 
{\bf    Nicholson}: $f(x,y)= -dx + pye^{-y} \Longrightarrow$  {\bf (H3')} with $\eta=1$ and $V(x)=x$. 

\bigskip 
\pause 

{\bf (H3')} is not enough! Reason: $x(t)-x(t-\tau)$ may be large.  

\end{frame}

\begin{frame}{A standard assumption in population models}

    {\bf (H4)} 
$\langle \nabla V(x),f(t,x,y)\rangle \ge -k V(x) $ for some constant $k$. 
\medskip{}

Thus, 
$$v'(t)= 
\langle \nabla V(x(t)), f(t,x(t),x(t-\tau)\rangle \ge -kv(t),
$$
whence
\begin{center}
    $v(t-\tau) \le e^{k\tau}v(t)$ for all $t\ge \tau$.
\end{center}
\medskip 

Consequence:  

\begin{center}
        {\bf (H1)} +     {\bf (H2')} +     {\bf (H3')} +     {\bf (H4)} $\Longrightarrow$ (SP). 
\end{center}    
 
 
 

\end{frame}

\begin{frame}{But... why is Nicholson (USP)?}
    
    Assume {\bf (H1)} +     {\bf (H2')} +     {\bf (H3')} +     {\bf (H4)} and $i=\liminf_{t\to+\infty} v(t) \ll 1$. Then $i>0$ and there are 3 cases: 
    \medskip 
    
\begin{enumerate}
    \item $v(t)\ge i$ for all $t\gg 0$. Then choose as before
        $t_n\to +\infty$ such that $v(t_n) \to i$, $v(t_n-\tau) \ge v(t_n)$ and $v'(t_n)\le 0$ and a contradiction yields.
        \pause 
       \item 
 $v(t)$ oscillates around $i$. Then we may choose 
    a sequence $t_n\to +\infty$ such that $v(t_n)\to i^-$  and $v'(t_n)\le 0$. However, it might happen that $v(t_n-\tau)< v(t_n)$ for $n$ large, then     {\bf (H3')} + {\bf (H4)} are not of any help.
 \pause  
   \item  
  $v(t)\to i^-$ as $t\to+\infty$: same situation.
    
\end{enumerate}       
    

    
     
    
\end{frame}

\begin{frame}{
}
 Wouldn't it be great if 
 
 $$\langle \nabla V(x),f(t,x,y)\rangle \ge c(i)>0
 $$
 whenever $V(x), V(y)$ are close to $i$? 
 \medskip 
 
 
 For example, if 
 $v(t)\to i^-$, then 
 $$v'(t)=\langle \nabla V(x(t)),f(t,x(t),x(t-\tau))\rangle \ge c
 $$
 for $t$ large, a contradiction. 
\medskip 

\pause 
\textbf{OK} for (\ref{nich}) but too ambitious when $N>1$:
the condition fails for example if $V(x)=|x|^2$ and $f(x,y)=y$.

\end{frame}

\begin{frame}{And yet it works
}
Let 
  $$\theta_i:= \limsup_{t\to+\infty, V(x), V(y)\to i^-} |f(t,x,y)|
$$
and observe that if $v(t)\to i^-$ then 
$$|x(t)-x(t-\tau)|\le \tau |x'(\xi)|\qquad \xi \in [t-\tau,t]$$
and hence: 
$$
\limsup_{t\to+\infty} |x(t)-x(t-\tau)| \le \tau \theta_i.
$$


\end{frame}

\begin{frame}{ At last we get (USP)}

{\bf (H2''')} $\exists \, r>0$ such that, $\forall \, i\in (0,r)$ and some $C_i>\tau \theta_i$ 
$$\liminf_{t\to+\infty, V(x),V(y)\to i^-, |y-x|\le C_i }\langle x,f(t,x,y)\rangle  > 0.
$$

\begin{theorem}

$$\hbox{ {\bf (H1)} $+$ 
{\bf (H2'')}
$+$ 
{\bf (H3')}
$+$ 
{\bf (H4)}
$\Longrightarrow$ {\bf (USP)}.}$$ 
More precisely, all solutions of (\ref{eq})-(\ref{ic}) with $\varphi_j(t)>0$ for all $j$ and $t\in [-\tau,0]$ satisfy 
$$\liminf_{t\to+\infty} V(x(t))\ge \min\{ r, e^{-k\tau}\eta \}.$$
\end{theorem}

\end{frame}

\section{Periodic solutions, existence and...}


\begin{frame}{Periodic solutions}
    
Consider a continuous function $a:[0,+\infty)\to (0,+\infty)$ and define
$$K(t,x,y):= 
\langle \nabla V(x),f(t,x,y)\rangle + a(t)V(x),
$$
$$\phi(t,r):= 
 \sup_{V(x),V(y)\le r} {\frac {K(t,x,y)}{a(t)}}.
$$

\medskip 

Autonomous system:  
{\bf (H1)} + 
{\bf (H2')}  and  
 $\phi(R)<R$ for some $R$, then there exists a positive equilibrium point in the region  
$$
V_\varepsilon^R:=\{ \varepsilon < V(x) < R\}.
$$
Indeed, the field $f(x,x)$ points outwards over $\partial V_\varepsilon^R$.


\end{frame}

\begin{frame}{
}

Faces $\overline{V_\varepsilon^R}\cap \{ 
x_i = 0\}:$ use {\bf (H1)}. 
\bigskip

Bottom cap $\{V=\varepsilon\}$: the outer normal is $-\nabla V$ and
{\bf (H2')} guarantees $\langle \nabla V(x),f(x,x)\rangle>0$.

\bigskip

Top cap $\{V=R\}$: the outer normal is $\nabla V$ and
the inequality  $\phi(R)<R$ implies 
$\langle \nabla V(x),f(x,x)\rangle <0$. 
 
\bigskip
\pause 

{\bf Important condition}: $\chi(V_\varepsilon^R)\ne 0$ (e.g. $\overline{V_\varepsilon^R}$
is a retract of $C\simeq \overline B$). 
    
\end{frame}

\begin{frame}{Closed orbits for the non-autonomous system}
    
Let $T\ge \tau$ and consider the assumption: 

\medskip

{\bf (H5)} 
There exists $R>0$ such that $\phi(t,R)<R$ for $0\le t\le T$. 

\bigskip 
\begin{theorem}
    {\bf (H1)} $+$ 
{\bf (H2')}
$+$ 
{\bf (H3')}
$+$ 
{\bf (H4)} $+$  {\bf (H5)}
$\Longrightarrow$ {Positive $T$-periodic solutions}.
\end{theorem}

\pause 
\textit{Idea of the proof}: $deg(I-K)= \chi(V_\varepsilon^R)$.
\end{frame}


\section{...nonexistence: 0 is a global attractor}

\begin{frame}{
Conversely...}
    
    Assume that 
    $$
\phi^*(r):= \sup_{t\ge 0} \phi(t,r).
$$
is continuous and

\medskip 

{\bf (H6)} 
For every $\varepsilon>0$
there exists $\mu>0$ such that $V^{-1}(0, \mu) \subset B_\varepsilon(0)$. 

\begin{theorem}

Assume  {\bf (H6)} and suppose  there exists $R_0$ such that  $\phi^*(r)<r$ for $0<r<R_0$. 
Then every solution with initial data $\varphi(t)\in (0,R_0)$ for all $t\in [-\tau,0]$ tends to $0$ as $t\to +\infty$.  

\end{theorem}

\end{frame}

\begin{frame}{Sketch of the proof}
    
If $v\le r$ on $[t-\tau, t]$ and
$v'(t)\ge 0$, then 
$$
v(t) \le \phi(r). 
$$
Let $R_{j+1}:= {\phi^*(R_j)}<R_j$, then two different situations may occur:

\begin{enumerate}
    \item There exists $t_j\to +\infty$ such that $v(t)\in (0,R_j]$ for $t\ge t_j$. Then $v(t)\to 0$ because
$$
\phi^*(\lim_{j\to \infty} R_j) = \lim_{j\to \infty} \phi^*(R_j)= \lim_{j\to \infty} R_j, 
$$.     
\item There exist $j$ and $t_j$ such that $v(t) \in (R_{j+1},R_j]$ and decreases strictly for $t\ge t_j$. Let 
 $r:=\lim_{t\to+\infty} v(t)$, fix $\tilde r>r$ such that $\phi^*(\tilde r)<r$ and $\tilde t$ such that $v(t)\le \tilde r$ for $t\ge \tilde t$. Thus 
$v(t)\le \phi^*(\tilde r)<r$ for $t\ge t^*+\tau$, a contradiction. 

\end{enumerate}

    
\end{frame}
\section{References}
\begin{frame}[allowframebreaks]{References}
\begin{thebibliography}{99}
{\fontsize{10}{10} \selectfont


\beamertemplatearticlebibitems

\bibitem{BBI} 
 L. Berezansky, E. Braverman, L.  Idels,  
 {\em Nicholson's blowflies differential equation revisited: main results and open problems}. Appl. Math. Model,  {\bf 34}, (2010) 1405--1417.
 
 \beamertemplatearticlebibitems

 \bibitem{FM}
H. Freedman, P. Moson, {\em Persistence definitions and their connections}, Proc. Am. Math. Soc. 109, 4 (1990), 1025--1033. 
 
\beamertemplatearticlebibitems

\bibitem{F}
A. Fonda, 
{\em Uniformly persistent semidynamical systems}
Proc. Am. Math. Soc.
104, 1 (1988)

\beamertemplatearticlebibitems

\bibitem{ST}
H. Smith, H. Thieme, {\em Dynamical Systems
and Population
Persistence}. 
American Mathematical Society, 2011. 

\beamertemplatearticlebibitems

 \bibitem{SY} 
  J. So, J. S. Yu, {\em Global attractivity and uniform persistence in Nicholson's blowflies}, Diff. Eqns. Dynam. Syst. {\bf 2} (1) (1994) 11--18
  
}

\end{thebibliography}
\end{frame}

\begin{frame}
 \begin{center}
  \LARGE{{\bf Thanks for your attention!}}
 \end{center}
\end{frame}

\end{document}
