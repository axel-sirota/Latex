\documentclass[11pt]{article}

\usepackage{amsfonts}
\usepackage{amsmath,accents,amsfonts, amssymb, mathrsfs }
\usepackage{tikz-cd}
\usepackage{graphicx}
\usepackage{syntonly}
\usepackage{color}
\usepackage{mathrsfs}
\usepackage[spanish]{babel}
\usepackage[latin1]{inputenc}
\usepackage{fancyhdr}
\usepackage[all]{xy}
\usepackage[at]{easylist}


\topmargin-2cm \oddsidemargin-1cm \evensidemargin-1cm \textwidth18cm
\textheight25cm


\newcommand{\B}{\mathcal{B}}
\newcommand{\Cont}{\mathcal{C}}
\newcommand{\F}{\mathcal{F}}
\newcommand{\inte}{\mathrm{int}}
\newcommand{\A}{\mathcal{A}}
\newcommand{\C}{\mathbb{C}}
\newcommand{\Q}{\mathbb{Q}}
\newcommand{\Z}{\mathbb{Z}}
\newcommand{\inc}{\hookrightarrow}
\renewcommand{\P}{\mathcal{P}}
\newcommand{\R}{{\mathbb{R}}}
\newcommand{\N}{{\mathbb{N}}}
\newcommand\tq{~:~}
\newcommand{\x}[3]{#1_#2^#3}
\newcommand{\xx}[4]{#1_#3#2_#4}
\newcommand\dd{\,\mathrm{d}}
\newcommand\norm[1]{\left\lVert#1\right\rVert}
\newcommand\abs[1]{\left\lvert#1\right\rvert}
\newcommand\ip[1]{\left\langle#1\right\rangle}
\renewcommand\tt{\mathbf{t}}
\newcommand\nn{\mathbf{n}}
\newcommand\bb{\mathbf{b}}                      % binormal
\newcommand\kk{\kappa}
\newcommand{\sett}[1]{\left\lbrace#1\right\rbrace}
\newcommand{\interior}[1]{\accentset{\smash{\raisebox{-0.12ex}{$\scriptstyle\circ$}}}{#1}\rule{0pt}{2.3ex}}
\fboxrule0.0001pt \fboxsep0pt
\newcommand{\Bigcup}[2]{\bigcup\limits_{#1}{#2}}
\newcommand{\Bigcap}[2]{\bigcap\limits_{#1}{#2}}
\newcommand{\Bigprod}[2]{\prod\limits_{#1}{#2}}
\newcommand{\Bigcoprod}[2]{\coprod\limits_{#1}{#2}}
\newcommand{\Bigsum}[2]{\sum\limits_{#1}{#2}}
\newcommand{\BigsumA}[3]{ \sideset{}{^#2}\sum\limits_{#1}{#3}}
\newcommand{\Biglim}[2]{\lim\limits_{#1}{#2}}
\newcommand{\quotient}[2]{{\raisebox{.2em}{$#1$}\left/\raisebox{-.2em}{$#2$}\right.}}



\def \le{\leqslant}	
\def \ge{\geqslant}
\def\noi{\noindent}
\def\sm{\smallskip}
\def\ms{\medskip}
\def\bs{\bigskip}
\def \be{\begin{enumerate}}
	\def \en{\end{enumerate}}
\def\deck{{\rm Deck}}
\def\Tau{{\rm T}}

\newtheorem{theorem}{Teorema}[section]
\newtheorem{lemma}[theorem]{Lema}
\newtheorem{proposition}[theorem]{Proposici\'on}
\newtheorem{corollary}[theorem]{Corolario}

\newenvironment{proof}[1][Demostraci\'on]{\begin{trivlist}
		\item[\hskip \labelsep {\bfseries #1}]}{\end{trivlist}}
\newenvironment{definition}[1][Definici\'on]{\begin{trivlist}
		\item[\hskip \labelsep {\bfseries #1}]}{\end{trivlist}}
\newenvironment{example}[1][Ejemplo]{\begin{trivlist}
		\item[\hskip \labelsep {\bfseries #1 }]}{\end{trivlist}}
\newenvironment{remark}[1][Observaci\'on]{\begin{trivlist}
		\item[\hskip \labelsep {\bfseries #1}]}{\end{trivlist}}
\newenvironment{declaration}[1][Afirmaci\'on]{\begin{trivlist}
		\item[\hskip \labelsep {\bfseries #1}]}{\end{trivlist}}


\newcommand{\qed}{\nobreak \ifvmode \relax \else
	\ifdim\lastskip<1.5em \hskip-\lastskip
	\hskip1.5em plus0em minus0.5em \fi \nobreak
	\vrule height0.75em width0.5em depth0.25em\fi}

\newcommand{\twopartdef}[4]
{
	\left\{
	\begin{array}{ll}
		#1 & \mbox{ } #2 \\
		#3 & \mbox{ } #4
	\end{array}
	\right.
}

\newcommand{\threepartdef}[6]
{
	\left\{
	\begin{array}{lll}
		#1 & \mbox{ } #2 \\
		#3 & \mbox{ } #4 \\
		#5 & \mbox{ } #6
	\end{array}
	\right.
}

\tikzset{commutative diagrams/.cd,
	mysymbol/.style={start anchor=center,end anchor=center,draw=none}
}
\newcommand\Center[2]{%
	\arrow[mysymbol]{#2}[description]{#1}}

\newcommand*\circled[1]{\tikz[baseline=(char.base)]{
		\node[shape=circle,draw,inner sep=2pt] (char) {#1};}}


\begin{document}
	
	\pagestyle{empty}
	\pagestyle{fancy}
	\fancyfoot[CO]{\slshape \thepage}
	\renewcommand{\headrulewidth}{0pt}
	
	
	
	\centerline{\bf An\'alisis Funcional - $1^{\circ}$ cuatrimestre $2017$}
	\centerline{\sc Final}
	
	\bigskip

\section{Espacios Vectoriales}

\subsection{Propiedades Elementales}

\begin{definition}
Si $\mathcal{X}$ es un espacio vectorial sobre un cuerpo $\mathbb{F}$, un conjunto $\mathcal{B} = \sett{v_i}_{i \in I}$ se dice:

\begin{enumerate}
\item \textit{Linealmente independiente} si dados $v_{i_1}, \dots, v_{i_k} \in \B$ y $\lambda_{i_1},  \dots, \lambda_{i_k} \in \F$ tal que $\Bigsum{i}{\lambda_{i_i}v_{i_i}} = 0$ implica que $\lambda_{i_i} = 0$ para todo $1 \leq i \leq k$.
\item \textit{Sistema de generadores} si dado $v \in \mathcal{X}$ entonces existen $v_{i_1}, \dots, v_{i_k} \in \B$ y $\lambda_{i_1},  \dots, \lambda_{i_k} \in \F$ tal que $\Bigsum{i}{\lambda_{i_i}v_{i_i}} = v$.
\item \textit{Base} si es a la vez un sistema de generadores linealmente independiente.
\end{enumerate}
\end{definition}

\begin{example}



\begin{itemize}
	\item $X=\R[X]$ es un espacio vectorial, si consideramos $\B = \sett{1,X,X^2, \dots} = \sett{X^j}_{j \in \N}$ es base.
	\item $X=\mathcal{C}[a,b]$ es un espacio vectorial, si consideramos $\B = \sett{e^{\alpha x}, \alpha \in [0,1]}$ veamos que es linealmente independiente.
	
	\begin{proof}
		Sean $\alpha_1, \dots, \alpha_n \in [0,1]$ y $\lambda_{1}, \dots, \lambda_{n} \in \R$ tal que $\Bigsum{i}{\lambda_i e^{\alpha_i x}} = 0$ para todo $x \in [a,b]$; luego si derivamos $n-1$ veces tenemos el sistema:
		
		\[
			\left(
				\begin{array}{cccc}
					e^{\alpha_1 x} &  e^{\alpha_2 x} &  \dots &  e^{\alpha_n x} 
				\end{array}
			\right)
			\left(
				\begin{array}{cccc}
					1 & \alpha_1 &  \dots &  \alpha_1^{n-1} \\
					\vdots & \vdots & \vdots & \vdots \\
					1 &  \alpha_n &  \dots &  \alpha_n^{n-1}					 
				\end{array}
			\right)
			\left(
			\begin{array}{c}
			\lambda_{1} \\
			\lambda_{2} \\
			\vdots \\
			\lambda_{n}
			\end{array}
			\right) = 
			\left(
			\begin{array}{c}
			0 \\
			0 \\
			\vdots \\
			0
			\end{array}
			\right)
		\]
		Y como los $\alpha_i$ son distintos entonces la matriz de Vandermonde es inversible y el sistema admite una \'unica soluci\'on, $\lambda_{1} = \lambda_{2} = \dots = \lambda_{n} = 0$. \qed
	\end{proof}

\end{itemize}
\end{example}

Recordemos:

\begin{proposition}[Lema de Zorn]
	\label{Lema de Zorn}
	
	Si $(P,\leq)$ es un conjunto parcialmente ordenado, no vac\'io, tal que todo subconjunto no vac\'io $S \subseteq P$ totalmente ordenado admite una cota superior; entonces existe un elemento maximal en $P$.
\end{proposition}

\begin{proposition}
	Si $E$ es un espacio vectorial, entonces $E$ admite una base.
\end{proposition}

\begin{proof}
	Consideremos $P = \sett{S \subseteq E \ / \ S \text{ es li}}$ y dotemoslo del orden dado por la inclusi\'ion, luego $P \neq \emptyset$ pues si $v \in E$ entonces $\sett{v} \in P$.
	
	Sea $\sett{S_i}$ una colecci\'on de subconjuntos de $P$ totalmente ordenada y sea $T = \Bigcup{i \in I}{S_i}$, luego es claro que $S_i \leq T$; faltar\'ia ver que $T \in P$.
	
	Para eso sean $v_{i_1}, \dots, v_{i_k} \in T$ y $\lambda_{i_1},  \dots, \lambda_{i_k} \ in \F$ tales que $\Bigsum{k}{\lambda_i v_i} = 0$. Como son finitos existe $k_0 \in \N$ tal que $v_i \in S_{k_0}$ para todo $i$, que al ser un conjunto linealmente independiente resulta que $\lambda_{1} = \lambda_{2} = \dots = \lambda_{n} = 0$. Conclu\'imos que $T \in P$, luego por \ref{Lema de Zorn} existe $M \in P$ elemento maximal.
	
	Finalmente, sea $v \in E \setminus <M>$ (el conjunto generado por combinaciones lineales de $M$), luego $M \cup \sett{v}$ ser\'ia un conjunto li lo que contradice la maximalidad de $M$; por ende no existe tal $v$ y $M$ resulta base. \qed
	
	
	
\end{proof}

\begin{proposition}
	\label{Dos bases tienen mismo cardinal, Hamel}
	Sea $E$ un espacio vectorial y sean $\B_1, \B_2$ dos bases de Hamel de $E$. Luego $\#B_1 = \#B_2$.
\end{proposition}

\begin{proof}
	Sea $x \in \B_1$ y llamemos $S(x)$ al conjunto de los elementos $v \in \B_2$ tal que al escribir a $x$ como combinaci\'on lineal de elementos de $\B_2$ aparece $v$, por lo que si $x = \Bigsum{k}{\lambda_{i_k} v_{i_k}}$ entonces $S(x) = \sett{v_{i_1}, \dots, v_{i_n}}$.
			
	\begin{lemma}
		\label{Lema de cardinalidad de bases}
		$\Bigcup{x \in \B_1}{S(x)} = \B_2$
	\end{lemma}
	\begin{proof}{Del lema}
		Si $v \in \Bigcup{x \in \B_1}{S(x)}$ luego existe $x_0 \in \B_1$ tal que $v \in S(x_0)$ por lo que $v \in \B_2$ por definici\'on de $S(x)$. Rec\'iprocamente, si $v \in \B_2$ pero no existe $x \in \B_1$ tal que $v \in S(x)$, entonces $v \not \in <\B_1> =E = <\B_2>$. \qed
	\end{proof}
	
	Por \ref{Lema de cardinalidad de bases} tenemos que $\# \B_2 \leq \Bigsum{x \in \B_1}{\#S(x)} \leq \#\N \#\B_1 \leq \#B_1$.
	
	Razonando al rev\'es obtenemos la otra desigualdad. \qed		
	
\end{proof}

\begin{definition}
	Si $E$ es un espacio vectorial, una norma definida en $E$ es una aplicaci\'on $\norm{.}: E \mapsto \R$ tal que:
	
	\begin{enumerate}
		\item $\norm{x} \geq 0$
		\item $\norm{x} = 0 \ \Longleftrightarrow x = 0$
		\item $\norm{\lambda x} = \abs{\lambda} \norm{x} $
		\item $\norm{x+y} \leq \norm{x} + \norm{y}$
	\end{enumerate}
	
\end{definition}

\begin{remark}
	Todo espacio normado es un espacio m\'etrico pero no viceversa.
\end{remark}

\begin{definition}
	Si $E$ es un espacio vectorial, un producto interno definido en $E$ es una aplicaci\'on $\ip{.,.}: E \times E \mapsto F$ tal que:
	
	\begin{enumerate}
		\item $\ip{.,z}$ es lineal
		\item $\ip{x,x} = 0 \ \Longleftrightarrow x = 0$
		\item $\ip{x,y} = \overline{\ip{y,x}}$
	\end{enumerate}
	
\end{definition}


\begin{remark}
	Todo espacio con producto interno es un espacio normado pero no viceversa.
\end{remark}

\begin{theorem}[Cauchy-Schwartz]
	\label{Desigualdad de Cauchy-Schwartz }
	Sea $E$ un espacio vectorial y $\ip{.}$ un producto interno definido en $E$; luego si $x,y \in E$ se tiene que $\abs{\ip{x,y}} \leq \norm{x} \norm{y}$.
\end{theorem}

\begin{proof}
	Sean $x,y \in E$, $\lambda \in \C$ y sea $z = x-\lambda y$, luego $\ip{z,z} = \ip{x,x} + \abs{\lambda^2}\ip{y,y} -2 \Re(\lambda \ip{y,x}) \geq 0$.
	
	Si $\ip{y,x} = re^{i \theta}$ sea $\lambda = e^{-i \theta}t$ con $t \in \R$; luego:
	
	\[
		0 \geq \ip{x,x} + t^2 \ip{y,y} - 2bt \equiv c -2bt + at^2 := q(t)
	\]
	
	Luego como la cuadr\'atica dada es positiva, eso implica que $0 \leq 4b^2 -4ac$ por lo que:
	
	\[
		0 \leq b^2 -ac = \abs{\ip{x,y}}^2 - \ip{x,x}\ip{y,y}
	\]
	
	Si $\abs{\ip{x,y}}= \norm{x} \norm{y}$, entonces $b^2 = \ip{x,x} \ip{y,y}$ por lo que $b^2 -ac = 0$. Esto implica que existe $t_0$ tal que $q(t_0) = 0$, por lo tanto eso implica que $\ip{x-e^{-i \theta}t_0 y,x- e^{-i \theta}t_0 y} \equiv 0$ y por lo tanto $x = e^{-i \theta}t_0 y$. \qed
	
\end{proof}

\end{document}