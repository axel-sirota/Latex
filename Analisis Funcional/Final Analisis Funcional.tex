\documentclass[11pt]{article}

\usepackage{amsfonts}
\usepackage{amsmath,accents,amsfonts, amssymb, mathrsfs }
\usepackage{tikz-cd}
\usepackage{graphicx}
\usepackage{syntonly}
\usepackage{color}
\usepackage{mathrsfs}
\usepackage[spanish]{babel}
\usepackage[latin1]{inputenc}
\usepackage{fancyhdr}
\usepackage[all]{xy}

\topmargin-2cm \oddsidemargin-1cm \evensidemargin-1cm \textwidth18cm
\textheight25cm


\newcommand{\B}{\mathcal{B}}
\newcommand{\F}{\mathcal{F}}
\newcommand{\Hil}{\mathcal{H}}
\newcommand{\X}{\mathcal{X}}
\newcommand{\f}{\mathbb{F}}
\newcommand{\inte}{\mathrm{int}}
\newcommand{\A}{\mathcal{A}}
\newcommand{\C}{\mathbb{C}}
\newcommand{\Q}{\mathbb{Q}}
\newcommand{\Z}{\mathbb{Z}}
\newcommand{\inc}{\hookrightarrow}
\renewcommand{\P}{\mathcal{P}}
\newcommand{\R}{{\mathbb{R}}}
\newcommand{\N}{{\mathbb{N}}}
\newcommand\norm[1]{\left\lVert#1\right\rVert}
\newcommand{\prodint}[1]{\langle #1 \rangle}
\newcommand{\sett}[1]{\{#1\}}
\newcommand{\interior}[1]{\accentset{\smash{\raisebox{-0.12ex}{$\scriptstyle\circ$}}}{#1}\rule{0pt}{2.3ex}}
\fboxrule0.0001pt \fboxsep0pt

\def \le{\leqslant}	
\def \ge{\geqslant}
\def\sen{{\rm sen} \, \theta}
\def\cos{{\rm cos}\, \theta}
\def\noi{\noindent}
\def\sm{\smallskip}
\def\ms{\medskip}
\def\bs{\bigskip}
\def \be{\begin{enumerate}}
\def \en{\end{enumerate}}
\def\deck{{\rm Deck}}

\newtheorem{theorem}{Teorema}[section]
\newtheorem{lemma}[theorem]{Lema}
\newtheorem{proposition}[theorem]{Proposici\'on}
\newtheorem{corollary}[theorem]{Corolario}

\newenvironment{proof}[1][Demostraci\'on]{\begin{trivlist}
\item[\hskip \labelsep {\bfseries #1}]}{\end{trivlist}}
\newenvironment{definition}[1][Definici\'on]{\begin{trivlist}
\item[\hskip \labelsep {\bfseries #1}]}{\end{trivlist}}
\newenvironment{example}[1][Ejemplo]{\begin{trivlist}
\item[\hskip \labelsep {\bfseries #1}]}{\end{trivlist}}
\newenvironment{remark}[1][Observaci\'on]{\begin{trivlist}
\item[\hskip \labelsep {\bfseries #1}]}{\end{trivlist}}
\newenvironment{declaration}[1][Afirmaci\'on]{\begin{trivlist}
\item[\hskip \labelsep {\bfseries #1}]}{\end{trivlist}}


\newcommand{\qed}{\nobreak \ifvmode \relax \else
      \ifdim\lastskip<1.5em \hskip-\lastskip
      \hskip1.5em plus0em minus0.5em \fi \nobreak
      \vrule height0.75em width0.5em depth0.25em\fi}

\newcommand{\twopartdef}[4]
{
	\left\{
		\begin{array}{ll}
			#1 & \mbox{ } #2 \\
			#3 & \mbox{ } #4
		\end{array}
	\right.
}

\newcommand{\threepartdef}[6]
{
	\left\{
		\begin{array}{lll}
			#1 & \mbox{ } #2 \\
			#3 & \mbox{ } #4 \\
			#5 & \mbox{ } #6
		\end{array}
	\right.
}


\begin{document}

\pagestyle{empty}
\pagestyle{fancy}
\fancyfoot[CO]{\slshape \thepage}
\renewcommand{\headrulewidth}{0pt}



\centerline{\bf Final An\'alisis Funcional}

\bigskip

\chapter{Espacios de Hilbert}

\section{Propiedades Elementales}

\begin{definition}
Si $\X$ es un espacio vectorial sobre un cuerpo $\f$, un semi-producto interno es $u: \X \times \X \rightarrow \f$ tal que $\forall \alpha, \beta \in \f$ y $x,y,z \in \X$:

\begin{itemize}
\item $u(\alpha x + \beta y, z) = \alpha u(x,z) + \beta u(y,z)$
\item $u(\alpha x + \beta y, z) = \bar{\alpha} u(x,z) + \bar{\beta} u(y,z)$
\item $u(x,x) \geq 0$
\item $u(x,y) = \bar{u(y,x)}$
\end{itemize}

\end{definition}

\begin{remark}
$u(0,y)=u(x,0)=0$
\end{remark}

Si $u(x,x)=0 \Longrightarrow x=0$ entonces $u$ es un producto interno, lo notaremos: $u(x,y)=\prodint{x,y}$

\begin{proposition}

Si $\prodint{.,.}$ es un semi producto interno en $\X$, entonces:
\[
|\prodint{x,y}|^2 \leq \prodint{x,x} \prodint{y,y} \quad \forall x,y \in \X
\]

Es m\'as, la igualdad se da si $\exists \alpha,\beta \neq 0 \ / \ \prodint{\alpha x + \beta y , \alpha x + \beta y} = 0$

\end{proposition}

\begin{proof}

Sea $\alpha \in \f$ y $x,y \in \X$, entonces:

\[
0 \leq \prodint{x - \alpha y , x - \alpha y} = \prodint{x,x} - \alpha \prodint{y,x} - \overline{\alpha}{\prodint{x,y}} + |\alpha|^2\prodint{y,y}
\]

Supongamos que $\prodint{y,x} = be^{i \theta} \ , \ b\geq 0$ y sea $\alpha = te^{-i \theta}$ con $t \in \R$, entonces:

\[
0 \leq \prodint{x,x} - 2bt + t^2 \prodint{y,y} \ \Longleftrightarrow \ 0 \geq 4b^2 - 4 \prodint{x,x} \prodint{y,y}
\\
\Longleftrightarrow \ |\prodint{x,y}|^2 \leq \prodint{x,x} \prodint{y,y}
\]
\qed
\end{proof}

\begin{corollary}
Si $\prodint{.,.}$ es un producto interno en $\X$, entonces $\norm{x} = \prodint{x,x}^{\frac{1}{2}}$ es una norma en $\X$.
\end{corollary}

\begin{proposition}
Sean $f_1,f_2,...,f_n \in \Hil$ espacio de Hilbert, entonces:
 
$$\norm{f_1+f_2 + ... + f_n} ^2 = \sum_{i=0}^{n}{\norm{f_i}^2}$$
 
\end{proposition}

\begin{proof}
Paja \qed
\end{proof}

\begin{proposition}
Si $\Hil$ es un espacio de Hilbert y $f,g \in \Hil$ entonces:

\begin{equation}\label{paralelogramo}
\norm{f+g}^2 + \norm{f-g}^2 = 2 (\norm{f} ^2 + \norm {g} ^2)
\end{equation}

Reciprocamente si $\Hil$ es un Banach tal que su norma $\norm{.}$ cumple \ref{paralelogramo}, entonces $\norm{.} = \prodint{.,.}^{\frac{1}{2}}$ para un producto interno tal que $\Hil$ es Hilbert.

\end{proposition}

\begin{proof}

$\Longrightarrow)$ Es f\'acil

$\Longleftarrow)$ Supongamos que vale \ref{paralelogramo} y que $\f = \R $ y sea $u(x,y)= \frac{1}{4} \norm{x+y}^2 - \frac{1}{4} \norm{x-y}^2$, veamos que es un producto interno.

\begin{itemize}

\item $u(x,y)=u(y,x)$

Trivial

\item $\norm{x}=u(x,x)^{\frac{1}{2}}$

$u(x,x) = \frac{1}{4} \norm{2x}^{2} = \frac{4}{4} \norm{x}^2 = \norm{x}^2$. Como ambos son positivos listo.

\item $u$ es $\X \times \X$ continua

Por definci\'on $\norm{.}$, sumar y restar son continuos y composici\'on de continuas es continua.

\item  $$

\end{itemize}
\end{proof}


\chapter{Operadores en Espacios de Hilbert}

\end{document}