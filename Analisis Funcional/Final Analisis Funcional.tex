\documentclass[11pt]{article}

\usepackage{amsfonts}
\usepackage{amsmath,accents,amsfonts, amssymb, mathrsfs }
\usepackage{tikz-cd}
\usepackage{graphicx}
\usepackage{syntonly}
\usepackage{color}
\usepackage{mathrsfs}
\usepackage[spanish]{babel}
\usepackage[latin1]{inputenc}
\usepackage{fancyhdr}
\usepackage[all]{xy}
\usepackage[at]{easylist}
\usepackage[colorlinks=true,linkcolor=blue,urlcolor=black,bookmarksopen=true]{hyperref}

\usepackage{bookmark}

\topmargin-2cm \oddsidemargin-1cm \evensidemargin-1cm \textwidth18cm
\textheight25cm


\newcommand{\B}{\mathcal{B}}
\newcommand{\Cont}{\mathcal{C}}
\newcommand{\F}{\mathcal{F}}
\newcommand{\inte}{\mathrm{int}}
\newcommand{\A}{\mathcal{A}}
\newcommand{\C}{\mathbb{C}}
\newcommand{\Q}{\mathbb{Q}}
\newcommand{\Z}{\mathbb{Z}}
\newcommand{\inc}{\hookrightarrow}
\renewcommand{\P}{\mathcal{P}}
\newcommand{\R}{{\mathbb{R}}}
\newcommand{\N}{{\mathbb{N}}}
\newcommand\tq{~:~}
\newcommand{\dual}[1]{\left(#1\right)^{\ast}}
\newcommand{\ortogonal}[1]{\left(#1\right)^{\perp}}
\newcommand{\ddual}[1]{\left(#1^{\ast}\right)^{\ast}}
\newcommand{\x}[3]{#1_#2^#3}
\newcommand{\xx}[4]{#1_#3#2_#4}
\newcommand\dd{\,\mathrm{d}}
\newcommand{\norm}[1]{\left\lVert#1\right\rVert}
\newcommand{\abs}[1]{\left\lvert#1\right\rvert}
\newcommand{\ip}[1]{\left\langle#1\right\rangle}
\renewcommand\tt{\mathbf{t}}
\newcommand\nn{\mathbf{n}}
\newcommand\bb{\mathbf{b}}                      % binormal
\newcommand\kk{\kappa}
\newcommand{\sett}[1]{\left\lbrace#1\right\rbrace}
\newcommand{\interior}[1]{\accentset{\smash{\raisebox{-0.12ex}{$\scriptstyle\circ$}}}{#1}\rule{0pt}{2.3ex}}
\fboxrule0.0001pt \fboxsep0pt
\newcommand{\Bigcup}[2]{\bigcup\limits_{#1}{#2}}
\newcommand{\Bigcap}[2]{\bigcap\limits_{#1}{#2}}
\newcommand{\Bigprod}[2]{\prod\limits_{#1}{#2}}
\newcommand{\Bigcoprod}[2]{\coprod\limits_{#1}{#2}}
\newcommand{\Bigsum}[2]{\sum\limits_{#1}{#2}}
\newcommand{\BigsumA}[3]{ \sideset{}{^#2}\sum\limits_{#1}{#3}}
\newcommand{\Biglim}[2]{\lim\limits_{#1}{#2}}
\newcommand{\quotient}[2]{{\raisebox{.2em}{$#1$}\left/\raisebox{-.2em}{$#2$}\right.}}
\DeclareMathOperator{\rank}{ran}
\DeclareMathOperator{\graf}{Gr}
\DeclareMathOperator{\ball}{ball}

\def \le{\leqslant}	
\def \ge{\geqslant}
\def\noi{\noindent}
\def\sm{\smallskip}
\def\ms{\medskip}
\def\bs{\bigskip}
\def \be{\begin{enumerate}}
	\def \en{\end{enumerate}}
\def\deck{{\rm Deck}}
\def\Tau{{\rm T}}

\newtheorem{mytheorem}{Theorem}
 %

\newtheorem{theorem}{Teorema}
\numberwithin{theorem}{subsection}
\newtheorem{lemma}[theorem]{Lema}

\newtheorem{proposition}[theorem]{Proposici\'on}

\newtheorem{corollary}[theorem]{Corolario}


\newenvironment{proof}[1][Demostraci\'on]{\begin{trivlist}
		\item[\hskip \labelsep {\bfseries #1}]}{\end{trivlist}}
\newenvironment{definition}[1][Definici\'on]{\begin{trivlist}
		\item[\hskip \labelsep {\bfseries #1}]}{\end{trivlist}}
\newenvironment{example}[1][Ejemplo]{\begin{trivlist}
		\item[\hskip \labelsep {\bfseries #1 }]}{\end{trivlist}}
\newenvironment{remark}[1][Observaci\'on]{\begin{trivlist}
		\item[\hskip \labelsep {\bfseries #1}]}{\end{trivlist}}
\newenvironment{declaration}[1][Afirmaci\'on]{\begin{trivlist}
		\item[\hskip \labelsep {\bfseries #1}]}{\end{trivlist}}


\newcommand{\qed}{\nobreak \ifvmode \relax \else
	\ifdim\lastskip<1.5em \hskip-\lastskip
	\hskip1.5em plus0em minus0.5em \fi \nobreak
	\vrule height0.75em width0.5em depth0.25em\fi}

\newcommand{\twopartdef}[4]
{
	\left\{
	\begin{array}{ll}
		#1 & \mbox{ } #2 \\
		#3 & \mbox{ } #4
	\end{array}
	\right.
}

\newcommand{\threepartdef}[6]
{
	\left\{
	\begin{array}{lll}
		#1 & \mbox{ } #2 \\
		#3 & \mbox{ } #4 \\
		#5 & \mbox{ } #6
	\end{array}
	\right.
}

\tikzset{commutative diagrams/.cd,
	mysymbol/.style={start anchor=center,end anchor=center,draw=none}
}
\newcommand\Center[2]{%
	\arrow[mysymbol]{#2}[description]{#1}}

\newcommand*\circled[1]{\tikz[baseline=(char.base)]{
		\node[shape=circle,draw,inner sep=2pt] (char) {#1};}}


\makeatletter
\newcommand{\xRightarrow}[2][]{\ext@arrow 0359\Rightarrowfill@{#1}{#2}}
\makeatother


\begin{document}
	
	\pagestyle{empty}
	\pagestyle{fancy}
	\fancyfoot[CO]{\slshape \thepage}
	\renewcommand{\headrulewidth}{0pt}
	
	
	
	\centerline{\bf An\'alisis Funcional}
	\centerline{\sc Final}
	\centerline{\sc Axel Sirota}
	
	\tableofcontents
	\newpage

\section{Espacios Vectoriales}

\subsection{Propiedades Elementales}

\begin{definition}
Si $\mathcal{X}$ es un espacio vectorial sobre un cuerpo $\mathbb{F}$, un conjunto $\mathcal{B} = \sett{v_i}_{i \in I}$ se dice:

\begin{enumerate}
\item \textit{Linealmente independiente} si dados $v_{i_1}, \dots, v_{i_k} \in \B$ y $\lambda_{i_1},  \dots, \lambda_{i_k} \in \F$ tal que $\Bigsum{i}{\lambda_{i_i}v_{i_i}} = 0$ implica que $\lambda_{i_i} = 0$ para todo $1 \leq i \leq k$.
\item \textit{Sistema de generadores} si dado $v \in \mathcal{X}$ entonces existen $v_{i_1}, \dots, v_{i_k} \in \B$ y $\lambda_{i_1},  \dots, \lambda_{i_k} \in \F$ tal que $\Bigsum{i}{\lambda_{i_i}v_{i_i}} = v$.
\item \textit{Base} si es a la vez un sistema de generadores linealmente independiente.
\end{enumerate}
\end{definition}

\begin{example}



\begin{itemize}
	\item $X=\R[X]$ es un espacio vectorial, si consideramos $\B = \sett{1,X,X^2, \dots} = \sett{X^j}_{j \in \N}$ es base.
	\item $X=\mathcal{C}[a,b]$ es un espacio vectorial, si consideramos $\B = \sett{e^{\alpha x}, \alpha \in [0,1]}$ veamos que es linealmente independiente.
	
	\begin{proof}
		Sean $\alpha_1, \dots, \alpha_n \in [0,1]$ y $\lambda_{1}, \dots, \lambda_{n} \in \R$ tal que $\Bigsum{i}{\lambda_i e^{\alpha_i x}} = 0$ para todo $x \in [a,b]$; luego si derivamos $n-1$ veces tenemos el sistema:
		
		\[
			\left(
				\begin{array}{cccc}
					e^{\alpha_1 x} &  e^{\alpha_2 x} &  \dots &  e^{\alpha_n x} 
				\end{array}
			\right)
			\left(
				\begin{array}{cccc}
					1 & \alpha_1 &  \dots &  \alpha_1^{n-1} \\
					\vdots & \vdots & \vdots & \vdots \\
					1 &  \alpha_n &  \dots &  \alpha_n^{n-1}					 
				\end{array}
			\right)
			\left(
			\begin{array}{c}
			\lambda_{1} \\
			\lambda_{2} \\
			\vdots \\
			\lambda_{n}
			\end{array}
			\right) = 
			\left(
			\begin{array}{c}
			0 \\
			0 \\
			\vdots \\
			0
			\end{array}
			\right)
		\]
		Y como los $\alpha_i$ son distintos entonces la matriz de Vandermonde es inversible y el sistema admite una \'unica soluci\'on, $\lambda_{1} = \lambda_{2} = \dots = \lambda_{n} = 0$. \qed
	\end{proof}

\end{itemize}
\end{example}

Recordemos:

\begin{proposition}[Lema de Zorn]
	\label{Lema de Zorn}
	
	Si $(P,\leq)$ es un conjunto parcialmente ordenado, no vac\'io, tal que todo subconjunto no vac\'io $S \subseteq P$ totalmente ordenado admite una cota superior; entonces existe un elemento maximal en $P$.
\end{proposition}

\begin{proposition}
	Si $E$ es un espacio vectorial, entonces $E$ admite una base.
\end{proposition}

\begin{proof}
	Consideremos $P = \sett{S \subseteq E \ / \ S \text{ es li}}$ y dotemoslo del orden dado por la inclusi\'ion, luego $P \neq \emptyset$ pues si $v \in E$ entonces $\sett{v} \in P$.
	
	Sea $\sett{S_i}$ una colecci\'on de subconjuntos de $P$ totalmente ordenada y sea $T = \Bigcup{i \in I}{S_i}$, luego es claro que $S_i \leq T$; faltar\'ia ver que $T \in P$.
	
	Para eso sean $v_{i_1}, \dots, v_{i_k} \in T$ y $\lambda_{i_1},  \dots, \lambda_{i_k} \ in \F$ tales que $\Bigsum{k}{\lambda_i v_i} = 0$. Como son finitos existe $k_0 \in \N$ tal que $v_i \in S_{k_0}$ para todo $i$, que al ser un conjunto linealmente independiente resulta que $\lambda_{1} = \lambda_{2} = \dots = \lambda_{n} = 0$. Conclu\'imos que $T \in P$, luego por \ref{Lema de Zorn} existe $M \in P$ elemento maximal.
	
	Finalmente, sea $v \in E \setminus <M>$ (el conjunto generado por combinaciones lineales de $M$), luego $M \cup \sett{v}$ ser\'ia un conjunto li lo que contradice la maximalidad de $M$; por ende no existe tal $v$ y $M$ resulta base. \qed
	
	
	
\end{proof}

\begin{proposition}
	\label{Dos bases tienen mismo cardinal, Hamel}
	Sea $E$ un espacio vectorial y sean $\B_1, \B_2$ dos bases de Hamel de $E$. Luego $\#B_1 = \#B_2$.
\end{proposition}

\begin{proof}
	Sea $x \in \B_1$ y llamemos $S(x)$ al conjunto de los elementos $v \in \B_2$ tal que al escribir a $x$ como combinaci\'on lineal de elementos de $\B_2$ aparece $v$, por lo que si $x = \Bigsum{k}{\lambda_{i_k} v_{i_k}}$ entonces $S(x) = \sett{v_{i_1}, \dots, v_{i_n}}$.
			
	\begin{lemma}
		\label{Lema de cardinalidad de bases}
		$\Bigcup{x \in \B_1}{S(x)} = \B_2$
	\end{lemma}
	\begin{proof}{Del lema}
		Si $v \in \Bigcup{x \in \B_1}{S(x)}$ luego existe $x_0 \in \B_1$ tal que $v \in S(x_0)$ por lo que $v \in \B_2$ por definici\'on de $S(x)$. Rec\'iprocamente, si $v \in \B_2$ pero no existe $x \in \B_1$ tal que $v \in S(x)$, entonces $v \not \in <\B_1> =E = <\B_2>$. \qed
	\end{proof}
	
	Por \ref{Lema de cardinalidad de bases} tenemos que $\# \B_2 \leq \Bigsum{x \in \B_1}{\#S(x)} \leq \#\N \#\B_1 \leq \#B_1$.
	
	Razonando al rev\'es obtenemos la otra desigualdad. \qed		
	
\end{proof}

\subsection{Normas y productos internos}

\begin{definition}
	Si $E$ es un espacio vectorial, una norma definida en $E$ es una aplicaci\'on $\norm{.}: E \mapsto \R$ tal que:
	
	\begin{enumerate}
		\item $\norm{x} \geq 0$
		\item $\norm{x} = 0 \ \Longleftrightarrow x = 0$
		\item $\norm{\lambda x} = \abs{\lambda} \norm{x} $
		\item $\norm{x+y} \leq \norm{x} + \norm{y}$
	\end{enumerate}
	
\end{definition}

\begin{remark}
	Todo espacio normado es un espacio m\'etrico pero no viceversa.
\end{remark}

\begin{definition}
	Si $E$ es un espacio vectorial, un producto interno definido en $E$ es una aplicaci\'on $\ip{.,.}: E \times E \mapsto F$ tal que:
	
	\begin{enumerate}
		\item $\ip{.,z}$ es lineal
		\item $\ip{x,x} = 0 \ \Longleftrightarrow x = 0$
		\item $\ip{x,y} = \overline{\ip{y,x}}$
	\end{enumerate}
	
\end{definition}


\begin{remark}
	Todo espacio con producto interno es un espacio normado pero no viceversa.
\end{remark}

\begin{theorem}[Cauchy-Schwartz]
	\label{Desigualdad de Cauchy-Schwartz }
	Sea $E$ un espacio vectorial y $\ip{.}$ un producto interno definido en $E$; luego si $x,y \in E$ se tiene que $\abs{\ip{x,y}} \leq \norm{x} \norm{y}$.
\end{theorem}

\begin{proof}
	Sean $x,y \in E$, $\lambda \in \C$ y sea $z = x-\lambda y$, luego $\ip{z,z} = \ip{x,x} + \abs{\lambda^2}\ip{y,y} -2 \Re(\lambda \ip{y,x}) \geq 0$.
	
	Si $\ip{y,x} = re^{i \theta}$ sea $\lambda = e^{-i \theta}t$ con $t \in \R$; luego:
	
	\[
		0 \geq \ip{x,x} + t^2 \ip{y,y} - 2bt \equiv c -2bt + at^2 := q(t)
	\]
	
	Luego como la cuadr\'atica dada es positiva, eso implica que $0 \leq 4b^2 -4ac$ por lo que:
	
	\[
		0 \leq b^2 -ac = \abs{\ip{x,y}}^2 - \ip{x,x}\ip{y,y}
	\]
	
	Si $\abs{\ip{x,y}}= \norm{x} \norm{y}$, entonces $b^2 = \ip{x,x} \ip{y,y}$ por lo que $b^2 -ac = 0$. Esto implica que existe $t_0$ tal que $q(t_0) = 0$, por lo tanto eso implica que $\ip{x-e^{-i \theta}t_0 y,x- e^{-i \theta}t_0 y} \equiv 0$ y por lo tanto $x = e^{-i \theta}t_0 y$. \qed
	
\end{proof}

\begin{definition}
	Un espacio normado que es completo respecto a la distancia inducida por la norma se llama \textit{Espacio de Banach}
\end{definition}

\begin{definition}
	Un \textit{Espacio de Hilbert} es un espacio de Banach donde la norma proviene de un producto interno mediante $\norm{x} = \sqrt{\ip{x,x}}$.
\end{definition}

\begin{proposition}
	\label{Identidad de poralizacion}
	Sea $E$ un espacio con producto interno, entonces:
	
	\begin{itemize}
		\item $\mathcal{R}(\ip{x,y}) = \frac{1}{4}\left(\norm{x+y}^2 - \norm{x-y}^2\right)$
		\item $\mathcal{I}(\ip{x,y}) = \frac{1}{4}\left(\norm{x+iy}^2 - \norm{x-iy}^2\right)$
	\end{itemize}

\end{proposition}

\begin{proof}
	Por un lado $\norm{x+y}^2 = \norm{x}^2 + \norm{y}^2 + 2\mathcal{R}(\ip{x,y})$ y $\norm{x-y}^2 = \norm{x}^2 + \norm{y}^2 - 2\mathcal{R}(\ip{x,y})$; por lo que restando se obtiene:
	
	\[
		4\mathcal{R}(\ip{x,y}) = \norm{x+y}^2 - \norm{x-y}^2
	\]
	
	Por el otro: 
	
	\[
	\begin{aligned}
		\norm{x+iy}^2 = & \ip{x+iy,x+iy} \\
					  = & \norm{x}^2 + \abs{i}\norm{y}^2 -i \ip{x,y} + i \overline{\ip{x,y}} \\
					  = & \norm{x}^2 + \norm{y}^2 -i 2 \mathcal{I}(\ip{x,y}) \\
		\norm{x-iy}^2 = & \ip{x-iy,x-iy} \\
			   		  = & \norm{x}^2 + \abs{i}\norm{y}^2 +i \ip{x,y} - i \overline{\ip{x,y}} \\
					  = & \norm{x}^2 + \norm{y}^2 +i 2 \mathcal{I}(\ip{x,y}) \\
	\end{aligned}
	\]
	
	Por lo tanto restando ambas obtenemos:
	
	\[
		4\mathcal{I}(\ip{x,y}) = \norm{x+iy}^2 - \norm{x-iy}^2
	\]
	\qed
	
\end{proof}

\begin{proposition}[Ley del paralelogramo]
	\label{Ley del paralelogramo}
	Sea $E$ un espacio normado real, entonces existe $\ip{.,.}: E \times E \rightarrow \C$ tal que $\norm{x}= \sqrt{\ip{x,x}}$ si y s\'olo si para todos $x,y \in E$ vale:
	
	\[
		\norm{x+y}^2 + \norm{x-y}^2 = 2\norm{x}^2 + 2\norm{y}^2
	\]
\end{proposition}

\begin{proof}
	Si $\norm{x} = \sqrt{\ip{x,x}}$ entonces de la demostraci\'on de \ref{Identidad de poralizacion} se da el resultado. Rec\'iprocamente definamos:
	
	 $$ \ip{x,y} := \frac{1}{4}\left(\norm{x+y}^2 - \norm{x-y}^2\right)$$
	
	Luego verifiquemos que es un producto interno.
	
	\begin{enumerate}
		\item $\sqrt{\ip{x,x}} = \norm{x}$
		\item Como $\norm{x+y} = \norm{y+x}$ y $\norm{x-y} = \norm{-(y-x)} = \norm{y-x}$ conclu\'imos que $\ip{x,y} = \ip{y,x}$.
		\item Dado que $\norm{.}, +, -, *$ son $\norm{.}$-continuas entonces $\ip{.,x},\ip{x,.}$ es $\norm{.}$-continua.
		\item Sean $x,y,z \in E$ entonces:
		
		\[
		\begin{aligned}
			\norm{x+y+z}^2 = & 2\norm{x+z}^2 + 2\norm{y}^2 - \norm{x-y+z}^2
						   = & 2\norm{y+z}^2 + 2\norm{x}^2 - \norm{y - x+z}^2
		\end{aligned}
		\]
		
		Luego como $A=B$ y $A=C$ implica $A=\frac{B+C}{2}$ se obtiene:
		
		\[
		\begin{aligned}
			\norm{x+y+z}^2 = & \norm{x+z}^2 + \norm{y}^2 - \frac{1}{2}\norm{x-y+z}^2 + \norm{y+z}^2 + \norm{x}^2 - \frac{1}{2}\norm{y - x+z}^2 \\
			\norm{x+y-z}^2 = & \norm{x-z}^2 + \norm{y}^2 - \frac{1}{2}\norm{x-y-z}^2 + \norm{y-z}^2 + \norm{x}^2 - \frac{1}{2}\norm{y - x-z}^2 \\
						   = & \norm{x-z}^2 + \norm{y}^2 - \frac{1}{2}\norm{-x+y+z}^2 + \norm{y-z}^2 + \norm{x}^2 - \frac{1}{2}\norm{-y + x+z}^2 
		\end{aligned}
		\]
		
		Por lo tanto:
		
		\[
			\begin{aligned}
				\ip{x+y,z} = & \frac{1}{4} \left(\norm{x+y+z}^2 - \norm{x+y-z}^2\right) \\
						   = & \frac{1}{4} \left(\norm{x+z}^2 - \norm{x-z}^2\right) + \frac{1}{4} \left(\norm{y+z}^2 - \norm{y-z}^2\right) \\
						   = & \ip{x,z} + \ip{y,z}
			\end{aligned}
		\]
		
		\item Por el item anterior es claro por inducci\'on que $\lambda \ip{x,y} = \ip{\lambda x,y}$ para todo $\lambda \in \N$ y como vale para $\lambda = -1$ tenemos que vale para todo $\lambda \in \Z$. Si $\lambda = \frac{p}{q} \in \Q$ entonces si llamamos $x' = \frac{x}{q}$ tenemos:
		
		\[
			q\ip{\lambda x, y} = q \ip{p x',y} = p \ip{qx',y} = p \ip{x,y}
		\]
		
		Luego $\lambda \ip{x,y} = \ip{\lambda x, y}$ para todo $\lambda \in \Q$. Por lo tanto probamos que fijados $x,y \in E$ la funci\'on $g(t)=\frac{1}{t} \ip{tx,y}$ y la funci\'on constante $h(t) = \ip{x,y}$ cumplen que $h\vert_{\Q} = g \vert_{\Q}$ y por continuidad entonces $h \equiv g$ para todo $t \in \R \setminus \sett{0}$; como el caso $\lambda = 0$ es trivial conclu\'imos que $\lambda \ip{x,y} = \ip{\lambda x,y}$. \qed
		
	\end{enumerate}
	
\end{proof}

\section{Espacios de Hilbert}

\subsection{Preliminares}

\begin{proposition}
	\label{Prod interno es continuo}
	Sea $E$ un espacio vectorial con producto interno, luego el producto interno es continuo.
\end{proposition}

\begin{proof}
	Sea $x_n, (y_n)$ tales que $x_n \rightarrow x, y_n \rightarrow y$, luego:
	
	\begin{equation*}
	\begin{aligned}
		\abs{\ip{x_n,y_n} -  \ip{x,y}} = & \abs{\ip{x_n - x,y_n} + \ip{x,y_n - y}} \\ 
		\leq & \abs{\ip{x_n-x,y_n}} + \abs{\ip{x,y_n-y}} \\
		\leq & \abs{\ip{x_n-x,y_n -y}} + \abs{\ip{x_n-x,y}} + \abs{\ip{x,y_n-y}} \\ 
		\leq & \norm{x_n - x}\norm{y} + \norm{x}\norm{y_n-y} + \norm{x_n - x}\norm{y_n-y} \rightarrow 0 
	\end{aligned}
	\end{equation*}
	\qed
\end{proof}

\subsection{Conjuntos ortogonales y ortonormales}

\begin{definition}
	Sea $E$ un espacio vectorial con producto interno, luego dados dos vectores $x,y \in E$ decimos que son \textit{ortogonales} si $\ip{x,y} = 0$.
	
	A su vez decimos que son \textit{ortonormales} si osn ortogonales y $\norm{x} = \norm{y} = 1$
	
	Finalmente dado un conjunto $S \subseteq E$ entonces decimos que es \textit{ortogonal / ortonormal} si dados cualesquiera $x,y \in S$ resulta que son \textit{ortogonales / ortonormales} 
\end{definition}

\begin{example}
	El conjunto $\sett{e^{inx} \ , \ n \in \N \ , \ x \in [0,2 \pi]}$ es ortonormal.
\end{example}

\begin{theorem}
	\label{Escritura de proyeccion a un conjunto ortonormal}
	Sea $E$ un espacio vectorial con producto interno y sea $S \subseteq E$ un conjunto ortonormal, luego si $x \in \ip{S}$ entonces existe una \'unica escritura de $x$ dada por:
	
	$$x = \sum\limits_{i = 1}^{n}{\ip{x,u_i}u_i} \qquad u_i \in S$$
	
\end{theorem}

\begin{proof}
	Como $x \in \ip{S}$ entonces existen \'unicos $\lambda_{1},\dots, \lambda_{n}$ tal que $x = \sum\limits_{i = 1}^{n}{\lambda_i u_i}$. Luego:
	
	
	\[
		\ip{x,u_j} = \sum\limits_{i = 1}^{n}{\lambda_i \ip{u_i,u_j}} = \lambda_j
	\]
	\qed
\end{proof}

\begin{theorem}[Desigualdad de Bessel]
	\label{Desigualdad de Bessel}
	Sea $E$ un espacio vectorial con producto interno y sea $S \subseteq E$ un conjunto ortonormal, luego:
	
	\begin{enumerate}
		\item SI $x \in E$ y $u_1, \dots, u_n \in S$ luego $\sum\limits_{i = 1}^{n}{\abs{\ip{x,u_i}}^2} \leq \norm{x}^2$
		\item Si $x \in E$ entonces $\sett{u \in S \ / \ \ip{x,u}\neq 0}$ es a lo sumo numerable
		\item Si $x,y \in E$ entonces $\abs{\Bigsum{u \in S}{\ip{x,u}\overline{\ip{y,u}}}} \leq \norm{x}\norm{y}$
	\end{enumerate}
\end{theorem}

\begin{proof}
	\begin{enumerate}
		\item Sean $u_1,\dots,u_n \in S$ y sea $z = x - \sum\limits_{i = 1}^{n}{{\ip{x,u_i}}}$, luego:
	
	\[
	\begin{aligned}
		0 \leq & \ip{z,z} \\
		= & \ip{x - \sum\limits_{i = 1}^{n}{\ip{x,u_i}},x - \sum\limits_{i = 1}^{n}{\ip{x,u_i}}} \\
		= & \norm{x}^2 + \norm{\sum\limits_{i = 1}^{n}{\ip{x,u_i}}}^2 - 2 \mathcal{R} \left(\ip{\sum\limits_{i = 1}^{n}{\ip{x,u_i}},x}\right) \\
		= & \norm{x}^2 + \sum\limits_{i = 1}^{n}{\norm{\ip{x,u_i}}^2} - 2 \mathcal{R} \left(\sum\limits_{i = 1}^{n}{\abs{\ip{x,u_i}}^2}\right) \\
		= & \norm{x}^2 - \sum\limits_{i = 1}^{n}{\norm{\ip{x,u_i}}^2}.
	\end{aligned}
	\]
	
	\item Notemos que $S = \sett{u \in S \ / \ \abs{\ip{x,u}} > 0} = \Bigcup{n \in \N}{\underbrace{\sett{u \in S \ / \ \abs{\ip{x,u}} \geq \frac{1}{m} }}_{T_m}}$.
	
	Ahora sean $u_1,\dots,u_n \in T$ por el item anterior sabemos que:
	
	\[
		\frac{n}{m^2} \leq \Bigsum{1 \leq k \leq n}{\abs{\ip{x,u_k}}^2} \leq \norm{x}^2
	\]
	
	Por lo que $n \leq m^2 \norm{x}^2$ y entonces $\# T_m \leq m^2 \norm{x}^2 < \infty$ para todo $m$, por lo tanto $\# S \leq \# \N * \# T_m \leq \# \N$.
	
	\item Sean $x,y \in E$ y $u_1,\dots,u_n \in S$, luego:
	\[
	 \begin{aligned}
	 	 \abs{\sum\limits_{i = 1}^{n}{\ip{x,u_i}\overline{\ip{y,u_i}}}} \leq_\text{C-S} & \sqrt{ \sum\limits_{i = 1}^{n}{\abs{\ip{x,u_i}}^2}}\sqrt{ \sum\limits_{i = 1}^{n}{\abs{\ip{y,u_i}}^2}} \\
	 	 \leq_\text{a} & \norm{x}\norm{y}
	 \end{aligned}
	\]
	\qed
	\end{enumerate}
\end{proof}

\begin{theorem}
	\label{Todo conjunto ortonormal en un separable es numerable}
	Si $E$ es un espacio vectorial con producto interno tal que $E$ es separable, entonces todo conjunto ortonormal es a lo sumo numerable
\end{theorem}

\begin{proof}
	Sea $S \subseteq E$ un conjunto ortonormal y sean $u \neq v \in S$, luego $\norm{u-v}^2 = \norm{u}^2 + \norm{v}^2 = 2$ y por lo tanto $B_{\frac{\sqrt{2}}{2}}(u) \cap B_{\frac{\sqrt{2}}{2}}(v) = \emptyset$.
	
	Sea $D \subseteq E$ un subconjunto denso numerable, luego $B_{\frac{\sqrt{2}}{2}}(u) \cap D \neq \emptyset$ para todo $u \in S$. Consideremos $f:S \rightarrow D$ dado por $f(u) \in B_{\frac{\sqrt{2}}{2}}(u) \cap D$, luego si $f(u) = f(v)$ entonces $f(v) \in B_{\frac{\sqrt{2}}{2}}(u) \cap B_{\frac{\sqrt{2}}{2}}(v) $ y por lo tanto $u = v$. Como $f$ es inyectiva conclu\'imos que $S$ es a lo sumo numerable. \qed
	
\end{proof}

\begin{theorem}
	\label{Proyeccion de un elemento en un ortonormal}
	Sean $H$ un espacio de Hilbert, $u_n$ una sucesi\'on de vectores ortonormales  y $c_n$ una sucesi\'on de numeros complejos. Luego:
	
	\begin{equation}
		\Bigsum{n \in \N}{c_nu_n} \in H \ \Longleftrightarrow \ \Bigsum{n \in \N}{\abs{c_n}^2} < \infty
	\end{equation}
	
	M\'as a\'un, $c_n = \ip{\Bigsum{n \in \N}{c_n u_n}, u_n}$
	
\end{theorem}

\begin{proof}
	Sea $S_k = \sum\limits_{i = 1}^{k}{c_i u_i}$, luego como $(u_n)$ son ortonormales dos a dos y $H$ es completo:
	
	\[
			\norm{\sum\limits_{i = k+ 1}^{k'}{c_n u_n}}^2 = \sum\limits_{i = k+ 1}^{k'}{\abs{c_n}^2}
	\]
	Por ende:
	\begin{equation*}
				\Bigsum{n \in \N}{c_nu_n} \in H \ \Longleftrightarrow \ \Bigsum{n \in \N}{\abs{c_n}^2} < \infty
	\end{equation*}
	
	Finalmente, notemos que $\ip{S_k,u_j} = c_j$ para todo $k \geq j$ y ,adem\'as si $(c_n) _in l^2$, entonces $S_k \rightarrow \Bigsum{n \in \N}{c_n u_n}=:x$; por lo tanto por \ref{Prod interno es continuo} $c_n = \ip{S_k,u_n} \rightarrow \ip{x,u_n}$. 
	\qed
	
\end{proof}

\begin{definition}
	Sea $E$ un espacio vectorial con producto interno y $M \subseteq E$, definimos \textit{el ortogonal a } $M$ como $M^{\perp} = \sett{x \in E \ / \ \ip{x,m} = 0 \ \forall m \in M}$.
\end{definition}

\begin{proposition}
	\label{El ortogonal es cerrado}
	$M^{\perp}$ es un subespacio cerrado de $E$
\end{proposition}

\begin{proof}
	Si $(x_n) \subset M$ es tal que $x_n \rightarrow x$ entonces por \ref{Prod interno es continuo} $0 = \ip{m,x_n} \rightarrow \ip{m,x}$, por lo que $x \in M$. \qed
\end{proof}

\begin{theorem}
	\label{Proyeccion a un conjunto ortonormal arbitrario}
	Sea $H$ un espacio de Hilbert y sea $S \subseteq H$ un conjunto ortonormal, luego:
	
	\begin{enumerate}
		\item Si $x \in H$ entonces $x_S = \Bigsum{u \in S}{\ip{x,u}u}$ esta bien definido
		\item Si $M = \ip{S}$ entonces $x \in M$ si y solo si $x = x_S$. Es m\'as si $x \in H$ entonces $x - x_S \in M^{\perp}$.
	\end{enumerate}
\end{theorem}

\begin{proof}
	\begin{enumerate}
		\item 	Dado $x \in H$, de \ref{Desigualdad de Bessel} sea $(u_n)$ una numeraci\'on de $S = \sett{u \in S \ / \ \ip{x,u}=0}$ y sea $(v_n)$ otra ordenaci\'on de los $u_n$; notemos $x_1 = \Bigsum{n}{\ip{x,u_n}u_n}$ y $x_2 = \Bigsum{n}{\ip{x,u_n}u_n}$ que por \ref{Proyeccion de un elemento en un ortonormal} y \ref{Desigualdad de Bessel} est\'an bien definidos.
	
	Luego:
	
	\[
	\begin{aligned}
		\ip{x_1 - x_2 , u_n} = & \ip{x_1,u_n} - \ip{x_2,u_n} \\
		\underbrace{=}_{u_n = v_{m_n} \text{ para alg\'un } m_n} & \ip{x,u_n} - \ip{x,v_{m_n}} \\
		= & \ip{x,u_n} - \ip{x,u_n}
		= & 0
	\end{aligned}
	\]
	
	Por ende, $\ip{x_1 - x_2,u_n} = \ip{x_1 - x_2,v_n} = 0 $ para todo $n \in \N$ y se concluye que $\ip{x_1 - x_2 , x_1 - x_2} = 0$ por lo que $x_1 = x_2$ y entonces $x_S$ esta bien definido y no depende del orden de la suma.
	
	\item Sea $x_{S_k} = \sum\limits_{i=1}^{k}\ip{x,u_i}u_i \in M$, luego como $M$ es cerrado se tiene que $x_{S_k} \rightarrow x_S \in M$. Ahora sea $s \in S$, entonces:
	
	\[
	\ip{x-x_S,v} = \ip{x,v} - \ip{x_S,v} = \ip{x,v} - \ip{x,v} = 0
	\]
	
	Por lo que $x - x_S \in M^{\perp}$. Finalmente, si $x \in M$ entonces como $x_S \in M$ entonces $x - x_S \in M \cap M^{\perp} = \sett{0}$, luego $x = x_S$. \qed
	
	\end{enumerate}
	
\end{proof}

\subsection{Conjuntos ortonormales completos}

\begin{definition}
	Sea $E$ un espacio vectorial con producto interno y sea $S \subseteq E$ ortonormal, diremos que $S$ es \textit{completo} si $S \subseteq T$ y $T$ es ortonormal, entonces $S = T$.
\end{definition}

\begin{proposition}
	\label{Ortogonal vacio es ser completo}
	Sea $S$ un conjunto ortonormal tal que $S^{\perp} = \sett{0}$, entonces $S$ es completo
\end{proposition}

\begin{proof}
	Sea $T$ ortonormal y sea $v \in T \setminus S$, luego $v \in S^{\perp} = 0$ por lo que $S$ es completo. \qed
\end{proof}

\begin{theorem}
	\label{Conjunto ortonormal completo genera en un Hilbert}
	Sea $E$ un espacio vectorial con producto interno, $S \subseteq E$ ortonormal y sea $M = \ip{S}$, entonces:
	
	\begin{enumerate}
		\item Si $M = E$ entonces $S$ es completo
		\item Si $S$ es completo y $E$ es de Hilbert entonces $M = E$
	\end{enumerate}
	
\end{theorem}

\begin{proof}
	\begin{enumerate}
		\item Si $x \in S^{\perp}$ entonces $x \in M^{\perp}=E^{\perp}=\sett{0}$, por lo tanto $S$ es completo
		\item Sea $x \in E$, luego por \ref{Proyeccion a un conjunto ortonormal arbitrario} $x_S$ esta bien definido y $x - x_S \in M^{\perp}$, luego como $S$ es completo $x - x_S = 0$ y por \ref{Proyeccion a un conjunto ortonormal arbitrario} se tiene que $x \in M$. \qed
	\end{enumerate}
\end{proof}

\begin{corollary}
	\label{Escritura de elemento de un HIlbert en un ortonormal completo}
	Sea $H$ de Hilbert y $S \subseteq H$ un conjunto ortonormal completo, luego si $x \in H$ entonces $x = \Bigsum{u \in S}{\ip{x,u}u}$.
\end{corollary}

\begin{proof}
	Como $H$ es Hilbert y $S$ es completo entonces por \ref{Conjunto ortonormal completo genera en un Hilbert} tenemos que $\ip{S} = H$, luego por \ref{Proyeccion a un conjunto ortonormal arbitrario} si $x \in H$ entonces $x = x_S$. \qed 
\end{proof}

\begin{theorem}[Identidad de Parseval]
	\label{Identidad de Parseval}
	Sea $E$ un espacio vectorial con producto interno y $S \subseteq E$ un conjunto ortonormal tal que para todo $x \in E$ vale:
	
	\begin{equation}
	\label{eq: Identidad de Parseval}
		\norm{x}^2 = \Bigsum{u \in S}{\abs{\ip{x,u}}^2}
	\end{equation}
	
	Luego $S$ es completo. M\'as a\'un si $E$ es Hilbert y $S$ es completo entonces vale \ref{eq: Identidad de Parseval}
\end{theorem}

\begin{proof}
	Sea $x \in E$ tal que $x \in S^{\perp}$, luego por \ref{eq: Identidad de Parseval} $\norm{x} = \Bigsum{u \in S}{\abs{\underbrace{\ip{x,u}}_{=0}}^2} = 0$, luego $x = 0$ y $S$ es completo.
	
	Si $E$ es Hilbert y $S$ es completo entonces por \ref{Escritura de elemento de un HIlbert en un ortonormal completo} y \ref{Desigualdad de Bessel} vale que $x = \Bigsum{n \in \N}{\ip{x,u_n}u_n}$ por lo que:
	
	\begin{equation*}
		\begin{aligned}
			\norm{x}^2 = & \ip{x,x}  \\
			= & \ip{\Bigsum{n \in \N}{\ip{x,u_n}u_n},\Bigsum{n \in \N}{\ip{x,u_n}u_n}} \\
			= & \Bigsum{n \in \N}{\ip{x,u_n}\overline{\ip{x,u_n}}} \\
			= & \Bigsum{n \in \N}{\abs{\ip{x,u_n}}^2} \\
			= & \Bigsum{u \in S}{\abs{\ip{x,u}}^2}
		\end{aligned}
	\end{equation*}
	\qed
\end{proof}

\begin{corollary}
	Sea $H$ Hilbert y $m \in M = \ip{S}$ con $S \subseteq H$ un conjunto ortonormal, luego $\norm{x-m} \geq \norm{x - x_S}$
\end{corollary}

\begin{proof}
	$\norm{x - m}^2 = \norm{\underbrace{x- x_S}_{\in M^{\perp}} + \underbrace{x_S - m}_{\in M}}^2 = \norm{x-x_S}^2 + \norm{x_S - M}^2 \geq \norm{x-x_S}^2$ \qed
\end{proof}

\subsection{Ortogonalizaci\'on de Gram Schmitt}

\begin{theorem}[Ortogonalizaci\'on de Gram-Schmidt]
	\label{Ortogonalizacion de GS}
	Sea $E$ un espacio vectorial con producto interno y sea $D$ un conjunto linealmente independiente no vac\'io, luego si $D$ es a lo sumo numerable existe $S	 \subseteq E$ ortonormal tal que:
	
	\begin{itemize}
		\item $\# S = \# D$
		\item $\ip{D} = \ip{S}$
	\end{itemize}
\end{theorem}

\begin{proof}
	Sea $D = \sett{x_n}$ y definamos:
	
	\[
		\begin{array}{cccccc}
			y_1 = & x_1 & \quad,\quad & u_1 = \dfrac{y_1}{\norm{y_1}} & \quad,\quad & S_1 = \sett{u_1} \\ 
			y_2 = & x_2 - x_{S_1} & \quad,\quad & u_2 = \dfrac{y_2}{\norm{y_2}} & \quad,\quad & S_2 = \sett{u_1,u_2} \\ 
			\vdots & \vdots & \quad,\quad & \vdots & \quad,\quad & \vdots \\ 
			y_n = & x_n - x_{S_{n-1}}  & \quad,\quad & u_n = \dfrac{y_n}{\norm{y_n}} & \quad,\quad & S_n = \sett{u_1, \dots, u_n} \\ 
		\end{array}
	\]
	
	Luego sea $S = \Bigcup{n \in \N}{S_n}$ y es claro verificar ambas propiedades. \qed
	
\end{proof}

\begin{proposition}
	\label{Aproximacion por ortonormal separable en Hilbert separable}
	Sea $E$ un espacio vectorial con producto interno de dimensi\'on infinita, luego si $E$ es separable existe $S \subseteq E$ ortonormal, completo y numerable tal que $\overline{\ip{S}} = E$.
\end{proposition}

\begin{proof}
	Como $E$ es separable existe $D = \sett{x_n} \subseteq E$ denso numerable. Sea $n_1 = \min\sett{n \in \N \ / \ x_n \neq 0}$ y $y_1 = x_{n_1}$ e inductivamente sea $n_k = \min\underbrace{\sett{n \in \N \ / \ x_n \not \in \ip{y_1, \dots, y_{k-1}} \ , \ n_k > n_{k-1}}}_{A_k}$ y $y_k = x_{n_k}$ , luego por la buena ordenaci\'on de $\N$ y el hecho que $D$ es denso y $dim(E)= \infty$ entonces $A_k \neq \emptyset$ para todo $k \in \N$ por lo que $n_k$ esta bien definido. Sea $Y = \Bigcup{k \in \N}{\sett{y_k}}$
	
	\begin{lemma}
		\label{Lemma: Aproximacion por ortonormal separable en Hilbert separable}
		$\ip{D} = \ip{Y}$ e $Y$ es linealmente independiente.
	\end{lemma}
	
	\begin{proof}[Demostraci\'on (del Lema)]
		
		Si $x_{n_0} \in D$ luego si $n_0 = n_k$ para alg\'un $k$ se concluye que $x_{n_0} \in Y$, si no entonces por Arquimedianidad $A = \sett{n \in \N \ / \ n > n_0 \ , \ \exists k \in \N \ , \ n = n_k \ } \neq \emptyset$ y sea $\hat{k} = \min A$; luego $x_{n_0} \in \ip{A_{\hat{k}-1}} \subseteq \ip{Y}$. Se concluye que $D \subseteq \ip{Y}$ y entonces $\ip{D} \subseteq \ip{Y}$.
		
		Rec\'iprocamente sea $y \in Y$, luego por definici\'on de $y$ se tiene que $y = x_{n_k} \in \ip{D}$ para alg\'un $k$ y entonces $Y \subseteq \ip{D}$ por lo que $\ip{Y} = \ip{D}$.
		
		Finalmente por construcci\'on $Y$ es linealmente independiente. \qed  
		
	\end{proof}
	
	Luego por \ref{Ortogonalizacion de GS} existe $S \subseteq E$ ortonormal tal que $\ip{S} = \ip{Y} \underbrace{=}_{\ref{Lemma: Aproximacion por ortonormal separable en Hilbert separable}} \ip{D}$, y como $D$ es denso se tiene que $\overline{\ip{S}} = \overline{\ip{D}} \supseteq \overline{D} = E$. \qed
	
	
\end{proof}

\subsection{Dimensi\'on de un espacio de Hilbert}

\begin{proposition}
	\label{Dos sistemas ortonormales y completos tienen el mismo cardinal}
	Sea $E$ un espacio vectorial con producto interno y sean $S_1,S_2 \subseteq E$ conjuntos ortonormales y completos, luego $\# S_1 = \# S_2$ 
\end{proposition}

\begin{proof}
	Si $S_1$ es finito entonces $S = \sett{u_1, \dots, u_k}$, sea $x \in E$ entonces $x - x_{S_1} = x - \sum\limits_{i=1}^{k}{\ip{x,u_i}u_i} \in S_1^{\perp} = \sett{0}$ por lo que $S_1$ es generador y ortonormal; conclu\'imos que $S_1$ es base y $\dim_E = k$. An\'alogamente $S_2 = \sett{v_1 , \dots , v_j}$ es base y finalmente sea $T \in End(E)$ dada por $T(u_i) = v_i$, luego $T$ es biyectiva y $\# S_1 = \# S_2$ .
	
	Si $S_1$ es infinito entonces para $x \in S_1$ sea $S_2(x) = \sett{u \in S_2 \ / \ \ip{u,x} \neq 0}$, luego por \ref{Desigualdad de Bessel} sabemos que $S_2(x)$ es a lo sumo numerable.
	
	\begin{lemma}
		$\Bigcup{x \in S_1}{S_2(x)} = S_2$
	\end{lemma}
	
	\begin{proof}[Demostraci\'on (del Lema)]
		Supongamos que existe $y \in S_2$ tal que $y \not \in S_2(x)$ para todo $x \in S_1$, luego $y \in S_{1}^{\perp} = \sett{0}$; conclu\'imos que $S_2 \subseteq \Bigcup{x \in S_1}{S_2(x)}$ pues $S_2$ es ortonormal.
		
		Trivialmente se da la otra inclusi\'on. \qed
	\end{proof}
	
	Por lo tanto $\# S_2 \leq \# \left(\N \times S_1\right) = \# S_1$; an\'alogamente $\# S_1 \leq \# S_2$ y se concluye el resultado. \qed
	
\end{proof}

\begin{definition}
	Se define $dim(E) = \# S$ donde $S \subseteq E$ es un sistema ortonormal completo.
\end{definition}

\begin{definition}
	Sean $E$ y $F$ dos espacios vectoriales con producto interno, decimos que son \textit{congruentes} si existe $T \in L(E,F)$ isomorfismo tal que $\norm{T(x)}_{F} = \norm{x}_E$
\end{definition}

\begin{definition}
	Sea $Q \neq \emptyset$, luego definimos $l^2(Q) = \sett{f : Q \rightarrow \R \ / \ \# \sett{q \in Q \ / \ f(q) \neq 0} \leq \aleph_0 \ , \ \Bigsum{q \in Q}{\abs{f(q)}^2} < \infty}$.
\end{definition}

\begin{proposition}
	Valen:
	
	\begin{enumerate}
		\item $l^2(Q)$ es un espacio de Hilbert con producto interno dado por $\ip{f,g} = \Bigsum{q \in Q}{f(q)\overline{g(q)}}$
		\item Sea $S = \sett{\chi_{\sett{q}}}_{q \in Q}$ es ortonormal y completo
		\item Si $\# Q > \# \N$ entonces $l^2(Q)$ no es separable
	\end{enumerate}
	
\end{proposition}

\begin{proposition}
	\label{Existencia de sistema ortonormal completo}
	Todo espacio vectorial con producto interno admite un sistema ortonormal completo.
\end{proposition}

\begin{proof}
	Sea $P = \sett{S \subseteq E \ / \ S \text{ ortonormal}}$ y dotemoslo del orden dado por la inclusi\'ion, luego $P \neq \emptyset$ pues si $v \in E$ entonces $\sett{v} \in P$.
	
	Sea $\sett{S_i}$ una colecci\'on de subconjuntos de $P$ totalmente ordenada y sea $T = \Bigcup{i \in I}{S_i}$, luego es claro que $S_i \leq T$; faltar\'ia ver que $T \in P$.
	
	Para eso sean $v_1,v_2 \in T$, luego existe $S_i$ tal que $v_1,v_2 \in S_i$ y como este es ortonormal resulta que $\ip{v_1,v_2} = 0$ y $\norm{v_1} = \norm{v_2} = 1$. Conclu\'imos que $T \in P$, luego por \ref{Lema de Zorn} existe $M \in P$ elemento maximal.
	
	Finalmente sea $v \in M^{\perp}$, luego $M \cup \sett{\dfrac{v}{\norm{v}}}$ ser\'ia un conjunto ortonormal lo que contradice la maximalidad de $M$; por ende no existe tal $v$ y $M$ resulta completo. \qed
\end{proof}

\begin{theorem}
	\label{HIlbert congruente a un l2}
	Sea $H$ Hilbert tal que $\dim H = \alpha$ entonces $H \cong l^2(Q)$ con $\# Q = \alpha$
\end{theorem}

\begin{proof}
	Sea $S_{\alpha} = \sett{u_i}_{i \in Q}$ un sistema ortonormal, completo de $H$ que existe por \ref{Existencia de sistema ortonormal completo}; luego $x \in H$ entonces $x = \Bigsum{i \in Q}{\ip{x,u_i}u_i}$, y debido a \ref{Identidad de Parseval} y \ref{Desigualdad de Bessel} $\sett{\ip{x,u_i}}_{i \in Q} \subset l^2(Q)$. Definimos $T : H \rightarrow l^2(Q)$ dado por $T(x) = (\ip{x,u_i})_{i \in Q}$ y veamos que es la indicada.
	
	\begin{enumerate}
		\item $T$ es lineal
		
		Sean $x,y \in H$ y $\lambda \in \mathbb{F}$, luego $T(x + \lambda y) = \left(\ip{x+\lambda y , u_i}\right) = \left( \ip{x,u_i} + \lambda \ip{y,u_i} \right) = T(x) + \lambda T(y)$.
		
		\item $T$ es monomorfismo
		
		Si $T(x) = (0)$ luego $\ip{x,u_i} = 0$ para todo $i \in Q$, luego $x \in S^{\perp} = \sett{0}$ pues $S$ es completo.
		
		\item $T$ es epimorfimso
		
		Si $(c_i) \in l^2(Q)$ luego por \ref{Proyeccion de un elemento en un ortonormal} $x = \Bigsum{i \in Q}{c_i u_i} \in H$ y $T(x) = (c_i)$
		
		\item $T$ es isometr\'ia
		
		Por \ref{Identidad de Parseval}
		
	\end{enumerate}
	\qed
\end{proof}


\begin{corollary}
	Sea $H$ Hilbert separable de dimensi\'on infinita, luego $H$ es congruente a $l^2$
\end{corollary}

\subsection{Proyecci\'on ortogonal}

\begin{example}
	El sistema $\sett{\dfrac{e^{int}}{\sqrt{2 \pi}} \ , \ t \in [0,2 \pi]}_{n \in \N}$ es completo.
\end{example}

\begin{proof}
	Supongamos que $\int_{-\pi}^{\pi}{f(t)e^{int} dt} = 0$ para todo $n \in \N$ y sea $g(t) = \int_{-\pi}^{t}{f(t) dt}$, luego $g$ es continua y $g' = f$ ctp por el teorema de diferenciaci\'on de Lebesgue. Notemos que:
	
	\[
	\begin{aligned}
		g(\pi) = & \int_{-\pi}^{\pi}{f(t) dt} = \int_{-\pi}^{\pi}{f(t) e^{i0t}dt} = 0 = g(-\pi) \\
		\int_{-\pi}^{\pi}{g(t) e^{int} dt} = & \dfrac{g(t)e^{int}}{ni} \vert_{-\pi}^{\pi} - \dfrac{\int_{-\pi}^{\pi}{f(t)e^{int}}}{in} = 0
	\end{aligned}	
	\]
	
	Por lo tanto tenemos que $\int_{-\pi}^{\pi}{g(t)e^{int}dt} = 0$ donde $g$ es continua y $g(-\pi) = g(\pi) = 0$, por Stone-Weirstrass existe $(p_n)_{n \in \N}$ sucesi\'on de polinomios trigonom\'etricos tal que $p_n \rightrightarrows g$, por lo tanto:
	
	\begin{equation*}
		\int_{-\pi}^{\pi}{p_k(t)e^{int}} \rightarrow \int_{-\pi}^{\pi}{g(t)e^{int}dt} = 0 \quad n \in \N
	\end{equation*}
	
	No obstante, si $p_k \neq cte$ entonces para todo $k \in \N$ $\ip{p_k,e^{int}} \neq 0$ para alg\'un $n$, luego $p_k = cte = g(\pi) = 0$. Conclu\'imos que $g = 0$ y entonces $f = 0$ ctp. \qed
	
\end{proof}

\begin{theorem}
	\label{Proyeccion unica en un cerrado convexo}
	Sea $H$ Hilbert y $K$ cerrado y convexo, luego si $x \in H$ entonces existe un \'unico $k \in K$ tal que $\norm{x-k} = d(x,K)$
\end{theorem}

\begin{proof}
	Sea $d_n = \norm{x-k_n}$ una sucesi\'on minimizante, luego para todo $n \geq N \in \N$ vale que $d + \frac{1}{N} \geq \norm{x-k_n}$ por lo que por \ref{Ley del paralelogramo}:
	
	\begin{equation*}
		\norm{(x-k_n) - (x-k_m)}^2 + \norm{(x-k_n) + (x-k_m)}^2 =  2 \norm{x-k_n}^2 + 2 \norm{x-k_m}^2
	\end{equation*}
	
	Por lo tanto:
	
	\[
	\begin{aligned}
		\norm{k_n-k_m}^2 = & 2 \norm{x-k_n}^2 + 2 \norm{x-k_m}^2  - \norm{2x-k_n -k_m}^2 \\
		= & 2 \norm{x-k_n}^2 + 2 \norm{x-k_m}^2  - 4\norm{x- \underbrace{\dfrac{k_n -k_m}{2}}_{\in K}}^2 \\
		\leq & 2 \left(d + \frac{1}{n}\right)^2 + 2 \left(d + \frac{1}{m}\right)^2 - 4d^2 \\
		= & \frac{4d}{n} + \frac{2}{n^2} + \frac{4d}{m} + \frac{2}{m^2} \quad  \overrightarrow{n,m \rightarrow \infty} \quad 0
	\end{aligned}
	\]
	
	Luego $(k_n)$ es de Cauchy y como $H$ es completo existe $k \in K$ tal que $k_n \rightarrow K$; por \ref{Prod interno es continuo} $d = \norm{x-k}$. Si $h \in K$ tal que $\norm{x-h} = d$ luego como $K$ es convexo $\frac{k+h}{2} \in K$ por lo que:
	
	\[
	d \leq \norm{x-\frac{k+h}{2}} \leq \dfrac{\norm{x-k} + \norm{x-h}}{2} = d
	\]
	
	Luego por \ref{Ley del paralelogramo}:
	
	\[
		\norm{k-h}^2 = 2 \norm{x-k}^2 + 2 \norm{x-h}^2  - 4\norm{x- \dfrac{k -h}{2}}^2 = 0 \\
	\]
	
	Por lo que $k = h$. \qed
\end{proof}

\begin{definition}
	Sea $M \subseteq H$ un subespacio cerrado de $H$ Hilbert, luego por \ref{Proyeccion unica en un cerrado convexo} existe un \'unico $f_0 \in M$ tal que para todo $x \in H$ vale $\norm{x-f_0} = d(x,M)$. A su vez como $M$ es cerrado tambi\'en es un espacio de Hilbert, luego por \ref{Existencia de sistema ortonormal completo} existe $S \subseteq M$ tal que $M = \ip{S}$, finalmente por \ref{Proyeccion de un elemento en un ortonormal} vale que $f_0 = x_S$.
	
	En resumen, dado $M \subseteq H$ subespacio cerrado y $h \in H$ existe un \'unico elemento $f_0$ tal que $h - f_0 \in M^{\perp}$. Definimos la \textit{proyecci\'on ortogonal sobre } $M$ $P_M:H \rightarrow M$ dado por $P_M(h) = f_0$.
\end{definition}

\begin{proposition}
	\label{Propiedades proyeccion ortogonal}
	Sea $M \subseteq H$ un subespacio cerrado en un Hilbert, sea $h \in H$ y $Ph := P_M(h)$ el \'unico elemento tal que $h - Ph \in M^{\perp}$, luego:
	
	\begin{enumerate}
		\item $P$ es lineal
		\item $\norm{Ph} \leq \norm{h}$
		\item $P^2 = P$
		\item $\ker P = M^{\perp}$ y $ran P = M$
	\end{enumerate}
	
	\begin{proof}
		\begin{enumerate}
			\item Sean $x,y \in H$, $\lambda \in \mathbb{F}$ y $m \in M$; luego $\ip{x + \lambda y - Px + \lambda Py, f} = \ip{x -Px,f} + \lambda \ip{y - Py,f} = 0$. Por unicidad en \ref{Proyeccion unica en un cerrado convexo} vale que $P(x + \lambda y) = Px + \lambda Py$.
		
		\item Notemos que $\norm{h}^2 = \norm{\underbrace{h - Ph}_{\in M^{\perp}} + \underbrace{Ph}_{\in M}}^2 = \norm{h-Ph}^2 + \norm{Ph}^2 \geq \norm{Ph}^2$.
		
		\item Como $P\vert_M = Id_M$ entonces $P(Ph) = Ph$ para todo $h \in H$.
		
		\item Si $Ph = 0$ entonces $h - Ph = h \in M^{\perp}$; rec\'iprocamente si $h \in M^{\perp}$ entonces $h-0 \in M^{\perp}$ por lo que $h \in \ker P$. \qed
	\end{enumerate}
	\end{proof}
	
\end{proposition}

\begin{corollary}
	Sea $M \subseteq H$ un subespacio cerrado en un Hilbert, entonces $(M^{\perp})^{\perp} = M$
\end{corollary}

\begin{proof}
	Primero notemos que:
	
	\begin{lemma}
		$Id - P_M = P_{M^{\perp}}$
	\end{lemma}
	
	\begin{proof}[Demostraci\'on del lema]
		Sea $m \in M^{\perp}$ y $h \in H$, luego $\ip{h - (Id - P_M)(h),m} = \ip{h - h + P_M(h),m} = \ip{P_M(h),m} = 0$, por la unicidad de \ref{Proyeccion unica en un cerrado convexo} vale que $P_{M^{\perp}} = Id - P_M$. \qed
	\end{proof}
	
	Luego por \ref{Propiedades proyeccion ortogonal} vale que $\left(M^{\perp}\right)^{\perp} = \ker P_{M^{\perp}} = \ker (Id - P_M) \underbrace{=}_{0 = h - Ph \Leftrightarrow h = Ph } ran \ P = M$. \qed
	
\end{proof}

\begin{corollary}
	\label{Calculo del doble complemento ortogonal}
	Sea $A \subseteq H$  un conjunto en un Hilbert, luego $(A^{\perp})^{\perp} = \overline{\ip{A}}$
\end{corollary}

\begin{proof}
	Para esto vamos a utilizar dos lemas:
	
	\begin{lemma}
		\label{Lemma1: Calculo del doble complemento ortogonal}
		$\ip{A}^{\perp} = A^{\perp}$
	\end{lemma}
	
	\begin{proof}
		Por un lado si $f \in A^{\perp}$ luego $\ip{f, \sum\limits_{i=1}^{n}{c_i a_i}} = \sum\limits_{i=1}^{n}{c_i\ip{f,a_i}} = 0$ por lo que $f \in \ip{A}^{\perp}$.
		
		Rec\'iprocamente si $f \in \ip{A}^{\perp}$ y sea $a \in A$, luego $\ip{f,\underbrace{a}_{A \subseteq \ip{A}}} = 0$ por lo que $f \in A^{\perp}$. \qed
		
	\end{proof}
	
	\begin{lemma}
		\label{Lemma2: Calculo del doble complemento ortogonal}
		Sea $U \subseteq H$ un conjunto en un Hilbert, entonces $U^{\perp} = \overline{U}^{\perp}$.
	\end{lemma}
	
	\begin{proof}
		Sea $h \in U^{ \perp}$, luego si $u \in \overline{U}$ entonces existe $\sett{u_n}_{n \in \N} \subset U$ tal que $u_n \rightarrow u$. Por \ref{Prod interno es continuo} entonces $0 = \ip{h,u_n} \rightarrow \ip{h,u}$ por lo que $h \in \overline{U}^{\perp}$.
		
		Rec\'iprocamente, si $h \in \overline{U}^{\perp}$ y $u \in U \subseteq \overline{U}$ entonces $\ip{h,u} = 0$; conclu\'imos que $h \in U^{\perp}. \qed$
	\end{proof}
	
	Luego por el corolario previo $\overline{\ip{A}} = \left(\overline{\ip{A}}^{\perp}\right)^{\perp} \underbrace{=}_{\ref{Lemma2: Calculo del doble complemento ortogonal}} \left({\ip{A}}^{\perp}\right)^{\perp} \underbrace{=}_{\ref{Lemma1: Calculo del doble complemento ortogonal}} \left({{A}}^{\perp}\right)^{\perp} $. \qed
	
\end{proof}

\begin{corollary}
	\label{Variedad es densa si el complemento es vacio}
	Sea $M \subseteq H$ una variedad lineal en un Hilbert, luego $M$ es denso si y s\'olo si $M^{\perp} = \sett{0}$
\end{corollary}

\begin{proof}
	Si $\overline{M} = H$ entonces $M^{\perp} \underbrace{=}_{\ref{Lemma2: Calculo del doble complemento ortogonal}} = \overline{M}^{\perp} = H^{\perp} = \sett{0}$.
	
	Rec\'iprocamente de \ref{Calculo del doble complemento ortogonal} sabemos que $\overline{M} = \left(M^{\perp}\right)^{\perp} = \sett{0}^{\perp} = H$. \qed
\end{proof}

\subsection{Teorema de representaci\'on de Riesz}

\begin{proposition}
	\label{Continuidad de un funcional}
	Sea $H$ un espacio de Hilbert y sea $L : H \rightarrow \mathbb{F}$ un funcional lineal, entonces son equivalentes:
	
	\begin{enumerate}
		\item $L$ es continua
		\item $L$ es continua en 0
		\item $L$ es continua en alg\'un punto
		\item Existe $c > 0$ tal que:
		
		\begin{equation}
		\label{eq: Funcional acotado}
			\abs{L(h)} \leq c \norm{h} \quad \forall h \in H
		\end{equation}  
	\end{enumerate}
\end{proposition}

\begin{proof}
	Es claro que $1) \Longrightarrow 2) \Longrightarrow 3)$ y que $4) \Longrightarrow 2)$, veamos las que faltan:
	
	\begin{enumerate}
		\item[$3) \Longrightarrow 1)$]Supongamos que $L$ es continua en $h_0 \in H$ y sea $h \in H$; luego si $h_n \rightarrow h$ entonces $h_n - h + h_0 \rightarrow h_0$, por lo tanto $L(h_0) = \lim L(h_n - h + h_0) = \lim L(h_n) - L(h) + L(h_0)$ y conclu\'imos que $L(h) = \lim L(h_n)$.
		
		\item[$ 2) \Longrightarrow 4) $] Como $L$ es continua en $0$ entonces si $V = \sett{\alpha \in \mathbb{F} \ / \ \abs{\alpha} < 1}$ entonces $L^{-1}(V)$ es abierto; es decir existe $\delta > 0$ tal que $\norm{h} < \delta$ implica $\abs{L(h)} < 1$.
		
		Si $h \in H$ y $\epsilon > 0$ entonces $\norm{\dfrac{\delta h}{\norm{h} + \epsilon}} < \delta$ por lo que:
		
		\[
			1 > \abs{L\left[\dfrac{\delta h}{\norm{h} + \epsilon}\right]} = \dfrac{\delta}{\norm{h} + \epsilon} \abs{L(h)}
		\]
		
		Por lo que si $\epsilon \rightarrow 0$:
		
		\[
			\abs{L(h)} < \dfrac{1}{\delta} \left(\norm{h}\right) := c \norm{h}
		\]
		\qed
	\end{enumerate}
\end{proof}

\begin{definition}
	Decimos que $L : H \rightarrow \mathbb{F}$ es \textit{acotado} si vale \ref{eq: Funcional acotado}. De \ref{Continuidad de un funcional} vemos que un funcional es acotado si y s\'olo si es continuo.
	
	En ese caso definimos:
	
	\begin{equation*}
		\norm{L} = \sup \sett{\abs{L(h)}: \ \norm{h} \leq 1}
	\end{equation*}
	
\end{definition}

\begin{proposition}
	Si $L$ es un funcional acotado entonces:
	\begin{equation}
	\label{eq: Normas equivalentes de un funcional}
	\begin{aligned}
		\norm{L} := & \sup \sett{\abs{L(h)}: \ \norm{h} \leq 1} \\
		= & \sup \sett{\abs{L(h)}: \ \norm{h} = 1}\\
		= & \sup \sett{\dfrac{\abs{L(h)}}{\norm{h}}: \ h \neq 0}  \\
		= & \inf \sett{c > 0: \ \abs{L(h)} \leq c \norm{h} \ h \in H} 
	\end{aligned}	
	\end{equation}
	
	Es m\'as, vale que $\abs{L(h)} \leq \norm{L} \norm{h}$ para todo $h \in H$.
	
\end{proposition}

\begin{proof}
	Notemos(solo por esta demostraci\'on):
	
	\[
	\begin{array}{ccc}
	 \norm{L}_2 & = & \sup \sett{\abs{L(h)}: \ \norm{h} = 1} \\
	 \norm{L}_3 & = & \sup \sett{\dfrac{\abs{L(h)}}{\norm{h}}: \ h \neq 0} \\
	 \norm{L}_4 & =  & \inf \sett{c > 0: \ \abs{L(h)} \leq c \norm{h} \ h \in H}
	\end{array}
	\]
	
	Vamos por partes, 
	
	\begin{itemize}
		\item Primero como $\sett{\abs{L(h)}: \ \norm{h} = 1} \subseteq \sett{\abs{L(h)}: \ \norm{h} \leq 1}$ entonces vale que $\norm{L}_2 \leq \norm{L}$. 
		
		Rec\'iprocamente, si $\norm{h} \leq 1$ entonces:
		
		\begin{equation*}
		\begin{array}{cc}
		& \abs{L\left(\dfrac{h}{\norm{h}}\right)} \leq \norm{L}_2 \\
		\Longrightarrow &	\dfrac{1}{\norm{h}}\abs{L(h)} \leq \norm{L}_2 \\
		\Longrightarrow &	\abs{L(h)} \leq \norm{L}_2 \norm{h} \leq \norm{L}_2 \\
		\Longrightarrow &	\sup\limits_{\norm{h} \leq 1} \abs{L(h)} \leq \norm{L}_2 \\
		\Longrightarrow &	\norm{L} \leq \norm{L}_2
		\end{array}
		\end{equation*}
		
		\item Si $h \neq 0$ entonces:
		
		\begin{equation*}
		\begin{array}{cc}
		& \abs{L\left(\dfrac{h}{\norm{h}}\right)}  \leq \norm{L}_2 \\
		\Longrightarrow &	\dfrac{1}{\norm{h}}\abs{L(h)} \leq \norm{L}_2 \\
		\Longrightarrow &	\sup\limits_{h \neq 0} \sett{\dfrac{1}{\norm{h}}\abs{L(h)}} \leq \norm{L}_2 \\
		\Longrightarrow &	\norm{L}_3 \leq \norm{L}_2
		\end{array}
		\end{equation*}
		
		Rec\'iprocamente notemos que $\sett{\abs{L(h)}: \ \norm{h} = 1} = \sett{\dfrac{\abs{L(h)}}{\norm{h}}: \ \norm{h} = 1} \subseteq \sett{\dfrac{\abs{L(h)}}{\norm{h}}: \ h \neq 0}$ por lo tanto $\norm{L}_2 \leq \norm{L}_3$.
		
		\item Sea $\epsilon > 0$, luego:
		
		\begin{equation*}
		\begin{array}{cc}
		& \abs{L\left(\dfrac{h}{\norm{h} + \epsilon}\right)} \leq \norm{L} \\
		\Longrightarrow  & \abs{L(h)} \leq \left(\norm{h} + \epsilon\right) \norm{L} \\
		\text{Si } \epsilon \rightarrow 0 \Longrightarrow  & \abs{L(h)} \leq \norm{L} \norm{h} \\
		\Longrightarrow  & \norm{L}_4 \leq \norm{L}
		\end{array}
		\end{equation*}
		
		Rec\'iprocamente, si $\norm{L(h)} \leq c \norm{h}$ entonces $\norm{L} \leq c$ por lo que $\norm{L} \leq \norm{L}_4$. \qed
		
	\end{itemize}
\end{proof}

\begin{theorem}[Teorema de Representaci\'on de Riesz]
	\label{Teorema de representacion de Riesz}
	Sea $L : H \rightarrow \mathbb{F}$ un funcional, entonces $L$ es acotado si y s\'olo si existe un \'unico $h_0 \in H$ tal que $L(h) = \ip{h,h_0}$. En ese caso $\norm{L} = \norm{h_0}$.
\end{theorem}

\begin{proof}
	Sea $M = \ker L$, como $L$ es acotada entonces $M$ es cerrado y como $L \neq 0$ (en cuyo caso $h_0 = 0$) entonces $M^{\perp} \neq \sett{0}$. Como $H = M \oplus M^{\perp}$ entonces existe $f_0 \in M^{\perp}$ tal que $L(f_0) = 1$.
	
	Sea $h \in H$, entonces $L(h - L(h)f_0) = 0$ por lo que $h - L(h)f_0 \in M$; de aqu\'i conclu\'imos:
	
	\begin{equation*}
	\begin{array}{cccc}
	& 0 & = & \ip{h - L(h)f_0, f_0} \\
	\Longrightarrow & 0 & = & \ip{h,f_0} - L(h)\norm{f_0}^2 \\
	\Longrightarrow & L(h) & = & \dfrac{1}{\norm{f_0}^2}\ip{h,f_0} \\
	\Longrightarrow & L(h) & \underbrace{=}_{h_0 = \dfrac{f_0}{\norm{f_0}^2}} & \ip{h,h_0} 
	\end{array}
	\end{equation*}
	
	Si $h_0'$ es tal que $\ip{h,h_0} = L(h) = \ip{h,h_0'}$ entonces $0 = \ip{h,h_0 - h_0'}$ para todo $h \in H$, en particular $0 = \ip{h_0 - h_0',h_0 - h_0'} = \norm{h_0 - h_0'}^2$ por lo que $h_0 = h_0'$.
	
	Rec\'iprocamente, si $L(h) = \ip{h,h_0}$ entonces por \ref{Desigualdad de Cauchy-Schwartz } $\abs{L(h)} \leq \norm{h} \norm{h_0}$ por lo tanto $\norm{L} \leq \norm{h_0}$.
	
	En ese caso, $L\left(\dfrac{h_0}{\norm{h_0}}\right) = \dfrac{1}{\norm{h_0}}\ip{h_0,h_0} = \norm{h_0}$ por lo que $\norm{L} = \norm{h_0}$. \qed
	
\end{proof}

\section{Espacios de Banach}

\subsection{Operadores entre espacios normados}

\begin{proposition}
	\label{La norma es continua}
	Sea $E$ un espacio normado, entonces:
	
	\begin{enumerate}
		\item La suma es continua
		\item El producto por un escalar es continuo
		\item La norma es continua
	\end{enumerate}
\end{proposition}

\begin{proof}
	\begin{enumerate}
		\item Si $x_n \rightarrow x$ y $y_n \rightarrow y$ entonces $\norm{x+y - x_n - y_n} \leq \norm{x_n -x} + \norm{y_n -y} \rightarrow 0$
		\item Si $x_n \rightarrow x$ y $\lambda \in \mathbb{F}$ entonces $\norm{\lambda x_n - \lambda x} = \abs{\lambda} \norm{x_n - x} \rightarrow 0$.
		\item Si $x_n \rightarrow x$ entonces por definici\'on $\norm{x_n -x} \rightarrow 0$. \qed
	\end{enumerate}
\end{proof}

\begin{proposition}
	Sea $E$ un espacio normado y $x_0 \in E$ entonces $\overline{B_r(x_0)} = B_r[x_0]$.
\end{proposition}

\begin{proof}
	Si $x \in \overline{B_r(x_0)}$ entonces existe $\sett{x_n} \subset B_r(x_0)$ tal que $x_n \rightarrow x$, como $\norm{x_n - x_0} < r$ entonces por \ref{La norma es continua} se tiene que $\norm{x_n - x_0} \rightarrow \norm{x - x_0}$ por lo que $x \in B_r[x_0]$.
	
	Rec\'iprocamente si $x \not \in \overline{B_r(x_0)}$ entonces existe $\epsilon > 0$ tal que $B_{\epsilon}(x) \cap B_r(x_0) = \emptyset$; luego $\norm{x - x_0} > \epsilon + r > r$ por lo que $x \not \in B_{r}[x_0]$. \qed
\end{proof}

\begin{theorem}
	\label{Banach si y solo si abs convergente es convergente}
	Sea $X$ un espacio normado, entonces $X$ es de Banach si y s\'olo si vale:
	
	\begin{equation}
		\label{eq: Condicion de Banach}
		\text{Si } \left(x_n\right) \text{ cumple que } \Bigsum{n \in \N}{\norm{x_n}} < \infty \ \Longrightarrow \ \Bigsum{n \in \N}{x_n} \in X
	\end{equation}
	
\end{theorem}

\begin{proof}
	Sea $S_k = \Bigsum{n \leq k}{x_n}$, entonces si $k > k'$, $\norm{S_{k} - S_{k'}} = \norm{\sum\limits_{n = k'+1}^{k}{x_n}} \leq \sum\limits_{n = k'+1}^{k}{\norm{x_n}} \ \xrightarrow{k,k' \rightarrow \infty} \ 0$. Luego $S_k$ es de Cauchy y como $X$ es Banach $S_k \rightarrow \Bigsum{n \in \N}{x_n} \in X$.
	
	Rec\'iprocamente, sea $\left(x_n\right) \subset X$ de Cauchy y para cada $k \in \N$ sea $\epsilon = \frac{1}{2^k}$ y $n_k \in \N$ tal que $\norm{x_{n} - x_m} < \frac{1}{2^k}$ si $n,m \geq n_k$. Luego si $z_k = x_{n_{k+1}} - x_{n_k}$ entonces $\Bigsum{k}{\norm{z_k}} < \Bigsum{k}{\frac{1}{2^k}} < \infty$; luego por hip\'otesis $S_m = \sum\limits_{k=1}^{m}{z_k}$ converge, pero $S_m = x_{n_{m+1}} - x_{n_1}$, luego $\lim\limits_{m} {x_{n_m}} = x_{n_1} + \lim S_m \in X$; como $x_n$ es de Cauchy y tiene una subsucesi\'on convergente entonces $(x_n)$ es convergente. \qed
	
\end{proof}

\begin{definition}
	Si $X,Y$ son espacios normados un \textit{isomorfismo topol\'ogico} es $T : X \rightarrow Y$ tal que:
	
	\begin{itemize}
		\item $T$ es isomorfismo lineal
		\item $T$ y $T^{-1}$ son continuas
	\end{itemize}
	
\end{definition}

\begin{proposition}
	\label{Continuidad de un operador}
	Seax $X,Y$ espacios normados y sea $T : X \rightarrow Y$ un operador lineal, entonces son equivalentes:
	
	\begin{enumerate}
		\item $T$ es continua
		\item $T$ es continua en 0
		\item $T$ es continua en alg\'un punto
		\item Existe $c > 0$ tal que:
		
		\begin{equation}
		\label{eq: Operador acotado}
		\norm{T(x)}_Y \leq c \norm{x}_X \quad \forall x \in X
		\end{equation}
		\item $T$ est\'a acotado en $B_1[0]$
		\item $T$ est\'a acotado en $B_r[x_0]$ para todos $x_0 \in X$ y $r > 0$
		\item $T$ est\'a acotado en $\partial B_r[x_0]$ para todos $x_0 \in X$ y $r > 0$
	\end{enumerate}
\end{proposition}

\begin{proof}
	Es claro que $1) \Longrightarrow 2) \Longrightarrow 3)$, que $4) \Longrightarrow 2)$ y que $6) \Longrightarrow 7)$, veamos las que faltan:
	
	\begin{enumerate}
		\item[$3) \Longrightarrow 1)$] Supongamos que $T$ es continua en $x_0 \in X$ y sea $x \in X$; luego si $x_n \rightarrow x$ entonces $x_n - x + x_0 \rightarrow x_0$, por lo tanto $T(x_0) = \lim T(x_n - x + x_0) = \lim T(x_n) - T(x) + T(x_0)$ y conclu\'imos que $T(x) = \lim T(x_n)$.
		
		\item[$ 2) \Longrightarrow 4) $] Como $T$ es continua en $0$ entonces si $V = \sett{y \in Y \ / \ \norm{y}_Y < 1}$ entonces $T^{-1}(V)$ es abierto; es decir existe $\delta > 0$ tal que $\norm{x}_X < \delta$ implica $\norm{T(x)}_Y < 1$.
		
		Si $x \in X$ y $\epsilon > 0$ entonces $\norm{\dfrac{\delta x}{\norm{x}_X + \epsilon}}_X < \delta$ por lo que:
		
		\[
		1 > \norm{T\left[\dfrac{\delta x}{\norm{x}_X + \epsilon}\right]}_Y = \dfrac{\delta}{\norm{x}_X + \epsilon} \norm{T(x)}_Y
		\]
		
		Por lo que si $\epsilon \rightarrow 0$:
		
		\[
		\norm{T(x)}_Y < \dfrac{1}{\delta} \left(\norm{x}_X\right) := c \norm{x}_X
		\]
		
		\item[$ 4) \Longrightarrow 5) $] Sea $x \in B_1[0]$, luego $\norm{T(x)}_Y \leq c \norm{x}_X \leq c$.
		
		\item[$ 5) \Longrightarrow 6) $] Sea $r > 0$ y $x_0 \in X$, luego si $x \in B_r[x_0]$ entonces existe $M > 0$ tal que $\norm{T\left(\dfrac{x- x_0}{r}\right)}_Y \leq M$ pues $\dfrac{x- x_0}{r} \in B_1[0]$
		
		Por lo tanto $\norm{T(x) - T(x_0)}_Y \leq Mr$ lo que implica que $\norm{T(x)}_Y \leq rM + \norm{T(x_0)}_Y := C$.
		
		\item[$ 7) \Longrightarrow 1) $] Sea $x_0 \in X$, luego por hip\'otesis si $\norm{x - x_0}_X = 1$ entonces $\norm{T(x - x_0)}_Y \leq C$; por lo tanto:
		
		\begin{equation*}
		\begin{array}{cc}
			& \norm{T \left(\dfrac{x- x_0}{\norm{x - x_0}_X}\right)}_Y \leq C \\
			\Longrightarrow & \norm{T(x) - T(x_0)}_Y \leq C\norm{x- x_0}_X
		\end{array}
		\end{equation*}
		
		Cuando $\norm{x - x_0}_X < \delta = \frac{\epsilon}{C}$.
		\qed
	\end{enumerate}
\end{proof}

\begin{example}
	Si $X = Y = C[a,b]$ dotados de la norma supremo entonces $T(f)(x) = \int_{a}^{x}{f(t)dt}$ es un operador lineal acotado que no es un isomorfismo topol\'ogico.
\end{example}

\begin{corollary}
	Sean $X,Y$ normados y sea $T:X \rightarrow Y$ un isomorfismo lineal. Entonces $T$ es isomorfismo topol\'ogico si y s\'olo si existen $C_1,C_2 > 0$ tal que $C_1 \norm{x} \underbrace{\leq}_{\star} \norm{T(x)} \underbrace{\leq}_{\ast} C_2\norm{x}$
\end{corollary}

\begin{proof}
	Si $T$ es isomorfimso topol\'ogico entonces:
	
	\begin{equation*}
	\begin{array}{ccc}
	T \text{ continua } & \Longrightarrow & \exists C_2 > 0 \ / \ \norm{T(x)} \leq C_2 \norm{x} \quad \forall x \in X \\
	T^{-1} \text{ continua } & \Longrightarrow & \exists D_1 > 0 \ / \ \norm{T^{-1}(y)} \leq D_1 \norm{y} \quad \forall y \in Y \\
	 & \Longrightarrow & \norm{x} \leq D_1 \norm{T(x)} \quad \forall x \in X \\
	 & \Longrightarrow & C_1\norm{x} \leq \norm{T(x)} \quad \forall x \in X 
	\end{array}
	\end{equation*}
	
	Por lo tanto vale que:
	
	\[
		C_1 \norm{x} \leq \norm{T(x)} \leq C_2\norm{x}
	\]
	
	Rec\'iprocamente, por $\ast$ se concluye que $T$ es acotado y por \ref{Continuidad de un operador} es continua; asimismo de $\star$ si $x = T^{-1}(y)$ se ve que $T^{-1}$ es continua.
	
\end{proof}

\subsection{Espacios vectoriales de dimensi\'on finita}

\begin{definition}
	Si $\norm{.}_1, \norm{.}_2$ son dos normas en un espacio vectorial $X$ entonces decimos que son \textit{equivalentes} si $1_X : (X,\norm{.}_1) \rightarrow (X,\norm{.}_2)$ es un isomorfismo topol\'ogico.
\end{definition}

\begin{theorem}
	\label{En dimension finita las normas son equivalentes}
	Sea $X$ un espacio vectorial de dimensi\'on finita, entonces:
	
	\begin{enumerate}
		\item Dos normas siempre son equivalentes
		\item $X$ es topol\'ogicamente isomorfo a $\R^n$ con $n = \dim X$
	\end{enumerate}
	
\end{theorem}

\begin{proof}
	\begin{enumerate}
		\item Sea $\norm{.}$ una norma en $X$ y veamos que $\norm{.}$ y $\norm{.}_{\infty}$ son equivalentes.
		
		Sea $a = \sum\limits_{i=1}^{k}{a_i e_i}$, luego $\norm{a} \leq \sum\limits_{i=1}^{k}{\abs{a_i}\norm{e_i}} \leq \norm{a}_{\infty}C$.
		
		Luego sea $id: (X,\norm{.}_{\infty}) \rightarrow (X,\norm{.})$
		
		Sabemos que $B_{1}[0]$ es compacta en $(X,\norm{.}_{\infty})$ y por la cuenta anterior $id$ es continua, por lo tanto $id(S) = S$ es compacta en $(X,\norm{.})$ y por ende alcanza m\'inimo y m\'aximo. 
		
		Sean $C_1 = \min\limits_{\norm{x}_{\infty} = 1} \norm{x}$ y $C_2 = \max\limits_{\norm{x}_{\infty} = 1} \norm{x}$, por lo tanto si $x \in X$ entonces:
		
		\[
		C_1 \leq \norm{\dfrac{x}{\norm{x}_{\infty}}} \leq C_2
		\]
		
		\item Si $x = \sum\limits_{i=1}^{k}{a_i e_i}$ definimos $T(x) = (a_1 , \dots, a_n)$, luego:
		
		$$C_1 \norm{x} \leq \norm{T(x)}_{\infty} = \norm{x}_{\infty} \leq C_2 \norm{x}$$
		
		Por lo que $T$ es isomorfismo topol\'ogico. \qed
	\end{enumerate}
\end{proof}

\begin{corollary}
	Todo espacio vectorial de dimensi\'on finita es Banach.
\end{corollary}

\begin{corollary}
	Si $X$ es normado de dimensi\'on finita, entonces todo subconjunto cerrado y acotado es compacto.
\end{corollary}

\begin{proof}
	Si $A\subseteq X$ es cerrado y acotado, entonces existe $x_0 \in X, r > 0$ tal que $A \subset B_{r}[x_0]$ y $B_{r}[x_0]$ es compacto pues $B_1[0]$ lo es. Por lo tanto $A$ es un cerrado en un compacto. \qed
\end{proof}

\begin{theorem}
	Si $X$ es un espacio normado de dimensi\'on infinita, entonces $B_1[0]$ no es compacta
\end{theorem}

\begin{proof}
	Veamos primero el siguiente lema:
	
	\begin{lemma}[Lema de Riesz]
		\label{Lema de Riesz}
		Sea $M \subseteq X$ un subespacio no denso en un Banach, dado $r \in (0,1)$ existe $x \in X$ tal que $\norm{x} = 1$ pero $d(x,M) \geq r$ 
	\end{lemma}
	
	\begin{proof}[Demostraci\'on del lema]
		Sea $y \in X \setminus \overline{M}$ y notemos$R = d(y,M)$, luego si $\epsilon > 0$ existe $m_1 \in M$ tal que $\norm{m_1 - y} < R + \epsilon$. Sea $x = \dfrac{y - m_1}{\norm{y - m_1}}$, luego $\norm{x}=1$ y:
		
		\[
		\begin{aligned}
		d(x,M) = & \inf\limits_{m \in M} \norm{x - m} \\ 
		= & \inf\limits_{m \in M} \norm{m - \dfrac{y}{\norm{y - m_1}} + \dfrac{m_1}{\norm{y - m_1}}} \\
		= & \dfrac{\inf\limits_{m \in M} \norm{m - y} }{\norm{m_1 - y}} \\
		= & \dfrac{R}{R + \epsilon} \nearrow 1
		\end{aligned}
		\]
		\qed
		
	\end{proof}
	
	Sea $x_1 \in \partial B_1[0]$, luego por \ref{Lema de Riesz} aplicado a $S_1 = \ip{x_1}$ existe $x_2 \in \partial B_1[0]$ tal que $\norm{x_1 - x_2} > \frac{1}{2}$.
	
	Inductivamente sea $x_n \in \partial B_1[0]$ tal que $d(x_n, S_{n-1}) = d(x_n , \sett{x_1 , x_2 , \dots , x_{n-1}}) > \frac{1}{2}$. Luego por construcci\'on $\sett{x_n}_{n \in \N} \subset B_1[0]$ es una sucesi\'on tal que $\norm{x_n - x_m} > \frac{1}{2}$ para todos $n \neq m$ por lo tanto es una sucesi\'on acotada que no admite subsucesi\'on convergente. Conclu\'imos que $B_1[0]$ no es compacto \qed
	
\end{proof}

\subsection{Espacio de Operadores entre espacios normados}

\begin{definition}
	Dados $X,Y$ normados decimos que $T : X \rightarrow Y$ es \textit{acotado} si vale \ref{eq: Operador acotado}. De \ref{Continuidad de un operador} vemos que un funcional es acotado si y s\'olo si es continuo.
	
	En ese caso definimos:
	
	\begin{equation*}
	\norm{T} = \sup \sett{\norm{T(x)}: \ \norm{x} \leq 1}
	\end{equation*}
	
\end{definition}

\begin{proposition}
	\label{Normas equivalentes de un operador}
	Si $T$ es un operador acotado entonces:
	\begin{equation}
	\label{eq: Normas equivalentes de un operador}
	\begin{aligned}
	\norm{T} := & \sup \sett{\norm{T(x)}: \ \norm{x} \leq 1} \\
	= & \sup \sett{\norm{T(x)}: \ \norm{x} = 1}\\
	= & \sup \sett{\dfrac{\norm{T(x)}}{\norm{x}}: \ x \neq 0}  \\
	= & \inf \sett{c > 0: \ \norm{T(x)} \leq c \norm{x} \ x \in X} 
	\end{aligned}	
	\end{equation}
	
	Es m\'as, vale que $\norm{T(x)} \leq \norm{T} \norm{x}$ para todo $x \in X$.
	
\end{proposition}

\begin{proof}
	Notemos(solo por esta demostraci\'on):
	
	\[
	\begin{array}{ccc}
	\norm{T}_2 & = & \sup \sett{\abs{T(x)}: \ \norm{x} = 1} \\
	\norm{T}_3 & = & \sup \sett{\dfrac{\abs{T(x)}}{\norm{x}}: \ x \neq 0} \\
	\norm{T}_4 & =  & \inf \sett{c > 0: \ \abs{T(x)} \leq c \norm{x} \ x \in X}
	\end{array}
	\]
	
	Vamos por partes, 
	
	\begin{itemize}
		\item Primero como $\sett{\abs{T(x)}: \ \norm{x} = 1} \subseteq \sett{\abs{T(x)}: \ \norm{x} \leq 1}$ entonces vale que $\norm{T}_2 \leq \norm{T}$. 
		
		Rec\'iprocamente, si $\norm{x} \leq 1$ entonces:
		
		\begin{equation*}
		\begin{array}{cc}
		& \abs{T\left(\dfrac{x}{\norm{x}}\right)} \leq \norm{T}_2 \\
		\Longrightarrow &	\dfrac{1}{\norm{x}}\abs{T(x)} \leq \norm{T}_2 \\
		\Longrightarrow &	\abs{T(x)} \leq \norm{T}_2 \norm{x} \leq \norm{T}_2 \\
		\Longrightarrow &	\sup\limits_{\norm{x} \leq 1} \abs{T(x)} \leq \norm{T}_2 \\
		\Longrightarrow &	\norm{T} \leq \norm{T}_2
		\end{array}
		\end{equation*}
		
		\item Si $x \neq 0$ entonces:
		
		\begin{equation*}
		\begin{array}{cc}
		& \abs{T\left(\dfrac{x}{\norm{x}}\right)}  \leq \norm{T}_2 \\
		\Longrightarrow &	\dfrac{1}{\norm{x}}\abs{T(x)} \leq \norm{T}_2 \\
		\Longrightarrow &	\sup\limits_{x \neq 0} \sett{\dfrac{1}{\norm{x}}\abs{T(x)}} \leq \norm{T}_2 \\
		\Longrightarrow &	\norm{T}_3 \leq \norm{T}_2
		\end{array}
		\end{equation*}
		
		Rec\'iprocamente notemos que $\sett{\abs{T(x)}: \ \norm{x} = 1} = \sett{\dfrac{\abs{T(x)}}{\norm{x}}: \ \norm{x} = 1} \subseteq \sett{\dfrac{\abs{T(x)}}{\norm{x}}: \ x \neq 0}$ por lo tanto $\norm{T}_2 \leq \norm{T}_3$.
		
		\item Sea $\epsilon > 0$, luego:
		
		\begin{equation*}
		\begin{array}{cc}
		& \abs{T\left(\dfrac{x}{\norm{x} + \epsilon}\right)} \leq \norm{T} \\
		\Longrightarrow  & \abs{T(x)} \leq \left(\norm{x} + \epsilon\right) \norm{T} \\
		\text{Si } \epsilon \rightarrow 0 \longrightarrow  & \abs{T(x)} \leq \norm{T} \norm{x} \\
		\longrightarrow  & \norm{T}_4 \leq \norm{T}
		\end{array}
		\end{equation*}
		
		Rec\'iprocamente, si $\norm{T(x)} \leq c \norm{x}$ entonces $\norm{T} \leq c$ por lo que $\norm{T} \leq \norm{T}_4$. \qed
		
	\end{itemize}
\end{proof}

\begin{definition}
	Sean $X,Y$ normados, definimos $L(X,Y) = \sett{T:X \rightarrow Y  \ / \ T \text{lineal y acotado}}$
\end{definition}

\begin{proposition}
	Si $X,Y$ son normados entonces $L(X,Y)$ es normado
\end{proposition}

\begin{proof}
	Probemos al desigualdad triangular pues las dem\'as son triviales:
	
	Sean $T,W : X \rightarrow Y$ lineales y acotadas, entonces $\norm{T + W} = \sup\limits_{\norm{x} \leq 1}{\norm{(T + W)(x)}} = \sup\limits_{\norm{x} \leq 1}{\norm{Tx + Wx}} \leq \sup\limits_{\norm{x} \leq 1}{\norm{Tx} + \norm{Wx}} \leq \sup\limits_{\norm{x} \leq 1}{\norm{T(x)}} + \sup\limits_{\norm{x} \leq 1}{\norm{W(x)}} = \norm{T} + \norm{W}$. \qed
	
\end{proof}

\begin{theorem}
	\label{Espacio de operadores es banach si lo es el codominio}
	Sean $X,Y$ normados, entonces $Y$ es de Banach si y s\'olo si $L(X,Y)$ es de Banach
\end{theorem}

\begin{proof}
	Sea $(T_n)_{n \in \N} \subset L(X,Y)$ una sucesi\'on de Cauchy, y sea $\epsilon > 0$ entonces existe $N \in \N$ tal que $\norm{T_n - T_m} < \epsilon$ para todos $n,m \geq N$.
	
	En particular dado $x \in B_{1}[0]$ vale que ${\norm{T_n(x) - T_m(x)}} \leq  \sup\limits_{\norm{x} \leq 1} {\norm{T_n(x) - T_m(x)}} = \norm{T_n - T_m} < \epsilon$ por lo que $\left(T_n(x)\right)_{n \in \N} \subset Y$ es una sucesi\'on de Cauchy; como $Y$ es Banach $\lim T_n(x) \in Y$. Adem\'as si $\norm{x} \geq 1$ entonces $\lim T_n(x) = \lim \norm{x}T_n\left(\dfrac{x}{\norm{x}}\right) = \norm{x} \lim T_n \left(\dfrac{x}{\norm{x}}\right) \in Y$; luego definimos:
	
	\[
	T(x) = \lim T_n(x) \quad \forall x \in X
	\]
	
	Veamos que $T_n \rightarrow T$ y que $T \in L(X,Y)$.
	
	\begin{itemize}
		
		\item Por \ref{La norma es continua} y la linealidad de $T_n$ vale que $T$ es lineal
		\item Sea $x \in B_1[0]$ y $\epsilon > 0$, luego sea $N \in \N$ tal que $\norm{T_n - T_m} < \epsilon$ para todos $n,m \geq N$; entonces $\norm{T(x)} = \norm{T(x) - T_N(x) + T_N(x)} \leq \norm{T(x) - T_N(x)} + \norm{T_N(x)} < \epsilon + C$.
		
		\item Sea $\epsilon > 0$, luego $\epsilon > \sup\limits_{\norm{x} \leq 1} {\norm{T_n(x) - T_m(x)}} \xrightarrow{m \rightarrow \infty} \sup\limits_{\norm{x} \leq 1} {\norm{T_n(x) - T(x)}} = \norm{T_n - T}$. 
	\end{itemize}
	
	La vuelta la probaremos con Hanh-Banach. \qed
	
\end{proof}

\begin{definition}
	Sea $X$ espacio normado, luego notamos $X^{\star} := L(X, \mathbb{F})$ y se llama \textit{espacio dual topol\'ogico}.
	
	Si pensamos a $X$ como espacio vectorial solamente tambi\'en esta definido $X^{'} := \sett{T:X \rightarrow \mathbb{F}, \ / \ T \text{ lineal}}$ el \textit{ dual algebraico}.

\end{definition}

\subsection{Espacios cocientes}

Sea $X$ un espacio vectorial normado y $S \subseteq X$ un subespacio cerrado. Definimos la siguiente relaci\'on de equivalencia en $X$:

\[
x \sim_S y \Longleftrightarrow \ x-y \in S
\]

y definimos $\norm{[x]}_S := \inf \sett{\norm{t} \ : \ x \in [t]}$.


\begin{proposition}
	El espacio $\left(\quotient{X}{S}, \norm{.}_S \right)$ es un espacio normado con la suma definida por $[x] + [y] = [x+y], [\lambda x] = \lambda.[x]$
\end{proposition}

\begin{proof}
	\begin{itemize}
		\item Sean $x,x' \in X$ tal que $[x] = [x']$, entonces $x - x' \in S$ por lo que $\lambda(x-x') \in S$; en conclusi\'on $\lambda . [x] := [\lambda x] = [\lambda x'] =: \lambda . [x']$.
		\item Sean $x,x'y,y' \in X$ tal que $[x] = [x']$, $[y] = [y']$, luego $x-x' \in S$ y $y-y' \in S$ por lo que $(x-x') + (y-y') = (x+y) - (x'+y') \in S$; en conclusi\'on $[x] + [y] := [x+y] = [x' + y'] =: [x'] + [y']$.
		\item Sean $[x] \in \quotient{X}{S}$, $\lambda \in \mathbb{F}$, luego $\norm{\lambda [x]} = \norm{[\lambda x]} = \inf\limits_{t \in [\lambda x]}{\norm{t}} = \inf\limits_{t \in [x]}{\norm{\lambda t}} = \abs{\lambda} \norm{[x]}$
		\item Sean $[x],[y] \in \quotient{X}{S}$, luego $\norm{[x] + [y]} = \norm{[x+y]} = \inf\limits_{t \in [x + y]}{\norm{t}} \leq \inf\limits_{t \in [x] \\ w \in [y]}{\norm{t+w}} \leq \norm{[x]} + \norm{[y]}$
		\item Si $\norm{[x]} = 0$ entonces existe $t_n$ tal que $\norm{t_n} < \frac{1}{n}$ con $t_n \in [x]$, por lo tanto $x + s_n = t_n \rightarrow 0$ y entonces $s_n \rightarrow -x$. Como $S$ es cerrado $-x \in S$ y como es subespacio $x \in S$; luego $[x]=[0]$
		
		Trivialmente si $x \in S$ entonces $[x] = [0]$ y entonces $\norm{[x]} = 0$. \qed
		
	\end{itemize}
\end{proof}

\begin{proposition}
	Sean $S \subseteq X$ un subespacio cerrado en un normado, entonces:
	
	\[
		\norm{[x]} = d(x,S)
	\]
	
\end{proposition}

\begin{proof}
	$d(x,S) = \inf\limits_{s \in S}{\norm{x - s}_X} = \inf\limits_{-s \in S}{\norm{x + s}_X} = \inf\limits_{t \in [x]}{\norm{t}_X}$.\qed
\end{proof}

\begin{theorem}
	\label{Propiedades del espacio cociente}
	Sea $M \subseteq X$ un subespacio cerrado de un espacio normado y notemos $Q: X \rightarrow \quotient{X}{M}$ la proyecci\'on can\'onica, entonces:
	
	\begin{enumerate}
		\item $Q$ es continua y $\norm{Q} \leq 1$.
		\item Si $X$ es de Banach entonces $\quotient{X}{M}$ lo es.
		\item Si $W \subset \quotient{X}{M}$ entonces $W$ es abierto si y s\'olo si $Q^{-1}(W)$ es abierto.
		\item Si $U \subset X$ es abierto entonces $Q(U) \subset \quotient{X}{M}$ es abierto.
	\end{enumerate}
	
\end{theorem}

\begin{proof}
	Vayamos de a partes:
	
	\begin{enumerate}
		\item $\norm{Q(x)} = \norm{[x]} = d(x,M) \leq \norm{x}$ pues $0 \in M$; conclu\'imos por \ref{Continuidad de un operador}.
		
		\item Sea $\left([x_n]\right) \subset \quotient{X}{M}$ una sucesi\'on tal que $\Bigsum{n \in \N}{\norm{[x_n]}} < \infty$, y para cada $n \in \N$ tal que $\norm{[x_n]} \neq 0$ sea $\epsilon_n = \norm{[x_n]}$. Luego $\norm{[x_n]} + \epsilon_n = 2\norm{[x_n]} >  \norm{x_n + m_n}$ para cierto $m_n \in M$  (Si $\norm{[x_n]} = 0$ entonces $x_n \in M$ y tomamos $m_n = -x_n \in M$), como $\Bigsum{n \in \N}{\norm{[x_n]}} < \infty$ entonces $\Bigsum{n \in \N}{\norm{m_n + x_n}} < \infty$ y por \ref{Banach si y solo si abs convergente es convergente} $\Bigsum{n \in \N}{m_n + x_n} \in X$. Como $S_p = \sum\limits_{n = 1}^{p}{x_n + m_n} \rightarrow \Bigsum{n \in \N}{m_n + x_n}:= v \in X$ y $Q$ es continua entonces $\sum\limits_{n = 1}^{p}{[x_n]} = Q\left(S_p\right) \rightarrow Q(v) \in \quotient{X}{M}$; conclu\'imos por \ref{Banach si y solo si abs convergente es convergente} que $\quotient{X}{M}$ es de Banach.
		
		\item Sea $W \subset \quotient{X}{M}$ tal que $Q^{-1}(W)$ es abierto, luego si $[x_0] \in W$ entonces $x_0 \in Q^{-1}(W)$ y existe un $r > 0$ tal que $x_0 + B_r(0) \subset Q^{-1}(W)$. Veamos el siguiente lema:
		
		\begin{lemma}
			$Q(B_r(0)) = B_r([0])$
		\end{lemma}
		
		\begin{proof}[Demostraci\'on del lema]
			Si $\norm{x} < r$, entonces $\norm{[x]} = \norm{Qx} \leq \norm{x} < r$. Rec\'iprocamente si $\norm{[x]} < r$ entonces existe $y \in M$ tal que $\norm{x+y} < r$ por lo que $[x] = Q(x+y) \in Q(B_r(0))$. \qed
		\end{proof}
		
		Por el lema $W = QQ^{-1}(W) \supset Q(x_0 + B_r(0)) = B_r([x_0])$ por lo que $W$ es abierto.
		
		\item Si $U \subset X$ es abierto entonces $Q^{-1}(Q(U)) = U + M = \Bigcup{m \in M}{U+y}$ que es una uni\'on de abiertos, por lo que $Q^{-1}(Q(U))$ es abierto; por el punto anterior $Q(U)$ es abierto. \qed
		
	\end{enumerate}
\end{proof}

\begin{proposition}
	Si $X$ es normado, $M \subseteq X$ es un subespacio cerrado y $N \subseteq X$ es de dimensi\'on finita entonces $M+N$ es un subespacio cerrado.
\end{proposition}

\begin{proof}
	Consideremos $Q: X \rightarrow \quotient{X}{M}$, como $\dim Q(N) \leq \dim N < \infty $ entonces $Q(N)$ es cerrado y como $Q$ es continua entonces $Q^{-1}(Q(N)) = N+M$ es cerrado .\qed
\end{proof}

\section{Teorema de Hahn-Banach}

\subsection{Funcionales Lineales}

\begin{definition}
	Sea $X$ un $\mathbb{F}$ espacio vectorial, un \textit{hiperplano} en $X$ es una variedad lineal $M$ tal que $\dim \quotient{X}{M}$
\end{definition}

\begin{proposition}
	Una variedad lineal es un hiperplano si y s\'olo si existe $f \neq 0 \in X^{'}$ tal que $M = \ker f$
\end{proposition}

\begin{proof}
	Si $f \in X^{'}$ es no nulo entonces $f$ induce $\bar{f}: \quotient{X}{\ker f} \rightarrow \mathbb{F}$ isomorfismo por lo que $\ker f$ es un hiperplano.
	
	Rec\'iprocamente, si $M$ es un hiperplano entonces existe $T: \quotient{X}{M} \rightarrow \mathbb{F}$ un isomorfismo; luego si consideramos $f = Q \circ T$ cumple que $f \in X^{'}$ y $M = \ker f$. \qed
\end{proof}

\begin{proposition}
	\label{Dos funcionales con mismo nucleo son proporcionales}
	Sean $f,g \in X^{'}$, luego $\ker f = \ker g$ si y s\'olo si $g = \alpha f$ con $\alpha \in \mathbb{F}$
\end{proposition}

\begin{proof}
	Sea $x_0 \in X$ tal que $f(x_0) = 1$, luego $g(x_0) = \alpha \neq 0$ y entonces $x - f(x)x_0 \in \ker f = \ker g$; por lo tanto $g(x) = \alpha f(x)$. \qed
\end{proof}

\begin{proposition}
	\label{Hiperplano en normado es cerrado o denso}
	Si $X$ es un espacio normado y $M$ es un hiperplano entonces $M$ es denso o cerrado.
\end{proposition}

\begin{proof}
	Sabemos que $\overline{M}$ es una variedad lineal y vale que $M \subset \overline{M}$ por lo que $\dim \quotient{X}{\overline{M}} \leq \dim \quotient{X}{M} = 1$. \qed
\end{proof}

\begin{theorem}
	\label{Un funcional es continuo sii su nucleo es cerrado}
	Si $X$ es normado y $f \in X^{'}$ entonces $f$ es acotada si y s\'olo si $\ker f$ es cerrado.
\end{theorem}

\begin{proof}
	Sea $M = \ker f$ cerrado, entonces por \ref{Propiedades del espacio cociente} $Q$ es continua y sea $T : \quotient{X}{\ker f} \rightarrow \mathbb{F}$ un isomorfismo entonces $T$ es continua pues $\norm{Tx} \leq \sum\limits_{i=1}^{k}{\abs{\lambda_i}\norm{Te_i}} \leq C\norm{x}_{\infty} \underbrace{\leq}_{\ref{En dimension finita las normas son equivalentes}} D \norm{x}$. Luego $g = T \circ Q \in X^{\ast}$ y $\ker g = \ker f$, por \ref{Dos funcionales con mismo nucleo son proporcionales} vale que $f = \alpha g \in X^{\ast}$. \qed
\end{proof}

\begin{proposition}
	\label{Operador desde un espacio de dim finita es continuo}
	Si $X$ es normado de dimensi\'on finita e $Y$ es normado, luego si $T: X \rightarrow Y$ es lineal entonces es continua.
\end{proposition}

\begin{proof}
	Ver arriba. \qed
\end{proof}

\subsection{El Teorema de Hanh-Banach}

\begin{definition}
	Sea $X$ un espacio vectorial, un \textit{funcional sublineal} es una $q : X \rightarrow \R$ tal que:
	
	\begin{enumerate}
		
		\item Dados $x,y \in X$ vale $q(x+y) \leq q(x) + q(y)$
		\item Dado $x \in X$ vale $q(\alpha x) = \alpha q(x)$ para todo $\alpha \geq 0$	
	\end{enumerate}
	
\end{definition}

\begin{proposition}
	\label{Escritura de funcionales lineales complejos}
	Sea $X$ un $\C$ espacio vectorial, entonces:
	
	\begin{enumerate}
		\item Si $f : X \rightarrow \R$ es un $\R$ funcional lineal, entonces $\tilde{f}(x) = f(x) - if(ix)$ es un $\C$ funcional lineal.
		\item Si $g : X \rightarrow \C$ es un $\C$ funcional lineal y  $f = \Re g$, entonces $g = \tilde{f}$.
		\item Si $p$ es una seminorma entonces $\abs{f} \leq p \ \Longleftrightarrow \ \abs{\tilde{f}} \leq p$
		\item Si $X$ es normado entonces $\norm{f} = \norm{\tilde{f}}$
	\end{enumerate}
	
\end{proposition}
	
\begin{proof}
	\begin{enumerate}
		\item Es claro que $\tilde{f}$ es $\R$ lineal y adem\'as notemos que $\tilde{f}(ix) = f(ix) - if(-x) = if(x) + f(ix) =i( f(x) - i f(ix)) = i \tilde{f}(x)$; por lo tanto $\tilde{f}$ es $\C$ lineal.
		\item Como $g$ es $\C$ lineal entonces $g(ix) = ig(x)$ y luego $\Im g(ix) = \Im ig(x) = \Re g = f(x)$ por lo que $-f(ix) = \Im g(x)$ y conclu\'imos que $g = \tilde{f}$.
		\item Si $\abs{f} \leq p$ luego como $\tilde{f} = e^{i\theta}\abs{\tilde{f}}$ entonces $\abs{\tilde{f}} = \tilde{f}(e^{-i\theta}x) = \Re \tilde{f}(e^{-i\theta}x) = f(e^{-i\theta}x) \leq p(e^{-i\theta}x) = \abs{e^{-i\theta}}p(x)$.
		
		Rec\'iprocamente si $\abs{\tilde{f}} \leq p$ entonces $\pm f( x)= \Re \tilde{f(\pm x)} \leq \abs{\tilde{f}(\pm x)} \leq p$ por lo que $\abs{f} \leq p$.
		
		\item Como $\norm{f}$ es una seminorma, entonces $\norm{\tilde{f}} \leq \norm{f}$. \qed
		
	\end{enumerate}
\end{proof}	

\begin{lemma}
	\label{Lemma: Hanh-Banach}
	
	Sea $X$ un $\R$ espacio vectorial y sea $q$ un funcional sublineal en $X$. Si $M \subseteq X$ es un hiperplano y $f : M \rightarrow \R$ es un funcional tal que $f \leq q$ para todo $x \in M$ entonces existe $F : X \rightarrow \R$ una extensi\'on tal que $F \leq q$.
		
\end{lemma}

\begin{proof}
	Sea $x_0 \in X \setminus M$ por lo que $X = M \oplus \ip{x_0}$, asumamos que existe tal extensi\'on $F$ y notemos $\alpha_0 = F(x_0)$. Si $t > 0$ y $y_1 \in M$ entonces $F(tx_0 + y_1) = t\alpha_0 + f(y_1) \leq q(tx_0 + y_1)$ por lo que $\alpha_0 \leq q(x_0 + \frac{y_1}{t}) - f(\frac{y_1}{t})$ para todo $y_1 \in M$ que se reduce a ($M$ variedad):
	
	\begin{equation*}
		\alpha_0 \leq q(x_0 + y_1) - f(y_1) \quad \forall y_1 \in M
	\end{equation*}
	
	Adem\'as si $t \geq 0$, $y_2 \in M$ entonces $F(-tx_0 + y_2) = -t\alpha_0 + f(y_2) \leq q(-tx_0 + y_2)$ y conclu\'imos:
	
	\begin{equation}
	\label{eq: Condicion Hanh-Banach}
		q(-x_0 + y_2) + f(y_2)  \leq \alpha_0 \leq q(x_0 + y_1) - f(y_1) \quad \forall y_1y_2 \in M
	\end{equation}
	
	Y rec\'iprocamente si $\alpha_0$ cumple \ref{eq: Condicion Hanh-Banach} entonces volviendo se cumple lo necesitado para $F$. Reordenando necesitamos probar que $f(y_1 + y_2) \leq q(x_0 + y_1) + q(-x_0 + y_2)$; pero:
	
	\begin{equation*}
	\begin{aligned}
		f(y_1 + y_2) \leq & q(y_1 + y_2) = q(x_0 + y_1 -x_0 + y_2) \\
		\leq & q(x_0 + y_1) + q(-x_0 + y_2)
	\end{aligned}
	\end{equation*}
	
	Luego, elegimos $\alpha_0$ tal que $\sup\limits_{y_2 \in M}\sett{f(y_2) - q(-x_0 + y_2)} \leq \alpha_0 \leq \inf\limits_{y_1 \in M}\sett{q(x_0 + y_1) - f(y_1)}$ y definimos $F(tx_0 + y) := t \alpha_0 + f(y)$ y $F$ es una extensi\'on de $f$ tal que $F \leq q$. \qed
	
\end{proof}

\begin{theorem}[Teorema de Hanh-Banach (Versi\'on real)]
	\label{Hanh-Banach real}

	Sea $X$ un $\R$ espacio vectorial y sea $q$ un funcional sublineal en $X$. Si $M \subseteq X$ es un subespacio y $f : M \rightarrow \R$ es un funcional tal que $f \leq q$ para todo $x \in M$ entonces existe $F : X \rightarrow \R$ una extensi\'on tal que $F \leq q$.
\end{theorem}

\begin{proof}
	Sea $\mathcal{S} = \sett{(M_1,f_1) \ , \ M_1 \supseteq M \ , \ M_1\text{ variedad } \ , \ f_1 \in M_1^{'} \ , \ f_1 \vert_{M} = f \ , \ f_1 \leq q \vert_{M_1}}$ y lo dotamos del orden dado por:
	
	\[
	(M_1,f_1) \leq_S (M_2,f_2) \ \Longleftrightarrow \ M_1 \subseteq M_2 \ , \ f_2 \vert_{M_1} = f_1
	\]
	
	Luego $(\mathcal{S}, \leq_S)$ es un poset. Sea $\mathcal{C} = \sett{(M_i,f_i)}_{i \in I}$ una cadena en $\mathcal{S}$ y sea $N = \Bigcup{i \in I}{M_i}$, luego $N$ es variedad y definimos $F: N \rightarrow \R$ dado por $F(x) = f_i(x)$ si $x \in M_i$.
	
	Si $x \in M_i,M_j$ entonces como $\mathcal{C}$ esa cadena $M_i \subseteq M_j$ y $F(x) = f_j(x) = f_i(x)$ pues $f_j \vert_{M_i} = f_i$ por lo que $F$ esta bien definida. Con una cuenta an\'aloga se ve que $F$ es lineal y que $F \leq q$ por lo que $(N,F) \in \mathcal{S}$ es una cota superior para $\mathcal{C}$.
	
	Por \ref{Lema de Zorn} existe $(Y,F)$ un elemento maximal y por \ref{Lemma: Hanh-Banach} $Y = X$. \qed
	
\end{proof}

\begin{theorem}[Teorema de Hanh-Banach]
	\label{Hanh-Banach}
	
	Sea $X$ un espacio vectorial (real o complejo) y sea $p$ una seminorma en $X$. Si $M \subseteq X$ es un subespacio y $f : M \rightarrow \C$ es un funcional tal que $\abs{f} \leq q$ para todo $x \in M$ entonces existe $F : X \rightarrow \C$ una extensi\'on tal que $\abs{F} \leq q$.
\end{theorem}

\begin{proof}
	Por \ref{Escritura de funcionales lineales complejos}(2) si notamos $f_1 = \Re f$ entonces $f(x) = f_1(x) - i f_1(ix)$ y adem\'as de la cuenta de la demostraci\'on de la vuelta de  \ref{Escritura de funcionales lineales complejos}(3) $\abs{f_1} \leq p$ para todo $x \in M$, luego por \ref{Hanh-Banach real} existe $F_1 \in X^{'}$ extensi\'on de $f_1$ tal que $\abs{F_1} \leq p$ por la misma cuenta que antes en \ref{Escritura de funcionales lineales complejos}(3).
	
	Sea $F = \tilde{F_1}$ y por \ref{Escritura de funcionales lineales complejos}(3) vale que $\abs{F} \leq p$. \qed
\end{proof}

\subsection{Corolarios de Hanh-Banach}

\begin{corollary}
	\label{La extension del funcional preserva norma}
	Si $X$ es normado, $M$ es subespacio y $f \in M^{\ast}$ entonces existe $F \in X^{\ast}$ tal que $F \vert_{M} = f$ y $\norm{F} = \norm{f}$
\end{corollary}

\begin{proof}
	Sea $p(x) = \norm{f}\norm{x}$, luego $\abs{f(x)}\leq p$ y por \ref{Hanh-Banach} existe $F: X \rightarrow \mathbb{F}$ extensi\'on tal que $\abs{F(x)} \leq p = \norm{f} \norm{x}$  por lo que $\norm{F} \leq \norm{f}$; finalmente $\norm{f} = \sup\limits_{\substack{x \in S \\ \norm{x} = 1}} \abs{f(x)} = \sup\limits_{\substack{x \in S \\ \norm{x} = 1}} \abs{F(x)} \leq \norm{F}$ entonces $\norm{f} = \norm{F}$. \qed
\end{proof}

\begin{corollary}
	Si $X$ es normado y $\sett{x_1, \dots , x_d}$ es un conjunto linealmente independiente y $\alpha_1 , \dots , \alpha_d \in \mathbb{F}$ entonces existe $f \in X^{\ast}$ tal que $f(x_i) = \alpha_i$ para todo $1 \leq i \leq d$.
\end{corollary}

\begin{proof}
	Sea $M = \ip{x_1 , \dots , x_d}$ y sea $g( \sum\limits_{i=1}^{k}{\beta_i x_i}) = \sum\limits_{i=1}^{k}{\beta_i \alpha_i}  \in M^{\ast}$ por \ref{Operador desde un espacio de dim finita es continuo}, sea $f$ la extensi\'on dada por \ref{La extension del funcional preserva norma}. \qed
\end{proof}


\begin{corollary}
	\label{Norma en funcion de funcionales}
	Si $X$ es normado y $x \in X$ entonces:
	
	\begin{equation}
		\label{eq: Norma en funcion de funcionales}
		\norm{x} = \max \sett{\abs{f(x)} \ , \ f \in X^{\ast} \ , \ \norm{f} \leq 1}
	\end{equation}
\end{corollary}

\begin{proof}
	Sea $f \in X^{\ast}$ tal que $\norm{f} \leq 1$, entonces $\abs{f(x)} \leq \norm{f}\norm{x} \leq \norm{x}$ por lo que $\sup \sett{\abs{f(x)} \ , \ f \in X^{\ast} \ , \ \norm{f} \leq 1} \leq \norm{x}$.
	
	Sea ahora $M = \ip{x}$ y definamos $g \in M^{\ast}$ dado por $g(\beta x) = \beta \norm{x}$ que es continua por \ref{Operador desde un espacio de dim finita es continuo} y $\norm{g} = 1$. Por \ref{La extension del funcional preserva norma} existe $f \in X^{\ast}$ tal que $\norm{f}=1$ y $f(x) = g(x) = \norm{x}$. \qed
\end{proof}

\begin{corollary}
	\label{Si x no esta ne la clasura de un subespacio existe un funcional que separa}
	Si $X$ es normado, $M \subseteq X$ es un subespacio cerrado, $x_0 \in X \setminus M$ y $d = dist\left(x_0 ,M\right)$ entonces existe $f \in X^{\ast}$ tal que $M \subset \ker f$, $f(x_0) = 1$ y $\norm{f} = \frac{1}{d}$
\end{corollary}

\begin{proof}
	Sea $Q : X \rightarrow \quotient{X}{M}$ la proyecci\'on al cociente, como $\norm{[x_0]} = d$ entonces por \ref{Norma en funcion de funcionales} existe $g \in \left(\quotient{X}{M}\right)^{\ast}$ tal que $g\left([x_0]\right) = d$ y $\norm{g} = 1$. Luego si consideramos $f =  d^{-1} \circ g \circ Q : X \rightarrow \mathbb{F}$ cumple que $f \in X^{\ast}$, $f(x_0) = 1$ y $M \subseteq \ker f$; adem\'as $\abs{f(x)} = d^{-1} \abs{g(Q(x))} \leq d^{-1} \norm{g}\norm{Q} \norm{x}$ por lo que $\norm{f} \leq d^{-1}$.
	
	Por otro lado, como $\norm{g}=1$ existe $[x_n] \subset B_1([0]) \subseteq \quotient{X}{M}$ tal que $\abs{g([x_n])} \rightarrow 1$, sea $y_n \in Q^{-1}[x_n]$ que cumple que $\norm{x_n + y_n} < 1$, entonces $\abs{f(x_n+y_n)} = d^{-1} \abs{g([x_n])} \rightarrow d^{-1}$ por lo que $\norm{f} = d^{-1}$. \qed
	
\end{proof}

\begin{corollary}
	\label{Clausura en funcion de funcionales}
	Si $X$ es normado y $M$ es un subespacio, entonces:
	
	\begin{equation}
	\label{eq: Clausura en funcion de funcionales}
		\overline{M} = \Bigcap{\substack{f \in X^{\ast} \\ M \subseteq \ker f}}{\ker f}	
	\end{equation}
	
\end{corollary}

\begin{proof}
	Si $f \in X^{\ast}$, $x \in \overline{M}$ y sea $\sett{x_n}_{n \in \N} \subset M$ con $x_n \rightarrow x$ tal que $f(x_n) = 0$, entonces como $f$ es continua $0 = f(x_n) \rightarrow f(x) = 0$ por lo que $x \in \ker f$.
	
	Rec\'iprocamente si $x_0 \not \in \overline{M}$ entonces por \ref{Si x no esta ne la clasura de un subespacio existe un funcional que separa} existe $f \in X^{\ast}$ tal que $M \subset \ker f \not \ni x_0$ por lo que $x_0 \not \in \Bigcap{\substack{f \in X^{\ast} \\ M \subseteq \ker f}}{\ker f}$. \qed
	
\end{proof}

\begin{corollary}
	\label{M es denso si el unico funcional que lo anula es el nulo}
	Si $X$ es normado y $M$ es subespacio, entonces $M$ es denso si y s\'olo si dado $f \in X^{\ast}$ tal que $M \subseteq \ker f$ implique que $f = 0$.
\end{corollary}

\begin{proof}
	Si $\overline{M} = X$ dado $x \in X$ existe $m_n \subset M$ tal que $m_n \rightarrow x$ por lo que $0 = f(m_n)  \rightarrow f(x) = 0$ por lo que $f(x) = 0$ para todo $x \in X$.
	
	Rec\'iprocamente, si $\Bigcap{\substack{f \in X^{\ast} \\ M \subseteq \ker f}}{\ker f} = \ker 0 = X$ entonces por \ref{Clausura en funcion de funcionales} vale que $X = \overline{M}$. \qed
	
\end{proof}

\subsection{Separabilidad y Reflexividad}

\begin{theorem}
	\label{Dual separable implica separable}
	Sea $X$ normado tal que $X^{\ast}$ es separable, entonces $X$ es separable	
\end{theorem}

\begin{proof}
	Como $X^{\ast}$ es separable entonces $B = \partial B_1[0]$ lo es, sea $D = \sett{f_n}_{n \in \N}$ un conjunto denso en $B$. Sea $0 < \epsilon < 1$, como $\norm{f_n} = 1$ entonces existe $x_n$ tal que $\norm{x_n} = 1$ y $\abs{f(x_n)} \geq \epsilon$ y sea $M = \overline{\ip{x_1 , \dots , x_n, \dots }}$.
	
	Sea $x_0 \in X \setminus M$, luego por \ref{M es denso si el unico funcional que lo anula es el nulo} existe $f \in X^{\ast}$ tal que $f(x_0) \neq 0 $ y $M \subseteq \ker f$. Luego $\epsilon \leq \abs{f_n(x_n)} = \abs{f_n(x_n) - \underbrace{f(x_n)}_{x_n \in M \subset \ker f}} = \abs{(f-f_n)(x_n)} \leq \norm{x_n} \norm{f-f_n} = \norm{f-f_n}$, por lo que $\norm{f-f_n} \geq \epsilon$ para todo $n \in \N$; como $D$ era denso conclu\'imos que $X = M$.
	
	Finalmente si $F = \ip{\sett{x_n}_{n \in  \N}}_{\mathbb{Q}}$ con las combinaciones lineales con escalares racionales de $M$, entonces $\overline{F} = M = X$ por lo que $X$ es separable. \qed
	
\end{proof}

\begin{theorem}
	\label{El dual de un subespacio es el cociente del dual por el polar}
	Sea $X$ normado, $S$ subespacio y $S^{\circ} = \sett{f \in X^{\ast} \ / \ S \subset \ker f}$; luego $S^{\circ}$ es un subespacio cerrado y $\quotient{X^{\ast}}{S^{\circ}} \simeq S^{\ast}$.
\end{theorem}

\begin{proof}
	Sea $s \in S$ y $\sett{f_n}_{n \in \N} \subset S^{\circ}$ tal que $f_n \rightarrow f$, luego $\abs{f(s)} = \abs{(f-f_n)(s) + \underbrace{f_n(s)}_{f_n \in S^{\circ}}} \leq \norm{f-f_n}\norm{s} \rightarrow 0 $ por lo que $f \in S^{\circ}$ y $S^{\circ}$ es cerrado.
	
	Sea entonces $F : S^{\ast} \rightarrow \quotient{X^{\ast}}{S^{\circ}}$ dada por $F(f) = [\hat{f}]$ donde $\hat{f}$ es la de \ref{La extension del funcional preserva norma}, $F$ esta bien definida pues si $g$ es otra extensi\'on continua entonces $[g] = [\hat{f}]$ pues $g \vert_{S} = \hat{f} \vert_{S} = f$.
	
	Es claro que $F$ es lineal e inyectiva, si $h \in \quotient{X^{\ast}}{S^{\circ}}$ entonces $F(f) = h$ si y s\'olo si $[\hat{f}] = h = [\hat{h}]$, luego si $f = \hat{h} \vert_{S}$ entonces $F(f)=h$ por lo que $F$ es sobreyectiva.
	
	Finalmente, si $f \in S^{\ast}$ entonces $\norm{F(f)} = \norm{[\hat{f}]} = dist (\hat{f}, S^{\circ}) \leq \norm{\hat{f}} \underbrace{=}_{\ref{La extension del funcional preserva norma}} \norm{f}$; por lo que $\norm{F} \leq 1$; rec\'iprocamente si $f \in S^{\ast}$ y $g \in S^{\circ}$ entonces $\norm{f} = \norm{f + g \vert_S} \leq \norm{\hat{f} + g}$, luego $\norm{f} \leq \inf\limits_{g \in S^{\circ}}{\norm{\hat{f} + g}} = \norm{[\hat{f}]} = \norm{F(f)}$. Conclu\'imos que $F$ es un isomorfismo.\qed
	
\end{proof}

\begin{definition}
	Sea $X$ normado y $\ddual{X}$ su doble dual, definimos la \textit{aplicaci\'on can\'onica} $p: X \rightarrow \ddual{X}$ dada por $P(x)(f) = f(x)$.
\end{definition}


\begin{remark}
	Si $p$ es la aplicaci\'on can\'onica luego vale que:
	
	\begin{enumerate}
		\item $p(x) \in \ddual{X}$ para todo $x \in X$
		\item $p$ es lineal y monomorfismo
		\item $\norm{p} = 1$
	\end{enumerate}
	
\end{remark}

\begin{proof}
	Si $x \in X$ luego $\abs{p(x)(f)} = \abs{f(x)} \leq \norm{f} \norm{x}$ por lo que $\norm{p} \leq 1$ y $p \in \ddual{X}$; es m\'as $\norm{p(x)} = \sup\limits_{\substack{f \in \dual{X} \\ \norm{f} = 1}}{\abs{p(x)(f)}} = \sup\limits_{\substack{f \in \dual{X} \\ \norm{f} = 1}}{\abs{f(x)}} \underbrace{=}_{\ref{Norma en funcion de funcionales}} \norm{x}$. \qed
\end{proof}

\begin{definition}
	Sea $X$ normado, decimos que es un espacio \textit{reflexivo} si $p$ es sobreyectiva.
\end{definition}

\begin{example}
	Todo espacio de dimensi\'on finita es reflexivo
\end{example}

\begin{proof}
	Como $\dim X = n$ y $p$ es monomorfismo entonces es isomorfismo y luego en particular es sobreyectiva.
\end{proof}

\begin{remark}
	Si $f,g \in \dual{H}$ entonces por \ref{Teorema de representacion de Riesz} existen $x_0,y_0 \in H$ tal que $f(z) =  \ip{z,x_0}, g(z) = \ip{z,y_0}$. Luego el producto interno que refiere a $\norm{}_{\dual{H}}$ es $\ip{f,g}_{\dual{H}} = \ip{y_0,x_0}_{H}$ 
\end{remark}

\begin{theorem}
	Sea $H$ Hilbert entonces $H$ es reflexivo.
\end{theorem}

\begin{proof}
	Ya sabemos de \ref{Teorema de representacion de Riesz} que la aplicaci\'on $F: H \rightarrow \dual{H}$ dada por $F(h)(x) = f(x) = \ip{x,h}$ es una isometr\'ia lineal. Sea $g \in \ddual{H}$ por lo que $g(f) = \ip{f,f_0}$ para alg\'un $f_0 \in \dual{H}$, que a su vez existe $z_0 \in H$ tal que $f_0(x) = \ip{x,z_0}$; luego $p(z_0)(f) = f(z_0) = \ip{x,z_0} = \ip{f,f_0}_{\dual{H}} = g(f)$.
\end{proof}

\begin{proposition}
	Si $\B = \sett{v_i}_{i \in I}$ es base ortonormal de $H$ Hilbert entonces si definimos $f_{v_i}(x) = \ip{x,v_i} \in \dual{H}$ vale que $\F = \sett{f_{v_i}}_{i \in I}$ es base ortonormal de $\dual{H}$
\end{proposition}

\begin{proof}
	Sean $f_{v{i}}, f_{v_j} \in \F$ luego $\ip{f_{v_i},f_{v_j}} = \ip{v_j,v_i} = \delta_{i,j}$ por lo que $\F$ es ortonormal. Finalmente si $f \in \dual{H}$ entonces existe $x \in H$ tal que $f(z) = \ip{z,x}$ y entonces $0 = \ip{f,f_{v_i}} = \ip{v_i,x}$ para todo $i \in I$, por la completitud de $\B$ conclu\'imos que $x = 0$ por lo que $f = 0$. \qed
\end{proof}

\begin{corollary}
	\label{Hilbert es autodual}
	Si $H$ es Hilbert entonces $H \simeq \dual{H}$
\end{corollary}

\begin{proof}
	Por \ref{HIlbert congruente a un l2} dos veces tenemos que $H,\dual{H} \simeq l^2(I)$, luego por transitividad $H \simeq \dual{H}$. \qed
\end{proof}

\begin{theorem}
	\label{Si X es reflexivo entonces su dual lo es}
	Sea $X$ normado y reflexivo, entonces $\dual{X}$ es reflexivo
\end{theorem}

\begin{proof}
	Sea $\dual{p}: \dual{X} \rightarrow \dual{\ddual{X}}$, $g \in \dual{\ddual{X}}$, $h \in \ddual{X}$ y definimos $f := g \circ p \in \dual{X}$. Como $h \in \ddual{X}$ entonces existe $x_h \in X$ tal que $p(x_h) = h$ y luego:
	
		\begin{itemize}
			\item $\dual{p}(f)(h) = h(f) = p(x_h)(f) = f(x_h)$
			\item $g(h) = g \circ p (x_h) = f(x_h)$
		\end{itemize}
	
	Luego $\dual{p}(f) = g$ y conclu\'imos que $\dual{X}$ es reflexivo. \qed
	
	
\end{proof}

\begin{theorem}
	\label{Cerrado en reflexivo es reflexivo}
	Si $X$ es reflexivo y $S \subseteq X$ subespacio cerrado entonces $S$ es reflexivo
\end{theorem}

\begin{proof}
	Sea $g \in \ddual{S}$ y consideremos $\ddual{x} \in \ddual{X}$ dada por $\ddual{x}(f) = g \circ T \circ \pi (f)$ donde $T: \quotient{\dual{X}}{S^{\circ}} \rightarrow \dual{S}$ es el isomorfismo isom\'etrico dado en \ref{El dual de un subespacio es el cociente del dual por el polar}, $\pi: \dual{X} \rightarrow \quotient{\dual{X}}{S^{\circ}}$ la proyecci\'on al cociente dada por \ref{Propiedades del espacio cociente}. Como $X$ es reflexivo entonces existe $x \in X$ tal que $p(x) = \ddual{x}$, si $x \not \in S$ entonces por \ref{Si x no esta ne la clasura de un subespacio existe un funcional que separa} existe $f \in \dual{X}$ tal que $f\vert_S = 0$ y $f(x) \neq 0$ por lo que $f(x) = p(x)(f) = \ddual{x}(f) = g \circ T \circ \pi (f) \underbrace{=}_{f \vert_S = 0} 0$ y conclu\'imos que $x \in S$.
	
	
	Sea $f \in \dual{S}$ y $\hat{f}$ su extensi\'on dada por \ref{La extension del funcional preserva norma}, entonces $g(f) = g \circ T \circ \pi (\hat{f}) = \ddual{x}(\hat{f}) = p(x) (\hat{f}) = \hat{f}(x) = f(x) = p(x)(f)$ y conclu\'imos que $p(x) = g$ por lo que $S$ es reflexivo. \qed
	
\end{proof}

\begin{remark}
	$l^1$ no es reflexivo
\end{remark}

\begin{proof}
	Si $l^1$ fuese reflexivo entonces $l^1 \simeq \dual{l^{\infty}}$ y como $l^1$ es separable entonces por \ref{Dual separable implica separable} conclu\'imos que $l^{\infty}$ es separable, conclu\'imos que $l^1$ no es reflexivo. \qed
\end{proof}

\subsection{Consecuencias Geom\'etricas de Hanh-Banach}

\begin{definition}
	Sea $X$ un espacio vectorial real y $K \subset X$ convexo, decimos que es \textit{absorbente} si para todo $x \in X$ existe $\epsilon > 0$ tal que $\alpha x \in K$ para todo $\alpha \in \R, \ \abs{\alpha} < \epsilon$
\end{definition}

\begin{definition}
	Sea $X$ un espacio vectorial real y $K \subset X$ convexo absorbente tal que $0 \in K$, definimos la \textit{funcional de Minkowsky asociada a } $K$ como la aplicaci\'on $p_K(x) = \inf \sett{\alpha > 0 \ / \ \quotient{x}{\alpha} \in K}$
\end{definition}

\begin{definition}
	Sea $X$ un espacio vectorial real y sea $f$ un funcional lineal, llamaremos \textit{hiperplano} a los conjuntos $f^{-1}(\alpha)$. Si $S \subset X$ decimos que $H = f^{-1}(\sett{\alpha})$ \textit{deja a un lado(estrictamente) a } $S$ si $S \subset \sett{x \in X \  \ f(x) \leq \alpha} \left(\sett{x \in X \ / \ f(x) < \alpha}\right)$ u el an\'alogo con los signos opuestos.
\end{definition}

\begin{proposition}
	\label{Lemma1: Hanh-Banach geometrico}
	Si $H$ es hiperplano y $K \subseteq X$ es convexo entonces $H$ deja a un lado estrictamente a $K$ si y s\'olo si $K \cap H = \emptyset$
\end{proposition}

\begin{proof}
	\label{Un hiperplano deja a un lado a un convexo disjunto}
	Sea $\alpha$ tal que $H = f^{-1}(\sett{\alpha})$ y supongamos que existe $x_1,x_2 \in K$ tal que $f(x_1) < \alpha$ y $f(x_2) > \alpha $, luego si consideramos $g(t) = f(tx_1 + (1-t)x_2) = t f(x_1 - x_2) + (1-t)f(x_2)$ por Bonzano existe $t_0 \in (0,1)$ tal que $g(t_0) = \alpha$ por lo que $t_0 x_1 + (1-t_0)x_2 \in H \cap K$ pues $K$ es convexo \qed
\end{proof}

\begin{proposition}
	\label{Lemma2: Hanh-Banach geometrico}
	Si $H$ es un hiperplano y $K$ es convexo absorbente disjunto de $H$ entonces existe $g$ funcional lineal tal que:
	
	\begin{enumerate}
		\item $H = g^{-1}(1)$
		\item $-p_K(-x) \leq g(x) \leq p(x)$ para todo $x \in X$
	\end{enumerate}
\end{proposition}

\begin{proof}
	Sea $\alpha \in \R$ y $f \in X'$ tal que $H = f^{-1}(\alpha)$, luego si $g = \dfrac{f}{\alpha}$ entonces $x \in H$ si y s\'olo si $f(x) = \alpha$ si y s\'olo si $\dfrac{f(x)}{\alpha} = g(x) = 1$ por lo que $H = g^{-1}(x)$. Como $K \cap H = \emptyset$ entonces por \ref{Un hiperplano deja a un lado a un convexo disjunto} entonces $K \subset g^{-1}((-\infty, 1))$ o $K \subset g^{-1}((1,\infty))$ pero como $g(\underbrace{0}_{0 \in K}) = 0$ entonces $K \subset g^{-1}((1,\infty))$.
	
	Como $K$ es absorbente dado $x \in X$ existe $\beta > 0$ tal que $\dfrac{x}{\beta} < 1$ por lo que $g(x) < \beta$ y conclu\'imos que $g(x) \leq p(x)$. \qed	
	
\end{proof}

\begin{proposition}
	\label{Lemma3: Hanh-Banach geometrico}
	Sea $H \subseteq X$ un hiperplano en un espacio normado, $S \subset X$ tal que $\interior{S} \neq \emptyset$ entonces $H$ es cerrado.
\end{proposition}

\begin{proof}
	Sea $\alpha \in \R$ y $f \in X'$ tal que $H_{\alpha} = f^{-1}(\alpha)$, si $H_{\alpha+1}$ no fuera cerrado entonces por \ref{Hiperplano en normado es cerrado o denso} es denso y como $\interior{S} \neq \emptyset$ eso implica que $S \cap H_{\alpha +1} \neq \emptyset$. Luego si $x_n \subset \ker f$ e $y \in H_{\alpha +1}$ entonces $H_{\alpha +1} \ni z \underbrace{=}_{x_n + y \in H_{\alpha + 1}} \lim \left(x_n + y\right) = \lim x_n + y$ pero $f(z-y) = f(z) - f(y) = 0$ por lo que $\lim x_n \in \ker f$ y por \ref{Un funcional es continuo sii su nucleo es cerrado} $f$ es continua; conclu\'imos que $H$ es cerrado. \qed
\end{proof}

\begin{definition}
	Sea $K \subset X$ un conjunto, decimos que es \textit{balanceado} si $\alpha x \in K$ para todo $x \in K$ y $\abs{\alpha} \leq 1$.
\end{definition}

\begin{proposition}
	\label{Lemma4: Hanh-Banach geometrico}
	Si $K \subset X$ es convexo, absorbente tal que $0 \in K$ entonces $p_K$ es sublineal. M\'as a\'un si $K$ es balanceado entonces $p_K$ es seminorma y $K = \sett{x \in X \ / \ p_K(x) < 1}$.
\end{proposition}

\begin{proof}
	Como $K$ es absorbente entonces $X = \Bigcup{n \in \N}{nK}$ por lo que $p_K(x)$ est\'a bien definido para todo $x \in X$. Como $p_K(0) = 0$ trivialmente, si $\alpha > 0$:
	
	\begin{equation*}
	\begin{aligned}
		p_K(\alpha x) = & \inf \sett{t \geq 0 \ / \ \alpha x \in tV} \\
					  = & \inf \sett{t \geq 0 \ / \ x \in  \left(\dfrac{t}{\alpha}V \right)} \\
					  = & \alpha \inf \sett{\dfrac{t}{\alpha} \geq 0 \ / \ x \in  \left(\dfrac{t}{\alpha}V \right)} \\
					  = & \alpha p_K(x)
	\end{aligned}
	\end{equation*}
	
	Por otro lado si $\alpha, \beta \geq 0$ y $a,b \in K$ entonces:
	
	\begin{equation*}
		\alpha a + \beta b = \left(\alpha + \beta\right) \left(\dfrac{\alpha}{\alpha + \beta} a + \dfrac{\beta}{\alpha + \beta} b\right) \in \left(\alpha + \beta\right) K
	\end{equation*}	
	
	Por la convexidad de $K$, luego si $x,y \in K$ y $p_K(x) = \alpha, p_K(y) = \beta$ y $\delta > 0$ entonces $x \in \left(\alpha + \delta\right)K, y \in \left(\beta + \delta\right)K$ pues $K$ es absorbente. Entonces por convexidad $x+y \in \left(\alpha + \beta + 2 \delta\right)K$ y si $\delta \rightarrow 0$ entonces $p_K(x+y) \leq \alpha + \beta = p_K(x) + p_K(y)$.
	
	Supongamos ahora que $K$ es balanceado, si $p_K(x) = \alpha < 1$ entonces para $\alpha < \beta < 1$ vale que $x \in \beta K \subset K$ por lo que $\sett{x \in X \ / \ p_K(x) < 1} \subseteq K$. Rec\'iprocamente si $x \in K$ entonces $p_K(x) \leq 1$ y como $K$ es absorbente existe $\epsilon > 0$ tal que si $0 < t < \epsilon$ entonces $y = (1+t)x = x + tx \in K$, luego $p_K(x) = (1+t)^{-1}p_K(y) \leq (1+t)^{-1} < 1$. \qed
	
\end{proof}

\begin{theorem}[Hanh-Banach Geom\'etrico]
	\label{Hanh-Banach geometrico}
	Si $X$ es normado, $K \subset X$ es un convexo absorbente abierto tal que $0 \in K$ y $V \subset X$ es una variedad lineal tal que $V \cap K = \emptyset$ entonces existe $H$ hiperplano cerrado tal que $V \subset H$ y $H$ deja a un lado estrictamente a $K$.
\end{theorem}

\begin{proof}
	Sea $x_0 \in V$ y $S = V - x_0$ un subespacio tal que $x_0 \not \in S$, definimos $T = \ip{S,x_0}$ entonces $\dim \quotient{T}{S} = 1$ y $K \cap S$ es un convexo absorbente abierto en $T$ tal que $V := S+x_0 \cap \left(K \cap S\right) = \emptyset$, luego por \ref{Lemma2: Hanh-Banach geometrico} existe $g \in T'$ tal que $V = g^{-1}(1)$ y $g \leq p_K$ que es sublineal por \ref{Lemma4: Hanh-Banach geometrico}. Por \ref{Hanh-Banach real} existe $f \in X'$ extensi\'on dominada por $p_K$, luego si $H = f^{-1}(1)$ es un hiperplano que contiene a $V$ y por \ref{Lemma1: Hanh-Banach geometrico} deja estrictamente a un lado a $K$. Finalmente como $K$ es abierto por \ref{Lemma3: Hanh-Banach geometrico} $f$ es continua. \qed
\end{proof}

\begin{theorem}
	\label{Dos convexos en un normado se separan si uno es abierto}
	Si $X$ es normado real y $A,B$ son conjuntos disjuntos convexos con $A$ abierto, entonces existe $f \in \dual{X}$ y $\alpha \in \R$ tal que $f \vert_A < \alpha$ y $f \vert_B \geq \alpha$. M\'as a\'un si $B$ es abierto entonces la separaci\'on es estricta.
\end{theorem}
\begin{proof}
	Sea $G = A - B$, entonces usemos:
	
	\begin{lemma}
		\label{Lemma1: Dos convexos en un normado se separan si uno es abierto}
		$G$ es convexo, abierto y $0 \not \in G$
	\end{lemma}
	
	\begin{proof}[Demostraci\'on del lema]
		Si $x,y \in G$ entonces $x = a_x-b_x, y = a_y-b_y$, luego $tx + (1-t)y = ta_x - t b_x + (1-t)a_y - (1-t)b_y = \left(\underbrace{ta_x + (1-t)a_y}_{\in A}\right) - \left(\underbrace{tb_x + (1-t)b_y}_{\in B}\right) = a_{tx + (1-t)y} - b_{tx + (1-t)y} \in G$.
		
		Adem\'as $G = \Bigcup{b \in B}{A-b}$ por lo que $G$ es abierto. Finalmente como $A \cap B = \emptyset$ enotnces $0 \not \in G$\qed
		
	\end{proof}
	
	Por \ref{Lemma1: Dos convexos en un normado se separan si uno es abierto} y \ref{Hanh-Banach geometrico} existe $H$ hiperplano cerrado tal que $H \cap G = \emptyset$ y sea $f \in \dual{X}$ tal que $H = f^{-1}(0)$ que existe por \ref{Lemma2: Hanh-Banach geometrico}. Luego $f(G)$ es convexo y $0 \not \in f(G)$, luego $f \vert_G > 0$ (sino lo es para $-f$), lo que implica que $f(a) > f(b)$ para todos $a \in A, \ b \in B$; sea $\alpha$ tal que:
	
	\begin{equation*}
		\sup \sett{f(b), \ b \in B} \leq \alpha \leq \inf \sett{f(a), \ a \in A}
	\end{equation*}
	
	Luego $\alpha$ es el buscado.
	
	
	Si $B$ es abierto, notemos:
	
	\begin{lemma}
		\label{Lemma2: Dos convexos en un normado se separan si uno es abierto}
		Si $f \in \dual{X}$ y $A$ es abierto convexo entonces $f(A)$ es un intervalo abierto
	\end{lemma}
	
	\begin{proof}[Demostraci\'on del lema]
		Sea $a,b \in A$, luego $tf(a) + (1-t)f(b)= f(ta + (1-t)b) \in f(A)$ por lo tanto $f(A)$ es convexo y como los \'unicos convexos de $\R$ son los intervalos, $f(A)$ es un intervalo.
		
		Sea $x_0$ tal que $f(x_0) = 1$ y $x \in A$, luego existe $\epsilon > 0$ tal que $x \pm \epsilon x_0 \in A$, luego $f(x \pm \epsilon x_0) = f(x) \pm \epsilon \in f(A)$, luego $\left(f(x) - \epsilon, f(x) + \epsilon\right) \subset f(A)$ y $f(A)$ es abierto. \qed
	\end{proof}
	
	Luego por \ref{Lemma2: Dos convexos en un normado se separan si uno es abierto} $f(A),f(B)$ son intervalo abiertos por lo que $\alpha$ separa de manera estricta. \qed
	
\end{proof}

\begin{theorem}
	\label{Un cerrado se separa de un compacto}
	Sea $X$ normado y $A,B \subset X$ dos subconjuntos cerrados, convexos y disjuntos; luego si $B$ es compacto entonces $A$ y $B$ se separan de manera estricta.
\end{theorem}

\begin{proof}
	Primero notemos:

\begin{lemma}
	\label{Lemma1: Un cerrado se separa de un compacto}
	Si $X$ es normado, $K \subseteq X$ es un compacto y $K \subseteq V \subseteq X$ es abierto entonces existe $U \ni 0$ un entorno tal que $K + U \subset V$
\end{lemma}

\begin{proof}[Demostraci\'on del lema]
	Sea $\mathcal{U}_0$ todos los entornos abiertos de $0$ y supongamos que para todo $U \in \mathcal{U}_0$, $K+U \not \subseteq V$, luego para cada $U$ existe $x_u \in K, \ y_u \in U$ tal que $x_u+y_u \in V^{c}$ y ordenemos $\mathcal{U}_0$ por la inclusi\'on inversa; luego $\sett{x_u},\sett{y_u}$ son redes en $X$, m\'as a\'un $y_u \rightarrow 0$ y existe $x \in X$ tal que $x$ es punto de acumulaci\'on de la red. Luego, $x$ es punto de acumulaci\'on de $x_u+y_u$ y entonces $x \in \overline{V^{c}} = V^{c}$; pero $x \in K \subset V$ por lo que existe dicho $U \in \mathcal{U}_0$. \qed
\end{proof}

Usando \ref{Lemma1: Un cerrado se separa de un compacto} sobre $B, A^{c}$ existe $U_1$ entorno de $0$ tal que $B+U_1 \subset A^c$ y por \ref{Lemma4: Hanh-Banach geometrico} existe $p$ seminorma tal que $\sett{x \in X, \ p(x) < 1} \subset U_1$. Sea $U = \sett{x \in X, \ p(x) < \dfrac{1}{2}}$, notemos:

\begin{lemma}
	\label{Lemma2: Un cerrado se separa de un compacto}
	 $\left(B+U\right)\cap \left(A + U\right) = \emptyset$ y ambos son abiertos convexos.
\end{lemma}

Luego por \ref{Lemma2: Un cerrado se separa de un compacto} y \ref{Dos convexos en un normado se separan si uno es abierto} se concluye que $A$ y $B$ se separan estrictamente.

\end{proof}


\section{Teoremas fundamentales de espacios de Banach}

\subsection{Teorema de la aplicaci\'on abierta}

\begin{definition}
	Un espacio topol\'ogico  $X$ es de \textit{primera cvategor\'ia} si $X = \Bigcup{n \in \N}{A_n}$ con $\interior{\overline{A_n}} = \emptyset$. Decimos que es de \textit{segunda categoria} si no es de primera categor\'ia.
\end{definition}

\begin{theorem}
	\label{Lemma: Teorema aplicacion abierta}
	Sean $X,Y$ espacios normados y $T \in L(X,Y)$ tal que $\rank T $ es de segunda categor\'ia. Entonces si $U \ni 0$ es un entorno abierto en $X$ luego existe $V \ni 0$ entorno abierto en $Y$ tal que $V \subseteq \overline{T(U)}$.
\end{theorem}

\begin{proof}
	Sea $U \ni 0$ un entorno abierto en $X$ y $\alpha > 0$ tal que $B_{\alpha}(0) \subseteq U$, finalmente sea $W = B_{\frac{\alpha}{2}}(0)$. Notemos:
	
	\begin{lemma}
		$\Bigcup{n \in \N}{T\left(nW\right)} = \Bigcup{n \in \N}{nT\left(W\right)} = \rank T$
	\end{lemma}
	
	\begin{proof}[Demostraci\'on del lema]
		
		Si $y \in \rank T$ entonces existe $x \in X$ tal que $T(x) = y$; por arquimedianidad esta bien definido $n_0 = \min \sett{n \in \N \tq \frac{\norm{x}}{n} < \frac{\alpha}{2}}$ por lo que $x \in n_0W$ y se concluye que $y \in T\left(n_0 W\right)$.
		
		Trivialmente $T \left(nW \right) \subset \rank T$ para todo $n \in \N$ por lo que $\Bigcup{n \in \N}{T\left(nW\right)} \subset \rank T$. \qed
		
	\end{proof}
	
	Como $\rank T$ es de segunda categor\'ia existe $n_0 \in \N$ tal que $n_0\interior{\overline{T \left(W\right)}} \neq \emptyset$; como adem\'as $f(x) = \frac{x}{n_0} \in \text{Hom}(Y)$ entonces conclu\'imos que $\interior{\overline{T \left(W\right)}} \neq \emptyset$. Por lo tanto existe $z_0 \in Y$ y $\delta > 0$ tal que $B_{\delta}(z_0) \cap T \left(W\right) \neq \emptyset$, sea $T(\underbrace{x_0}_{\in W}) = y_0 \in B_{\delta}(z_0) \cap T \left(W\right)$ y $r > 0$ tal que $B_{r}(y_0) \subseteq B_{\delta}(z_0) \subseteq \overline{T \left(W\right)}$.
	
	Como $B_{r}(y_0) \subseteq \overline{T(W)}$ entonces $B_{r}(0) \subseteq \overline{T\left(W\right)} - y_0 = \overline{T \left(W\right) - y_0} = \overline{T \left(W- x_0\right) } \subseteq \overline{T \left(B_{\alpha}(0)\right)} \subseteq \overline{T \left(U\right)}$. \qed 
	
\end{proof}

\begin{theorem}[Teorema de la aplicaci\'on abierta]
	\label{Teorema de la aplicacion abierta}
	Sean $X,Y$ normados con $X$ Banach y sea $T \in L(X,Y)$ tal que $\rank T$ es de segunda categor\'ia, entonces:
	
	\begin{enumerate}
		\item Si $\alpha > 0 $ entonces existe $\beta > 0$ tal que $B_{\beta}[0] \subseteq T(B_{\alpha}[0])$
		\item $\rank T = Y$
		\item $T$ es abierta.
	\end{enumerate}
	
\end{theorem}

\begin{proof}
	Vayamos de a partes:
	
	\begin{enumerate}
		
		\item Sea $\alpha > 0$ y consideremos $B_{\frac{\alpha}{2}}(0) \subset B_{\alpha}[0]$, por \ref{Lemma: Teorema aplicacion abierta} existe $\beta > 0$ tal que $B_{\beta}(0) \subseteq \overline{T(B_{\frac{\alpha}{2}}(0))}$, sea $t > \frac{\alpha}{2}$ y veamos que $B_{\beta}(0) \subset T \left(B_{t}(0)\right)$.
		
		Sea $\sett{\epsilon_n}_{n \in \N} \subset \R_{+}$ tal que $\Bigsum{n \in \N}{\epsilon_n} < t - \frac{\alpha}{2}$, por \ref{Lemma: Teorema aplicacion abierta} para cada $n \in \N$ existe $\delta_n > 0$ tal que $B_{\delta_n}(0) \subseteq \overline{T_{\epsilon_n}(0)}$ y llamando $\tilde{\delta_n} = \dfrac{\delta_n}{2^{n}}$ podemos suponer sin p\'erdida de generalidad que $\delta_n \rightarrow 0$.
		
		Sea $y \in B_{\beta}(0)$, luego $B_{\delta_1}(y) \cap T \left(B_{\frac{\alpha}{2}}(0)\right) \neq \emptyset$ por lo que existe $T(x_0) = y_0 \in Y$ tal que $\norm{y - y_0} < \delta_1$ con $\norm{x_0} \leq \frac{\alpha}{2}$. Como $y-y_0 \in B_{\delta_1}(0) \subseteq \overline{T \left(B_{\epsilon_1}(0)\right)}$ entonces existe $T(\underbrace{x_1}_{\in B_{\epsilon_1}(0)})=y_1 \in Y$ tal que $\norm{y-y_0-y_1} < \delta_2$. Inductivamente contru\'imos $\sett{y_n}_{n \in \N} \subset Y$ tal que $\norm{y- \left(\Bigsum{1 \leq n \leq k}{y_n}\right)} < \delta_{k+1}$, $y_n = T(x_n)$ y $x_n \in B_{\epsilon_n}(0)$. Notemos que como $\delta_n \rightarrow 0$ entonces $\Bigsum{n \in \N_0}{y_n} = y$; por otro lado $\Bigsum{n \in \N}{\norm{x_n}} < \Bigsum{n \in \N}{\epsilon_n} < t-\frac{\alpha}{2}$ luego por \ref{Banach si y solo si abs convergente es convergente} $\Bigsum{n \in \N_0}{x_n} = x \in X$ y como $T$ es continua, $T(x) = \Bigsum{n \in \N_0}{T(x_n)} = \Bigsum{n \in \N_0}{y_n} = y$.
		
		Finalmente notemos que $\norm{x} \leq \Bigsum{n \in \N_0}{\norm{x_n}} = \underbrace{\norm{x_0}}_{< \frac{\alpha}{2}} + \underbrace{\Bigsum{n \in \N}{\norm{x_n}}}_{< t - \frac{\alpha}{2}} < t$; por lo tanto $B_{\beta}(0) \subseteq T \left(B_t(0) \right)$.
		
		\item Sea $y \in Y$ y $\alpha,\beta > 0$ que cumplan 1, sea $n \in \N$ tal que $\frac{y}{n} \in B_{\frac{\beta}{2}}[0]$, entonces por 1 $\frac{y}{n} \in \rank T$ por lo que $y \in \rank T$.
		
		\item Sea $U \subseteq X$ abierto y sea $T(x_0) = y_0 \in T(U)$, luego $U - x_0 \ni 0$ es un entorno abierto. Sea $r > 0$ tal que $B_r(0) \subseteq U-x_0$ luego $T(B_{\frac{r}{2}[0]}) \subseteq T \left(B_r(0)\right) \subseteq T \left(U\right) - y_0$; por 1 existe $\delta > 0$ tal que $B_{\delta[0]} \subseteq T \left(B_{\frac{r}{2}}[0]\right) \subseteq T(U) - y_0$ por lo que $B_{\frac{\delta}{2}(y_0)} \subseteq T(U)$. \qed
		
		
				
	\end{enumerate}
	
\end{proof}

\begin{corollary}[Teorema de la inversa acotada]
	\label{Teorema de la inversa acotada}
	Sean $X,Y$ Banach y sea $T \in L(X,Y)$ tal que $T$ es isomorfismo lineal, entonces $T$ es isomorfismo de Banach.	
\end{corollary}

\begin{proof}
	Como $T$ es isomorfismo lineal entonces $\rank T = Y$ y como $Y$ es Banach, es completo y por el teorema de Baire es de segunda categor\'ia; luego por \ref{Teorema de la aplicacion abierta} $T$ es abierta si y s\'olo si $T^{-1}$ es acotada. \qed
\end{proof}

\subsection{Teorema del Gr\'afico cerrado}

\begin{definition}
	Sean $X,Y$ normados y $T:X \rightarrow Y$ un operador lineal cuyo dominio $D(T) \subset X$ es subespacio y $\rank T \subset Y$ es subespacio a su vez. Luego diremos que $T$ es \textit{cerrado} si $\graf T \subset X \times Y$ es cerrado.
\end{definition}

\begin{proposition}
	\label{Lemma 1: Teo Grafico cerrado}
	Sean $X,Y$ normados y $T \in L(D(T),Y)$ con $D(T)$ subespacio cerrado de $X$, entonces $T$ es cerrado.
\end{proposition}

\begin{proof}
	Si $x_n \rightarrow x$ y $T(x_n) \rightarrow y$, luego por \ref{Continuidad de un operador} $T(x) \leftarrow T(x_n) \rightarrow y$ por lo que $y = T(X)$. \qed
\end{proof}

\begin{proposition}
	\label{Lemma 2: Teo Grafico cerrado}
	Sean $X,Y$ normados con $Y$ Banach y $T: X \rightarrow Y$ un operador cerrado y acotado, entonces $D(T)$ es subespacio cerrado.
\end{proposition}

\begin{proof}
	Sea $x_n \subset D(T)$ tal que $x_n \rightarrow x$, luego como $T$ es acotado y lineal $\norm{T(x_n) - T(x_m)} = \norm{T \left(x_n-x_m\right)} \leq \norm{T} \norm{x_n-x_m} \rightarrow 0$ y como $Y$ es Banach existe $y = \lim T(x_n)$.
	
	Luego como $x_n \rightarrow x$, $T(x_n) \rightarrow y$ y $T$ es cerrada entonces $x \in D(T)$ y $T(x) = y$. \qed
	
\end{proof}

\begin{theorem}[Teorema del gr\'afico cerrado]
	\label{Teorema del grafico cerrado}
	Sean $X,Y$ Banach y $T:X \rightarrow Y$ lineal, entonces $T \in L(X,Y)$ si y s\'olo si $T$ es cerrada. 
\end{theorem}

\begin{proof}
	Como $T$ es cerrada entonces $\graf T$ es un subespacio cerrado de $X \times Y$ que es Banach con la norma suma directa. Consideremos $\pi: \graf T \rightarrow X$ dada por $\pi(x,T(x)) = x$, entonces:
	
	\begin{itemize}
		\item $\pi$ es lineal
		\item $\norm{\pi(x,T(x))} = \norm{x} \leq \norm{x} + \norm{T(x)} = \norm{(x,T(x))}$
		\item $\pi$ es isomorfismo lineal
		
	\end{itemize}
	
	Entonces por \ref{Teorema de la aplicacion abierta} resulta que $\pi^{-1}$ es continua. Por lo tanto si $x_n \rightarrow x$ entonces:
	
	$$(x_n,T(x_n)) = \pi^{-1}(x_n) \rightarrow \pi^{-1}(x) = (x,T(x))$$
	
	Por lo que $T(x_n) \rightarrow T(x)$ y por \ref{Continuidad de un operador} $T \in L(X,Y)$.
	
	Rec\'iprocamente, por \ref{Lemma 1: Teo Grafico cerrado} y \ref{Lemma 2: Teo Grafico cerrado} vale que $T$ es cerrada. \qed
	
\end{proof}

\begin{example}
	Sea $\sett{a_n} \subset \C$ tal que para todo $p \in \left(1,\infty\right)$ toda sucesi\'on $x_n \in l^p$ cumple $\Bigsum{n \in \N}{a_n x_n} \in \C$ entonces $a_n \in l^q$
\end{example}

\begin{proof}
	Sea $T:l^p \rightarrow l^{\infty}$ dado por $T(x_n)(k) = \Bigsum{1 \leq n \leq k}{a_n x_n}$ y veamos que esta bien definida, es lineal y continua.
	
	\begin{enumerate}
		\item SI $x_n \in l^p$ entonces $\Bigsum{n \in \N}{a_n x_n}$ converge, en particular est\'a acotada por lo que $\sett{\Bigsum{1 \leq n k}{a_n x_n}}_{k \in \N} \in l^{\infty}$.
		\item Trivial
		\item Veamos que $T$ es cerrado:
		
		Sea $x_n \in l^p$, $x \in l^p$ e $y \in l^{\infty}$ tal que $(x_n, T(x_n)) \xrightarrow{\norm{.}_{l^p \times l^{\infty}}} (x,y)$, escribamos $x_n = \left(x_1^n, x_2^n, \dots, x_j^n, \dots\right)\in l^p$ y consideremos:
		
		\[
		\begin{aligned}
			\abs{T(x_n)(k) - \Bigsum{1 \leq j \leq k}{a_j x_j}} = & \abs{\Bigsum{1 \leq j \leq k}{a_j(x_j^n - x_j)}} \\
			& \leq \left(\Bigsum{1 \leq j \leq k}{\abs{x_j^n - x_j}^p}\right)^{\frac{1}{p}} \left(\Bigsum{1 \leq j \leq k}{\abs{a_j}}^q\right)^{\frac{1}{q}} \\
			& \leq \norm{x_n-x}_{p} C_a \xrightarrow{n \rightarrow \infty} 0
		\end{aligned}
		\]
		
		Es decir que $T(x_n)(k) \rightarrow \Bigsum{1 \leq j \leq k}{a_j x_j}$ para todo $k \in \N$; si y s\'olo si $T(x_n) \xrightarrow{\norm{.}_{\infty}} T(x)$ y por unicidad del l\'imite $T(x) = y$ y $T$ es cerrada.
		
		Finalmente como $T:l^p \rightarrow l^{\infty}$ es cerrada y ambos son Banach entonces por \ref{Teorema del grafico cerrado} $T \in L(l^p, l^{\infty})$
		
	\end{enumerate}
	
	Luego, si $x \in l^p$ entonces $\norm{T(x)}_{\infty} \leq \norm{T}\norm{x}_p$. Sea:
	
	\begin{equation*}
		x_j^k  =  \left\lbrace \begin{array}{cc}
									\overline{a_j}\abs{a_j}^{q-2} & \text{si } 1 \leq j \leq k \text{ y } a_j \neq 0 \\
									0 & \text{ si no}
								\end{array} \right.
	\end{equation*}
	
	Luego $a_jx_j = \abs{a_j}^q$ y $\norm{x}_p = \left(\Bigsum{1 \leq j \leq k}{\abs{a_j}^{(q-1)p}}\right)^{\frac{1}{p}} = \left(\Bigsum{1 \leq j \leq k}{\abs{a_j}^{q}}\right)^{\frac{1}{p}}$ y por lo tanto:
	
	\[
	\begin{aligned}
		\Bigsum{1 \leq j \leq k}{\abs{a_j}^q} & = \norm{T(x)}_{\infty} \\
		& \leq \norm{T}\left(\Bigsum{1 \leq j \leq k}{\abs{a_j}^{q}}\right)^{\frac{1}{p}} \\
		& \Longrightarrow \left(\Bigsum{1 \leq j \leq k}{\abs{a_j}^q}\right)^{\frac{1}{q}} \leq \norm{T} \qquad \forall \ k \in \N
	\end{aligned}
	\]
	
	Luego si $k \rightarrow \infty$ vemos que $a_n \in l^q$ y $\norm{a_n}_q \leq \norm{T}$. \qed
	
\end{proof}

\subsection{Principio de Acotaci\'on Uniforme}

\begin{theorem}[Banach-Steinhaus, Principio de Acotaci\'on Uniforme]
	\label{PAU operadores}
	Sean $X,Y$ Banach y sea $\sett{T_i}_{i \in I} \subset L(X,Y)$ tal que:
	
	\begin{equation*}
		\sup\limits_{\substack{i \in I}} \norm{T_i(x)}_Y < \infty \qquad \forall x \in X
	\end{equation*}
	
	Entonces $\sup\limits_{i \in I} \norm{T_i}_{L(X,Y)} < \infty$
	
\end{theorem}

\begin{proof}
	
	Sea $n \in N$ y consideremos $X_n = \sett{x \in X \tq \norm{T_i(x)}_Y \leq n \quad \forall i \in I}$, luego $X_n$ es cerrado y por hip\'otesis:
	
	\[
	X = \Bigcup{n \in \N}{X_n}
	\]
	
	Como $X$ es Banach por el teorema de Baire existe $n_0$ tal que $\interior{X_{n_0}} \neq \emptyset$. Sea $x_0 \in X, r > 0$ tal que $B_r(x_0) \subset \interior{X_{n_0}}$, luego $\norm{T(x_0 + rz)}_{Y} \leq n_0$ para todo $i \in I$ y $z \in B_1(0)$; por lo tanto:
	
	
	\begin{equation*}
		\norm{T_i}_{L(X,Y)} \leq \frac{1}{r} \left(n_0 + \norm{T_ix_0}_Y\right)
	\end{equation*}
	
	.\qed
	
\end{proof}

\begin{corollary}
	\label{PAU X}
	Sea $X$ un espacio de Banach y $F \subset X$ tal que:
	
	\begin{equation*}
		\sup\limits_{x \in F} \abs{f(x)} < \infty \quad \forall f \in \dual{X}
	\end{equation*}
	
	Entonces $F$ es acotado.
	
\end{corollary}

\begin{proof}
	Sea $Y = \dual{X}$, $Z=\R$ y $I = F$, luego para todo $x \in F$ consideremos $T_x : Y \rightarrow Z$ dado por:
	
	\[
		T_x(f) = f(x)
	\]
	
	Luego por hip\'otesis para todo $f \in Y = \dual{X}$:
	
	\[
	\begin{aligned}
		\sup\limits_{x \in F} \abs{T_x(f)} = \sup\limits_{x \in F} \abs{f(x)} < \infty
	\end{aligned}
	\]
	
	Por \ref{PAU operadores} vale que $\sup\limits_{x \in F} \norm{T_x} = C < \infty$, por lo que:
	
	\[
	\sup\limits_{x \in F} \norm{x} \underbrace{=}_{\ref{Norma en funcion de funcionales}} \sup\limits_{\substack{x \in F \\ \norm{f} = 1}} \abs{T_x(f)} \leq \sup\limits_{\substack{x \in F \\ \norm{f} = 1}} \norm{T_x}\norm{f} \leq \sup\limits_{\substack{x \in F }} \norm{T_x} = C < \infty
	\]
	.\qed
	
	
\end{proof}

\begin{corollary}
	\label{PAU X dual}
	Sea $X$ un espacio de Banach y $F \subset \dual{X}$ tal que
	
	\begin{equation*}
	\sup\limits_{f \in F} \abs{f(x)} < \infty \quad \forall x \in X
	\end{equation*}
	
	Entonces $F$ es acotado.
	
\end{corollary}

\begin{proof}
	Sea $Z=\R$ y $I = F$, luego para todo $f \in F$ consideremos $T_f : X \rightarrow Z$ dado por:
	
	\[
	T_f(x) = f(x)
	\]
	
	Luego por hip\'otesis para todo $x \in X$:
	
	\[
	\begin{aligned}
	\sup\limits_{f \in F} \abs{T_f(x)} = \sup\limits_{f \in F} \abs{f(x)} < \infty
	\end{aligned}
	\]
	
	Por \ref{PAU operadores} vale que $\sup\limits_{f \in F} \norm{T_f} = C < \infty$, por lo que:
	
	\[
	\sup\limits_{f \in F} \norm{f} = \sup\limits_{\substack{f \in F \\ \norm{x} = 1}} \abs{T_f(x)} \leq \sup\limits_{\substack{f \in F \\ \norm{x} = 1}} \norm{T_f}\norm{x} \leq \sup\limits_{\substack{f \in F }} \norm{T_f} = C < \infty
	\]
	.\qed
\end{proof}

\begin{corollary}
	\label{Si una sucesion converge debilmente entonces esta fuertemente acotada}
	Si $X$ Banach y $\sett{x_n}_{n \in \N} \subset X$ tal que $\lim\limits_{n \rightarrow \infty}{f(x_n)} < \infty$ para todo $f \in \dual{X}$ entonces $\sup\limits_{n \in \N}{\norm{x_n}} < \infty$
\end{corollary}

\begin{proof}
	Sea $F = \sett{x_n}_{n \in \N} \subset X$, luego si $\lim\limits_{n \rightarrow \infty}{f(x_n)} < \infty$ para todo $f \in \dual{X}$ entonces en particular $\sup\limits_{x_n \in F}{\abs{f(x_n)}} < \infty$ para todo $f \in \dual{X}$ y por \ref{PAU X} entonces $\sup\limits_{n \in \N}{\norm{x_n}} = \sup\limits_{x_n \in F}{\norm{x_n}} < \infty$. \qed
\end{proof}

\begin{corollary}
	\label{Si una sucesion converge debilmente estrella entonces esta fuertemente acotada}
	Si $X$ Banach y $\sett{f_n}_{n \in \N} \subset \dual{X}$ tal que $\lim\limits_{n \rightarrow \infty}{f_n(x)} < \infty$ para todo $x \in X$ entonces $\sup\limits_{n \in \N}{\norm{f_n}} < \infty$
\end{corollary}

\begin{proof}
	Sea $F = \sett{f_n}_{n \in \N} \subset \dual{X}$, luego si $\lim\limits_{n \rightarrow \infty}{f_n(x)} < \infty$ para todo $x \in X$ entonces en particular $\sup\limits_{f_n \in F}{\abs{f_n(x)}} < \infty$ para todo $x \in X$ y por \ref{PAU X dual} entonces $\sup\limits_{n \in \N}{\norm{f_n}} = \sup\limits_{f_n \in F}{\norm{f_n}} < \infty$. \qed
\end{proof}

\begin{corollary}
	\label{Sucesion de operadores que convergen puntualmente definen un operador limite puntual}
	Sean $X,Y$ Banach y sea $\sett{T_n}_{n \in \N} \subset L(X,Y)$ tal que para todo $x \in X$ existe $\lim\limits_{n \rightarrow \infty} T_n(x) = T(x)$, luego:
	
	\begin{enumerate}
		\item $\sup\limits_{n \in \N}{\norm{T_n}_{L(X,Y)}} < \infty$
		\item $T \in L(X,Y)$
		\item $\norm{T}_{L(X,Y)} \leq \liminf \norm{T_n}_{L(X,Y)}$
	\end{enumerate}
	
\end{corollary}

\begin{proof}
	Por \ref{PAU operadores} existe $C$ tal que:
	
	\[
	\begin{aligned}
		\norm{T_n(x)} & \leq C \norm{x} \\
		\xRightarrow{n \rightarrow \infty} & \norm{Tx} \leq C \norm{x} 
	\end{aligned}
	\]
	
	Luego eso prueba los primeros dos items pues por unicidad del l\'imite $T$ resulta lineal. Finalmente como $\norm{T_n(x)} \leq \norm{T_n}\norm{x}$ entonces eso implica que:
	
	$$\norm{T(x)} = \liminf \norm{T_n(x)} \leq \norm{x} \liminf \norm{T_n}$$
	
	Por lo que:
	
	$$\sup\limits_{\norm{x} \neq 0}{\dfrac{\norm{T(x)}}{\norm{x}}} \underbrace{=}_{\ref{Normas equivalentes de un operador}} \norm{T} \leq \liminf \norm{T_n}_{L(X,Y)}$$ \qed
	
\end{proof}


\section{Topolog\'ias d\'ebiles}

\subsection{Topolog\'ia D\'ebil: Definici\'on y propiedades}

\textbf{Notaci\'on:} Sea $X$ Banach y $x \in X, f \in \dual{X}$ entonces notamos:

$$\ip{x,f} = \ip{f,x} := f(x)$$


\begin{definition}
	Sea $X$ normado, la \textit{topolog\'ia d\'ebil} en $X$ es la topolog\'ia inicial respecto a la familia $\B = \sett{p_{\dual{x}} \tq \dual{x} \in \dual{X}}$, donde:
	
	\begin{equation*}
		p_{\dual{x}}(x) = \abs{\ip{x,\dual{x}}}
	\end{equation*}
	
	Y la notaremos $wk$ o $\sigma(X,\dual{X})$. 
	
\end{definition}

\begin{remark}
	$U \subset X$ es $wk-$abierto si y s\'olo si para todo $x_0 \in U$ existen $\epsilon > 0$ y $\dual{x_1}, \dots, \dual{x_n} \in \dual{X}$ tal que:
	
	\begin{equation*}
		\Bigcap{1 \leq k \leq n}{\sett{x \in X \tq p_{\dual{x_k}}(x-x_0) < \epsilon}} \subset U
	\end{equation*}
	
	Por lo que $\sett{x_i} \subset X$ converge a $x_0$ en esta topolog\'ia si y s\'olo si $\ip{x_i,\dual{x}} \rightarrow \ip{x_0,\dual{x}}$ para todo $\dual{x} \in \dual{X}$. En este caso, diremos que $x_i$ \textit{converge d\'ebilmente} a $x_0$ y lo notaremos $x_i \xrightarrow{d} x_0$, $x_i \xrightarrow{wk} x_0$ o $x_i \rightarrow x_0 \ (wk)$
\end{remark}	



\begin{proposition}
	\label{El dual a la topologia debil es la dual fuerte}
	Sea $X$ normado, entonces $\dual{X,wk} = \dual{X}$
\end{proposition}

\begin{proof}
	Sea $f \in \dual{X,wk}$ y $V \subset \mathbb{F}$ un abierto, luego $f^{-1}(V)$ es $wk$-abierto en $X$, luego al ser intersecci\'on e abiertos (pues $p_{\dual{x}}$ son continuas) $f^{-1}(V)$ es abierto fuertemente por lo que $f \in \dual{X}$.
	
	Rec\'iprocamente, sea $f \in \dual{X}$ y $\sett{x_i} \subset X$ una red tal que $x_i \ \xrightarrow{wk} \ x_0$, luego $f(x_i) \rightarrow f(x_0)$ si y s\'olo si $\ip{f(x_i),\dual{x}} \rightarrow \ip{f(x_0), \dual{x}}$ si y s\'olo si $p_{\dual{x}} \circ f(x_i) \rightarrow p_{\dual{x}} \circ f (x_0)$ que vale pues ambas son continuas. \qed
	
\end{proof}

\begin{proposition}
	\label{Topologia debil es Hausdorff}
	Sea $X$ normado, entonces $\sigma(X,\dual{X})$ es Hausdorff	
\end{proposition}

\begin{proof}
	Sea $x \neq y \in X$, como $\sett{x}, \sett{y}$ son espacios compactos por \ref{Un cerrado se separa de un compacto} existe $f \in \dual{X}$ y $\alpha \in \mathbb{F}$ tal que:
	
	
	\begin{equation*}
	\ip{f, x} < \alpha < \ip{f,y}
	\end{equation*}
	
	Sean $U = f^{-1}(\left(-\infty, \alpha\right)), V = f^{-1}(\left(\alpha, \infty\right))$ luego $U \cap V = \emptyset$ y $x \in U, y \in V$. \qed
	
\end{proof}

\begin{proposition}
	\label{Resultados de convergencia debil}
	Sean $X$ normado y $\sett{x_n}_{n \in \N} \subset X$ una sucesi\'on. Luego:
	
	
	\begin{enumerate}
		\item Si $x_n \xrightarrow{\norm{.}_X} x_0$ entonces $x_i \xrightarrow{d} x_0$.
		\item Si $x_i \xrightarrow{d} x_0$ entonces $\sett{\norm{x_n}}$ est\'a acotado y $\norm{x_0} \leq \liminf \norm{x_n}$
		\item Si $x_i \xrightarrow{d} x_0$  y $f_n \xrightarrow{\norm{.}_{\dual{X}}} f_0$ entonces $f_n(x_n) \rightarrow f_0(x_0)$
	\end{enumerate}
	
\end{proposition}

\begin{proof}
	Vamos por partes:
	
	\begin{enumerate}
		\item Sea $f \in \dual{X}$, luego $\abs{\ip{f(x_n) - f(x_0)}} \leq \norm{f}_{\dual{X}} \norm{x_n-x_0}_X \rightarrow 0$; por lo tanto $x_i \xrightarrow{d} x_0$.
		\item Sea $T_n : \dual{X} \rightarrow \R$ dado por $T_n(f) = f(x_n)$, luego $\lim\limits_{n \rightarrow \infty}{T_n(f)} = f(x_0) = T(f)$ para todo $f \in \dual{X}$; como $\dual{X}, \R$ son Banach entonces por \ref{PAU operadores} vale la conclusi\'on.
		\item $\abs{\ip{f_n,x_n} - \ip{f_0,x_0}} = \abs{\ip{f_n,x_n} - \ip{f_0,x_n} + \ip{f_0,x_n} - \ip{f_0,x_0}} \leq \abs{\ip{f_n -f_0,x_n}} + \abs{\ip{f_0,x_n - x_0}} \leq \underbrace{\norm{x_n}_X}_{\leq C} \underbrace{\norm{f_n-f_0}_{\dual{X}}}_{\rightarrow 0} + \underbrace{\abs{\ip{f_0,x_n - x_0}}}_{\rightarrow 0} \rightarrow 0$. \qed
	\end{enumerate}
	
\end{proof}

\subsection{Conjuntos convexos y operadores}

\begin{proposition}
	\label{Clausura fuerta es clausura debil en un convexo}
	Si $X$ es normado y $A \subset X$ es convexo entonces $\overline{A}^{\norm{.}} = \overline{A}^{wk}$
\end{proposition}

\begin{proof}
	Sea $x \in \left(\overline{A}^{\norm{.}}\right)^c$, luego por \ref{Dos convexos en un normado se separan si uno es abierto} existe $\dual{x} \in \dual{X}$ y $\alpha \in \mathbb{R}$ tal que para todo $a \in \overline{A}^{\norm{.}}$ vale:
	
	\begin{equation*}
	\abs{\ip{a,\dual{x}}} < \alpha < \abs{\ip{x,\dual{x}}}
	\end{equation*}
	
	Luego $\overline{A}^{\norm{.}} \subseteq B = \sett{y \in X \tq 	\abs{\ip{y,\dual{x}}} \leq \alpha}$ que es $wk$ cerrado; por lo tanto $\overline{A}^{wk} \subseteq B$ y entonces $x \in \left(\overline{A}^{wk}\right)^c$; conclu\'imos que $\overline{A}^{wk} \subseteq \overline{A}^{\norm{.}}$ y como la rec\'iproca es trivial finalizamos. \qed
\end{proof}

\begin{corollary}[Teorema de Mazur]
	\label{Teorema de Mazur}
	Sea $X$ normado y $\sett{x_n}_{n \in \N} \subset X$ una sucesi\'on tal que $x_i \xrightarrow{d} x$, entonces existe $y_n$ hecho de combinaciones convexas de los $x_n$ tal que $y_n \xrightarrow{\norm{.}} x$. 
\end{corollary}

\begin{proof}
	Sea $C = conv \left(\Bigcup{n \in \N}{\sett{x_n}}\right)$ que es convexo, luego por hip\'otesis $x \in \overline{C}^{wk}$ y por \ref{Clausura fuerta es clausura debil en un convexo} $x \in \overline{C}^{\norm{.}}$. \qed
\end{proof}

\begin{corollary}
	Sea $\phi : X \rightarrow \R \cup \sett{\infty}$ convexa y semicontinua inferiormente, luego $\phi$ es $wk$ semi continua inferiormente.
\end{corollary}

\begin{proof}
	Sea $\lambda \in \R$, luego $A = \phi^{-1}\left((-\infty, \lambda]\right)$ es convexo y fuertemente cerrado, luego por \ref{Clausura fuerta es clausura debil en un convexo} es convexo y $wk$ cerrado. \qed
	
\end{proof}

\begin{corollary}
	\label{Convexa y fuertemente continua es debilmente semicontinua inferiormente}
	Sea $\phi : X \rightarrow \R \cup \sett{\infty}$ convexa y continua, luego $\phi$ es $wk$ semi continua inferiormente.
\end{corollary}

\begin{theorem}
	\label{T es fuerte-fuerte continua entonces es debil-debil continua}
	Sean $X,Y$ Banach. Luego $T \in L(X,Y)$ si y s\'olo si $T \in L(\left(X,wk\right),\left(Y,wk\right))$
\end{theorem}

\begin{proof}
	Supongamos que $T \in L(X,Y)$, luego por la propiedad universal de la topolog\'ia inicial basta ver que $p_{\dual{y}} \circ T : \left(X,wk\right) \rightarrow \R$ dado por $x \mapsto \ip{y,T(x)}$ es continua para todo $y \in Y$. Como $T \in L(X,Y)$, $p\in \dual{Y}$ y \ref{El dual a la topologia debil es la dual fuerte} conclu\'imos que $p_{\dual{y}} \circ T \in \dual{\left(X,wk\right)}$.
	
	Rec\'iprocamente, si $T \in L(\left(X,wk\right),\left(Y,wk\right))$ entonces por \ref{Teorema del grafico cerrado} $\graf T$ es cerrado en:
	
	\begin{equation*}
	\left(X \times Y, \sigma(X,\dual{X}) \times \sigma\left(Y,\dual{Y}\right)\right) = \left(X \times Y, \sigma(X \times Y,\dual{X \times Y})\right)
	\end{equation*}
	
	y por lo tanto como $\tau_{wk} \subset \tau_{\norm{.}}$ entonces $\graf T$ es fuertemente cerrado; finalmente por \ref{Teorema del grafico cerrado} conclu\'imos que $T \in L(X,Y)$. \qed
\end{proof}

\subsection{Topolog\'ia D\'ebil Estrella: Definici\'on y Propiedades}


\begin{definition}
	Sea $X$ normado, la \textit{topolog\'ia d\'ebil estrella} en $\dual{X}$ es la topolog\'ia inicial respecto a la familia $\F = \sett{p_x \tq x \in X}$ donde:
	
	\begin{equation*}
	p_x(\dual{x}) = \abs{\ip{x,\dual{x}}}
	\end{equation*}
	
	Y la notaremos $\dual{wk}$ o $\sigma(\dual{X},X)$. 
	
\end{definition}
	
\begin{remark}
		
		$U \subset \dual{X}$ es $\dual{wk}-$abierto si y s\'olo si para todo $\dual{x_0} \in U$ existen $\epsilon > 0$ y $x_1, \dots, x_n \in X$ tal que:
		
		\begin{equation*}
		\Bigcap{1 \leq k \leq n}{\sett{\dual{x} \in \dual{X} \tq p_{x_k}(\dual{x}-\dual{x_0}) < \epsilon}} \subset U
		\end{equation*}
		
		Por lo que $\sett{\dual{x_i}} \subset \dual{X}$ converge a $\dual{x_0}$ en esta topolog\'ia si y s\'olo si $\ip{\dual{x_i},x} \rightarrow \ip{\dual{x_0},x}$ para todo $x \in X$. En este caso, diremos que $\dual{x_i}$ \textit{converge d\'ebil estrella} a $\dual{x_0}$ y lo notaremos $\dual{x_i} \xrightarrow{\dual{d}} \dual{x_0}$, $\dual{x_i} \xrightarrow{\dual{wk}} \dual{x_0}$ o $\dual{x_i} \rightarrow \dual{x_0} \ (\dual{wk})$
\end{remark}

\begin{proposition}
	\label{El dual a la topologia debil estrella es X}
	Sea $X$ normado, entonces $\dual{(\dual{X}, \dual{wk})} \simeq X$
\end{proposition}

\begin{proof}
	Sea $x \in X$, afirmo que $ev_x \in dual{(\dual{X}, \dual{wk})}$ pues si $\dual{x_i} \xrightarrow{\dual{d}} \dual{x_0}$ entonces por definici\'on $ev_x (\dual{x_i}) := \ip{\dual{x_i}, x} \rightarrow \ip{\dual{x_0}, x} := ev_x(\dual{x_0})$. Por lo tanto si notamos $J:X \rightarrow \ddual{X}$ a la inyecci\'on can\'onica tenemos que $X \simeq J(X) \subseteq \dual{(\dual{X}, \dual{wk})}$.
	
	Rec\'iprocamente, sea $f \in \dual{(\dual{X}, \dual{wk})}$, luego $U = \sett{\dual{x} \in \dual{X} \tq \abs{f(\dual{x})} < \delta} \in \tau_{\dual{wk}}$, si y s\'olo si:existe $\epsilon > 0$ y $x_1 , \dots, x_n \in X$ tal que:
	
	\begin{equation*}
	\Bigcap{1 \leq k \leq n}{\sett{\dual{x} \in \dual{X} \tq \dual{x}(x_k) < \epsilon}} := \Bigcap{1 \leq k \leq n}{\sett{\dual{x} \in \dual{X} \tq ev_{x_k}(\dual{x}) < \epsilon}}  \subset U
	\end{equation*}
	
	Luego vale que $\Bigcap{1 \leq k \leq n}{\ker x_k} \subseteq \ker f$. Notemos el siguiente resultado:
	
	
	
	\begin{lemma}
		\label{Lemma: El dual a la topologia debil estrella es X}
		Sea $f, f_1, \dots, f_n$ funcionales lineales en $X$ tal que $\Bigcap{1 \leq k \leq n}{\ker x_k} \subseteq \ker f$, entonces existen $\alpha_1, \dots, \alpha_n$ tal que $f = \Bigsum{1 \leq i \leq n}{\alpha_i f_i}$
	\end{lemma}
	
	
	
	\begin{proof}[Demostraci\'on del lema]
		Sea $l : X \rightarrow \R^n$ dado por $l(x) = \left(f_1(x), \dots, f_n(x)\right)$ y  sea $V = \rank l$. Luego $\ker l = \Bigcap{1 \leq k \leq n}{\ker x_k} $ y existe $\phi : V \rightarrow \R$ tal que $f = \phi \circ l$ dado por $\phi(v) = f(x)$ donde $x \in l^{-1}(v)$ que esta bien definido por hip\'otesis.
		
		Por \ref{La extension del funcional preserva norma} existe $\tilde{\phi} \in \dual{\R^k}$ y $\tilde{\phi}(y) = \Bigsum{1 \leq i \leq n}{\alpha_i y_i}$. Por lo tanto existen $\alpha_1 , \dots , \alpha_n$ tal que $f(x) = \phi \circ l (x) = \phi \left(f_1(x), \dots, f_n(x)\right) = \tilde{\phi} \left(f_1(x), \dots, f_n(x)\right) = \Bigsum{1 \leq i \leq n}{\alpha_i f_i(x)}$. \qed
		
	\end{proof}
	
	
	
	Por \ref{Lemma: El dual a la topologia debil estrella es X} existen $\alpha_1 , \dots, \alpha_n \in \C$ tal que $f = \Bigsum{1 \leq i \leq n}{\alpha_i ev_{x_i}} \in J(X)$. Por lo tanto $\dual{(\dual{X}, \dual{wk})} \simeq X$. \qed
	
\end{proof}

\begin{corollary}
	\label{Caracterizacion de los funcionales del dual con debil estrella}
	Sea $\phi \in \dual{\dual{X}, \dual{wk}}$, entonces existe un \'unico $x_0 \in X$ tal que $\phi(f) = \ip{f,x_0}$
\end{corollary}

\begin{proposition}
	\label{Topologia debil estrella es Hausdorff}
	Sea $X$ normado, entonces $\sigma(\dual{X},X)$ es Hausdorff	
\end{proposition}

\begin{proof}
	Sea $f \neq g \in \dual{X}$, luego existe $x \in X$ y $\alpha \in \mathbb{R}$ tal que:
	
	
	\begin{equation*}
	\ip{f, x} < \alpha < \ip{g,x}
	\end{equation*}
	
	Sean $U = p_x^{-1}(\left(-\infty, \alpha\right)), V = p_x^{-1}(\left(\alpha, \infty\right))$ luego $U \cap V = \emptyset$ y $f \in U, g \in V$. \qed
	
\end{proof}

\begin{proposition}
	\label{Resultados de convergencia debil estrella}
	Sean $X$ Banach y $\sett{f_n}_{n \in \N} \subset \dual{X}$ una sucesi\'on. Luego:
	
	
	\begin{enumerate}
		\item Si $f_n \xrightarrow{\norm{.}_{\dual{X}}} f$ entonces $f_n \xrightarrow{d} f$.
		\item Si $f_n \xrightarrow{d} f$ entonces $f_n \xrightarrow{\dual{d}} f$.
		\item Si $f_n \xrightarrow{\dual{d}} f$ entonces $\sett{\norm{f_n}}$ est\'a acotado y $\norm{f} \leq \liminf \norm{f_n}$
		\item Si $f_n \xrightarrow{\dual{d}} f$  y $x_n \xrightarrow{\norm{.}_{X}} x$ entonces $f_n(x_n) \rightarrow f(x)$
	\end{enumerate}
	
\end{proposition}

\begin{proof}
	Vamos por partes:
	
	\begin{enumerate}
		\item Sea $x \in \ddual{X}$, luego $\abs{\ip{x(f_n) - x(f_0)}} \leq \norm{x}_{\ddual{X}} \norm{f_n-f_0}_{\dual{X}} \rightarrow 0$; por lo tanto $f_n \xrightarrow{d} f$.
		\item Sea $x \in X$, luego $\abs{\ip{f_n(x) - f(x)}} \underbrace{=}_{J:X \rightarrow \ddual{X}} \abs{\ip{J(x)(f_n) - J(x)(f_0)}} \rightarrow 0$; por lo tanto $f_n \xrightarrow{\dual{d}} f$.
		\item Sea $T_n : X \rightarrow \R$ dado por $T_n(x) = f_n(x)$, luego $\lim\limits_{n \rightarrow \infty}{T_n(x)} = f(x) = T(f)$ para todo $x \in X$; como $X, \R$ son Banach entonces por \ref{PAU operadores} vale la conclusi\'on.
		\item 
		\begin{equation*}
		\begin{aligned}
		\abs{\ip{f_n,x_n} - \ip{f,x}} = & \abs{\ip{f_n,x_n} - \ip{f_n,x} + \ip{f_n,x} - \ip{f,x}} \leq \abs{\ip{f_n ,x_n-x}} + \abs{\ip{f_n - f,x}} \\
		\leq & \underbrace{\norm{f_n}_{\dual{X}}}_{\leq C} \underbrace{\norm{x_n-x}_{X}}_{\rightarrow 0} + \underbrace{\abs{\ip{f_n -f,x}}}_{\rightarrow 0} \rightarrow 0
		\end{aligned}
		\end{equation*}\qed
	\end{enumerate}
	
\end{proof}

\subsection{El teorema de Alaoglu}

\begin{theorem}[Teorema de Alaoglu]
	\label{Teorema de Alaoglu}
	Sea $X$ normado, luego $B_1(0) \subset \dual{X}$ es $\dual{wk}$ compacta.
\end{theorem}

\begin{proof}
	Sea $x \in X$, llamemos $B=B_1(0) \subset \dual{X}$, $D_x = \sett{\alpha \in \mathbb{F} \tq \abs{\alpha} \leq 1}$ y sea $D = \Bigprod{x \in X}{D_x}$ que por Tychonoff es compacto. Sea $\tau: B \rightarrow D$ dado por:
	\begin{equation*}
	\tau \left(f\right)(x) \ip{f, x}
	\end{equation*}
	
	Afirmo que $\tau$ es un homeomorfismo de $\left(B,\dual{wk}\right)$ a $\tau(B)$ como subespacio de $D$.
	
	En efecto, si $\tau(f) = \tau(g)$ entonces $f(x) = \ip{f,x} = \ip{g,x} = g(x)$ para todo $x \in X$ por lo que $f = g$. Por otro lado, si $f_n \xrightarrow{\dual{d}} f$ entonces $\tau(f_n)(x) = \ip{f_n,x} \rightarrow \ip{f,x} = \tau(f)(x)$ para todo $x \in X$ si y s\'olo si $\tau(f_n) \xrightarrow{\tau_{prod}} \tau(f)$ por lo que $\tau$ es continua. Finalmente $\tau^{-1}: \tau(B) \rightarrow \left(B, \dual{wk}\right)$ es continua si y s\'olo si $p_{x} \circ \tau^{-1} : \tau(B) \rightarrow \R$ es continua para todo $x \in X$ por la propiedad universal de la topolog\'ia inicial; pero $p_{x} \circ \tau^{-1}(w) = \ip{\tau^{-1}(w), x} \underbrace{=}_{w = \tau(f)} \ip{f,x} = w_x$ por lo que $p_{x} \circ \tau^{-1} = \pi_x$ la proyecci\'on a la $x$ coordenada que es continua pues $\tau(B) \inc D$ que tiene la topolog\'ia producto. Conclu\'imos que $\tau : \left(B,\dual{wk}\right) \rightarrow \tau(B)$ es un homeomorfismo.
	
	Finalmente afirmo que $\tau(B)$ es $\tau_{prod}$ cerrado.
	
	Sean $\sett{f_i}_{i \in I} \subset B$ y $g \in D$ tal que $\tau(f_i) \rightarrow g$ por lo que $h(x) = \lim \ip{f_i,x}$ esta bien definido y $h \in \dual{X}$. Adem\'as, si $\norm{x} \leq 1$ entonces $\abs{h(x)} \leq 1$ pues $\ip{f_i,x} \in D_x$ por lo tanto $h \in B$ y $\tau(h) = g$. Esto demuestra que $\tau(B)$ es $\tau_{prod}$ cerrado.
	
	Como $D$ es compacto y $\tau(B)$ es cerrado entonces $\tau(B)$ es compacto y como $\tau$ es homeomorfismo conclu\'imos que $B$ es compacto.\qed
	
\end{proof}

\subsection{El teorema de Kukatani}

\textbf{Notaci\'on:} Si $X$ es normado entonces notaremos $\ball X$ a la bola cerrada unitaria, ie: $ball X = \sett{x \in X \tq \norm{x} \leq 1}$

\begin{proposition}
	\label{Lemma: Teorema de Kukatani}
	Si $X$ es normado entonces $\ball X$ es $\sigma\left(\ddual{X}, \dual{X}\right)$ densa en $\ball \ddual{X}$
\end{proposition}

\begin{proof}
	Sea $B = \overline{J(\ball X)}^{\sigma(\ddual{X}, \dual{X})} \subset \ball \ddual{X}$ y supongamos que existe $f_0 \in \ball \ddual{X} \setminus B$,  luego por \ref{Dos convexos en un normado se separan si uno es abierto} existe $\dual{x} \in \dual{X}$ y $\alpha \in \R$ tal que para todo $x \in \ball X$ vale que:
	
	\[
	\text{Re} {\ip{x,\dual{x}}} < \alpha < \text{Re}\ip{\dual{x}, f}
	\]
	
	Reemplazando $\dual{x}$ por $\alpha^{-1} \dual{x}$ obtenemos que:
	
	\[
	\text{Re} {\ip{x,\dual{x}}} < 1 < \text{Re}\ip{\dual{x}, f}
	\]
	
	Como $e^{i\theta}x \in \ball X$ si $x \in \ball X$ entonces $\abs{\ip{x,\dual{x}}} \leq 1$ si $\norm{x} \leq 1$ entonces $\dual{x} \in \ball \dual{X}$. Pero entonces $1 < \text{Re} \ip{\dual{x}, f} \leq \abs{\ip{\dual{x}, f}} \leq \norm{f} \leq 1$ por lo que $\overline{J(\ball X)}^{\sigma(\ddual{X}, \dual{X})} = \ball \ddual{X}$\qed
	
\end{proof}

\begin{theorem}[Teorema de Kukatani]
	\label{Teorema de Kukatani}
	Sea $X$ Banach, entonces las siguientes son equivalentes:
	
	\begin{enumerate}
		\item $X$ es reflexivo
		\item $\dual{X}$ es reflexivo
		\item $\sigma\left(\dual{X}, X\right) = \sigma \left(\dual{X}, \ddual{X}\right)$
		\item $\ball X$ es $wk$ compacto
	\end{enumerate}
	
\end{theorem}

\begin{proof}
	Vayamos por partes:
	
	\begin{itemize}
		
		\item[$(1) \Longrightarrow (3)$] Es claro pues $X = \ddual{X}$.
		\item[$(4) \Longrightarrow (1)$] Afirmo que $\sigma \left(\ddual{X}, \dual{X}\right) \vert_{X} = \sigma \left(X, \dual{X}\right)$
		
		En efecto, $U \ni 0$ es un entorno de $0$ referido a $\tau_{\left(\ddual{X}, \dual{wk}\right)} \vert_{J(X)}$ si y s\'olo si existen $\epsilon > 0$ y $f_1 , \dots , f_n \in \dual{X}$ tal que $\Bigcap{1 \leq k \leq n}{\sett{\ddual{x} \in J(X) \subset \ddual{X} \tq {Jx}(f_k) < \epsilon)}} \subset U$ si y s\'olo si existen $\epsilon > 0$ y $f_1 , \dots , f_n \in \dual{X}$ tal que $\Bigcap{1 \leq k \leq n}{\sett{x \in X \tq p_{f_k}(x) < \epsilon)}} \subset U$ si y s\'olo si $U$ es un entorno de $0$ referido a $\tau_{\sigma \left(X, \dual{X}\right)}$
		
		Luego, por hip\'otesis y la afirmaci\'on $\ball X$ es $\sigma(\ddual{X}, \dual{X})$ cerrada en $\ball \ddual{X}$ y por \ref{Lemma: Teorema de Kukatani} adem\'as es densa; por lo tanto $\ball X = \ball \ddual{X}$ y $X$ es reflexivo.
		
		\item[$(3) \Longrightarrow (2)$] Por \ref{Teorema de Alaoglu} $\ball \dual{X}$ es $\sigma \left(\dual{X}, X\right)$ compacto y por hip\'otesis entonces es $\sigma \left(\dual{X}, \ddual{X}\right)$ compacta. Por $(4) \Longrightarrow (1)$ vale que $\dual{X}$ es reflexivo.
		
		\item[$(2) \Longrightarrow (1)$] Como $\ball X$ es cerrado en $\ddual{X}$ (en realidad $J(\ball X)$) por \ref{Clausura fuerta es clausura debil en un convexo} es $\sigma \left(\ddual{X}, \dual{\ddual{X}}\right)$ cerrado. Por hip\'otesis $\dual{X} = \dual{\ddual{X}}$ por lo tanto $\ball X$ es $\sigma \left(\ddual{X}, \dual{X}\right)$ cerrado en $\ddual{X}$, como por \ref{Lemma: Teorema de Kukatani} $\ball X$ es $\sigma \left(\ddual{X}, \dual{X}\right)$ denso en $\ball \ddual{X}$ entonces $\ball X  = \ball \ddual{X}$ y $X$ es reflexivo.
		
		\item[$(1) \Longrightarrow (4)$] Por \ref{Teorema de Alaoglu} $\ball \ddual{X}$ es $\sigma \left(\ddual{X}, \dual{X}\right)$ compacta y por hip\'otesis $\ddual{X} = X$ por lo que $\ball X$ es $\sigma \left(X, \dual{X}\right)$ compacta. \qed
		
	\end{itemize}
	
\end{proof}


\begin{corollary}
	\label{Subespacio de Reflexivo es reflexivo}
	Sea $X$ Banach reflexivo y $M \subseteq X$ un subespacio cerrado, entonces $M$ es un espacio de Banach reflexivo.
\end{corollary}

\begin{proof}
	Como $\ball M = M \cap \ball X$ entonces $\ball M$ es $\sigma(X, \dual{X})$ compacto por \ref{Teorema de Kukatani}, queda ver que $\sigma \left(X, \dual{X}\right) \vert_{M}  = \sigma(M, \dual{M})$.
	
	Sea $U \ni 0$ un entorno de $0$ referido a $\tau_{\sigma(M, \dual{M})}$ sy i s\'olo si existe $\epsilon > 0$ y $f_1,\dots, f_n \in \dual{M}$ tal que $\Bigcap{1 \leq k \leq n}{\sett{m \in M \tq f_i(m) < \epsilon}} \subset U$, debido a \ref{La extension del funcional preserva norma} sean $F_1, \dots, F_n \in \dual{X}$ las extensiones respectivas, luego $\Bigcap{1 \leq k \leq n}{\sett{m \in M \tq f_i(m) < \epsilon}} = \Bigcap{1 \leq k \leq n}{\sett{m \in M \tq F_i(m) < \epsilon}} \in \tau_{\sigma \left(X, \dual{X}\right) \vert_{M}}$. \qed
	
\end{proof}

\begin{corollary}
	\label{Banach reflexivo es debilmente secuencialmente completo}
	Sea $X$ Banach reflexivo, entonces es $wk$ secuencialmente completo.
\end{corollary}

\begin{proof}
	Sea $\sett{x_n}_{n \in \N} \subset X$ una sucesi\'on $wk$ de Cauchy, luego $\sett{\ip{x_n,f}} \subset \mathbb{F}$ es una sucesi\'on de Cauchy (y por ende converge) para todo $f \in \dual{X}$. Por lo tanto $\sett{x_n}$ es $wk$ acotado y por \ref{Si una sucesion converge debilmente entonces esta fuertemente acotada} existe $M$ tal que $\norm{x_n}_X \leq M$ para todo $n \in \N$; no obstante $\sett{x \in X \tq \norm{x} \leq M} = M \ball X$ es $wk$ compacto por \ref{Teorema de Kukatani} por lo que existe $x \in X$ y $\left(x_n\right)_k$ subsucesi\'on tal que $x_{n_k} \xrightarrow[cl]{d} x$ y como para todo $f \in \dual{X}$ $\lim \ip{f,x_n} < \infty$ entonces $x_{n_k} \xrightarrow{d} x$. \qed
\end{proof}

\begin{corollary}
	\label{En banach reflexivo se realiza la distancia}
	Sea $X$ Banach reflexivo, $M \subseteq X$ un subespacio cerrado y $x_0 \in M^c$, entonces existe $y_0 \in M$ tal que $d(x_0, M) = \norm{x_0-y_0}$
\end{corollary}

\begin{proof}
	Por \ref{Convexa y fuertemente continua es debilmente semicontinua inferiormente} se tiene que $x \mapsto \norm{x-x_0}$ es $wk$ semicontinua inferiormente y si $d = d(x_0,M)$ entonces $M \cap B_{2d}(x_0)$ es $wk$ compacto por \ref{Teorema de Kukatani} y \ref{Subespacio de Reflexivo es reflexivo}; conclu\'imos porque una funci\'on semicontinua inferiormente alcanza el m\'inimo en un compacto.\qed
\end{proof}

\begin{corollary}
	\label{Cerrado, acotado y convexo es debilmente compacto}
	Sea $X$ Banach reflexivo y sea $K \subseteq X$ cerrado, acotado y convexo; luego $K$ es $\sigma \left(X,\dual{X}\right)$ compacto.
\end{corollary}

\begin{proof}
	Como es cerrado y convexo entonces por $\ref{Clausura fuerta es clausura debil en un convexo}$ es $wk$ cerrado, y por ser acotado $K \subset M \ball X$ para alg\'un $M$. Por \ref{Teorema de Kukatani} $M \ball X$ es $wk$ compacto y como $K$ es $wk$ cerrado en un $wk$ compacto, es $wk$ compacto. \qed
\end{proof}

\subsection{Separabilidad}


\begin{theorem}
	\label{Si el dual es separable entonces X lo es}
	Sea $X$ Banach tal que $\dual{X}$ es separable, entonces $X$ es separable
\end{theorem}

\begin{proof}
	Sea $\sett{f_n}_{n \in \N} \subset \dual{X}$ un conjunto denso numerable, y sea $\sett{x_n}_{n \in \N} \subset X$ el conjunto de los $x_n$ tal que $\norm{x_n}_{X} = 1$ y cumple $\ip{f_n,x_n} \geq \frac{1}{2} \norm{f_n}_{\dual{X}}$; a su vez sea $L_0 = \sett{\Bigsum{J \text{ finito}}{\alpha_j x_j} \tq \alpha_j \in \Q, \ J \subset \N} = \ip{\sett{x_n}_{n \in \N}}_{\Q}$ el espacio vectorial generado sobre $\Q$ de las $x_n$.
	
	Es sabido que $L_0$ es numerable y si llamamos $L = \ip{\sett{x_n}_{n \in \N}}_{\R}$ entonces $L_0$ es denso en $L$; afirmo que $L$ es denso en $X$ a su vez.
	
	En efecto, sea $f \in \dual{X}$ tal que $f \vert_{L} = 0$ y sea $\epsilon > 0$, entonces existe $N \in \N$ tal que $\norm{f - f_N}_{\dual{X}} < \epsilon$ y adem\'as:
	
	\begin{equation*}
		\frac{1}{2} \norm{f_N}_{\dual{X}} \leq \ip{f_N, x_N} = \ip{f_N -f, x_N} < \epsilon
	\end{equation*}
	
	Por lo tanto $\norm{f}_{\dual{X}} \leq \norm{f-f_N}_{\dual{X}} + \norm{f_N}_{\dual{X}} < 3\epsilon$ y conclu\'imos que $f = 0$. De \ref{M es denso si el unico funcional que lo anula es el nulo} conclu\'imos que $X$ es separable \qed
	 
\end{proof}

\begin{remark}
	La vuelta no vale pues $X=L^1$ es separable pero su dual $L^{\infty}$ no lo es.
\end{remark}

\begin{corollary}
	\label{X es reflexivo y separable si y solo si dual es reflexivo y separable}
	Sea $X$ un Banach, entonces $X$ es reflexivo y separable si y s\'olo si $\dual{X}$ es reflexivo y separable
\end{corollary}

\begin{proof}
	Por \ref{Teorema de Kukatani} y \ref{Si el dual es separable entonces X lo es} conclu\'imos que si $\dual{X}$ es reflexivo y separable entonces $X$ lo es. Rec\'iprocamente, si $X$ es reflexivo y separable entonces $\ddual{X}$ lo es y por la anterior implicaci\'on $\dual{X}$ lo es. \qed
\end{proof}

\begin{theorem}[Metrizabilidad de la topolog\'ia d\'ebil estrella]
	\label{Separable si y solo si la bola dual es metrizable estrella}
	Sea $X$ Banach, entonces $X$ es separable si y s\'olo si $\ball \dual{X}$ es $\dual{wk}$ metrizable
\end{theorem}

\begin{proof}
	Sea $\sett{x_n}_{n \in \N} \subset \ball X$ un conjunto denso y numerable de $\ball X$ y para cada $f \in \dual{X}$ definamos:
	
	\begin{equation*}
		[f] = \Bigsum{n \in \N}{\dfrac{1}{2^n}{\abs{\ip{f, x_n}}}}
	\end{equation*}
	
	De C\'alculo Avanzado sabemos que $[.]$ es una norma en $\dual{X}$ y que $[f] \leq \norm{f}_{\dual{X}}$; llamemos $d(f-g) = [f-g]$ la correspondiente m\'etrica y veamos que $\left(\ball \dual{X}, \tau_d\right) = \left(\ball \dual{X}, \dual{wk}\right)$.
	
	Sea $f_0 \in \ball \dual{X}$ y $V \ni f_0$ un entorno seg\'un $\dual{wk}$ y asumamos que $V$ es b\'asico, ie:
	
	\[
		V = \sett{f \in \ball \dual{X} \tq \abs{\ip{f - f_0, y_i}} < \epsilon}
	\]
	
	Para algunos $\epsilon > 0$ y $y_1, \dots, y_k \in X$; es m\'as asumamos que $\norm{y_i}_X \leq 1$ y que (bajo reordenamiento de los \'indices de ser necesario) $\norm{y_i - x_{n_i}}_X < \frac{\epsilon}{4}$. Sea $r > 0$ tal que $2^{n_i}r < \frac{\epsilon}{2}$ para todo $1 \leq i \leq k$ y afirmo que $U = B^{d}_r(f_0) \subset V$ donde la bola es tomada seg\'un $d$ (de ah\'i el superscripto $d$).
	
	En efecto, si $d(f,f_0) < r$ entonces:
	
	\begin{equation*}
	\begin{array}{ccc}
		& \dfrac{1}{2^{n_i}} \abs{\ip{f-f_0,x_{n_i}}} < r & 1 \leq i \leq k \\
		\Longrightarrow & \abs{\ip{f-f_0, y_i}} = \abs{\ip{f-f_0, y_i-x_{n_i}}} + \abs{\ip{f-f_0,x_{n_i}}} < \epsilon & 1 \leq i \leq k
	\end{array}
	\end{equation*} 
	
	Y conclu\'imos que $f \in V$.
	
	Para el otro lado, sea $r > 0$ y encontremos $V \in \tau_{\dual{wk}}$ tal que $f_0 \in V \subset B^{d}_{r}(f_0)$; para esto afirmo que $V = \sett{f \in \ball \dual{X} \tq \abs{\ip{f - f_0, x_i}} < \epsilon \quad 1 \leq  i\leq k}$ con $k,\epsilon$ a determinar. Si $f \in V$ entonces:
	
	\begin{equation*}
		d(f,f_0)  =  \Bigsum{1 \leq  n \leq k}{\dfrac{1}{2^{n}}\abs{\ip{f-f_0, x_n}}} + \Bigsum{n \geq k+1}{\dfrac{1}{2^{n}}\abs{\ip{f-f_0, x_n}}} < \epsilon + 2 \Bigsum{n \geq  k+1}{\dfrac{1}{2^n}} = \epsilon + \dfrac{1}{2^{k-1}}
	\end{equation*}
	
	Por ende si $\epsilon = \dfrac{r}{2}$ y $k$ es tal que $\dfrac{1}{2^{k-1}} < \dfrac{r}{2}$ conclu\'imos que $f_0 \in V \subset U$; conclu\'imos que si $X$ es separable entonces $\ball \dual{X}$ es metrizable.
	
	Rec\'iprocamente, supongamos que $\ball \dual{X}$ es $\dual{wk}$ metrizable y sea:
	
	\begin{equation*}
		U_n  =  \sett{f \in \ball X \tq d(f,0) < \frac{1}{n}}
	\end{equation*}
	
	Y sea $V_n \ni 0$ un entorno de $0$ respecto a $\dual{wk}$ tal que $V_n \subset U_n$. Asumamos que:
	
	\begin{equation*}
		V_n = \sett{f \in \ball \dual{X} \tq \abs{\ip{f,x_n}} < \epsilon_n \quad \forall x \in \Phi_n}
	\end{equation*}
	
	Donde $\epsilon_n > 0$ y $\Phi_n \subset X$ es finito. Sea $D = \Bigcup{n \in \N}{\Phi_n}$ y sea $f \in \dual{X}$ tal que $f \vert_{D} = 0$; entonces $f \in V_n$ para todo $n \in \N$ por lo que $f \in U_n$ para todo $n \in \N$. Por ende, $f = 0$ y por \ref{M es denso si el unico funcional que lo anula es el nulo} conclu\'imos que $D$ es denso en $X$ y por lo tanto separable \qed
	
\end{proof}

\begin{theorem}[Metrizabilidad de la topolog\'ia d\'ebil]
	\label{Dual es separable si y solo si la bola es debil metrizable}
	Sea $X$ Banach, entonces $\dual{X}$ es separable si y s\'olo si $\ball X$ es $wk$ metrizable
\end{theorem}

\begin{proof}
	Si $\dual{X}$ es separable entonces replicando la demostraci\'on \ref{Separable si y solo si la bola dual es metrizable estrella} intercambiando los roles de $X$ y $\dual{X}$ se prueba que $\ball X$ es $wk$ metrizable. Para la vuelta esta fuera del alcance de este apunte en el caso general (Notar que a priori no podemos usar Hanh Banach aqu\'i) [Ver An Introduction to Nonlinear Analysis: Theory
	By Zdzislaw Denkowski, Stanislaw Mig�rski, Nikolaos S. Papageorgiou]. En el caso reflexivo es trivial por usar \ref{Separable si y solo si la bola dual es metrizable estrella} en $\ball \ddual{X} = \ball X$. \qed
\end{proof}

\begin{corollary}
	\label{En un banach separable la bola dual es secuencialmente compacta}
	Sea $X$ Banach separable y sea $\sett{f_n}_{n \in \N} \subset \dual{X}$ una sucesi\'on acotada, entonces existe $\left(f_{n_k}\right)_{k \in \N} \subset \left(f_n \right)_{n \in \N}$ subsucesi\'on $\dual{wk}$ convergente.
\end{corollary}

\begin{proof}
	Podemos suponer sin p\'erdida de generalidad que $\norm{f_n}_{\dual{X}} \leq 1$ y como por \ref{Teorema de Alaoglu} y \ref{Separable si y solo si la bola dual es metrizable estrella} $\ball \dual{X}$ es $\dual{wk}$ compacta y metrizable el resultado sigue de la caracterizaci\'on de espacios m\'etricos compactos \qed
\end{proof}

\begin{theorem}[Teorema de Eberlein-Smulian]
	\label{Teorema de Eberlein-Smulian}
	Sea $X$ un espacio de Banach, entonces $X$ es reflexivo si y s\'olo si toda sucesi\'on $\norm{.}_X$ acotada admite una subsucesi\'on $wk$ convergente
\end{theorem}

\begin{proof}
	Sea $M_0 = \ip{\sett{x_n}_{n \in \N}}_{\mathbb{F}}$ y sea $M = \overline{M_0}$, luego $M$ es separable y reflexivo por \ref{Subespacio de Reflexivo es reflexivo}. Entonces por \ref{X es reflexivo y separable si y solo si dual es reflexivo y separable} $\dual{M}$ es reflexivo y separable, por lo que por \ref{Teorema de Kukatani} y \ref{Dual es separable si y solo si la bola es debil metrizable} $\ball M$ es un espacio m\'etrico compacto por lo que existe $\left(x_{n_k}\right)_{k \in \N} \subset \left(x_n\right)_{n \in \N}$ subsucesi\'on $\sigma \left(M, \dual{M}\right)$ convergente a un $x_0$; pero como probamos que $\sigma \left(M, \dual{M}\right) = \sigma \left(X, \dual{X}\right) \vert_{M}$ entonces $\left(x_{n_k}\right)_{k \in \N}$ es $wk$ convergente a $x_0$.
	
	\bigskip
	
	La vuelta esta por fuera del alcance de este apunte, ver 
	
	\begin{itemize}
		\item Albiac?Kalton, Topics in Banach Space Theory (GTM 233), Corollary 1.6.4, page 24
		\item Whitley, An elementary proof of the Eberlein-?mulian theorem, Mathematische Annalen 172 (2), 1967, 116-118
	\end{itemize}
	.\qed
	
\end{proof}

\section{Operadores lineales en Espacios de Banach}

\subsection{Adjunto de un operador lineal}

\begin{definition}
	Sean $X,Y$ espacios vectoriales y $T: X \rightarrow Y$ una tranformaci\'on lineal. Definimos el \textit{adjunto algebraico} de $T$ como $T' : Y' \rightarrow X'$ dado por $T'(y') = y' \circ T$
\end{definition}

\begin{proposition}
	Sean $X,Y$ Banach y $T:X \rightarrow Y$ lineal, entonces son equivalentes:
	
	\begin{enumerate}
		\item $T$ es acotada
		\item $T'(\dual{Y}) = \dual{X}$
	\end{enumerate}
	
\end{proposition}

\begin{proof}
	Si $g \in \dual{Y}$ entonces $\abs{T'(g)(x)} = \abs{g (T(x))} \leq \norm{T} \norm{g}_{\dual{Y}} \norm{x}_X$ por lo que $T'(g) \in \dual{X}$
	
	Rec\'iprocamente, sea $g \in \dual{Y}$ y sea $f = T'(g) \in \dual{X}$ por hip\'otesis; luego si $x \in \ball X$ entonces $\abs{\ip{T(x), g}} = \abs{\ip{f, x}} \leq \norm{f}_{\dual{X}} < \infty$ por lo que $ \sup \sett{\abs{\ip{T(x), g}} \tq x \in \ball X} < \infty$ y por \ref{PAU X} $T$ es acotado. \qed 
	
\end{proof}

\begin{definition}
	Sean $X,Y$ Banach y $T \in L\left(X, Y\right)$. Definimos el \textit{adjunto } de $T$ como $\dual{T} : \dual{Y} \rightarrow \dual{X}$ dado por $\dual{T} = T' \vert_{\dual{Y}}$ y cumple que:
	
	\begin{equation*}
		\ip{\dual{T}(\dual{y}), x} = \ip{\dual{y}, T(x)} \qquad x \in X, \dual{y} \in \dual{Y}
	\end{equation*}
	
\end{definition}

\begin{proposition}
	Sean $X,Y$ Banach, $A,B \in L(X,Y)$ y $\alpha, \beta \in \mathbb{F}$, entonces $\dual{\alpha A + \beta B} = \alpha \dual{A} + \beta \dual{B}$
\end{proposition}

\begin{proposition}
	\label{Propiedades del operador adjunto}
	Sean $X,Y$ Banach, $T \in L(X,Y)$ entonces:
	
	\begin{itemize}
		\item $\ddual{T} \vert_{X} = T$
		\item Si $Z$ es Banach y $W \in L(Y,Z)$ entonces $\dual{WT} = \dual{T}\dual{W}$
		\item $\norm{\dual{T}} = \norm{T}$
		\item $\dual{T}$ es $\dual{wk}$ continua
	\end{itemize}
	
\end{proposition}

\begin{proof}
	Vayamos por partes:
	
	\begin{itemize}
		\item Sea $x \in X$, entonces $\ip{\ddual{T}(J(x)), \dual{Y}} = \ip{J(x), \dual{T}(\dual{y})} = \ip{Tx, \dual{y}}$
		
		\item Sea $\dual{z} \in \dual{Z}$, $x \in X$ entonces:
		
		\begin{equation*}
			\ip{\dual{T}\dual{W} (\dual{z}), x} = \ip{\dual{W} (\dual{z}), Tx} =  \ip{\dual{z}, WTx} = \ip{\dual{WT} \left(\dual{z} \right), x}
		\end{equation*}
		
		Por lo tanto $\dual{WT} = \dual{T}\dual{W}$.
		
		
		\item Si $\norm{\dual{y}}_{\dual{Y}} \leq 1$ y $\norm{x}_X \leq 1$ entonces $\abs{\ip{\dual{T}(\dual{y})}, x} = \abs{\ip{\dual{y}, Tx}} \leq \norm{Tx}_Y \leq \norm{T} \norm{x}_X$. Deducimos que:
		
		\begin{equation*}
			\norm{\dual{T}} = \sup\limits_{\dual{y} \in \ball \dual{Y}} \norm{\dual{T}(\dual{y})}_{\dual{X}} = \sup\limits_{\dual{y} \in \ball \dual{Y}} \sup\limits_{x \in \ball X} \abs{\ip{x, \dual{T}(\dual{y})}} \leq \norm{T}
		\end{equation*}
		
		Rec\'iprocamente, tenemos que $\norm{\ddual{T}} \leq \norm{\dual{T}}$ por lo que si $x \in \ball X$ entonces por (a):
		
		\begin{equation*}
			\norm{T} = \sup\limits_{x \in \ball X} \norm{Tx}_Y = \sup\limits_{x \in \ball X} \norm{\ddual{T}x}_Y \leq \sup\limits_{x \in \ball X} \norm{\dual{T}x}_Y = \norm{\dual{T}}
		\end{equation*}
		
		\item Por la propiedad universal de la topolog\'ia inicial $\dual{T} : \left(\dual{Y}, \dual{wk}\right) \rightarrow  \left(\dual{X}, \dual{wk}\right)$ es continua si y s\'olo si $p_x \circ \dual{T} :  \left(\dual{Y}, \dual{wk}\right) \rightarrow \R$ es continua para todo $x \in X$, pero $p_x \circ \dual{T} (\dual{y}) = \ip{\dual{T}(\dual{y}), x} = \ip{\dual{y}, Tx} = \dual{y} \circ T (x)$ que es continua para todo $x \in X$. \qed
		 
	\end{itemize}
	
\end{proof}

\begin{definition}
	Sea $X$ normado y $M \subseteq X$ subespacio, definimos el \textit{ortogonal} a $M$ como:
	
	\begin{equation*}
		\ortogonal{M} = \sett{\dual{x} \in \dual{X} \tq \ip{\dual{x}, m} = 0 \quad \forall m \in M}
	\end{equation*}
	
\end{definition}

\begin{definition}
	Sea $X$ normado y $M \subseteq \dual{X}$ subespacio, definimos el \textit{pre-ortogonal} a $M$ como:
	
	\begin{equation*}
	^{\perp}\left(M\right) = \sett{x \in X \tq \ip{m, x} = 0 \quad \forall m \in M}
	\end{equation*}
	
\end{definition}

\begin{proposition}
	\label{Relaciones de ortogonalidad}
	Sean $X,Y$ Banach y $T \in L(X,Y)$, entonces $\ker \dual{T} = \ortogonal{\rank T}$ y $\ker T = ^{\perp}\left(\rank \dual{T}\right)$
\end{proposition}

\begin{proof}
	Sea $f \in \dual{Y}$ y $x \in X$ tal que $f \in \ker \dual{T}$, entonces $\ip{f, T(x)} = \ip{\dual{T}(f), x} = 0$ por lo tanto $f \in \ortogonal{\rank T}$; rec\'iprocamente si $f \in \dual{Y}$ es tal que $\ip{f, T(x)} = \ip{\dual{T}(f), x} = 0$ para todo $x \in X$ entonces $\norm{\dual{T}(f)}_{\dual{X}} = \sup\limits_{x \in \ball X} \abs{\ip{\dual{T}(f), x}} = 0$ por lo que $f \in \ker \dual{T}$.
	
	Por otro lado, si $x \in \ker T$ entonces para $f \in \dual{Y}$ vale que $\ip{\dual{T}(f), x} = \ip{f, T(x)} = 0$ por lo que $x \in ^{\perp}\left(\rank \dual{T}\right)$; rec\'iprocamente $x \in X$ es tal que $\ip{f, T(x)} = \ip{\dual{T}(f), x} = 0$ para todo $f \in \dual{Y}$ entonces $\norm{T(x)}_{Y} \underbrace{=}_{\ref{Norma en funcion de funcionales}} \sup\limits_{f \in \ball \dual{Y}} \abs{\ip{f, T(x)}} = 0$ por lo que $x \in \ker T$. \qed
	
\end{proof}

\begin{proposition}
	\label{T es invertible si y solo si el adjunto lo es}
	Sean $X,Y$ Banach y $T \in L(X,Y)$, entonces $T$ es invertible si y s\'olo si $\dual{T}$ lo es. Es m\'as en este caso $\left(\dual{T}\right)^{-1} = \dual{T^{-1}}$
\end{proposition}

\begin{proof}
	En efecto, supongamos que $T$ es invertible y sea $x \in X, f \in \dual{Y}$, entonces:
	
	\begin{equation*}
	\begin{aligned}
		\ip{f,x} = & \ip{f, T^{-1} \circ T(x)} = \ip{\dual{T^{-1}} (f), Tx} = \ip{\dual{T} \circ \dual{T^{-1}} (f), x} \\
		= & \ip{f, T \circ T^{-1}(x)} = \ip{\dual{T} (f), T^{-1}x} = \ip{\dual{T^{-1}} \circ \dual{T} (f), x}
	\end{aligned}		 
	\end{equation*}
	
	Por lo tanto $\dual{T}$ es inversible y $\left(\dual{T}\right)^{-1} = \dual{T^{-1}}$; conclu\'imos que $\dual{T}$ es invertible pues:
	
	\begin{equation*}
		\norm{\left(\dual{T}\right)^{-1}} = \norm{\dual{T^{-1}}} \underbrace{=}_{\ref{Propiedades del operador adjunto}} \norm{T^{-1}} \leq \dfrac{1}{\norm{T}}
	\end{equation*}
	
	 
	
	Rec\'iprocamente, si $\dual{T}$ es invertible entonces por \ref{Teorema de la aplicacion abierta} existe $c > 0$ tal que:
	
	\begin{equation*}
		\dual{T} \left( \ball \dual{Y} \right) \supseteq c \ball \dual{X}
	\end{equation*}
	
	Por lo tanto si $x \in X$ entonces:
	
	\begin{equation*}
	\begin{array}{ccc}
		\norm{Tx}_Y & \underbrace{=}_{\ref{Norma en funcion de funcionales}} & \sup\limits_{\dual{y} \in \ball \dual{Y}} \abs{\ip{\dual{y}, Tx}} \\
		& = &  \sup\limits_{\dual{y} \in \ball \dual{Y}} \abs{\ip{\dual{T} \left(\dual{y} \right), x}} \\
		& \geq  &  \sup\limits_{\dual{x} \in c \ball \dual{X}} \abs{\ip{\dual{x} , x}} \\
		& = & \frac{1}{c} \norm{x}_X
	\end{array}
	\end{equation*}
	
	Usemos el siguiente lema:
	
	\begin{lemma}
		\label{Lemma: T es invertible si y solo si el adjunto lo es}
		Sean $X,Y$ Banach y $T \in L(X,Y)$ tal que existe $d > 0$ que cumple $\norm{Tx}_Y \geq d\norm{x}_X$, entonces $\rank T$ es cerrado y $\ker T = 0$
	\end{lemma}
	
	\begin{proof}[Demostraci\'on del lema]
		En efecto, si $Tx = 0$ entonces $\norm{x}_X \leq \frac{1}{d} \norm{Tx}_Y = 0$ por lo que $\ker T = 0$. Asimismo sea $\sett{x_n}_{n \in \N} \subset X$ una sucesi\'on tal que $T(x_n) \rightarrow y \in Y$, entonces $\norm{x_n - x_m}_{X} \leq \frac{1}{d} \norm{T(x_n - x_m)}_Y  = \frac{1}{d} \norm{T(x_n) - T(x_m)}_Y \xrightarrow{n,m \rightarrow \infty} 0$; por lo tanto $\sett{x_n}_{n \in \N}$ es de Cauchy y como $X$ es Banach existe $x \in X$ tal que $x_n \rightarrow x$, finalmente como $T$ es continua conclu\'imos que $T(x_n) \rightarrow T(x) = y$ por la unicidad del l\'imite. \qed
	\end{proof}
	
	Por lo tanto usando \ref{Lemma: T es invertible si y solo si el adjunto lo es} sabemos que $T$ es inyectiva de rango cerrado. Por otro lado por \ref{Relaciones de ortogonalidad} $\ortogonal{\rank T} = \ker \dual{T} = 0$ pues $\dual{T}$ es invertible, de \ref{M es denso si el unico funcional que lo anula es el nulo} conclu\'imos que $\rank T$ adem\'as es denso por lo que $T$ resulta suryectiva, inyectiva y ya era continua; conclu\'imos que $T$ es invertible. \qed 
	
\end{proof}

\begin{proposition}
	Sean $X,Y$ Banach y $S : \dual{Y} \rightarrow \dual{X}$ lineal, entonces existe $T \in L(X,Y)$ tal que $S = \dual{T}$ si y s\'olo si $S : \left(\dual{Y}, \dual{wk}\right) \rightarrow  \left(\dual{X}, \dual{wk}\right)$ es continua
\end{proposition}

\begin{proof}
	Para un lado si $S = \dual{T}$ entonces ya probamos en $\ref{Propiedades del operador adjunto}$ que $S$ es $\dual{wk}-\dual{wk}$ continua.
	
	Rec\'iprocamente, primero usemos el siguiente lema:
	
	\begin{lemma}
		\label{Lemma: Condiciones de ser adjunto}
		Sean $X,Y$ Banach y $T : \left(\dual{Y}, \dual{wk}\right) \rightarrow  \left(\dual{X}, \dual{wk}\right)$ continua, entonces $T \in L(X,Y)$
	\end{lemma}
	
	\begin{proof}
		Sea $\sett{f_n}_{n \in \N} \subset \dual{Y}$ una sucesi\'on tal que $f_n \rightarrow f$ y $S(f_n) \rightarrow g \in \dual{X}$, entonces por \ref{Resultados de convergencia debil estrella} $f_n \xrightarrow{\dual{d}_{\dual{Y}}} f$ y $S(f_n) \xrightarrow{\dual{d}_{\dual{X}}} g$.
		
		Como $S$ es $\dual{wk}-\dual{wk}$ continua sabemos que $S(f_n) \xrightarrow{\dual{d}} Sf$, luego por \ref{Topologia debil estrella es Hausdorff} deducimos que $S(f) = g$ y por \ref{Teorema del grafico cerrado} conclu\'imos que $S \in L \left((\dual{Y}, \dual{X} \right))$ \qed
	\end{proof}
	
	Sea $x \in X$ y consideremos $\alpha_x: \dual{Y} \rightarrow \R$ dado por $\alpha_x (f) = S(f) (x)$, entonces:
	
	\begin{equation*}
	\begin{aligned}
		\norm{\alpha_x}_{\ddual{Y}} = & \sup\limits_{f \in \ball \dual{Y}} \abs{\alpha_x (f)} \\
		 = & \sup\limits_{f \in \ball \dual{Y}} \abs{ \underbrace{S(f)}_{\in \dual{X}} (x)} \\
		 \leq & \sup\limits_{f \in \ball \dual{Y}} \norm{S(f)}_{\dual{X}} \norm{x}_X \\
		 \underbrace{\leq}_{\ref{Lemma: Condiciones de ser adjunto}} & \sup\limits_{f \in \ball \dual{Y}} \norm{S} \norm{f}_{\dual{X}} \norm{x}_X 
		 = & \norm{S} \norm{x}
	\end{aligned}
	\end{equation*}
	
	Por lo que $\alpha_x \in \dual{\left(\dual{Y}, \dual{wk}\right)}$ y por \ref{Caracterizacion de los funcionales del dual con debil estrella} existe un \'unico $y \in Y$ tal que $\alpha_x(f) = \ip{f,y}$, definamos $T(x) = y$. Es claro que $T$ es lineal y $\ip{S(f), x} = S(f)(x) = \alpha_x(f) = \ip{f, T(x)}$ por lo que basta ver que $T$ es acotada. En pos de eso:
	
	\begin{equation*}
	\begin{aligned}
		\norm{T(x)}_Y \underbrace{=}_{\ref{Norma en funcion de funcionales}} & \sup\limits_{f \in \ball \dual{Y}} \abs{\ip{f, T(x)}} \\
		= &  \sup\limits_{f \in \ball \dual{Y}} \abs{\ip{S(f), x}} \\
		= & \sup\limits_{f \in \ball \dual{Y}} \abs{\alpha_x (f)} \\
		= & \norm{\alpha_x}_{\ddual{Y}} \\
		\leq & \norm{S} \norm{x}
	\end{aligned}
	\end{equation*}
	
	Por lo que $T \in L(X,Y)$ y $S = \dual{T}$ \qed
	
\end{proof}

\subsection{Operadores compactos en espacios de Banach}

\end{document}