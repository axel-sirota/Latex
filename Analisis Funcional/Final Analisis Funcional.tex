\documentclass[11pt]{article}

\usepackage{amsfonts}
\usepackage{amsmath,accents,amsfonts, amssymb, mathrsfs }
\usepackage{tikz-cd}
\usepackage{graphicx}
\usepackage{syntonly}
\usepackage{color}
\usepackage{mathrsfs}
\usepackage[spanish]{babel}
\usepackage[latin1]{inputenc}
\usepackage{fancyhdr}
\usepackage[all]{xy}
\usepackage[at]{easylist}
\usepackage[colorlinks=true,linkcolor=blue,urlcolor=black,bookmarksopen=true]{hyperref}

\usepackage{bookmark}

\topmargin-2cm \oddsidemargin-1cm \evensidemargin-1cm \textwidth18cm
\textheight25cm


\newcommand{\B}{\mathcal{B}}
\newcommand{\Cont}{\mathcal{C}}
\newcommand{\F}{\mathcal{F}}
\newcommand{\inte}{\mathrm{int}}
\newcommand{\A}{\mathcal{A}}
\newcommand{\C}{\mathbb{C}}
\newcommand{\Q}{\mathbb{Q}}
\newcommand{\Z}{\mathbb{Z}}
\newcommand{\inc}{\hookrightarrow}
\renewcommand{\P}{\mathcal{P}}
\newcommand{\R}{{\mathbb{R}}}
\newcommand{\N}{{\mathbb{N}}}
\newcommand\tq{~:~}
\newcommand{\x}[3]{#1_#2^#3}
\newcommand{\xx}[4]{#1_#3#2_#4}
\newcommand\dd{\,\mathrm{d}}
\newcommand{\norm}[1]{\left\lVert#1\right\rVert}
\newcommand{\abs}[1]{\left\lvert#1\right\rvert}
\newcommand{\ip}[1]{\left\langle#1\right\rangle}
\renewcommand\tt{\mathbf{t}}
\newcommand\nn{\mathbf{n}}
\newcommand\bb{\mathbf{b}}                      % binormal
\newcommand\kk{\kappa}
\newcommand{\sett}[1]{\left\lbrace#1\right\rbrace}
\newcommand{\interior}[1]{\accentset{\smash{\raisebox{-0.12ex}{$\scriptstyle\circ$}}}{#1}\rule{0pt}{2.3ex}}
\fboxrule0.0001pt \fboxsep0pt
\newcommand{\Bigcup}[2]{\bigcup\limits_{#1}{#2}}
\newcommand{\Bigcap}[2]{\bigcap\limits_{#1}{#2}}
\newcommand{\Bigprod}[2]{\prod\limits_{#1}{#2}}
\newcommand{\Bigcoprod}[2]{\coprod\limits_{#1}{#2}}
\newcommand{\Bigsum}[2]{\sum\limits_{#1}{#2}}
\newcommand{\BigsumA}[3]{ \sideset{}{^#2}\sum\limits_{#1}{#3}}
\newcommand{\Biglim}[2]{\lim\limits_{#1}{#2}}
\newcommand{\quotient}[2]{{\raisebox{.2em}{$#1$}\left/\raisebox{-.2em}{$#2$}\right.}}

\def \le{\leqslant}	
\def \ge{\geqslant}
\def\noi{\noindent}
\def\sm{\smallskip}
\def\ms{\medskip}
\def\bs{\bigskip}
\def \be{\begin{enumerate}}
	\def \en{\end{enumerate}}
\def\deck{{\rm Deck}}
\def\Tau{{\rm T}}

\newtheorem{theorem}{Teorema}[section]
\newtheorem{lemma}[theorem]{Lema}
\newtheorem{proposition}[theorem]{Proposici\'on}
\newtheorem{corollary}[theorem]{Corolario}

\newenvironment{proof}[1][Demostraci\'on]{\begin{trivlist}
		\item[\hskip \labelsep {\bfseries #1}]}{\end{trivlist}}
\newenvironment{definition}[1][Definici\'on]{\begin{trivlist}
		\item[\hskip \labelsep {\bfseries #1}]}{\end{trivlist}}
\newenvironment{example}[1][Ejemplo]{\begin{trivlist}
		\item[\hskip \labelsep {\bfseries #1 }]}{\end{trivlist}}
\newenvironment{remark}[1][Observaci\'on]{\begin{trivlist}
		\item[\hskip \labelsep {\bfseries #1}]}{\end{trivlist}}
\newenvironment{declaration}[1][Afirmaci\'on]{\begin{trivlist}
		\item[\hskip \labelsep {\bfseries #1}]}{\end{trivlist}}


\newcommand{\qed}{\nobreak \ifvmode \relax \else
	\ifdim\lastskip<1.5em \hskip-\lastskip
	\hskip1.5em plus0em minus0.5em \fi \nobreak
	\vrule height0.75em width0.5em depth0.25em\fi}

\newcommand{\twopartdef}[4]
{
	\left\{
	\begin{array}{ll}
		#1 & \mbox{ } #2 \\
		#3 & \mbox{ } #4
	\end{array}
	\right.
}

\newcommand{\threepartdef}[6]
{
	\left\{
	\begin{array}{lll}
		#1 & \mbox{ } #2 \\
		#3 & \mbox{ } #4 \\
		#5 & \mbox{ } #6
	\end{array}
	\right.
}

\tikzset{commutative diagrams/.cd,
	mysymbol/.style={start anchor=center,end anchor=center,draw=none}
}
\newcommand\Center[2]{%
	\arrow[mysymbol]{#2}[description]{#1}}

\newcommand*\circled[1]{\tikz[baseline=(char.base)]{
		\node[shape=circle,draw,inner sep=2pt] (char) {#1};}}


\begin{document}
	
	\pagestyle{empty}
	\pagestyle{fancy}
	\fancyfoot[CO]{\slshape \thepage}
	\renewcommand{\headrulewidth}{0pt}
	
	
	
	\centerline{\bf An\'alisis Funcional - $1^{\circ}$ cuatrimestre $2017$}
	\centerline{\sc Final}
	
	\tableofcontents
	\bigskip

\section{Espacios Vectoriales}

\subsection{Propiedades Elementales}

\begin{definition}
Si $\mathcal{X}$ es un espacio vectorial sobre un cuerpo $\mathbb{F}$, un conjunto $\mathcal{B} = \sett{v_i}_{i \in I}$ se dice:

\begin{enumerate}
\item \textit{Linealmente independiente} si dados $v_{i_1}, \dots, v_{i_k} \in \B$ y $\lambda_{i_1},  \dots, \lambda_{i_k} \in \F$ tal que $\Bigsum{i}{\lambda_{i_i}v_{i_i}} = 0$ implica que $\lambda_{i_i} = 0$ para todo $1 \leq i \leq k$.
\item \textit{Sistema de generadores} si dado $v \in \mathcal{X}$ entonces existen $v_{i_1}, \dots, v_{i_k} \in \B$ y $\lambda_{i_1},  \dots, \lambda_{i_k} \in \F$ tal que $\Bigsum{i}{\lambda_{i_i}v_{i_i}} = v$.
\item \textit{Base} si es a la vez un sistema de generadores linealmente independiente.
\end{enumerate}
\end{definition}

\begin{example}



\begin{itemize}
	\item $X=\R[X]$ es un espacio vectorial, si consideramos $\B = \sett{1,X,X^2, \dots} = \sett{X^j}_{j \in \N}$ es base.
	\item $X=\mathcal{C}[a,b]$ es un espacio vectorial, si consideramos $\B = \sett{e^{\alpha x}, \alpha \in [0,1]}$ veamos que es linealmente independiente.
	
	\begin{proof}
		Sean $\alpha_1, \dots, \alpha_n \in [0,1]$ y $\lambda_{1}, \dots, \lambda_{n} \in \R$ tal que $\Bigsum{i}{\lambda_i e^{\alpha_i x}} = 0$ para todo $x \in [a,b]$; luego si derivamos $n-1$ veces tenemos el sistema:
		
		\[
			\left(
				\begin{array}{cccc}
					e^{\alpha_1 x} &  e^{\alpha_2 x} &  \dots &  e^{\alpha_n x} 
				\end{array}
			\right)
			\left(
				\begin{array}{cccc}
					1 & \alpha_1 &  \dots &  \alpha_1^{n-1} \\
					\vdots & \vdots & \vdots & \vdots \\
					1 &  \alpha_n &  \dots &  \alpha_n^{n-1}					 
				\end{array}
			\right)
			\left(
			\begin{array}{c}
			\lambda_{1} \\
			\lambda_{2} \\
			\vdots \\
			\lambda_{n}
			\end{array}
			\right) = 
			\left(
			\begin{array}{c}
			0 \\
			0 \\
			\vdots \\
			0
			\end{array}
			\right)
		\]
		Y como los $\alpha_i$ son distintos entonces la matriz de Vandermonde es inversible y el sistema admite una \'unica soluci\'on, $\lambda_{1} = \lambda_{2} = \dots = \lambda_{n} = 0$. \qed
	\end{proof}

\end{itemize}
\end{example}

Recordemos:

\begin{proposition}[Lema de Zorn]
	\label{Lema de Zorn}
	
	Si $(P,\leq)$ es un conjunto parcialmente ordenado, no vac\'io, tal que todo subconjunto no vac\'io $S \subseteq P$ totalmente ordenado admite una cota superior; entonces existe un elemento maximal en $P$.
\end{proposition}

\begin{proposition}
	Si $E$ es un espacio vectorial, entonces $E$ admite una base.
\end{proposition}

\begin{proof}
	Consideremos $P = \sett{S \subseteq E \ / \ S \text{ es li}}$ y dotemoslo del orden dado por la inclusi\'ion, luego $P \neq \emptyset$ pues si $v \in E$ entonces $\sett{v} \in P$.
	
	Sea $\sett{S_i}$ una colecci\'on de subconjuntos de $P$ totalmente ordenada y sea $T = \Bigcup{i \in I}{S_i}$, luego es claro que $S_i \leq T$; faltar\'ia ver que $T \in P$.
	
	Para eso sean $v_{i_1}, \dots, v_{i_k} \in T$ y $\lambda_{i_1},  \dots, \lambda_{i_k} \ in \F$ tales que $\Bigsum{k}{\lambda_i v_i} = 0$. Como son finitos existe $k_0 \in \N$ tal que $v_i \in S_{k_0}$ para todo $i$, que al ser un conjunto linealmente independiente resulta que $\lambda_{1} = \lambda_{2} = \dots = \lambda_{n} = 0$. Conclu\'imos que $T \in P$, luego por \ref{Lema de Zorn} existe $M \in P$ elemento maximal.
	
	Finalmente, sea $v \in E \setminus <M>$ (el conjunto generado por combinaciones lineales de $M$), luego $M \cup \sett{v}$ ser\'ia un conjunto li lo que contradice la maximalidad de $M$; por ende no existe tal $v$ y $M$ resulta base. \qed
	
	
	
\end{proof}

\begin{proposition}
	\label{Dos bases tienen mismo cardinal, Hamel}
	Sea $E$ un espacio vectorial y sean $\B_1, \B_2$ dos bases de Hamel de $E$. Luego $\#B_1 = \#B_2$.
\end{proposition}

\begin{proof}
	Sea $x \in \B_1$ y llamemos $S(x)$ al conjunto de los elementos $v \in \B_2$ tal que al escribir a $x$ como combinaci\'on lineal de elementos de $\B_2$ aparece $v$, por lo que si $x = \Bigsum{k}{\lambda_{i_k} v_{i_k}}$ entonces $S(x) = \sett{v_{i_1}, \dots, v_{i_n}}$.
			
	\begin{lemma}
		\label{Lema de cardinalidad de bases}
		$\Bigcup{x \in \B_1}{S(x)} = \B_2$
	\end{lemma}
	\begin{proof}{Del lema}
		Si $v \in \Bigcup{x \in \B_1}{S(x)}$ luego existe $x_0 \in \B_1$ tal que $v \in S(x_0)$ por lo que $v \in \B_2$ por definici\'on de $S(x)$. Rec\'iprocamente, si $v \in \B_2$ pero no existe $x \in \B_1$ tal que $v \in S(x)$, entonces $v \not \in <\B_1> =E = <\B_2>$. \qed
	\end{proof}
	
	Por \ref{Lema de cardinalidad de bases} tenemos que $\# \B_2 \leq \Bigsum{x \in \B_1}{\#S(x)} \leq \#\N \#\B_1 \leq \#B_1$.
	
	Razonando al rev\'es obtenemos la otra desigualdad. \qed		
	
\end{proof}

\subsection{Normas y productos internos}

\begin{definition}
	Si $E$ es un espacio vectorial, una norma definida en $E$ es una aplicaci\'on $\norm{.}: E \mapsto \R$ tal que:
	
	\begin{enumerate}
		\item $\norm{x} \geq 0$
		\item $\norm{x} = 0 \ \Longleftrightarrow x = 0$
		\item $\norm{\lambda x} = \abs{\lambda} \norm{x} $
		\item $\norm{x+y} \leq \norm{x} + \norm{y}$
	\end{enumerate}
	
\end{definition}

\begin{remark}
	Todo espacio normado es un espacio m\'etrico pero no viceversa.
\end{remark}

\begin{definition}
	Si $E$ es un espacio vectorial, un producto interno definido en $E$ es una aplicaci\'on $\ip{.,.}: E \times E \mapsto F$ tal que:
	
	\begin{enumerate}
		\item $\ip{.,z}$ es lineal
		\item $\ip{x,x} = 0 \ \Longleftrightarrow x = 0$
		\item $\ip{x,y} = \overline{\ip{y,x}}$
	\end{enumerate}
	
\end{definition}


\begin{remark}
	Todo espacio con producto interno es un espacio normado pero no viceversa.
\end{remark}

\begin{theorem}[Cauchy-Schwartz]
	\label{Desigualdad de Cauchy-Schwartz }
	Sea $E$ un espacio vectorial y $\ip{.}$ un producto interno definido en $E$; luego si $x,y \in E$ se tiene que $\abs{\ip{x,y}} \leq \norm{x} \norm{y}$.
\end{theorem}

\begin{proof}
	Sean $x,y \in E$, $\lambda \in \C$ y sea $z = x-\lambda y$, luego $\ip{z,z} = \ip{x,x} + \abs{\lambda^2}\ip{y,y} -2 \Re(\lambda \ip{y,x}) \geq 0$.
	
	Si $\ip{y,x} = re^{i \theta}$ sea $\lambda = e^{-i \theta}t$ con $t \in \R$; luego:
	
	\[
		0 \geq \ip{x,x} + t^2 \ip{y,y} - 2bt \equiv c -2bt + at^2 := q(t)
	\]
	
	Luego como la cuadr\'atica dada es positiva, eso implica que $0 \leq 4b^2 -4ac$ por lo que:
	
	\[
		0 \leq b^2 -ac = \abs{\ip{x,y}}^2 - \ip{x,x}\ip{y,y}
	\]
	
	Si $\abs{\ip{x,y}}= \norm{x} \norm{y}$, entonces $b^2 = \ip{x,x} \ip{y,y}$ por lo que $b^2 -ac = 0$. Esto implica que existe $t_0$ tal que $q(t_0) = 0$, por lo tanto eso implica que $\ip{x-e^{-i \theta}t_0 y,x- e^{-i \theta}t_0 y} \equiv 0$ y por lo tanto $x = e^{-i \theta}t_0 y$. \qed
	
\end{proof}

\begin{definition}
	Un espacio normado que es completo respecto a la distancia inducida por la norma se llama \textit{Espacio de Banach}
\end{definition}

\begin{definition}
	Un \textit{Espacio de Hilbert} es un espacio de Banach donde la norma proviene de un producto interno mediante $\norm{x} = \sqrt{\ip{x,x}}$.
\end{definition}

\begin{proposition}
	\label{Identidad de poralizacion}
	Sea $E$ un espacio con producto interno, entonces:
	
	\begin{itemize}
		\item $\mathcal{R}(\ip{x,y}) = \frac{1}{4}\left(\norm{x+y}^2 - \norm{x-y}^2\right)$
		\item $\mathcal{I}(\ip{x,y}) = \frac{1}{4}\left(\norm{x+iy}^2 - \norm{x-iy}^2\right)$
	\end{itemize}

\end{proposition}

\begin{proof}
	Por un lado $\norm{x+y}^2 = \norm{x}^2 + \norm{y}^2 + 2\mathcal{R}(\ip{x,y})$ y $\norm{x-y}^2 = \norm{x}^2 + \norm{y}^2 - 2\mathcal{R}(\ip{x,y})$; por lo que restando se obtiene:
	
	\[
		4\mathcal{R}(\ip{x,y}) = \norm{x+y}^2 - \norm{x-y}^2
	\]
	
	Por el otro: 
	
	\[
	\begin{aligned}
		\norm{x+iy}^2 = & \ip{x+iy,x+iy} \\
					  = & \norm{x}^2 + \abs{i}\norm{y}^2 -i \ip{x,y} + i \overline{\ip{x,y}} \\
					  = & \norm{x}^2 + \norm{y}^2 -i 2 \mathcal{I}(\ip{x,y}) \\
		\norm{x-iy}^2 = & \ip{x-iy,x-iy} \\
			   		  = & \norm{x}^2 + \abs{i}\norm{y}^2 +i \ip{x,y} - i \overline{\ip{x,y}} \\
					  = & \norm{x}^2 + \norm{y}^2 +i 2 \mathcal{I}(\ip{x,y}) \\
	\end{aligned}
	\]
	
	Por lo tanto restando ambas obtenemos:
	
	\[
		4\mathcal{I}(\ip{x,y}) = \norm{x+iy}^2 - \norm{x-iy}^2
	\]
	\qed
	
\end{proof}

\begin{proposition}[Ley del paralelogramo]
	\label{Ley del paralelogramo}
	Sea $E$ un espacio normado real, entonces existe $\ip{.,.}: E \times E \rightarrow \C$ tal que $\norm{x}= \sqrt{\ip{x,x}}$ si y s\'olo si para todos $x,y \in E$ vale:
	
	\[
		\norm{x+y}^2 + \norm{x-y}^2 = 2\norm{x}^2 + 2\norm{y}^2
	\]
\end{proposition}

\begin{proof}
	Si $\norm{x} = \sqrt{\ip{x,x}}$ entonces de la demostraci\'on de \ref{Identidad de poralizacion} se da el resultado. Rec\'iprocamente definamos:
	
	 $$ \ip{x,y} := \frac{1}{4}\left(\norm{x+y}^2 - \norm{x-y}^2\right)$$
	
	Luego verifiquemos que es un producto interno.
	
	\begin{enumerate}
		\item $\sqrt{\ip{x,x}} = \norm{x}$
		\item Como $\norm{x+y} = \norm{y+x}$ y $\norm{x-y} = \norm{-(y-x)} = \norm{y-x}$ conclu\'imos que $\ip{x,y} = \ip{y,x}$.
		\item Dado que $\norm{.}, +, -, *$ son $\norm{.}$-continuas entonces $\ip{.,x},\ip{x,.}$ es $\norm{.}$-continua.
		\item Sean $x,y,z \in E$ entonces:
		
		\[
		\begin{aligned}
			\norm{x+y+z}^2 = & 2\norm{x+z}^2 + 2\norm{y}^2 - \norm{x-y+z}^2
						   = & 2\norm{y+z}^2 + 2\norm{x}^2 - \norm{y - x+z}^2
		\end{aligned}
		\]
		
		Luego como $A=B$ y $A=C$ implica $A=\frac{B+C}{2}$ se obtiene:
		
		\[
		\begin{aligned}
			\norm{x+y+z}^2 = & \norm{x+z}^2 + \norm{y}^2 - \frac{1}{2}\norm{x-y+z}^2 + \norm{y+z}^2 + \norm{x}^2 - \frac{1}{2}\norm{y - x+z}^2 \\
			\norm{x+y-z}^2 = & \norm{x-z}^2 + \norm{y}^2 - \frac{1}{2}\norm{x-y-z}^2 + \norm{y-z}^2 + \norm{x}^2 - \frac{1}{2}\norm{y - x-z}^2 \\
						   = & \norm{x-z}^2 + \norm{y}^2 - \frac{1}{2}\norm{-x+y+z}^2 + \norm{y-z}^2 + \norm{x}^2 - \frac{1}{2}\norm{-y + x+z}^2 
		\end{aligned}
		\]
		
		Por lo tanto:
		
		\[
			\begin{aligned}
				\ip{x+y,z} = & \frac{1}{4} \left(\norm{x+y+z}^2 - \norm{x+y-z}^2\right) \\
						   = & \frac{1}{4} \left(\norm{x+z}^2 - \norm{x-z}^2\right) + \frac{1}{4} \left(\norm{y+z}^2 - \norm{y-z}^2\right) \\
						   = & \ip{x,z} + \ip{y,z}
			\end{aligned}
		\]
		
		\item Por el item anterior es claro por inducci\'on que $\lambda \ip{x,y} = \ip{\lambda x,y}$ para todo $\lambda \in \N$ y como vale para $\lambda = -1$ tenemos que vale para todo $\lambda \in \Z$. Si $\lambda = \frac{p}{q} \in \Q$ entonces si llamamos $x' = \frac{x}{q}$ tenemos:
		
		\[
			q\ip{\lambda x, y} = q \ip{p x',y} = p \ip{qx',y} = p \ip{x,y}
		\]
		
		Luego $\lambda \ip{x,y} = \ip{\lambda x, y}$ para todo $\lambda \in \Q$. Por lo tanto probamos que fijados $x,y \in E$ la funci\'on $g(t)=\frac{1}{t} \ip{tx,y}$ y la funci\'on constante $h(t) = \ip{x,y}$ cumplen que $h\vert_{\Q} = g \vert_{\Q}$ y por continuidad entonces $h \equiv g$ para todo $t \in \R \setminus \sett{0}$; como el caso $\lambda = 0$ es trivial conclu\'imos que $\lambda \ip{x,y} = \ip{\lambda x,y}$. \qed
		
	\end{enumerate}
	
\end{proof}

\section{Espacios de Hilbert}

\subsection{Preliminares}

\begin{proposition}
	\label{Prod interno es continuo}
	Sea $E$ un espacio vectorial con producto interno, luego el producto interno es continuo.
\end{proposition}

\begin{proof}
	Sea $x_n, (y_n)$ tales que $x_n \rightarrow x, y_n \rightarrow y$, luego:
	
	\begin{equation*}
	\begin{aligned}
		\abs{\ip{x_n,y_n} -  \ip{x,y}} = & \abs{\ip{x_n - x,y_n} + \ip{x,y_n - y}} \\ 
		\leq & \abs{\ip{x_n-x,y_n}} + \abs{\ip{x,y_n-y}} \\
		\leq & \abs{\ip{x_n-x,y_n -y}} + \abs{\ip{x_n-x,y}} + \abs{\ip{x,y_n-y}} \\ 
		\leq & \norm{x_n - x}\norm{y} + \norm{x}\norm{y_n-y} + \norm{x_n - x}\norm{y_n-y} \rightarrow 0 
	\end{aligned}
	\end{equation*}
	\qed
\end{proof}

\subsection{Conjuntos ortogonales y ortonormales}

\begin{definition}
	Sea $E$ un espacio vectorial con producto interno, luego dados dos vectores $x,y \in E$ decimos que son \textit{ortogonales} si $\ip{x,y} = 0$.
	
	A su vez decimos que son \textit{ortonormales} si osn ortogonales y $\norm{x} = \norm{y} = 1$
	
	Finalmente dado un conjunto $S \subseteq E$ entonces decimos que es \textit{ortogonal / ortonormal} si dados cualesquiera $x,y \in S$ resulta que son \textit{ortogonales / ortonormales} 
\end{definition}

\begin{example}
	El conjunto $\sett{e^{inx} \ , \ n \in \N \ , \ x \in [0,2 \pi]}$ es ortonormal.
\end{example}

\begin{theorem}
	\label{Escritura de proyeccion a un conjunto ortonormal}
	Sea $E$ un espacio vectorial con producto interno y sea $S \subseteq E$ un conjunto ortonormal, luego si $x \in \ip{S}$ entonces existe una \'unica escritura de $x$ dada por:
	
	$$x = \sum\limits_{i = 1}^{n}{\ip{x,u_i}u_i} \qquad u_i \in S$$
	
\end{theorem}

\begin{proof}
	Como $x \in \ip{S}$ entonces existen \'unicos $\lambda_{1},\dots, \lambda_{n}$ tal que $x = \sum\limits_{i = 1}^{n}{\lambda_i u_i}$. Luego:
	
	
	\[
		\ip{x,u_j} = \sum\limits_{i = 1}^{n}{\lambda_i \ip{u_i,u_j}} = \lambda_j
	\]
	\qed
\end{proof}

\begin{theorem}[Desigualdad de Bessel]
	\label{Desigualdad de Bessel}
	Sea $E$ un espacio vectorial con producto interno y sea $S \subseteq E$ un conjunto ortonormal, luego:
	
	\begin{enumerate}
		\item SI $x \in E$ y $u_1, \dots, u_n \in S$ luego $\sum\limits_{i = 1}^{n}{\abs{\ip{x,u_i}}^2} \leq \norm{x}^2$
		\item Si $x \in E$ entonces $\sett{u \in S \ / \ \ip{x,u}\neq 0}$ es a lo sumo numerable
		\item Si $x,y \in E$ entonces $\abs{\Bigsum{u \in S}{\ip{x,u}\overline{\ip{y,u}}}} \leq \norm{x}\norm{y}$
	\end{enumerate}
\end{theorem}

\begin{proof}
	\begin{enumerate}
		\item Sean $u_1,\dots,u_n \in S$ y sea $z = x - \sum\limits_{i = 1}^{n}{{\ip{x,u_i}}}$, luego:
	
	\[
	\begin{aligned}
		0 \leq & \ip{z,z} \\
		= & \ip{x - \sum\limits_{i = 1}^{n}{\ip{x,u_i}},x - \sum\limits_{i = 1}^{n}{\ip{x,u_i}}} \\
		= & \norm{x}^2 + \norm{\sum\limits_{i = 1}^{n}{\ip{x,u_i}}}^2 - 2 \mathcal{R} \left(\ip{\sum\limits_{i = 1}^{n}{\ip{x,u_i}},x}\right) \\
		= & \norm{x}^2 + \sum\limits_{i = 1}^{n}{\norm{\ip{x,u_i}}^2} - 2 \mathcal{R} \left(\sum\limits_{i = 1}^{n}{\abs{\ip{x,u_i}}^2}\right) \\
		= & \norm{x}^2 - \sum\limits_{i = 1}^{n}{\norm{\ip{x,u_i}}^2}.
	\end{aligned}
	\]
	
	\item Notemos que $S = \sett{u \in S \ / \ \abs{\ip{x,u}} > 0} = \Bigcup{n \in \N}{\underbrace{\sett{u \in S \ / \ \abs{\ip{x,u}} \geq \frac{1}{m} }}_{T_m}}$.
	
	Ahora sean $u_1,\dots,u_n \in T$ por el item anterior sabemos que:
	
	\[
		\frac{n}{m^2} \leq \Bigsum{1 \leq k \leq n}{\abs{\ip{x,u_k}}^2} \leq \norm{x}^2
	\]
	
	Por lo que $n \leq m^2 \norm{x}^2$ y entonces $\# T_m \leq m^2 \norm{x}^2 < \infty$ para todo $m$, por lo tanto $\# S \leq \# \N * \# T_m \leq \# \N$.
	
	\item Sean $x,y \in E$ y $u_1,\dots,u_n \in S$, luego:
	\[
	 \begin{aligned}
	 	 \abs{\sum\limits_{i = 1}^{n}{\ip{x,u_i}\overline{\ip{y,u_i}}}} \leq_\text{C-S} & \sqrt{ \sum\limits_{i = 1}^{n}{\abs{\ip{x,u_i}}^2}}\sqrt{ \sum\limits_{i = 1}^{n}{\abs{\ip{y,u_i}}^2}} \\
	 	 \leq_\text{a} & \norm{x}\norm{y}
	 \end{aligned}
	\]
	\qed
	\end{enumerate}
\end{proof}

\begin{theorem}
	\label{Todo conjunto ortonormal en un separable es numerable}
	Si $E$ es un espacio vectorial con producto interno tal que $E$ es separable, entonces todo conjunto ortonormal es a lo sumo numerable
\end{theorem}

\begin{proof}
	Sea $S \subseteq E$ un conjunto ortonormal y sean $u \neq v \in S$, luego $\norm{u-v}^2 = \norm{u}^2 + \norm{v}^2 = 2$ y por lo tanto $B_{\frac{\sqrt{2}}{2}}(u) \cap B_{\frac{\sqrt{2}}{2}}(v) = \emptyset$.
	
	Sea $D \subseteq E$ un subconjunto denso numerable, luego $B_{\frac{\sqrt{2}}{2}}(u) \cap D \neq \emptyset$ para todo $u \in S$. Consideremos $f:S \rightarrow D$ dado por $f(u) \in B_{\frac{\sqrt{2}}{2}}(u) \cap D$, luego si $f(u) = f(v)$ entonces $f(v) \in B_{\frac{\sqrt{2}}{2}}(u) \cap B_{\frac{\sqrt{2}}{2}}(v) $ y por lo tanto $u = v$. Como $f$ es inyectiva conclu\'imos que $S$ es a lo sumo numerable. \qed
	
\end{proof}

\begin{theorem}
	\label{Proyeccion de un elemento en un ortonormal}
	Sean $H$ un espacio de Hilbert, $u_n$ una sucesi\'on de vectores ortonormales  y $c_n$ una sucesi\'on de numeros complejos. Luego:
	
	\begin{equation}
		\Bigsum{n \in \N}{c_nu_n} \in H \ \Longleftrightarrow \ \Bigsum{n \in \N}{\abs{c_n}^2} < \infty
	\end{equation}
	
	M\'as a\'un, $c_n = \ip{\Bigsum{n \in \N}{c_n u_n}, u_n}$
	
\end{theorem}

\begin{proof}
	Sea $S_k = \sum\limits_{i = 1}^{k}{c_i u_i}$, luego como $(u_n)$ son ortonormales dos a dos y $H$ es completo:
	
	\[
			\norm{\sum\limits_{i = k+ 1}^{k'}{c_n u_n}}^2 = \sum\limits_{i = k+ 1}^{k'}{\abs{c_n}^2}
	\]
	Por ende:
	\begin{equation*}
				\Bigsum{n \in \N}{c_nu_n} \in H \ \Longleftrightarrow \ \Bigsum{n \in \N}{\abs{c_n}^2} < \infty
	\end{equation*}
	
	Finalmente, notemos que $\ip{S_k,u_j} = c_j$ para todo $k \geq j$ y ,adem\'as si $(c_n) _in l^2$, entonces $S_k \rightarrow \Bigsum{n \in \N}{c_n u_n}=:x$; por lo tanto por \ref{Prod interno es continuo} $c_n = \ip{S_k,u_n} \rightarrow \ip{x,u_n}$. 
	\qed
	
\end{proof}

\begin{definition}
	Sea $E$ un espacio vectorial con producto interno y $M \subseteq E$, definimos \textit{el ortogonal a } $M$ como $M^{\perp} = \sett{x \in E \ / \ \ip{x,m} = 0 \ \forall m \in M}$.
\end{definition}

\begin{proposition}
	\label{El ortogonal es cerrado}
	$M^{\perp}$ es un subespacio cerrado de $E$
\end{proposition}

\begin{proof}
	Si $(x_n) \subset M$ es tal que $x_n \rightarrow x$ entonces por \ref{Prod interno es continuo} $0 = \ip{m,x_n} \rightarrow \ip{m,x}$, por lo que $x \in M$. \qed
\end{proof}

\begin{theorem}
	\label{Proyeccion a un conjunto ortonormal arbitrario}
	Sea $H$ un espacio de Hilbert y sea $S \subseteq H$ un conjunto ortonormal, luego:
	
	\begin{enumerate}
		\item Si $x \in H$ entonces $x_S = \Bigsum{u \in S}{\ip{x,u}u}$ esta bien definido
		\item Si $M = \ip{S}$ entonces $x \in M$ si y solo si $x = x_S$. Es m\'as si $x \in H$ entonces $x - x_S \in M^{\perp}$.
	\end{enumerate}
\end{theorem}

\begin{proof}
	\begin{enumerate}
		\item 	Dado $x \in H$, de \ref{Desigualdad de Bessel} sea $(u_n)$ una numeraci\'on de $S = \sett{u \in S \ / \ \ip{x,u}=0}$ y sea $(v_n)$ otra ordenaci\'on de los $u_n$; notemos $x_1 = \Bigsum{n}{\ip{x,u_n}u_n}$ y $x_2 = \Bigsum{n}{\ip{x,u_n}u_n}$ que por \ref{Proyeccion de un elemento en un ortonormal} y \ref{Desigualdad de Bessel} est\'an bien definidos.
	
	Luego:
	
	\[
	\begin{aligned}
		\ip{x_1 - x_2 , u_n} = & \ip{x_1,u_n} - \ip{x_2,u_n} \\
		\underbrace{=}_{u_n = v_{m_n} \text{ para alg\'un } m_n} & \ip{x,u_n} - \ip{x,v_{m_n}} \\
		= & \ip{x,u_n} - \ip{x,u_n}
		= & 0
	\end{aligned}
	\]
	
	Por ende, $\ip{x_1 - x_2,u_n} = \ip{x_1 - x_2,v_n} = 0 $ para todo $n \in \N$ y se concluye que $\ip{x_1 - x_2 , x_1 - x_2} = 0$ por lo que $x_1 = x_2$ y entonces $x_S$ esta bien definido y no depende del orden de la suma.
	
	\item Sea $x_{S_k} = \sum\limits_{i=1}^{k}\ip{x,u_i}u_i \in M$, luego como $M$ es cerrado se tiene que $x_{S_k} \rightarrow x_S \in M$. Ahora sea $s \in S$, entonces:
	
	\[
	\ip{x-x_S,v} = \ip{x,v} - \ip{x_S,v} = \ip{x,v} - \ip{x,v} = 0
	\]
	
	Por lo que $x - x_S \in M^{\perp}$. Finalmente, si $x \in M$ entonces como $x_S \in M$ entonces $x - x_S \in M \cap M^{\perp} = \sett{0}$, luego $x = x_S$. \qed
	
	\end{enumerate}
	
\end{proof}

\subsection{Conjuntos ortonormales completos}

\begin{definition}
	Sea $E$ un espacio vectorial con producto interno y sea $S \subseteq E$ ortonormal, diremos que $S$ es \textit{completo} si $S \subseteq T$ y $T$ es ortonormal, entonces $S = T$.
\end{definition}

\begin{proposition}
	\label{Ortogonal vacio es ser completo}
	Sea $S$ un conjunto ortonormal tal que $S^{\perp} = \sett{0}$, entonces $S$ es completo
\end{proposition}

\begin{proof}
	Sea $T$ ortonormal y sea $v \in T \setminus S$, luego $v \in S^{\perp} = 0$ por lo que $S$ es completo. \qed
\end{proof}

\begin{theorem}
	Sea $E$ un espacio vectorial con producto interno, $S \subseteq E$ ortonormal y sea $M = \ip{S}$, entonces:
	
	\begin{enumerate}
		\item Si $M = E$ entonces $S$ es completo
		\item Si $S$ es completo y $E$ es de Hilbert entonces $M = E$
	\end{enumerate}
	
\end{theorem}

\begin{proof}
	\begin{enumerate}
		\item Si $x \in S^{\perp}$ entonces $x \in M^{\perp}=E^{\perp}=\sett{0}$, por lo tanto $S$ es completo
		\item Sea $x \in E$, luego por \ref{Proyeccion a un conjunto ortonormal arbitrario} $x_S$ esta bien definido y $x - x_S \in M^{\perp}$, luego como $S$ es completo $x - x_S = 0$ y por \ref{Proyeccion a un conjunto ortonormal arbitrario} se tiene que $x \in M$. \qed
	\end{enumerate}
\end{proof}

\end{document}