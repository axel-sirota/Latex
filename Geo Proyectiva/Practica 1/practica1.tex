\documentclass[11pt]{article}

\usepackage{amsfonts}
\usepackage{amsmath,accents,amsfonts, amssymb, mathrsfs }
\usepackage{tikz-cd}
\usepackage{graphicx}
\usepackage{syntonly}
\usepackage{color}
\usepackage{mathrsfs}
\usepackage[spanish]{babel}
\usepackage[latin1]{inputenc}
\usepackage{fancyhdr}
\usepackage[all]{xy}
\usepackage[at]{easylist}


\topmargin-2cm \oddsidemargin-1cm \evensidemargin-1cm \textwidth18cm
\textheight25cm


\newcommand{\B}{\mathcal{B}}
\newcommand{\Cont}{\mathcal{C}}
\newcommand{\F}{\mathcal{F}}
\newcommand{\inte}{\mathrm{int}}
\newcommand{\A}{\mathcal{A}}
\newcommand{\C}{\mathbb{C}}
\newcommand{\Q}{\mathbb{Q}}
\newcommand{\Z}{\mathbb{Z}}
\newcommand{\inc}{\hookrightarrow}
\renewcommand{\P}{\mathcal{P}}
\newcommand{\R}{{\mathbb{R}}}
\newcommand{\N}{{\mathbb{N}}}
\newcommand\tq{~:~}
\newcommand{\x}[3]{#1_#2^#3}
\newcommand{\xx}[4]{#1_#3#2_#4}
\newcommand\norm[1]{\left\lVert#1\right\rVert}
\newcommand\abs[1]{\left\lvert#1\right\rvert}
\newcommand\ip[1]{\left\langle#1\right\rangle}
\newcommand{\sett}[1]{\left\lbrace#1\right\rbrace}
\newcommand{\interior}[1]{\accentset{\smash{\raisebox{-0.12ex}{$\scriptstyle\circ$}}}{#1}\rule{0pt}{2.3ex}}
\fboxrule0.0001pt \fboxsep0pt
\newcommand{\Bigcup}[2]{\bigcup\limits_{#1}{#2}}
\newcommand{\Bigcap}[2]{\bigcap\limits_{#1}{#2}}
\newcommand{\Bigprod}[2]{\prod\limits_{#1}{#2}}
\newcommand{\Bigcoprod}[2]{\coprod\limits_{#1}{#2}}
\newcommand{\Bigsum}[2]{\sum\limits_{#1}{#2}}
\newcommand{\BigsumA}[3]{ \sideset{}{^#2}\sum\limits_{#1}{#3}}
\newcommand{\Biglim}[2]{\lim\limits_{#1}{#2}}
\newcommand{\quotient}[2]{{\raisebox{.2em}{$#1$}\left/\raisebox{-.2em}{$#2$}\right.}}



\def \le{\leqslant}	
\def \ge{\geqslant}
\def\sen{{\rm sen} \, \theta}
\def\cos{{\rm cos}\, \theta}
\def\noi{\noindent}
\def\sm{\smallskip}
\def\ms{\medskip}
\def\bs{\bigskip}
\def \be{\begin{enumerate}}
	\def \en{\end{enumerate}}
\def\deck{{\rm Deck}}
\def\Tau{{\rm T}}

\newtheorem{theorem}{Teorema}[section]
\newtheorem{lemma}[theorem]{Lema}
\newtheorem{proposition}[theorem]{Proposici\'on}
\newtheorem{corollary}[theorem]{Corolario}

\newenvironment{proof}[1][Demostraci\'on]{\begin{trivlist}
		\item[\hskip \labelsep {\bfseries #1}]}{\end{trivlist}}
\newenvironment{definition}[1][Definici\'on]{\begin{trivlist}
		\item[\hskip \labelsep {\bfseries #1}]}{\end{trivlist}}
\newenvironment{example}[1][Ejemplo]{\begin{trivlist}
		\item[\hskip \labelsep {\bfseries #1 }]}{\end{trivlist}}
\newenvironment{remark}[1][Observaci\'on]{\begin{trivlist}
		\item[\hskip \labelsep {\bfseries #1}]}{\end{trivlist}}
\newenvironment{declaration}[1][Afirmaci\'on]{\begin{trivlist}
		\item[\hskip \labelsep {\bfseries #1}]}{\end{trivlist}}


\newcommand{\qed}{\nobreak \ifvmode \relax \else
	\ifdim\lastskip<1.5em \hskip-\lastskip
	\hskip1.5em plus0em minus0.5em \fi \nobreak
	\vrule height0.75em width0.5em depth0.25em\fi}

\newcommand{\twopartdef}[4]
{
	\left\{
	\begin{array}{ll}
		#1 & \mbox{ } #2 \\
		#3 & \mbox{ } #4
	\end{array}
	\right.
}

\newcommand{\threepartdef}[6]
{
	\left\{
	\begin{array}{lll}
		#1 & \mbox{ } #2 \\
		#3 & \mbox{ } #4 \\
		#5 & \mbox{ } #6
	\end{array}
	\right.
}

\tikzset{commutative diagrams/.cd,
	mysymbol/.style={start anchor=center,end anchor=center,draw=none}
}
\newcommand\Center[2]{%
	\arrow[mysymbol]{#2}[description]{#1}}

\newcommand*\circled[1]{\tikz[baseline=(char.base)]{
		\node[shape=circle,draw,inner sep=2pt] (char) {#1};}}


\begin{document}
	
	\pagestyle{empty}
	\pagestyle{fancy}
	\fancyfoot[CO]{\slshape \thepage}
	\renewcommand{\headrulewidth}{0pt}
	
	
	
	\centerline{\bf Geometr\'ia Proyectiva - $2^{\circ}$ cuatrimestre $2016$}
	\centerline{\sc Pr\'actica 1}
	
	\bigskip

\textbf{Aclaraci\'on:} Notemos $\Bigsum{}{}$ a la suma en un espacio vectorial $V$ y $\Bigsum{}{}^{A}$ a la suma en $V_A$ con $A \in V$

\begin{enumerate}
	
	\item Sea $V$ un espacio vectorial de dimensi�n $n$. 
	Un \emph{sistema de coordenadas afines} en $V$ es un par 
	$S= (A, \{v_1,\ldots, v_n\})$, con si $A\in V$ y $\{v_1,\ldots, v_n\}$
	una base de $V_A$. Notaremos $S=\{A,v_1,\ldots,v_n\}$. Mostrar que son 
	equivalentes las siguientes dos afirmaciones
	
	\begin{itemize}
		
		\item$S=\{A,v_1,\ldots,v_n\}$ es un sistema de coordenadas afines en $V$.
		
		\item $\{v_1-A;\ldots,v_n-A\}$  es una base de $V$.
		
	\end{itemize}
	
	\label{Ejercicio 1}
	
	\begin{proof}
		
		
		
		Por un lado supongamos que $S = \sett{A,v_1, \dots, v_n}$ es un sistema de coordenadas afines de $V$, y veamos que $\B = \sett{v_1 -A , \dots , v_n -A}$ es base de $V$.
		
		Sea $0 = \Bigsum{1 \leq  i \leq n}{x_i (v_i -A)}$, entonces $A = A +  \Bigsum{1 \leq i \leq n}{x_i . (v_i -A)} = \BigsumA{1 \leq i \leq n}{A}{x_i ._A v_i}$. Por lo tanto se tiene que $ 0 =_A \BigsumA{1 \leq i \leq n}{A}{{x_i} ._A v_i}$ y como $S$ es un sistema de coordenadas afines en $V$ se tiene que $x_i = 0$ para todo $1 \leq i \leq n$. Conclu\'imos que $\B$ es linealmente independiente.
		
		Por el teorema de la dimensi\'on, como $n = dim \ V = \vert \B \vert$ y $\B$ es linealmente independiente, se concluye que $\B$ es base.
		
		
		Para el otro lado, sea $A = \BigsumA{1 \leq i \leq n}{A}{x_i ._A v_i} = A + \Bigsum{1 \leq i \leq n}{x_i .(v_i -A)}$ y por lo tanto $0 = \Bigsum{1 \leq i \leq n}{x_i . (v_i -A)}$. Como $\B$ es base de $V$ entonces $x_i = 0$ para todo $1 \leq i \leq n$ y por lo tanto $\sett{v_1 -A, \dots , v_n -A}$ es linealmente independiente en $V_A$; por el mismo razonamiento de dimensi\'on conclu\'imos que $S$ es un sistema de coordenadas afines en $V$. \qed
		
	\end{proof}
	
	\item Sean $V$ un espacio vectorial de dimensi�n $n$ y $S=\{A,v_1,\ldots,v_n\}$ un sistema de
	coordenadas afines en $V$. Dado $v\in V$, notaremos con $[v]_S$ al vector
	de coordenadas de $v$ con respecto a la base $\{v_1,\ldots,v_n\}$ de $V_A$; 
	esto es, $[v]_S=(a_1,\ldots,a_n)$ si y solo si
	\[
	v = a_1\cdot_Av_1 +_A\cdots+_A a_n\cdot_Av_n.
	\]
	
	\begin{itemize}
		\item Hallar $[v]_S$ en los casos siguientes.
		
		\begin{itemize}
			\item $V = \R^3$, $S=\{(2,1,0);(0,1,0),(2,0,1),(0,0,-3)\}$  y $v=(0,0,0)$.
			
			\item $V = \R_2[X]$, $S=\{X^2;X+1,X^2+3X,X^2+2\}$  y  $v=2X$.
			
			\item 
			$V = \R^{2\times 2} $,
			\[
			S = 
			\left\{
			\begin{pmatrix}
			0&1\\ %%@
			0&0
			\end{pmatrix}
			;
			\begin{pmatrix}
			1&0\\
			0&0
			\end{pmatrix}
			,
			\begin{pmatrix}
			1&0\\ %%@
			0&1
			\end{pmatrix}
			,
			\begin{pmatrix}
			0&0\\
			2&0
			\end{pmatrix}
			,
			\begin{pmatrix}
			0&0\\ %%@
			0&1\end{pmatrix}
			\right\}
			\]  
			y $v=\begin{pmatrix}3&2\\ -1&4\end{pmatrix}$.
		\end{itemize}
		
		\item Sea $S=\{(0,-2,1);(1,0,0),(0,1,0),(0,0,1)\}$.
		Calcular $v$, sabiendo que $[v]_S=(-2,0,4)$.
		
		\item Sean $S=\{A,v_1,\ldots,v_n\}$  y $S'=\{B,w_1,\dots,w_n\}$
		dos sistemas afines en $ V $. Si $v\in  V $, expresar $[v]_{S'}$ en
		funci\'on de $[v]_S$.
		
	\end{itemize}
	
	\label{Ejercicio 2}
	
	\begin{proof}		
		
		\begin{easylist}
			
			\ListProperties(Style1*=\bfseries,Numbers2=l,Mark1={},Mark2={)},Indent2=1em)
			  
			@ Primer item
				
			@@ $V = \R^3 \quad  S = \sett{(2, 1, 0) ; (0, 1, 0),(2, 0, 1),(0, 0, -3) }$ y $v = (0, 0, 0)$
							
			Sea $\B = \sett{(-2,0,0), (0,-1,1), (-2,-1,-3)}$, por el ejercicio \ref{Ejercicio 1} se tiene que $\B$ es una base de $\R^3$ y notemos que $[t_{-A}(v)]_{\B} = [(-2,-1,0)]_{\B} = (\frac{3}{4}, \frac{3}{4}, \frac{1}{4})$. Por la te\'orica tenemos entonces que $[t_{A}(t_{-A}(v))]_S = [v]_S = (\frac{3}{4}, \frac{3}{4}, \frac{1}{4})$.
							
			@@ $V = \R_2[X] \quad S = \sett{X^2; X + 1, X^2 + 3X, X^2 + 2}$ y $v = 2X$
							
			Nuevamente sea $\B = \sett{-X^2 +X +1 , 3X, 2}$ que por \ref{Ejercicio 1} es base de $\R_2[X]$ y notemos que $[v]_S = [t_{-A}(2X)]_{\B} = [-X^2 + 2X]_{\B} = (1,\frac{1}{3},-\frac{1}{6})$.
				
			@@ Igual
				
			@@ Igual
			

			@ Sea $\B = \sett{(1,2,-1), (0,3,-1) , (0,2,0)}$, luego $[(0,-2,1)]_{\B} = (0,-1,\frac{1}{2})$, por lo tanto sabemos de la te\'orica que $[v]_{S} = [v-(0,-2,1)]_{\B}$; por lo tanto $[v]_{\B} = (0,-1,\frac{1}{2}) + (-2,0,4)$. 
			
			@ Sea $\B= \sett{v_1 -A , \dots , v_n -A}$ y $\B' = \sett{w_1 -B , \dots , v_n - B}$ y luego $[v]_{S'} = [t_{-B}(v)]_{\B'} = C_{\B' , \B} [t_{-B}(v)]^{t}_{\B}$; por otro lado $[v]_{S} = [t_{-A}(v)]_{\B}$ y por lo tanto si consideremos $t(v) = v + A - B$ se tiene que $[v]_{S'} = C_{\B' , \B} [t(t_{-A}(v))]^{t}_{\B} = C_{\B' , \B} \norm{\phi_t}_{\B} [t_{-A}(v)]_{\B}^{t} = C_{\B' , \B} \norm{\phi_t}_{\B} [v]_{S}^{t}$. Donde como $A \neq B$ entonces $\norm{\phi_t},C_{\B' , \B} \in GL_3(\R)$. \qed
			
			
		\end{easylist}
		
	\end{proof}
	
	\item Sea $m\geq 2$. El conjunto 
	$\{v_1,\dots,v_m \}\subset V$ es \emph{af�nmente independiente}
	si $\{v_2-v_1,\dots,v_m-v_1 \}$ es linealmente independiente.
	Probar que son
	equivalentes las siguientes afirmaciones.
	\begin{enumerate}
		
		\item El conjunto $\{v_1,\dots,v_m\}$ es af�nmente independiente.
		
		\item Si $\sum_{i=1}^m\lambda_iv_i=0$ con
		$\sum_{i=1}^m\lambda_i=0$ entonces $\lambda_i=0$ para
		$1\leq i\leq m$.
		
		\item Dado $1\leq j \leq m$, el conjunto
		$\{v_1,\ldots,\hat{v}_j,\ldots,v_m\}$ es linealmente independiente
		en $V_{v_j}$.
		
	\end{enumerate}
	
	\label{Ejercicio 3}
	
	\begin{proof}
		
		Vayasmo de a partes:
		
		\begin{enumerate}
			\item [i) $\Longrightarrow$ ii)] Sea $0 = \Bigsum{1 \leq i \leq n}{\lambda_i v_i}$, luego $0 = \Bigsum{1 \leq i \leq n}{\lambda_i v_i} - \Bigsum{1 \leq i \leq n}{\lambda_i v_1}$ pues $\Bigsum{1 \leq i \leq n}{\lambda_i} = 0$. Por lo tanto $0 = \Bigsum{2 \leq i \leq n}{\lambda_i (v_i - v_1)}$ y como $\sett{v_1, \dots , v_n}$ es afinmente independiente se concluye que $\lambda_i = 0$ para todo $2 \leq i \leq n$; finalizamos pues $\Bigsum{1 \leq i \leq n}{\lambda_i} = \lambda_1 = 0$.
		
			\item [ii) $\Longrightarrow$ iii)] Notemos que la hip\'otesis implica que el conjunto $\sett{(v_1,1) , \dots, (v_m,1)}$ es linealmente independiente pues si $0 = \Bigsum{1 \leq i \leq m}{\mu_i (v_i,1)}$ entonces se tiene que $\Bigsum{1 \leq i \leq m}{\mu_i v_i} = 0$ y que $\Bigsum{1 \leq i \leq m}{\mu_i} = 0$ luego se tiene que $\mu_i = 0$ para todo $1 \leq i \leq m$. Por lo tanto $\B = \sett{v_1 - v_j , \dots , v_{j-1} - v_j , v_{j+1} - v_j , \dots , v_m - v_j}$ es linealmente independiente, por \ref{Ejercicio 1} se tiene que $\sett{v_1 , \dots, v_{j-1}, v_{j+1}, \dots , v_m}$ es linealmente independiente en $V_{v_j}$.
			
			\item [iii) $\Longrightarrow$ i)] Por \ref{Ejercicio 1} esto vale. \qed
			
		\end{enumerate}
		
		
	\end{proof}
	
	\item Un conjunto $S\subseteq\R^n$ est� en \emph{posici�n general} si todo 
	subconjunto $A\subseteq S$ de cardinal menor o igual a $n+1$ es af�nmente
	independiente. Probar que para todo $n\in\N$ el conjunto infinito
	$S = \{ (t,t^2,\ldots, t^n)~:~t\in\R\}\subseteq\R^n$ est� en posicion 
	general en $\R^n$ 
			
	\label{Ejercicio 4}
	
	\begin{proof}
		
		Sea $A \subseteq S$ tal que $|A| = m \leq n+1$, luego existen $t_1 \neq \dots \neq t_{m} \neq 0$ (error pr\'actica) tal que $A = \sett{(t_1, \dots , t_{1}^{n}), \dots , (t_m, \dots , t_{m}^{n}) }$ y queremos ver que este conjunto es afinmente independiente. Por \ref{Ejercicio 3} hab\'iamos visto que esto es equivalente a que el conjunto $\sett{(1,t_1, \dots , t_{1}^{n}), \dots , (1,t_m, \dots , t_{m}^{n}) }$ sea linealmente independiente, que es equivalente a que sea inversible:
		
		\[
		V = \left(
		\begin{array}{cccc}
		1 & t_1 & \dots & t_{1}^{n} \\ 
		\vdots  & \ddots &  & \vdots \\
		\vdots & & \ddots & \vdots \\
		1 & t_m & \dots  & t_{m}^{n}  
		\end{array} 
		\right)
		\]
		
		Pero como los $t_i$ son diferentes, entonces es sabido que la matriz de Vandermonde es inversible. \qed
		
	\end{proof}	
	
	\item Un subconjunto no vac�o $M$ de un espacio vectorial $V$ se dice
	\emph{variedad lineal} si existe $A\in V$ tal que $M$ es un subespacio de
	$V_A$. Probar que, dado $M$ un subconjunto no vac�o de $V$, son 
	equivalentes las siguientes afirmaciones.
	
	\begin{enumerate}
		
		\item $M$ es una variedad lineal.
		
		\item $M$ es un subespacio de $V_B$ para todo $B\in M$.
		
		\item $M-_CA$ es un subespacio de $V_C$ para todo $A\in M$ y todo $C\in V$.
		
		\item $M-A$ es un subespacio de $V$ para todo $A\in M$.
		
		\item $M-A$ es un subespacio de $V$ para alg�n $A\in M$.
		
		\item Existen $A\in V$ y $S$ subespacio de $V$ tales que $M=A+S$.
		
	\end{enumerate}
	
	\label{Ejercicio 5}
	
	\begin{proof}
		
		Vayamos de a partes:
		
		\begin{itemize}
			
			\item [i) $\Longrightarrow$ ii)] Sea $B \in M$, como $M$ es variedad lineal existe $A \in V$ tal que $M$ es subespacio de $V_{A}$, luego $t_{B-A}(M)$ es subespacio de $t_{B-A}(V_A) = V_B$. Finalizamos notando que como $B \in M$ y $M$ es subespacio de $V_A$ entonces $B-A \in M$ y luego $t_{B-A}(M) = M$.
			
			\item [ii) $\Longrightarrow$ iii)] Sean $A \in M$ y $C \in V$, luego $M$ es subespacio de $V_A$ y notando que $M-_C A = M-A + C$ tenemos que $M-A$ es subespacio de $V$, luego $t_{C}(M-A) = M -_C A$ es subespacio de $V_C$
			
			\item [iii) $\Longrightarrow$ iv)] Supongamos que existe un $A \in M$ tal que $M - A$ no es subespacio de $V$, entonces por definici\'on $t_C(M -A) = M-A + C = M -_C A$ no es subespacio de $V_C$; conclu\'imos que $M-A$ es subespacio de $V$ para todo $A \in V$.
			
			\item[iv) $\Longrightarrow$ v)] Trivial
			
			\item [v) $\Longrightarrow$ vi)] Sabemos que existe $A \in M \subseteq V$ tal que $M-A = S$ es subespacio de $V$, luego $M = A +S$
			
			\item [vi) $\Longrightarrow$ i)] Trivial \qed
			
		\end{itemize}
		
	\end{proof}	
	
	\item Sea $V$ un espacio vectorial y $v_1,\dots,v_k\in V$.
	Llamamos al subconjunto de $V$
	\[
	\sigma(v_1,\dots,v_k)
	:= \left\{
	\sum_{i=1}^k\lambda_iv_i \tq \sum_{i=1}^k\lambda_i=1
	\right\}
	\]
	el conjunto de las combinaciones
	afines de  $\{v_1,\dots,v_k\}$. Probar que
	$\sigma(v_1,\dots,v_k)$ es una variedad lineal y que es la menor
	que incluye a $\{v_1,\dots,v_k\}$. �Qu� dimensi�n tiene?
	
	\label{Ejercicio 6}
	
	\begin{proof}
		
		Veamos primero que si $a,b \in \sigma(v_1, \dots , v_n)$ y $\lambda \in \R$ entonces $\lambda a + (1-\lambda) b \in \sigma(v_1 , \dots , v_n)$.
		
		Para esto existen $\mu_{1}^{a} , \dots , \mu_{n}^{a} , \mu_{1}^{b} , \dots , \mu_{n}^{b}$ tal que $a = \Bigsum{1 \leq i \leq n}{\mu_{i}^{a} v_i}, b = \Bigsum{1 \leq i \leq n}{\mu_{i}^{b} v_i}$; luego $\lambda a + (1-\lambda) b = \Bigsum{1 \leq i \leq n}{(\lambda \mu_{i}^{a} + (1- \lambda)\mu_{i}^{b})v_i}$ y $\Bigsum{1 \leq i \leq n}{\lambda \mu_{i}^{a} + (1- \lambda)\mu_{i}^{b}} = \lambda \Bigsum{1 \leq i \leq n}{\mu_{i}^{a}} + (1- \lambda) \Bigsum{1 \leq i \leq n}{\mu_{i}^{b}} = \lambda + 1 - \lambda = 1$. Por lo tanto tenemos que $\lambda a + (1-\lambda) b \in \sigma(v_1 , \dots , v_n)$.
		
		Sea $A \in V$, $S = \sett{v -A \ / \ v \in \sigma(v_1 , \dots , v_n)}$ y $s \in S$; luego $s + A \in \sigma(v_1, \dots , v_n)$ y entonces $\lambda(s+A) + (1-\lambda)A = \lambda s +A \in \sigma(v_1 , \dots , v_n)$; por lo atnto $\lambda s \in S$.
		
		Finalmente, sean $x,y \in S$ y luego $\frac{1}{2}(x+y) +A = \frac{1}{2}(x+A) + \frac{1}{2}(y+A)$ y por lo tanto $\frac{1}{2}(x+y) \in S$; por lo anterior $(x+y) \in S$.
		
		Conclu\'imos que $S$ es un subespacio y entonces $M = S +A$ y por \ref{Ejercicio 5} es una variedad lineal.
		
		Claramente es la m\'as chica que contiene a $\sett{v_1, \dots , v_n}$ y finalmente tiene dimensi\'on $n-1$. \qed
		
		
	\end{proof}
	
	\item Hallar un conjunto de generadores af�nmente independientes
	de las siguientes variedades lineales.
	
	\begin{itemize}
		
		\item  $M=\{x\in \R^3 \tq x_1-x_3=2 \ ,\ 2x_1+x_2-x_3=1\}$.
		\item  $M=\{x\in \R^3 \tq 2x_1+x_2=0\}$.
		\item  $M=\{x\in \R^2 \tq x_1-x_2=2\ \ 2x_1-3x_2=1\}$.
		\item  $M=\{x\in \R^5 \tq x_1-x_2+x_3-x_4+x_5=-2\}$.
		\item  $M\subseteq\R_2[X]$ la menor variedad lineal que contiene al conjunto
		\[
		\{ 4X^2+2X, 2X^2+X, 3X^2+X+1, 5X^2+2X+1\}.
		\]
	\end{itemize}
	
	\label{Eercicio 7}
	
	\begin{proof}
		
		\begin{itemize}
			
			\item $M = \sett{x \in \R^3 \ / \ x_1 - x_3 = 2 , 2x_1 + x_2 - x_3 = 1}$
			
			Notemos que $A = (2,-3,0) \in M$ y luego $M - A = S$ es un subespacio dado por $S =  \sett{x \in \R^3 \ / \ (x_1 +2) - x_3 = 2 , 2(x_1 +2 ) + (x_2 -3) - x_3 = 1} = \\ \sett{ x \in \R^3 \ / \ x_1 - x_3 = 0 , 2x_1 +x_2 - x_3 = 0} = \left\langle (1,-1,1) \right\rangle$
			
			\item Igual
			
			\item Igual
			
			\item Igual
			
			\item $M \subseteq \R_2[X]$ la menor variedad lineal que tiene a $\sett{4X^2 + 2X, 2X^2 + X, 3X^2 + X + 1, 5X^2 + 2X + 1}$. Es claro que el conjunto no esa finmente independiente pues $x_2 + v_3 = v_4$, luego $M = 2X^2 + x + \langle 4X^2 + 2X, 3X^2 + X + 1 \rangle$.
					
			
		\end{itemize}
		
	\end{proof}
	
	\item Sea $ f: V \longrightarrow V $ una transformaci�n af�n $ A \in V $ y $g: V_A \longrightarrow V_A$ dada por $$ g(v) = f(v) \underset{A}{-} f(A).$$ Probar que $g$ es lineal.
	
	 \label{Ejercicio 8}
	 
	 \begin{proof}
	 	
	 	Notemos primero que como $f$ es af\'in entonces existe $h : V \rightarrow V$ transformaci\'on lineal y $p \in V$ tal que $f = t_{p} \circ h$, por otro lado consideremos $t_{-A} : V_A \rightarrow V$ dado por $t_{-A}(v + A) = v$.
	 	
	 	Sea entonces $v+ A\in V_A$, luego $f \circ t_{-A}(v + A) = f(v) = p + h(v)$ y por otro lado $f \circ t_{-A} (A) = p + h(0) = p$; por lo tanto $f \circ t_{-A} = f \circ t_{-A} (A) + h$. Consideremos fionalmente $t : V_p \rightarrow V_A$ dado por $t(x + p) = x + A$, luego $g(v + A) = f \circ t_{-A} (A) + h(v) -  f \circ t_{-A} (A) + A = h(v) + A$.
	 	
	 	Conclu\'imos que si notamos laa coordenada de una transformaci\'on af�n $z : V_p \rightarrow V_q$ tal que $z(v + p) = h(z) + q$ como $[z] = h$; se tiene que $[g] = h$ que es lineal, y por ende $g$ es lineal.
	 	
	 	En efecto, si $v,w \in V_A$ y $\lambda \in \R$ luego $g(v+_A \lambda._Aw) = g(v+ \lambda w+A) = A + h(v + \lambda w) = h(v) +_A \lambda h(w) = g(v) +_A \lambda._A g(w)$. \qed
	 	
	 \end{proof}
	
	\item Probar que $ f: V \longrightarrow V $ es una transformaci�n af�n si y solo si existen $ A \in V $ y $ g: V \longrightarrow V $ lineal tales que $f(v) = g(v) + A $. Deducir que $f$ es isomorfismo af�n si y solo si $g$ es isomorfismo lineal.
	
	\label{Ejercicio 9}
	
	\begin{proof}
		
		Para un lado si existe $p \in V$ tal que $f : V \rightarrow V_p$ es lineal, entonces $t_{-p} \circ f : V \rightarrow V$ es lineal y por lo tanto si llamamos $g = t_{-p} \circ f$ se tiene que $f = t_p \circ g = g + p$.
		
		Para el otro es justamente \ref{Ejercicio 8} \qed
		
	\end{proof}
	
	\item Sea $f: \R^n \longrightarrow \R^n$ una funci�n que satisface $ f(\lambda \underset{x}{\cdot} y) = \lambda \underset{f(x)}{\cdot}f(y) $ para cada $x,y \in \R^n$ y $\lambda \in \R$. Mostrar que $f$  es una transformaci�n af�n.
	
	\label{Ejercicio 10}
	
	\begin{proof}
		
		Por \ref{Ejercicio 9} debemos ver que $f(v) - A = g$ es lineal para $A = f(0)$. por lo tanto veamos $f(0)$. Notemos que $\lambda_x . y = x + \lambda (y-x)$ por lo tanto sabemos que para todos $x,y \in \R^n$ se tiene que $f(x + \lambda (y-x)) = f(x) + \lambda (f(y) - f(x))$.
		
		Sea $x,y \in \R^n$ tal que $x = 0$, luego $f(\lambda y) = f(\lambda ._{0} y) = \lambda ._{f(0)} f(y)$ por lo que llamemos $p = f(0)$ debemos probar que $f : V \rightarrow V_{f(0)}$ es lineal.
		
		Finalmente si $\lambda = \frac{1}{2}$ entonces queda que $f(\frac{1}{2}(x+y)) = \frac{1}{2} (f(x) + f(y))$ y por lo tanto $f(0) + \frac{1}{2}(f(x+y) - f(0)) = f(\frac{1}{2}(x+y))  = \frac{1}{2} (f(x) + f(y))$ y juntando los extremos $\frac{1}{2} f(x+y) = \frac{1}{2} (f(0) + f(x) + f(y)))$ con lo que $f(x+y) = f(x) +_{f(0)} f(y)$. \qed
		
	\end{proof}
	
	\section{Cu�dricas}
	
	
	\underline{Notaci�n}: Dado un polinomio $P$ de grado dos en $n$ variables reales notaremos a la cu�drica que genera como $$ \mathcal{C}(P) = \{ x \in \R^n : P(x) = 0 \},$$ y a su centro como $\mbox{cent}(P) = \mbox{cent} \Big( \mathcal{C}(P) \Big).$
	
	\item Sea $\phi:V\times V\longrightarrow \R$ una forma bilineal sim\'etrica. Encontrar una base $B$ de
	$V$ de modo que la matriz de $\phi$ en $B$, $\|\phi\|_B$,  sea
	diagonal.
	
	\begin{itemize}
		
		\item $V=\R^3$, $\|\phi(x)\|_E=\left(\begin{matrix}1 & 1 & 0 \\  1 & 0 & -1 \\ 0 & -1
		&-1\end{matrix}\right)$, donde $E$ es la base can\'onica de
		$\R^3$.
		
		\item $V=\R^4$, $\phi(x,y)=3x_1y_1+2x_1y_2+2x_2y_1$.
		
		\item $V=\R^3$, $\|\phi(x)\|_B=\left(\begin{matrix}1 & 0 & 1 \\  0 & 1 & 0 \\ 1 &
		0 & 1\end{matrix}\right)$, donde $B=\{(1,0,0),(1,0,1),(0,1,1)\}$.
		
	\end{itemize}
	
	\label{Ejercicio 11}
	
	\begin{proof}
		
		Hay que diagonalizar. \qed
		
	\end{proof}
	
	\item Sea $F:\R^3\longrightarrow\R$ la funci\'on cuadr\'atica
	cuya expresi\'on en la base can\'onica es
	\[
	F(x)=x_1^2-x_2^2+3x_1x_2+x_2x_3-x_1+3x_2-10.
	\]
	Encontrar la
	expresi\'on de $F$ en $\R^3_{(1,0,2)}$, $\R^3_{(1,1,5)}$ y
	$\R^3_{(0,0,1)}$.
	
	\label{Ejercicio 12}
	
	\begin{proof}
		
		\begin{enumerate}
			
			\item Notemos que $\|\phi(x)\|_E=\left(\begin{matrix}1 & 3 & 0 \\  0 & -1 & 1 \\ 0 & 0
				& 0 \end{matrix}\right)$ y por lo visto en la pr\'actica podemos tomar la simetrizaci\'on de $\phi$ tal que $\tilde{\phi}(x,y) := \frac{\phi(x,y) + \phi(y,x)}{2}$ y entonces $\|\tilde{\phi}(x)\|_E=\left(\begin{matrix}1 & \frac{3}{2} & 0 \\  \frac{3}{2} & -1 & \frac{1}{2} \\ 0 & \frac{1}{2}
				& 0 \end{matrix}\right)$. Adem\'as, $\varphi(x) = \frac{1}{2} \langle (-1,3,0 ) , {x} \rangle$ y $c = -10$.
				
				Por lo tanto si $A = (1,0,2)$ entonces $c_A = F(A) = 1 - 0 +0+0-1+0-10 = -10$; adem\'as:
				
				\[
				\begin{array}{ccc}
					\phi(X-A,X-A) & = & (x_1 - 1, x_2, x_3 - 2). \left(\begin{matrix}1 & 3 & 0 \\  0 & -1 & 1 \\ 0 & 0 & 0 \end{matrix}\right) . (x_1 - 1, x_2, x_3 - 2)^t \\
					\phi(X-A,X-A) & = & (x_1 -1)^2 + 3(x_1 -1)x_2 - x_{2}^{2} + x_2(x_3-2)
				\end{array} 				
				\]
				
				Y finalmente $\varphi_A = \phi(X-A,A) + \varphi(X-A)$ con lo que:
			
				\[
				\begin{array}{ccc}
				\varphi_A & = & (x_1-1) 2x_2 + \frac{1}{2} \langle (-1,3,0 ) , {x} \rangle \\
				\varphi_A & = & \frac{1}{2} x_1 + \frac{7}{2} x_2 -1
				\end{array} 				
				\]
				
				Finalmente entonces $F_A = (x_1 -1)^2 + 3(x_1 -1)x_2 - x_{2}^{2} + x_2(x_3-2) + x_1 + 7x_2 - 2 - 10$. No pienso hacer en los otros dos puntos... \qed
			
		\end{enumerate}
		
	\end{proof}
	
	\item Dado un polinomio de grado dos en $n$ variables reales $P$ probar que para cualquier transformaci�n af�n inversible $f$ ($f(x) = T x + b$ con $T \in GL(n,\R))$)
	vale que:
	\begin{itemize}
		\item $\mbox{cent}(P \circ f) = f^{-1} \mbox{cent}(P)$.
		\item $ \mathcal{C}(P \circ f) = f^{-1} \mathcal{C}(P) $.
	\end{itemize}
	
	\label{Ejercicio 13}
	
	\begin{proof}
		
		Vayamos de a partes:
		
		\begin{enumerate}
			
			\item Sea $p \in \mbox{cent}(P \circ f)$, luego si $P \circ f = \psi_f + 2 \varphi_f + c_f$ tenemos que $\norm{\phi_f} . p^t = b^t$. Pero $\phi_f(x,y) = \phi(f(x),f(y))$ y por lo tanto si $f = Tx + c$ tenemos que $\norm{\phi_f} = T^t\norm{\phi}T $. Esto junto nos dice que $T^{-1}b^t = \norm{\phi}T p^t = \norm{\phi}g^t$ con $g^t = Tp^t$, o sea que $f(p) \in \mbox{cent}(P)$. Es claro que todos los pasos eran si y s\'olo si por ende vale la rec\'iproca tomando $f^{-1} = f$.
			
			\item Si $P(f(v)) = 0$ entonces $f(v) \in \mathcal{C}(P)$, y para el otro lado si $f(v)$ es tal que $P(f(v)) = 0$ entonces $v \in \mathcal{C}(P \circ f)$. \qed
			
		\end{enumerate}
		
	\end{proof}
	
	\item Dado un polinomio de grado dos en $n$ variables reales $P$ probar que 
	$$ \mbox{cent}(P) = \{ y \in \R^n : \frac{\partial P}{\partial x_i} (y)= 0 \; \forall \; i= 1, \ldots, n \}. $$
	
	\label{Ejercicio 14}
	
	\begin{proof}
		
		Sea $p \in \mbox{cent}(P)$ si y s\'olo si $P_p = \psi_p + c_p$, luego $\dfrac{\partial P}{\partial x_i}(p) = \dfrac{\partial \psi_p}{\partial x_i}(p) = (x-P) \norm{\phi} |_{p} + \norm{\phi} (x-P)^t |_{P} = 0$.
		
		Rec\'iprocamente si $\dfrac{\partial P}{\partial x_i}(p) = 0$ entonces $P$ no es lineal en $x - P$ y luego se tiene que $\varphi_p = 0$, o sea $p \in \mbox{cent}(P)$. \qed
		
	\end{proof}
	
	Ahora es en el resto de la gu\'ia usar \ref{Ejercicio 14} y hacer a lo analisis 1 los puntos de gradiente 0, ya tiene cero gracia hacer la forma a lo keilhauer (sirvio pa las demos)
	
	\item En cada uno de los siguientes casos encontrar el conjunto de centros de la cu\'adrica $Q$.
	
	\begin{itemize}
		
		\item $Q:\x x12+2\x x22-2\xx xx12+2x_2-1=0$\hfill (en $\R^2$)
		
		\item $Q:\x x12+4\x x22-4\xx xx12+2x-4x_2-5=0$ \hfill (en $\R^2$)
		
		\item $Q:\x x12-2\xx xx12+2\x x22+2x_1+3=0$ \hfill (en $\R^2$)
		
		\item $Q:\x x12-\x x22+\x x32+1=0$ \hfill (en $\R^3$)
		
		\item $Q:\x x12+\x x22-\x x32+2\xx xx12+2x_3+1=0$ \hfill (en $\R^3$)
		
		\item $Q:2\x x12-\x x22-\x x32+2\xx xx12+2\xx xx13-2\xx xx23+2x_1+4x_2+6x_3
		-4=0$ \hfill (en $\R^3$)
		
		\item $Q:\x x12+\x x22-\xx xx12+x_3-7=0$ \hfill (en $\R^4$)
		
	\end{itemize}
	
	\begin{proof}
		
		Para pr\'acticar vamos a llevar a la forma normal a las cu\'adricas 1 y 6 pues parece que son con y sin centros.
		
		\begin{enumerate}
			
			\item  $Q:\x x12+2\x x22-2\xx xx12+2x_2-1=0$\hfill (en $\R^2$)
			
			Primero notemos que  $ \nabla F = (2(x_1-x_2) , 4x_2 - 2x_1 + 2)$ y entonces $\nabla F = 0$ si y s\'olo si:
			
			\[
			\begin{array}{ccc}
			x_1 - x_2 & = & 0 \\
			4x_2 - 2x_1 + 2 & = & 0
			\end{array}
			\]
			
			Que pasa si y s\'olo si $(x_1,x_2) = (-1,-1)$, luego por \ref{Ejercicio 14} se tiene que $\emph{cent}(P) = \sett{(-1,-1)}$. Como $F((-1,-1)) = -2 \neq 0$ entonces el centro no est\'a en la cu\'adrica y estamos en el caso de las esferas. Sea $A = (-1,-1)$, luego por un lado $c_{A} = F(A) = -2$ y  adem\'as $\psi_{A} = \psi(X-A) = (x_1 +1 ,x_2 +1) 
			\left( \begin{array}{cc} 
			1 & 0 \\
			-2 & 2
			\end{array} \right)
			\left( \begin{array}{c}
			x_1 +1 \\
			x_2 +1
			\end{array} \right) = (x_1 +1)^2 + 2(x_2 +1)^2 -2(x_1+1)(x_2+1)$. 
			
			Finalmente, $\varphi_{A} = \phi(A,X-A) + \varphi(X-A) = (-1,-1) \left( \begin{array}{cc} 
			1 & -1 \\
			-1 & 2
			\end{array} \right)
			\left( \begin{array}{c}
			x_1 +1 \\
			x_2 +1
			\end{array} \right) + (x_2+1) = -(x_2 +1) + (x_2+1) = 0$; conclu\'imos que:
			
			\[
			F_A = (x_1 +1)^2 + 2(x_2 +1)^2 -2(x_1+1)(x_2+1) - 2
			\]
			
			
			Ahora para llevar $F_A$ a la forma normal  en vez de diagonalizar $\norm{\phi}$ vamos a ver el signo de los autovalores con el m\'etodo de Pancho, para eso notemos que:
			
			\[
			\left( \begin{array}{cc} 
			1 & -1 \\
			-1 & 2
			\end{array} \right)  \xrightarrow{F_2 -> F_2 + F_1} \left( \begin{array}{cc} 
			1 & -1 \\
			0 & 1
			\end{array} \right) \xrightarrow{C_2 -> C-2 + C_1} \left( \begin{array}{cc} 
			1 & 0 \\
			0 & 1
			\end{array} \right) 			
			\]
			
			Entonces sabemos que ambos autovalores son positivos y entonces $F \simeq x_1^2 + x_2^2 - 1$. Ahora supongamos que nos piden la base $\B$ tal que presenta la equivalencia, entonces deer\'iamos diagonalizar, para eso veamos: 
			
			\[
			\begin{aligned}
			\chi(\lambda) = & det 
			\left( \begin{array}{cc} 
			\lambda - 1 & +1 \\
			+1 & \lambda - 2
			\end{array} \right) \\
			 = & (\lambda -1)(\lambda -2) - 1 \\
			 = & \lambda^2 -3\lambda +1 \\
			 = & (\lambda - (\frac{1}{2}(3 + \sqrt{5}) ))(\lambda - (\frac{1}{2}(3 - \sqrt{5}))) 
			\end{aligned}
			\]
			
			Y claramente los autovectores son demasiado feos para analizar.
			
			\item $Q:2\x x12-\x x22-\x x32+2\xx xx12+2\xx xx13-2\xx xx23+2x_1+4x_2+6x_3
			-4=0$ \hfill (en $\R^3$)
			
			Nuevamente primero notemos que $\nabla F = (4x_1 + 2x_2 + 2x_3 + 2 , -2x_2 +2x_1 -2x_3 +4 , -2x_3 +2x_1 -2x_2 +6) = 0$ si y s\'olo si:
			
			\[
			\begin{array}{ccc}
			4x_1 + 2x_2 + 2x_3 + 2 & = & 0 \\
			-2x_2 +2x_1 -2x_3 +4  & = & 0 \\
			-2x_3 +2x_1 -2x_2 +6 & = & 0
			\end{array}
			\]
			
			si y s\'olo si:
			
			\[
			\left( \begin{array}{ccc}
			2 &  1 & 1  \\
			1 & -1 & -1 \\
			1 & -1 & -1
			\end{array} \right) \left(
			\begin{array}{c}
			x_1 \\
			x_2 \\
			x_3
			\end{array} \right) = \left(
			\begin{array}{c}
			-1 \\
			-2 \\
			-3
			\end{array}
			\right)
			\]
			
			Y se ve claramente de las condiciones 2 y 3 que el conjunto de soluciones es vac\'io y entonces la c\'uadrica no tiene centro. Para ver la equivalencia af\'in veamos el signo de los autovalores con el m\'etodo de Pancho:
			
			\[
			\begin{aligned}
			\left( \begin{array}{ccc}
			2 &  1 & 1  \\
			1 & -1 & -1 \\
			1 & -1 & -1
			\end{array} \right) & \xrightarrow{F_3 -> F_3 - F_2} \left( \begin{array}{ccc}
			2 &  1 & 1  \\
			1 & -1 & -1 \\
			0 & 0 & 0
			\end{array} \right) \xrightarrow{C_3 -> C_3 - C_2} \left( \begin{array}{ccc}
			2 &  1 & 0  \\
			1 & -1 & 0 \\
			0 & 0 & 0
			\end{array} \right) \\ 
								& \xrightarrow{F_1->F_1 + F_2} \left( \begin{array}{ccc}
			3 &  0 & 0  \\
			1 & -1 & 0 \\
			0 & 0 & 0
			\end{array} \right) \xrightarrow{C_1->C_1 + C_2} \left( \begin{array}{ccc}
			3 &  0 & 0  \\
			0 & -1 & 0 \\
			0 & 0 & 0
			\end{array} \right) \\ 
								& \xrightarrow{F_1 -> F_1 \frac{1}{\sqrt{3}}} \left( \begin{array}{ccc}
			\frac{3}{\sqrt{3}} &  0 & 0  \\
			0 & -1 & 0 \\
			0 & 0 & 0
			\end{array} \right) \xrightarrow{C_1 -> C_1 \frac{1}{\sqrt{3}}} \left( \begin{array}{ccc}
			1 &  0 & 0  \\
			0 & -1 & 0 \\
			0 & 0 & 0
			\end{array} \right)
			\end{aligned}
			\]
			
			Por lo tanto sabemos que $F \simeq x_1^2 - x_2^2 -2x_3$. 
			
		\end{enumerate}
		
	\end{proof}
	
	\item Determinar el conjunto de puntos singulares $Q_S=Q_c\cap Q$ para 
	cada una de las siguientes cu\'adricas de $\R^n$.
	
	\begin{itemize}
		
		\item $Q:2\x x12+\x x22+\x x32-4x_2+1=0$ \hfill ($n=3$)
		
		\item $Q:\x x12-\x x22+\x x32-2\xx xx13-2\xx xx24+4x_4=0$ \hfill ($n=4$)
		
		\item $Q:\x x12-\x x22-\x x42-2\xx xx13+2\xx xx23-2\xx xx34+x_1-x_2+2x_3+x_
		4=0$\hfill ($n=4$)
		
		\item $Q:\xx xx12+\xx xx34+\x x52=0$\hfill ($n=5$)
		
		\item $Q:\x x12-2x_1+1=0$\hfill ($n=2$)
		
	\end{itemize}
	
	
	\item Determinar los $a\in\R$ para los cuales la cu\'adrica $Q$
	de $\R^3$ de ecuaci\'on
	\[
	Q:
	\x x12+(a^2+3)\x x22+(a^2-3)\x x32+(2a+4)\xx xx12+2\xx xx13+x_1-2x_2+1
	=0
	\]
	tiene centro \'unico.
	
	
	\item Sean $L$ una recta y $Q$ una cu\'adrica. Probar que el conjunto
	$L\cap Q$ bien es vac�o, tiene solo un punto, tiene solo dos puntos, o es
	todo $L$.

\end{enumerate}


\end{document}