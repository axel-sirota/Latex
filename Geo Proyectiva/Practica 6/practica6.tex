\documentclass[11pt]{article}

\usepackage{amsfonts}
\usepackage{amsmath,accents,amsfonts, amssymb, mathrsfs }
\usepackage{tikz-cd}
\usepackage{graphicx}
\usepackage{syntonly}
\usepackage{color}
\usepackage{mathrsfs}
\usepackage[spanish]{babel}
\usepackage[latin1]{inputenc}
\usepackage{fancyhdr}
\usepackage[all]{xy}
\usepackage[at]{easylist}


\topmargin-2cm \oddsidemargin-1cm \evensidemargin-1cm \textwidth18cm
\textheight25cm


\newcommand{\B}{\mathcal{B}}
\newcommand{\Cont}{\mathcal{C}}
\newcommand{\F}{\mathcal{F}}
\newcommand{\inte}{\mathrm{int}}
\newcommand{\A}{\mathcal{A}}
\newcommand{\C}{\mathbb{C}}
\newcommand{\Q}{\mathbb{Q}}
\newcommand{\Z}{\mathbb{Z}}
\newcommand{\inc}{\hookrightarrow}
\renewcommand{\P}{\mathcal{P}}
\newcommand{\R}{{\mathbb{R}}}
\newcommand{\N}{{\mathbb{N}}}
\newcommand\tq{~:~}
\newcommand{\x}[3]{#1_#2^#3}
\newcommand{\xx}[4]{#1_#3#2_#4}
\newcommand\dd{\,\mathrm{d}}
\newcommand\norm[1]{\left\lVert#1\right\rVert}
\newcommand\abs[1]{\left\lvert#1\right\rvert}
\newcommand\ip[1]{\left\langle#1\right\rangle}
\renewcommand\tt{\mathbf{t}}
\newcommand\nn{\mathbf{n}}
\newcommand\bb{\mathbf{b}}                      % binormal
\newcommand\kk{\kappa}
\newcommand{\sett}[1]{\left\lbrace#1\right\rbrace}
\newcommand{\interior}[1]{\accentset{\smash{\raisebox{-0.12ex}{$\scriptstyle\circ$}}}{#1}\rule{0pt}{2.3ex}}
\fboxrule0.0001pt \fboxsep0pt
\newcommand{\Bigcup}[2]{\bigcup\limits_{#1}{#2}}
\newcommand{\Bigcap}[2]{\bigcap\limits_{#1}{#2}}
\newcommand{\Bigprod}[2]{\prod\limits_{#1}{#2}}
\newcommand{\Bigcoprod}[2]{\coprod\limits_{#1}{#2}}
\newcommand{\Bigsum}[2]{\sum\limits_{#1}{#2}}
\newcommand{\BigsumA}[3]{ \sideset{}{^#2}\sum\limits_{#1}{#3}}
\newcommand{\Biglim}[2]{\lim\limits_{#1}{#2}}
\newcommand{\quotient}[2]{{\raisebox{.2em}{$#1$}\left/\raisebox{-.2em}{$#2$}\right.}}



\def \le{\leqslant}	
\def \ge{\geqslant}
\def\noi{\noindent}
\def\sm{\smallskip}
\def\ms{\medskip}
\def\bs{\bigskip}
\def \be{\begin{enumerate}}
	\def \en{\end{enumerate}}
\def\deck{{\rm Deck}}
\def\Tau{{\rm T}}

\newtheorem{theorem}{Teorema}[section]
\newtheorem{lemma}[theorem]{Lema}
\newtheorem{proposition}[theorem]{Proposici\'on}
\newtheorem{corollary}[theorem]{Corolario}

\newenvironment{proof}[1][Demostraci\'on]{\begin{trivlist}
		\item[\hskip \labelsep {\bfseries #1}]}{\end{trivlist}}
\newenvironment{definition}[1][Definici\'on]{\begin{trivlist}
		\item[\hskip \labelsep {\bfseries #1}]}{\end{trivlist}}
\newenvironment{example}[1][Ejemplo]{\begin{trivlist}
		\item[\hskip \labelsep {\bfseries #1 }]}{\end{trivlist}}
\newenvironment{remark}[1][Observaci\'on]{\begin{trivlist}
		\item[\hskip \labelsep {\bfseries #1}]}{\end{trivlist}}
\newenvironment{declaration}[1][Afirmaci\'on]{\begin{trivlist}
		\item[\hskip \labelsep {\bfseries #1}]}{\end{trivlist}}


\newcommand{\qed}{\nobreak \ifvmode \relax \else
	\ifdim\lastskip<1.5em \hskip-\lastskip
	\hskip1.5em plus0em minus0.5em \fi \nobreak
	\vrule height0.75em width0.5em depth0.25em\fi}

\newcommand{\twopartdef}[4]
{
	\left\{
	\begin{array}{ll}
		#1 & \mbox{ } #2 \\
		#3 & \mbox{ } #4
	\end{array}
	\right.
}

\newcommand{\threepartdef}[6]
{
	\left\{
	\begin{array}{lll}
		#1 & \mbox{ } #2 \\
		#3 & \mbox{ } #4 \\
		#5 & \mbox{ } #6
	\end{array}
	\right.
}

\tikzset{commutative diagrams/.cd,
	mysymbol/.style={start anchor=center,end anchor=center,draw=none}
}
\newcommand\Center[2]{%
	\arrow[mysymbol]{#2}[description]{#1}}

\newcommand*\circled[1]{\tikz[baseline=(char.base)]{
		\node[shape=circle,draw,inner sep=2pt] (char) {#1};}}


\begin{document}
	
	\pagestyle{empty}
	\pagestyle{fancy}
	\fancyfoot[CO]{\slshape \thepage}
	\renewcommand{\headrulewidth}{0pt}
	
	
	
	\centerline{\bf Geometr\'ia Proyectiva - $2^{\circ}$ cuatrimestre $2016$}
	\centerline{\sc Pr\'actica 6}
	
	\bigskip
	
\begin{enumerate}
	
	\item Ejercicio 1
	
	\label{Ejercicio 1}
	
	\begin{proof}
		
		Notemos que si $X,Y$ son campos paralelos entonces $\ip{X,Y} = cte$ pues $\dfrac{d}{dt} (\ip{X,Y}) = \ip{\dot{X},Y} + \ip{X,\dot{Y}} = \ip{\dot{X}^{T},Y} + \ip{\dot{X}^{\perp},Y} + \ip{X,\dot{Y}^{T}} + \ip{X,\dot{Y}^{\perp}} = \ip{\nabla_D X ,Y} + \ip{X,\nabla_DY} = 0$. Por lo tanto $\mathnormal{ang}(X,Y) = \dfrac{\ip{X,Y}}{\sqrt{\ip{X,X} \ip{Y,Y}}} = cte$. \qed
		
	\end{proof}
	
	\item Ejercicio 2
	
	\label{Ejercicio 2}
	
	\begin{proof}
		
		Sea $p=c(t) \in M$ y $(U,x)$ una carta centrada en $p$, luego si tomamos $X \in \mathcal{X}_c$ entonces  $X = \Bigsum{i = 1,2}{X^i \partial x_i}$ pues $\sett{\partial{x_1, \partial x_2}}$ es una base de $T_pM$. Luego $\dot{X} = \Bigsum{i = 1,2}{\dfrac{dX^i}{dt} \partial x_i} + \Bigsum{i = 1,2}{X^i \dfrac{d}{dt}\left(\partial x_i \right)}$. Si llamamos $x^{-1} = f$ entonces ya probamos previamente que $\partial{x_i}|_{c(t)} = f_{u_i}(x \circ c(t))$ por lo que $\dot{X} = \Bigsum{i = 1,2}{\dfrac{dX^i}{dt}|_{t} f_{u_i}(x \circ c(t))} + \Bigsum{i = 1,2}{X^i(t) \Bigsum{j =1,2}{f_{u_i u_j}(x \circ c(t)) \dfrac{d (x_j \circ c)}{dt}|_{t}}}$. Como por otro lado:
		
		\[
		f_{u_iu_j} = \Bigsum{k = 1,2}{\Gamma_{i,j}^{k} f_{u_k} + l_{i,j} \circ f \eta}
		\]
		
		Conclu\'imos que:
		
		\[
		\begin{aligned}
		\dot{X} = & \Bigsum{k = 1,2}{\left(\dot{X^k} + \Bigsum{i,j = 1,2}{ X^i \dfrac{d(x_j \circ c)}{dt} \Gamma_{i,j}^{k}(c(t))} \right) \partial x_k} \\
				  & + \Bigsum{i,j = 1,2}{X^i l_{i,j} \circ f} N
		\end{aligned}
		\]
		
		Por lo que:
		
		
		\begin{equation*}
		\nabla_D X = 0 \qquad \Longleftrightarrow \quad \dot{X^k} + \Bigsum{i,j = 1,2}{ X^i \dfrac{d(x_j \circ c)}{dt} \Gamma_{i,j}^{k}(c(t))} = 0 \quad k = 1,2
		\end{equation*}
		\qed
		
	\end{proof}
	
	\item Ejercicio 3
	
	\label{Ejercicio 3}
	
	\begin{proof}
		
		Sean $X,Y$ campos paralelos y $a \in \R$, luego $\nabla_D (X + aY) = (\dot{X} + a \dot{Y})^{T} = \nabla_D X + a \nabla_D Y = 0$. \qed
		
	\end{proof}
	
	\item Ejercicio 4
	
	\label{Ejercicio 4}
	
	\begin{proof}
		
		Sea $\alpha \in \pi \cap S$ y supongamos que esta parametrizada por longitud de arco, luego sea $\sett{\tt,\nn \times \tt, \nn}$ al referencia m\'ovil de $\alpha$ respecto de $\nn$ la normal de la superficie en una carta $(U,x)$ de un punto $p \in S \cap \pi$. Por un lado sabemos que $\dot{\tt} \perp \tt$ por lo que $\dot{\tt} = a \nn + b \nn \times \tt$; pero por el otro lado $\alpha$ es una curva plana por lo que por Serret-Frenet tenemos que $\dot{\tt} = \dfrac{e_2}{k_c} = \dfrac{\ddot{\alpha}}{k_c}$ el vector normal a la curva. Finalmente por simetr\'ia $\nn \in \pi$ pues si reflejo $\pi$ entonces $\nn$ debe ser igual; en conclusi\'on tenemos que $\dot{\alpha},\ddot{\alpha},\nn$ son generadores de $\pi$ y por lo tanto $k_g = C\ip{\ddot{\alpha} , \nn \times \dot{\alpha}} = 0$. \qed
		
	\end{proof}
	
	\item Ejercicio 5
	
	\label{Ejercicio 5}
	
	\begin{proof}
		
		\begin{itemize}
			\item 		Sea $\phi(s,v) = \left(x(s)\cos(v) , x(s) \sin(v) , z(s)\right)$ una parametrizaci\'on de la superficie de revoluci\'on generada por $\alpha$, donde asumamos que $s$ es el par\'ametrod e longitud de arco. Luego notemos que los generadores de $T_pS$ son:
		
		\[
		\begin{aligned}
		\phi_{s} = \left(\dot{x}\cos(v) , \dot{x}\sin(v),\dot{z} \right) \\
		\phi_{v} = \left(-x\sin(v) , x\cos(v),0\right) 
		\end{aligned} 
		\]
		
		Pero por el otro:
		
		\[
		\begin{aligned}
		\dot{\phi(s,v_0)} = \left(\dot{x}\cos(v_0) , \dot{x}\sin(v_0),\dot{z} \right)  \\
		\ddot{\phi(s,v_0)} = \left(\ddot{x}\cos(v_0) , \ddot{x}\sin(v_0),\ddot{z} \right)  
		\end{aligned} 
		\]
		
		Por lo que:
		
		\[
		\begin{aligned}
		\ip{\ddot{\phi}(s,v_0),\phi_s(s,v_0)} = \dot{x}\ddot{x} + \dot{z}\ddot{z} = \ip{\dot{\alpha} , \ddot{\alpha}} = \dfrac{d \left(\norm{\dot{\alpha}}^2\right)}{dt} = 0 \\
		\ip{\ddot{\phi}(s,v_0),\phi_v(s,v_0)} = 0
		\end{aligned} 
		\]
		
		Conclu\'imos que $\ddot{\gamma} = kN$ y por lo tanto $k_g = 0$.
		
		\item Para ver los paralelos veamos la misma cuenta!
		
		\[
		\begin{aligned}
		\dot{\phi}(s_0,v) = \left(-x(s_0)\sin(v) , x(s_0)\cos(v), 0\right)  \\
		\ddot{\phi}(s_0,v) = \left(-x(s_0)\cos(v) , -x(s_0)\sin(v), 0\right)  
		\end{aligned} 
		\]
		
		Por lo que:
		
		\[
		\begin{aligned}
		\ip{\ddot{\phi}(s_0,v),\phi_s(s_0,v)} = -x(s_0)\dot{x}(s_0) = -\dfrac{d \left(\frac{x^2}{2}\right) }{dt}\\
		\ip{\ddot{\phi}(s_0,v),\phi_v(s_0,v)} = 0
		\end{aligned} 
		\]
		
		Por lo que para que un paralelo sea geod\'esica debemos tener que $x(s_0) = cte$. \qed
		
		\end{itemize}
		
	\end{proof}
	
	\item Ejercicio 6
	
	\label{Ejercicio 6}
	
	\begin{proof}
		
		Sea $\alpha = at +b$ la recta recorrida a velocidad constante, trivialmente $\ddot{\alpha} = 0$ por lo que $k_g = 0$. \qed
		
	\end{proof}
	
	\item Ejercicio 7
	
	
	\label{Ejercicio 7}
	
	\begin{proof}
		
		Notemos que por definici\'on $\alpha$ es l\'inea de curvatura sii $dN_{\gamma(t)}(\tt) = \dfrac{d}{dt} \left(N \circ \gamma\right) = \lambda \tt$, pero por otro lado sabemos que $\gamma$ es geod\'esica sii $k_g = 0$ sii $\nn = N$. Por lo tanto $\tau_{\gamma} = \ip{\dfrac{d \nn}{dt} , \bb} = \ip{\lambda \tt , \bb} = 0$ por lo que $\gamma$ es planar. \qed
		
	\end{proof}
	
	\item Ejercicio 8
	
	\label{Ejercicio 8}
	
	\begin{proof}
		
		Como todas las geod\'esicas son planares, entonces por \ref{Ejercicio 7} tenemos que todas las geod\'esicas son lineas de curvatura; no obstante dado $v \in T_pM$ entonces por el teorema de existencia y unicidad de geod\'esicas (asumimos conexi\'on de $M$) se tiene que existe una geod\'esica $\gamma_v$ tal que $\dot{\gamma_v}(0) = v$, luego $dN_p(v) = k_vv$ para todo $v \in T_pM$ por lo que $p$ es umb\'ilico.
		
		Cmo segunda etapa veamos que el hecho que todo punto sea umb\'ilico implica la conclusi\'on. Tenemos que:
		
		
		\[
		\begin{aligned}
		-N_u = k \sigma_u \\
		-N_v = k \sigma_v
		\end{aligned} 
		\]		
		
		Para $\sigma(u,v)$ una parametrizaci\'on de $M$ alrededor de $p$, luego:
		
		
		\[
		\begin{aligned}
		-N_{uv}(\sigma_u) = k_v \sigma_u + k \sigma_{uv}\\
		-N_{vu}(\sigma_v) = k_u \sigma_v + k \sigma_{vu}
		\end{aligned} 
		\]		
		
		Y como $\sigma_{uv} = \sigma_{vu}$, $N_{uv} = N_{vu}$ y $\sett{\sigma_u , \sigma_v}$ son base de $T_pM$ tenemos que $k_u = k_v = 0$. Por lo tanto $N = k \sigma + c$ con $k \in \R, c \in \R^3$ constantes.
		
		Por lo tanto si $k = 0$ tenemos que $N$ es constante y por lo tanto $M \subset \pi$ o $\norm{\sigma + \dfrac{c}{\abs{k}}} = \norm{\frac{N}{k}} = \dfrac{1}{\abs{k}} = cte$ y entonces $M \subset S_{-c}(\dfrac{1}{\abs{k}})$. \qed
		
		
	\end{proof}
	
\end{enumerate}
	

\end{document}