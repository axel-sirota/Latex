\documentclass[11pt]{article}

\usepackage{amsfonts}
\usepackage{amsmath,accents,amsfonts, amssymb, mathrsfs }
\usepackage{tikz-cd}
\usepackage{graphicx}
\usepackage{syntonly}
\usepackage{color}
\usepackage{mathrsfs}
\usepackage[spanish]{babel}
\usepackage[latin1]{inputenc}
\usepackage{fancyhdr}
\usepackage[all]{xy}


\topmargin-2cm \oddsidemargin-1cm \evensidemargin-1cm \textwidth18cm
\textheight25cm


\newcommand{\B}{\mathcal{B}}
\newcommand{\Cont}{\mathcal{C}}
\newcommand{\F}{\mathcal{F}}
\newcommand{\inte}{\mathrm{int}}
\newcommand{\A}{\mathcal{A}}
\newcommand{\C}{\mathbb{C}}
\newcommand{\Q}{\mathbb{Q}}
\newcommand{\Z}{\mathbb{Z}}
\newcommand{\inc}{\hookrightarrow}
\renewcommand{\P}{\mathcal{P}}
\newcommand{\R}{{\mathbb{R}}}
\newcommand{\N}{{\mathbb{N}}}
\newcommand\norm[1]{\left\lVert#1\right\rVert}
\newcommand{\sett}[1]{\left\lbrace#1\right\rbrace}
\newcommand{\interior}[1]{\accentset{\smash{\raisebox{-0.12ex}{$\scriptstyle\circ$}}}{#1}\rule{0pt}{2.3ex}}
\fboxrule0.0001pt \fboxsep0pt
\newcommand{\Bigcup}[2]{\bigcup\limits_{#1}{#2}}
\newcommand{\Bigcap}[2]{\bigcap\limits_{#1}{#2}}
\newcommand{\Bigprod}[2]{\prod\limits_{#1}{#2}}
\newcommand{\Bigcoprod}[2]{\coprod\limits_{#1}{#2}}
\newcommand{\Bigsum}[2]{\sum\limits_{#1}{#2}}
\newcommand{\Biglim}[2]{\lim\limits_{#1}{#2}}
\newcommand{\quotient}[2]{{\raisebox{.2em}{$#1$}\left/\raisebox{-.2em}{$#2$}\right.}}



\def \le{\leqslant}	
\def \ge{\geqslant}
\def\sen{{\rm sen} \, \theta}
\def\cos{{\rm cos}\, \theta}
\def\noi{\noindent}
\def\sm{\smallskip}
\def\ms{\medskip}
\def\bs{\bigskip}
\def \be{\begin{enumerate}}
	\def \en{\end{enumerate}}
\def\deck{{\rm Deck}}
\def\Tau{{\rm T}}

\newtheorem{theorem}{Teorema}[section]
\newtheorem{lemma}[theorem]{Lema}
\newtheorem{proposition}[theorem]{Proposici\'on}
\newtheorem{corollary}[theorem]{Corolario}

\newenvironment{proof}[1][Demostraci\'on]{\begin{trivlist}
		\item[\hskip \labelsep {\bfseries #1}]}{\end{trivlist}}
\newenvironment{definition}[1][Definici\'on]{\begin{trivlist}
		\item[\hskip \labelsep {\bfseries #1}]}{\end{trivlist}}
\newenvironment{example}[1][Ejemplo]{\begin{trivlist}
		\item[\hskip \labelsep {\bfseries #1 }]}{\end{trivlist}}
\newenvironment{remark}[1][Observaci\'on]{\begin{trivlist}
		\item[\hskip \labelsep {\bfseries #1}]}{\end{trivlist}}
\newenvironment{declaration}[1][Afirmaci\'on]{\begin{trivlist}
		\item[\hskip \labelsep {\bfseries #1}]}{\end{trivlist}}


\newcommand{\qed}{\nobreak \ifvmode \relax \else
	\ifdim\lastskip<1.5em \hskip-\lastskip
	\hskip1.5em plus0em minus0.5em \fi \nobreak
	\vrule height0.75em width0.5em depth0.25em\fi}

\newcommand{\twopartdef}[4]
{
	\left\{
	\begin{array}{ll}
		#1 & \mbox{ } #2 \\
		#3 & \mbox{ } #4
	\end{array}
	\right.
}

\newcommand{\threepartdef}[6]
{
	\left\{
	\begin{array}{lll}
		#1 & \mbox{ } #2 \\
		#3 & \mbox{ } #4 \\
		#5 & \mbox{ } #6
	\end{array}
	\right.
}

\tikzset{commutative diagrams/.cd,
	mysymbol/.style={start anchor=center,end anchor=center,draw=none}
}
\newcommand\Center[2]{%
	\arrow[mysymbol]{#2}[description]{#1}}

\newcommand*\circled[1]{\tikz[baseline=(char.base)]{
		\node[shape=circle,draw,inner sep=2pt] (char) {#1};}}


\begin{document}
	
	\pagestyle{empty}
	\pagestyle{fancy}
	\fancyfoot[CO]{\slshape \thepage}
	\renewcommand{\headrulewidth}{0pt}
	
	
	
	\centerline{\bf Topolog\'ia }
	\centerline{\sc Final}
	
	\bigskip
	
	
\part{Espacios Topol\'ogicos}

\section{Ordenes}

\begin{definition}{}
	Una relaci\'on $\mathcal{R}$ en un conjunto $A$ se dice un \textit{orden(parcial)} si satisface:
	
\begin{itemize}
	\item \underline{Reflexividad}: $a \mathcal{R} a \quad \forall a \in A $
	\item \underline{Transitividad}: $a \mathcal{R} b \ \wedge b \mathcal{R} c \Longrightarrow a \mathcal{R} c$
	\item \underline{Antisimetr\'ia}: $a \mathcal{R} b \ \wedge b \mathcal{R} a \Longrightarrow a = b$
\end{itemize}	
	
\end{definition}

En este caso decimos que $(A,\mathcal{R})$ es un conjunto parcialmente ordenado, o \textit{poset}.

\begin{remark}
	En el caso que una relaci\'on en un conjunto $A$ cumpla la antisimetr\'ia y la \textbf{no} reflexividad, decimos que $(A,\mathcal{R})$ es un \textit{orden parcial estricto}. 
\end{remark}

\begin{proposition}
	Dado un conjunto $A \neq \emptyset$ existe una biyecci\'on entre los ordenes de $A$ y los ordenes parciales estrictos de $A$
\end{proposition}

\begin{proof}
	Si $\mathcal{R}$ es un orden en $A$, definamos $\mathcal{R'}$ dado por:
	
\[
a \mathcal{R'} b \Longleftrightarrow a \mathcal{R} b \ \wedge a \neq b 
\]

Entonces notemos que $\mathcal{R'}$ es un orden parcial estricto.

Rec\'iprocamente, si $\mathcal{R'}$ es un orden parcial estricto, definimos $\mathcal{R}$ dado por:

\[
a \mathcal{R} b \Longleftrightarrow a \mathcal{R'} b \ \vee a \neq b 
\]

Entonces es claro que $\mathcal{R}$ es un orden. \qed

\end{proof}

\begin{example}
	$\N$ con la relaci\'on $a \mathcal{R} b$ si $a \vert b$ es un poset.	
\end{example}

\textbf{Nota}: De ahora en m\'as vamos a notar $\leq$ a los \'ordenes parciales y \textless  a los \'ordenes parciales estrictos

\begin{definition}
	Un orden parcial en un conjunto se dice un \underline{orden total} si $\forall a \in A, \ a \leq b \ \vee b \leq a$, y similarmente con \'ordenes parciales estrictos.
\end{definition}

\begin{example}{De \'ordenes:}
	
	\begin{itemize}
		\item $(\R, \leq)$ es un orden total
		\item $(\N, \vert)$ no lo es, pues $2 \nmid 3 $ ni $3 \nmid 2$.
	\end{itemize}
\end{example}

\begin{definition} {Sea $A$ un conjunto ordenado, un elemento $a \in A$ se dice:}
	
	 \begin{itemize}
	 	\item \textit{Maximal} si $b \geq a$ implica que $a = b$
	 	\item \textit{Minimal} si $b \leq a$ implica $b = a$.
		\item \textit{M\'aximo} si $b \leq a \ \forall b \in A$
		\item \textit{M\'inimo} si $b \geq a \ \forall b \in A$.
		\item Si $S \subset A, S\neq \emptyset$, $a$ se dice \textit{cota superior de S} si $s \leq a \ \forall s \in S$
		\item Si $S \subset A, S\neq \emptyset$, $a$ se dice \textit{cota inferior de S} si $s \geq a \ \forall s \in S$
	\end{itemize}
\end{definition}

\begin{example} {De m\'inimos y minimales}
	\begin{itemize}
		\item $(\N, \mid)$ tiene m\'inimo, el $1 \in \N$
		\item $(\N \setminus \sett{1} , \mid)$ no tiene m\'inimo, pero los minimales son los primos.
	\end{itemize}
\end{example}

\begin{definition}
	Sean $(A, \leq) \ y \ (B, \preceq)$ \'ordenes parciales. Un \textit{morfismo de orden} es una funci\'on $f: A \rightarrow B$ tal que $a \leq a' \ \Longrightarrow f(a) \preceq f(a')$. Similarmente un morfismo de \'ordenes estrictos.
\end{definition}

En cualquier caso decimos que $f$ preserva el orden.

\begin{definition}
	Sean $A,B$ ordenes totales estrictos, un morfismo de \'ordenes estrictos $f A \rightarrow B$ se dice un \textit{isomorfismo} si existe $g: B \rightarrow A$ tal que $gf = 1_A$ y $fg = 1_B$. En caso de existir tal morfismo, decimos que $A$ y $B$ tienen el mismo tipo de orden.
\end{definition}

\begin{proposition}
	
	\label{Iso de ordenes es biyectiva}

	Un morfismo de \'ordenes estrictos $f : A \rightarrow B$ es un isomorfismo sii $f$ es biyectiva
\end{proposition}

\begin{proof}
	
	Supongamos que $f$ es biyectiva y llamemos $g$ a su inversa, queremos ver que $g$ es morfismo de orden y preserva el orden.
	
	Sea $b \prec b'$, como $g$ es inyectiva entonces $g(b) \neq g(b')$. Como $A$ es un orden total estricto, $g(b) < g(b')$ o $g(b) > g(b')$, supongamos \'este \'ultimo. Entonces como $f$ es morfismo de \'ordenes estrictos, $b=f(g(b)) > f(g(b'))=b'$, por lo que $g$ es morfismo de \'ordenes estrictos. \qed
	
\end{proof}

\begin{definition}
	Sean $A,B$ \'ordenes totales estrictos, el \textit{orden lexicogr\'afico} en el producto $A \times B$ viene dado por $(a,b) < (a',b')$ si $a < a'$ o $a=a' \ , \ b < b'$. Es claro que el orden lexicogr\'afico es un orden total estricto en $A \times B$

\end{definition}

\begin{example}
	Consideremos a $\N$ con el orden usual y a $\N \times \N$ con el orden lexicogr\'afico. Veamos que no tienen el mismo tipo de orden.
	
	Si $f : \N \times \N \rightarrow \N$ fuese un isomorfismo de orden, entonces si llamamos $A := \left\lbrace (a,b) \in \N \times \N \ / \ (a,b) < (2,1) \right\rbrace $, entonces $f \vert_{A} : A \rightarrow \sett{n \ / \ n < f(2,1)}$ ser\'ia un isomorfismo de orden, y por la proposici\'on \ref{Iso de ordenes es biyectiva} tenemos que es biyectiva.
	
	Pero $\sett{(a,n) \ , \ n \in \N} \subset A$ por lo que $A$ es infinito, y $\sett{n \ / \ n < f(2,1)}$ es finito por arquimedianidad. Por lo tanto $f$ no era isomorfismo de orden.
	
\end{example}

\begin{definition}
	Sea $A$ un orden total estricto. Una \textit{secci\'on} de $A$ es un conjunto $S \subset A$ tal que $a \in S \ , \ b < a \ \Longrightarrow b \in S$. Si $a \in A$, la secci\'on por $a$ es $S_a = \sett{b \in A \ / \ b<a} $, notemos que $S_a$ es secci\'on de $A$.
\end{definition}

\begin{example}
	$S = \sett{x \in \Q \ / \ x^2 < 2}$ es una secci\'on de $\Q$ pero $S \neq S_q \  \forall q \in \Q$
\end{example}

\begin{definition}
	Un orden total estricto $A$ se dice un \textit{buen orden} si todo subconjunto $S \subset A$ tiene un m\'inimo.
\end{definition}


\begin{proposition}
	
	\label{En buen orden toda seccion es por un elemento}
	
	Sea $A$ y $S \subsetneq A$ una secci\'on propia, entonces existe $a \in A $ tal que $S_a = S$
\end{proposition}

\begin{proof}
	Sea $B = A \setminus S$ entonces $B \neq \emptyset$, sea $b_0 = min(B)$ que existe pues $A$ es bien ordenado; veamos que $S = S_{b_0}$.
	
	En efecto, si $b \in S_{b_0}$ entonces $b < b_0$, por lo tanto $b \not\in B$. Por ende, $b \in S$.
	
	Por el otro lado, si $s \in S$ y $s \geq b_0$, entonces como $S$ es secci\'on tenemos que $b_0 \in S$ por lo que $b_0 \not\in S_{b_0}$. Entonces $s < b_0$. \qed
	
\end{proof}

\begin{proposition}
	
	\label{Producto de ordenados es bien ordenado}
	
	Si $A,B$ estan bien ordenados, entonces $A \times B$ con el orden lexicogr\'afico tambi\'en.
\end{proposition}

\begin{proof}
	Sea $S \neq \emptyset \subset A \times B$ y consideremos $S_a = \sett{a \in A \ / \ \exists b \in B \ , \ (a,b) \in S} \subset A$. Como $S \neq \emptyset$, existe $(a',b') \in S$, por lo que $a' \in S_a \neq \emptyset$. Sea $a_0 = min(S_a)$ y consideremos $S_b = \sett{b \in B \ / \ (a_0,b) \in S} \neq \emptyset$ por lo visto antes. Llamemos $b_0 = min (S_b)$ y veamos que $(a_0 , b_0) = min(S)$:
	
	\begin{enumerate}
		\item $(a_0,b_0) \in S$
		\item Sean $(c,d) \in S$ tal que $(c,d) < (a_0 ,b_0)$; si $c < a_0$ como $d \in B$ y $(c,d) \in S$ tenemos que $c \in S_a$ y es menor al m\'inimo, por lo que $c = a_0$. Supongamos entonces $d < b_0$, notemos que $d \in S_b$ pues $S \ni (c,d) = (a_0,d)$ pero $b_0$ era el m\'inimo de $S_b$. Por lo tanto $(c,d) \geq (a_0,b_0)$
	\end{enumerate}
	
	Conclu\'imos que $S$ tiene m\'inimo y entonces $A \times B$ est\'a bien ordenado. \qed
	
\end{proof}

\begin{proposition}
	
	\label{Subconjunto de bien ordenado es bien ordenado}
	
	Si $A$ esta bien ordenado y $B \subset A$, entonces $B$ esta bien ordenado
\end{proposition}

\begin{theorem}
	
	\label{Teorema del buen orden}
	
	Para todo conjunto $A$, existe una relaci\'on \textless que es un orden total estricto y tal que $(A,<)$ es un buen orden.
\end{theorem}

\begin{corollary}
	
	\label{Construccion seccion omega}
	
	Existe un conjunto bien ordenado $S_{\Omega}$ que es no numerable pero tal que toda secci\'on propia es numerable.
\end{corollary}

\begin{proof}
	Sea $A$ no numerable, por el teorema \ref{Teorema del buen orden} existe un orden \textless tal que es bien ordenado.
	
	Consideremos el orden lexicogr\'afico en $\sett{1,2} \times A$ donde $1<2$. Por \ref{Producto de ordenados es bien ordenado} sabemos que es bien ordenado. Sea $B = \sett{(n,a) \in \sett{1,2} \times A \ / \ S_{(n,a)} \emph{es no numerable}}$. Notemos que $(2, min(A)) \in B$ por lo que $B \neq \emptyset$ y sea $\Omega = min(B)$, consideremos $S_{\Omega}$.
	
	Por \ref{Subconjunto de bien ordenado es bien ordenado} sabemos que $S_{\Omega}$ es bien ordenado y como $\Omega \in B$ adem\'as es no numerable. Si $S \subsetneq S_{\Omega}$ es secci\'on propia, entonces por \ref{En buen orden toda seccion es por un elemento} sabemos que existe $\alpha \in B$ tal que $S = S_{\alpha}$ y $\alpha < \Omega$. Como $\Omega = min(B)$ sabemos que $\alpha \not\in B$, por lo que $S$ es numerable \qed
		
	
\end{proof}

\begin{corollary}
	\label{Si A es numerable en SOmega entonces tiene cota superior}
	
	Si $A \subset S_{\Omega}$ es numerable, entonces $A$ tiene cota superior
\end{corollary}

\begin{proof}
	Si $A$ es numerable, entonces $B = \bigcup_{a \in A}{S_a}$ es numerable por \ref{Construccion seccion omega}. Adem\'as $B \subset S_{\Omega}$ y $S_{\Omega}$ es no numerable por lo que existe $b \in S_{\Omega} \setminus B$, afirmo que $b$ es cota superior de A.
	
	En efecto, si existe $a \in A$ tal que $b < a$, entonces $b \in S_a$ y entonces $b \in B$. \qed
\end{proof}

\begin{definition}
	Sea $A$ un poset, un subconjunto $C \subset A$ se dice \textit{cadena} si es un orden total con el orden inducido
\end{definition}

\begin{theorem}{Lema de Zorn}
	
	\label{Lema de Zorn}
	
	Sea $A \neq \emptyset$ un poset, si toda cadena en $A$ tiene una cota superior en $A$, entonces $A$ tiene al menos un elemento maximal
\end{theorem}

\begin{theorem}{Axioma de elecci\'on}
	
	\label{Axioma de eleccion}
	
	Si $A$ es un conjunto entonces existe $f: \mathcal{P}(A) \setminus \emptyset \rightarrow A$ tal que $f(B) \in B \ \forall B \in \mathcal{P}(A) \setminus \emptyset$
\end{theorem}

\pagebreak

\section{Espacios topol\'ogicos: Introducci\'on}

\subsection{Interior, clausura y frontera de un conjunto}

\begin{definition}
	Sea $X$ un conjunto, una \textit{topolog\'ia} en $X$ es una colucci\'on $\tau$ de subconjuntos de $X$ que satisface:
	
	\begin{enumerate}
		
		\item $\emptyset,X \in \tau$
		\item Si $\sett{U_i}_{i \in I}$ es una colecci\'on de elementos de $\tau$, entonces $\bigcup_{i \in I}{U_i} \in \tau$
		\item Si $U,V \in \tau$ entonces $U\cap V \in \tau$
		
	\end{enumerate}
\end{definition}

El par $(X,\tau)$ se llama \textit{espacio topol\'ogico} y si queda claro la topolog\'ia diremos simplemente $X$ un espacio topol\'ogico. Un conjunto $U \subset X$ se dice abierto si $U \in \tau$ y se dice cerrado si $U^c \in \tau$.

\begin{example}{Topolog\'ias discreta e indiscreta}
	\begin{itemize}
		\item $X=\emph{cualquiera}$ y $\tau_d = \mathcal{P}(X)$
	
	A esta topolog\'ia la llamaremos \textit{topolog\'ia discreta}.
	
		\item $X=\emph{cualquiera}$ y $\tau_i = \sett{\emptyset, X}$
	A esta topolog\'ia la llamaremos \textit{topolog\'ia indiscreta}
	\end{itemize}
\end{example}

\begin{remark}
	Sea $A = \sett{\tau \in \mathcal{P}(X) \ / \ \tau\emph{ es topolog\'ia}}$ y lo ordenamos por la inclusi\'on, entonces $\tau_d = max(A)$ y $\tau_i = min(A)$
\end{remark}

\begin{example}{Topolog\'ia m\'etrica}
	
	Sea $(X,\tilde{d})$ un espacio m\'etrico y consideremos:
	
	\[
	\tau = \tau_{\tilde{d}} = \sett{U \subset X \ / \ \forall x \in U, \exists r>0 \ / \ B_r(x) \subset U}
	\]
	
	A esta topolog\'ia la llamaremos \textit{ topolog\'ia m\'etrica}, concluimos que todo espacio m\'etrico es un espacio topol\'ogico (pero si tomamos $X$ finito y $\tau = \tau_d$ vemos que no todo espacio topol\'ogico es m\'etrico).
	
\end{example}

\begin{example}{Espacio de Sierpinsky}
	
	Sea $X = \sett{0,1}$ y $\tau = \sett{\emptyset,X,\sett{0}}$, notamos a este espacio $\mathfrak{S} = (X,\tau)$ el \textit{espacio de Sierpinsky}
	
\end{example}

\begin{example}{Topolog\'ia del complemento finito}
	Sea $X$ un conjunto y $\tau = \sett{\emptyset} \cup \sett{U \subset X \ / \ U^c \emph{es finito}}$, veamos que es una topolog\'ia:
	
	\begin{itemize}
		\item $X^c = \emptyset \in \tau$ y $\emptyset \in \tau$.
		\item Si $U_i \in \tau$ entonces $\left(\bigcup_{i}{U_i} \right)^c = \bigcap_{i}{U_{i}^{c}} \subset U_{i_0}^c$ y $U_{i_0}^c$ es finito pues $U_{i_0} \in \tau$. Entonces $\bigcup_{i}{U_i} \in \tau $
		\item Si $U,V \in \tau$ entonces $\left(U \cap V \right)^c = U^c \cup V^c$ es finito, por lo que $U \cap V \in \tau$
	\end{itemize}
	
	A esta topolog\'ia la llamaremos \textit{topolog\'ia del complemento finito}
	
\end{example}

\begin{definition}{Topolog\'ias mas finas}
	Dadas $\tau_1,\tau_2$ topolog\'ias en un conjunto arbitrario $X$, decimos que $\tau_1$ \textit{es m\'as fina que } $\tau_2$ si $\tau_2 \subset \tau_1$
\end{definition}

\begin{definition}{Interior, clausura y frontera de un conjunto}
	
	Sea $X$ un espacio topol\'ogico y $A \subset X$. 
	
	\medskip
	
	El \textit{interior} de $A$ es $\interior{A} = \bigcup\limits_{\substack{U \subset A \\ U \emph{ abierto}}}{U}$
	
	\medskip
	
	La \textit{clausura} de $A$ es $\overline{A} = \bigcap\limits_{\substack{A \subset F \\ F \emph{ cerrado}}}{F}$
	
	\medskip
	
	La \textit{frontera} de $A$ es $\partial A = \overline{A} \setminus \interior{A}$ 
	
\end{definition}

\begin{remark}
	$\interior{A}$ es el abierto m\'as grande contenido en $A$ y $\overline{A}$ es el cerrado m\'as chico que contiene a $A$
\end{remark}

\begin{example}
	Sea $\mathfrak{S}$ el espacio de Sierpinsky, entonces:
	
	\begin{itemize}
		
		\item $\overline{\sett{0}} = \sett{0,1}$
		\item $\overline{\sett{1}} = \sett{1}$
		\item $\interior{\sett{0}} = \sett{0}$
		\item $\interior{\sett{1}} = \emptyset$
		
	\end{itemize}
	
\end{example}

\begin{definition}
	 
	Sea $X$ un espacio topol\'ogico y sea $x \in X$. Un \textit{entorno} de $x$ es un subconjunto $A \subset X$ tal que existe un abierto $U$ con $x \in U \subset A$. Un \textit{entorno abierto} es un entorno que es abierto.

\end{definition}	

\begin{proposition}
	
	\label{Definicion clausura por entornos abiertos}
	
	Sean $X$ un espacio topol\'ogico, $A \subset X$ y  $x \in X$. Luego $x \in \overline{A}$ si y s\'olo si todo entorno $U$ de $x$ cumple $U \cap A \neq \emptyset$
	
\end{proposition}

\begin{proof}

	Por el contrarec\'iproco, supongamos que existe $U$ entorno de $x$ tal que $U \cap A = \emptyset$, entonces existe $V \ni x$ abierto tal que $V \subset U$ y $V \cap A = \emptyset$. Si consideramos $F = V^c$ tenemos que $F$ es cerrado y por lo anterior $A \subset F$, como $x \not\in F$ entonces $x \not \in \overline{A}$.
	
	Por el otro lado, supongamos que $x \not\in \overline{A}$, entonces $x \in V := \left( \overline{A} \right) ^c$ que es abierto  y por ende entorno. Sin embargo, $A \cap V = \emptyset$, por lo que $V$ es un entorno abierto de $x$ que no interseca con $A$ \qed
	  

\end{proof}

\subsection{Bases y subbases}

\begin{definition}
	Sea $X$ un conjunto, una \textit{base para una topolog\'ia en } $X$ es una colecci\'on $\B$ de subconjuntos de $X$ que satisface:
	
	\begin{enumerate}
		\item[i)] $\bigcup\limits_{U \in \B}{U} = X$
		\item[ii)] Si $U,V \in \B$ y $x \in U\cap V$ entonces existe $W \in B$ tal que $x \in W \subset U \cap V$
	\end{enumerate}
\end{definition}

\begin{example}
	Si $(X,d)$ es un espacio m\'etrico, la colecci\'on $\B = \sett{B_r(x)}_{\substack{x \in X \\ r >0}}$ de bolas abiertas es una base para una topolog\'ia.
\end{example}

\begin{proposition}
	
\label{Topologia dada por base}	

	Si $\B$ es una base para una topolog\'ia de un conjunto $X$ entonces $\tau = \sett{U \subset X \ / \ \forall x \in U \ \exists V \in \B \ , \ x \in V \subset U}$ es una topolog\'ia.
	
\end{proposition}

\begin{proof}
	\begin{enumerate}
		\item[i)] $\emptyset \in \tau$ trivialmente
		\item[ii)] Como $X = \bigcup\limits_{U \in \B}{U}$ entonces dado $x \in X$ existe $U \in \B$ tal que $x \in U \subset X$. Por lo tanto $X \in \tau$.
		\item[iii)] Si $U_i \in \tau$ sea $x \in \bigcup\limits_{i \in I}{U_i}$. Luego existe $i_0 \in I$ tal que $x \in U_{i_0} \in \tau$, por lo que existe $V \in \B$ tal que $x \in V \subset U_{i_0} \subset \bigcup\limits_{i \in I}{U_i}$. Por lo tanto $\bigcup\limits_{i \in I}{U_i} \in \tau$
		\item[iv)] Si $U,V \in \tau$ entonces dado $x \in U \cap V$ existen $U',V' \in \B$ tal que $x \in U' \subset V'$ y $x \in V' \subset V$. Como $\B$ es base y $x \in U' \cap V'$ existe $W \in \B$ tal que $x \in W \subset U' \cap V' \subset U \cap V$. Luego $U \cap V \in \tau$
	\end{enumerate}
	
	Por lo tanto $\tau$ es base, la llamaremos \textit{topolog\'ia generada por } $\B$ 
	\qed
	
\end{proof}

\begin{proposition}
	
	\label{Caracterizacion de la topologia generada por una base}
	
	En la notaci\'on anterior se tiene que $U \in \tau \ \Longleftrightarrow  \  U = \bigcup\limits_{B \in \B}{B}$
\end{proposition}

\begin{proof}
	
	Si $U \in \tau$ entonces por \ref{Topologia dada por base} $\forall x \in U$ existe $V \in \B$ tal que $x \in V_x \subset U$, por lo tanto $U = \bigcup\limits_{x \in U}{V_x}$.
	
	Rec\'iprocamente si $U = \bigcup\limits_{B \in \B}{B}$ entonces dado $x \in U$ existe $B' \in \B$ tal que $x \in B' \subset \bigcup\limits_{B \in \B}{B} = U$, por lo que $U \in \tau$ \qed
	
\end{proof}

\begin{remark}
	Si $\B$ es una base, entonces la topolog\'ia generada por $\B$ es la m\'as chica que contiene a $\B$.
\end{remark}

\begin{proposition}
	
	\label{Criterio de finura de topologia por bases}
	
	Si $X$ es un conjunto y $\tau_1 , \tau_2$ son dos topolog\'ias en $X$ con bases $\B_1,\B_2$ respectivamente, entonces son equivalentes:
	
	\begin{enumerate}
		\item $\tau_1$ es m\'as fina que $\tau_2$
		\item $\forall U \in \B_2$ y $x \in U$ existe $V \in \B_1$ tal que $x \in V \subset U$
	\end{enumerate}
	
\end{proposition}

\begin{proof}
	
	Supongamos que $\tau_1$ es m\'as fina que $\tau_2$, entonces dado $U \in \B_2$ tenemos que $U \in \tau_2 \subset \tau_1$, luego por \ref{Topologia dada por base} dado $x \in U$ existe $V \in \B_1$ tal que $x \in V \subset U$.
	
	Por el otro lado, antes vimos justamente que \textit{ii)} es equivalente a que $\B_2 \in \tau_1$y como $\tau_2$ es la topolog\'ia m\'as chica que contiene a $\B_2$ tenemos que $\tau_2 \subset \tau_1 $ \qed
	
	
\end{proof}

\begin{definition}{Topolog\'ia del l'imite inferior}
	
	Sea $\B = \sett{[a,b) \ , \ a < b]}$ la colecci\'on de intervalos semiabiertos y notemos que es una base. La topolog\'ia generada por esta base se llama \textit{topolog\'ia del l\'imite inferior} y el espacio $\R$ con esta topolog\'ia se notar\'a $(\R,\tau_L) := \R_L$
	
\end{definition}

\begin{proposition}
	Si notamos a $\tau$ como la topolog\'ia usual de $\R$, entonces $\tau \subsetneq \tau_L$
\end{proposition}

\begin{proof}
	Si notamos $\B, \B_L$ a las bases de las topolog\'ias $\tau , \tau_L$ respectivamente, por \ref{Criterio de finura de topologia por bases} basta ver que dado $x \in I \in \tau$ existe $V \in \tau_L$ tal que $x \in V \subset I$ y que no se puede la rec\'iproca.
	
	En efecto, sea $x \in (a,b) \in \tau$ entonces tenemos que $x \in [x,b) \subset (a,b)$ por lo que $\tau \subset \tau_L$.
	
	Por el otro lado, si tomamos $0 \in [0,1) \in \B_L$ entonces si $0 \in I \in  \B$ entonces dado $0 < \epsilon < \vert I \vert$ tenemos que $-\epsilon \in I$ y $-\epsilon \not\in [0,1)$. Por lo tanto $\tau \subseteq \tau_L$ \qed 
	
\end{proof}

\begin{definition}{Topolog\'ia del orden}
	
	Sea $A$ un orden total estricto, sea $a_0 = min(A)$ (en caso de existir) y $b_0 = max(A)$ (en caso de existir). Definimos $\B$ como:
	
	\[
		\B =  \left\lbrace
			\begin{array}{cc}
				\sett{(a,b) \ , \ a < b \ , a,b \in A}	& \emph{si no existen } a_0,b_0 	\\ 
				\sett{(a,b) \ , \ a < b \ , a,b \in A} \cup \sett{[a_0,b) \ , b \in A}	&  \emph{si no existe } b_0 \\ 
				\sett{(a,b) \ , \ a < b \ , a,b \in A} \cup \sett{(a,b_0] \ , a \in A}	&  \emph{si no existe } a_0 \\ 
				\sett{(a,b) \ , \ a < b \ , a,b \in A} \cup \sett{(a,b_0] \ , a \in A} \cup \sett{[a_0,b) \ , b \in A}	& \emph{si existen ambos} 
			\end{array} 
		\right.
	\]
	
	Entonces es f\'acil ver que es una base y la  topolog\'ia que genera la llamaremos la \textit{topolog\'ia del orden} y notaremos al espacio topol\'ogico como $(A,\tau_0) := A_O$

\end{definition}

\begin{remark}
	Dado que en $(\R, <)$ es un conjunto con un orden total estricto, entonces aplicando lo anterior obtenemos un espacio topol\'ogico $(\R, \tau_O)$ que al no haber ni m\'aximo ni m\'inimo en $\R$ resulta $(\R, \tau)$ la topolog\'ia usual.
\end{remark}

\begin{example}{La topolog\'ia del orden en $\N$ es la topolog\'ia discreta}
	Consideremos $\N$ con $\tau_d , \tau_O$ las topolog\'ias discreta y del orden respectivamente. Afirmo que ambas son equivalentes:
	
	En efecto, sea $U \in \B_d$ un abierto para la base de la topolog\'ia discreta y $n \in U$. Como $\B_d = \sett{\sett{n} \ , \ n \in \N}$ entonces $U = \sett{n_0}$ para alg\'un $n_0 \in \N$. Por otro lado, $\B_O \ni (n_0 - 1 , n_0 + 1) = \sett{n \in \N \ / \ n_0 -1 < n < n_0 + 1} = \sett{n}$ por lo que $n \in (n_0 - 1 , n_0 + 1) \subseteq U $. Por \ref{Criterio de finura de topologia por bases} tenemos que $\tau_d \subseteq \tau_O$. 
	
	Rec\'iprocamente como la topolog\'ia discreta es el m\'aximo rrespecto a la inclusi\'on de las topolog\'ias	de $\N$ obtenemos que $\tau_O \subseteq \tau_d $. Concluimos que $\tau_d = \tau_O$. \qed
\end{example}

\begin{definition}{Subbase de un espacio topol\'ogico}
	
	Dado un conjunto $X$, una colecci\'on $\B \subseteq \mathcal{P}(X)$ se dice \textit{subbase} si $\bigcup\limits_{U \in \B}{U} = X$
	
\end{definition}

\begin{proposition}
	
	\label{Interseccion de topologias es topologia}
	
	Si $\sett{\tau_i}_{i \in I}$ es una familia de topolog\'ias para un conjunto $X$, entonces $\tau = \bigcap\limits_{i \in I}{\tau_i}$ es una topolog\'ia para $X$.
\end{proposition}

\begin{proof}
	
	Verifiquemos los axiomas que tiene que cumplir una topolog\'ia:
	
	\begin{itemize}
		
		\item[i)] Como $\tau_i$ son topolog\'ias, $\emptyset \in \tau_i \ \forall i \in I$ entonces $\emptyset \in \tau$.
		\item[ii)] Similarmente $X \in \tau$
		\item[iii)] Sean $U,V \in \tau$ entonces $U,V \in \tau_i \ \forall i \in I$; como son todas topolog\'ias $U \cap V \in \tau_i \ \forall i \in I$ y entonces $U \cap V \in \tau$.
		\item[iv)] Sean $\sett{U_i}_{i \in I} \subset \tau$ una familia de abiertos de $\tau$. Entonces $\sett{U_i}_{i \in I} \subset \tau_i$ para todo $i$, por lo que $\bigcup\limits_{i \in I}{U_i} \in \tau_i$ para todo $i$. Por ende, $\bigcup\limits_{i \in I}{U_i} \in \tau$. 
		
	\end{itemize}
	
	Por lo tanto $\bigcap\limits_{i \in I}{\tau_i}$ es una topolog\'ia. \qed
	
\end{proof}

\begin{corollary}
	
	Dado una colecci\'on $\B \subset \mathcal{P}(X)$, el conjunto de todas las topolog\'ias de $X$ que contienen a $B$ tiene un m\'inimo (en el sentido de la inclusi\'on). Se llama la \textit{topolog\'ia generada por $\B$} y se la denota $\sigma(B)$.
	
\end{corollary}

\begin{proof}
	Es claro que si $\sett{\tau_i}_{i \in I}$ es dicho conjunto de topolog\'ias, entonces por \ref{Interseccion de topologias es topologia} tenemos que $\tau = \bigcap\limits_{i \in I}{\tau_i}$ es la topolog\'ia buscada \qed
\end{proof}

\begin{proposition}
	
	\label{Construccion de topologia desde subbase}
	
	Sea $X$ un conjunto y $\B$ una subbase de una topolog\'ia en $X$, entonces 
	
	
	\begin{equation*}
		\sigma(B) = \left\lbrace \bigcup\limits_{i \in I}{\bigcap\limits_{\substack{j \in J \\ J \emph{finito}}} {U_{i,j}}} \ / \ U_{i,j} \in \B \right\rbrace
	\end{equation*}
	
\end{proposition}

\begin{proof}
	
	Consideremos $\B' = \left\lbrace \bigcap\limits_{\substack{j \in J \\ J \emph{finito}}} {U_{j}} \ / \ U_i \in \B \right\rbrace $ y veamos que $\B'$ es una base:
	
	\begin{itemize}
		
		\item[i)] Notemos que tomando $J = {j_0}$ tenemos que $\B \subset \B'$, por ende $X = \bigcup\limits_{U \in \B}{U} \subseteq \bigcup\limits_{U \in \B'}{U} \subseteq X$ por lo que $\bigcup\limits_{U \in \B'}{U} = X$.
		\item[ii)] Sean $U,V \in \B'$ y $x \in U \cap V$, entonces por definici\'on $U \cap V \in \B'$ por lo que $x \in U \cap V \subseteq U \cap V$.
		
	\end{itemize}
	
	Finalmente veamos que $\sigma(B) = \sigma(B')$ lo que probar\'ia la proposici\'on.
	
	\begin{itemize}
		
		\item Por un lado $\B \subseteq \B'$ por lo que $\sigma(B) \subseteq \sigma(B')$
		
		\item Por el otro, $\B \subseteq \sigma(B)$ y entonces (tomando primar como el conjunto de intersecciones finitas del conjunto) $\B' \subseteq \sigma(B)' = \sigma(B)$. Por lo tanto por la definici\'on de $\sigma(B')$, tenemos que $\sigma(B') \subset \sigma(B)$. \qed.
		
	\end{itemize}
	
\end{proof}

\begin{definition}{Punto de acumulaci\'on}
	
	Sea $X$ un espacio topol\'ogico y $A \subseteq X$ un subconjunto. Un punto $x \in X$ se dice \textit{punto de acumulaci\'on} de $A$ si para todo entorno abierto $U \ni x$ se tiene que $( U \cap A ) \setminus \sett{x} \neq \emptyset$ o  equivalentemente si $x \in \overline{A \setminus \sett{x}}$. Al conjunto de puntos de acumulaci\'on de $A$ lo notaremos $A\prime$
	
\end{definition}

\begin{proposition}
	
	Sea $X$ un espacio topol\'ogico y $A \subseteq X$, entonces $\overline{A} = A\prime \cup A$
	
\end{proposition}

\begin{proof}
	
	Por un lado si $x \in A\prime$ entonces $x \in \overline{A \setminus \sett{x}} \subseteq \overline{A}$; adem\'as $A \subseteq \overline{A}$ por lo que $A \cup A\prime \subseteq \overline{A}$.
	
	Rec\'iprocamente si $x \not \in A\prime \cup A$ entonces $x \not \in A$ y existe $U \ni x$ entorno abierto tal que $\emptyset = ( U \cap A ) \setminus \sett{x} = \left( U \setminus \sett{x} \right) \cap \left( A \setminus \sett{x} \right) = \left( U \setminus \sett{x} \right) \cap A$, entonces $ U \cap A = \left( U \setminus \sett{x} \right) \cap A ) \cup ( \sett{x} \cap A ) = \emptyset$. Por lo tanto $U \ni x$ es un entorno abierto de $x$ tal que $U \cap A = \emptyset$, con lo que $x \not \in \overline{A}$ \qed
	
\end{proof}

\subsection{Redes y sucesiones en espacios topol\'ogicos}

\begin{definition}{Sucesi\'on convergente en $X$}
	
	
	Si $X$ es un espacio topol\'ogico y $(x_n)_{n \in \N}$ es una sucesi\'on en $X$ diremos que $(x_n)_{n \in \N}$ converge a un punto $x \in X$ si para todo entorno $U$ de $x$ existe $n_0 \in \N$ tal que si $n \geq n_0$ implica $x_n \in U$.
	
\end{definition}

\begin{example}
	Sea $X = \mathfrak{S}$ y $(x_n)_{n \in \N} \in \mathfrak{S}^{\N}$ tal que $x_n = 0$ para todo $n \in \N$, afirmo que $x_n \rightarrow 0$ y $x_n \rightarrow 1$.
	
	En efecto, sea $U$ un entorno de $0$, entonces tomando $V = \sett{0}$ es un abierto que cumple $x_n \in V \subseteq U$; por lo tanto $x_n \rightarrow 0$.
	
	Por el otro lado, sea $V$ un entorno de $1$, pero entonces $V = X$ es el \'unico entorno abierto de $1$ que cumple $x_n \in X = V$; por lo tanto $x_n \rightarrow 1$.
	
\end{example}

\begin{remark}{Las sucesiones no definen la topolog\'ia en general}
	
	Sea $X$ un espacio topol\'ogico y $A \subseteq X$, si $x \in X$ es tal que existe $(x_n)_{n \in \N} \in A^{\N}$ con $x_n \rightarrow x$ entonces $x \in \overline{A}$. No obstante, si $x \in \overline{A}$ entonces puede no existir $(x_n)_{n \in \N} \in A^{\N}$ tal que $x_n \rightarrow x$
	
\end{remark}

\begin{proof}{De la observaci\'on}
	
	Por un lado, sea $U \ni x$ un entorno abierto de $x$, entonces como $x_n \rightarrow x$ por definici\'on existe $n_0 \in \N$ tal que $x_n \in U$ para todo $n \geq n_0$. En particular $x_{n_0 +1} \in U \cap A \neq \emptyset$, por lo que $x \in \overline{A}$.
	
	Para ver que no vale la vuelta, sea $S_{\Omega}$ el conjunto dado por \ref{Construccion seccion omega} y sea $\widetilde{S_{\Omega}} = S_{\Omega} \cup \sett{\Omega}$ donde $\Omega > a$ para todo $a \in S_{\Omega}$. Finalmente consideremos $(\widetilde{S_{\Omega}}, \tau_O)$ la topolog\'ia del orden en dicho conjunto. Afirmo que $\Omega \in \overline{S_{\Omega}}$ pero no existe $(x_n)_{n \in \N}$ sucesi\'on en $S_{\Omega}$ tal que $x_n \rightarrow \Omega$.
	
	En efecto, si $U \ni \Omega$ es un entorno abierto de $\Omega$ entonces existe $a \in S_{\Omega}$ tal que $\Omega \in (a, \Omega] \subseteq U$ por definici\'on de la topolog\'ia del orden. Ahora por la construcci\'on de $S_{\Omega}$, $S_a \cup \sett{a}$ es numerable y entonces existe $b \in S_{\Omega} \setminus S_a \cup \sett{a}$ por lo que $b \in (a , \Omega] \cap S_{\Omega}$ lo que implica que $U \cap S_{\Omega} \neq \emptyset$. Esto prueba que $\Omega \in \overline{S_{\Omega}}$.
	
	Sin embargo, si $(x_n)_{n \in \N} \in {S_{\Omega}}^{\N}$ entonces $\sett{x_n \ , \ n \in \N}$ es numerable y por lo tanto (por construcci\'on de $S_{\Omega}$) existe $b \in S_{\Omega}$ tal que $b > x_n \ \forall n \in \N$. Si consideramos $U= (b , \Omega]$ obtenemos que $x_n \not \in U \ \forall n \in \N$ y entonces $x_n \not \rightarrow \Omega$. \qed
	
	
\end{proof}

\begin{remark}{De la Observaci\'on}
	Notemos que en particular $\widetilde{S_{\Omega}} = \overline{S_{\Omega}}$, y que $S_{\Omega}$ no puede ser metrizable ya que esta propiedad si la cumplen los espacios m\'etricos.
\end{remark}

\begin{definition}{Conjunto dirigido}
	Un conjunto parcialmente ordenado $(\Lambda , \leq)$ se dice un \textit{conjunto dirigido} si $\forall \alpha,\beta \in \Lambda$ existe un $\omega \in \Lambda$ tal que $\alpha,\beta \leq \omega$.
\end{definition}

\begin{definition}{Red en espacios topol\'ogicos}
	Una \textit{red} en un espacio topol\'ogico $X$ es una funci\'on $f: \Lambda \rightarrow X$ donde $\Lambda$ es un conjunto dirigido. A la red tambi\'en se la notar\'a $(x_{\alpha})_{\alpha \in \Lambda}$ donde $x_{\alpha} = f(\alpha)$.
\end{definition}

\begin{definition}{De convergencia de redes}
	Diremos que una red $(x_{\alpha})$ converge a un punto $x \in X$ si para todo entorno $U \ni x$ existe $\alpha_0$ tal que $x_{\alpha} \in U \ / \forall \alpha \geq \alpha_0$ y lo notaremos $x_{\alpha} \rightarrow x$
\end{definition}

\begin{example}
	Sea $(X, \tau_i)$ un espacio topol\'ogico y $x_{\alpha}$ una red en $X$, entonces $x_{\alpha} \rightarrow x$ $\forall x \in X$.
\end{example}

\begin{proposition}
	
	\label{Caracterizacion de la clausura por redes}
	
	Sea $X$ un espacio topol\'ogico, $A \subseteq X$ un subconjunto y $x \in X$. Entonces $x \in \overline{A}$ si y s\'olo si existe $x_{\alpha}$ red en $X$ tal que $x_{\alpha} \in A \ \forall \alpha \in \Lambda$ y $x_{\alpha} \rightarrow x$ 
	
\end{proposition}

\begin{proof}
	
	Replicando la demostraci\'on de sucesiones se puede ver la necesidad de la condici\'on.
	
	Para el otro lado, sea $\Lambda = \sett{u \subseteq X \ / \ U \emph{entorno de } x}$ ordenado por la inclusi\'on inversa, ie: $U \leq V \ \leftrightarrow V \subseteq U$. Notemos que $\Lambda$ es dirigido pues si $U,V \in \Lambda$ entonces $U \cap V \in \Lambda$, y $U,V \leq U\cap V$. Como $x \in \overline{A}$ entonces para todo $U \in \Lambda$ se tiene $A \cap U \neq \emptyset$, sea entonces $f : \Lambda \rightarrow X$ dado por $f(U) \in U \cap A$. Notemos que $f$ es una red y converge a $x$; en efecto, sea $V \ni x$ entorno de $x$, entonces si $U \geq V$ entonces $x_{U} = f(U) \in U \cap A \subseteq V \cap A \subseteq V$ \qed
	
\end{proof}

\begin{example}{Red que converge a $\Omega$ en $S_{\Omega}$}
	Recordemos que no existe sucesi\'on que tiende a $\Omega$ como vimos previamente. Sea ahora $f : S_{\Omega} \rightarrow S_{\Omega}$ dado por $f(a) = a$ donde en el dominio vemos a $S_{\Omega}$ como conjunto dirigido y en el codominio como espacio topol\'ogico. Sea $U=(b,\Omega] \ni \Omega$ el entorno b\'asico de $\Omega$, entonces si $a \geq b$ trivialmente $f(a) \in U$ por lo que $f \rightarrow \Omega$.
\end{example}

\begin{definition}{Funci\'on cofinal}
	
	Sean $\Gamma, \Lambda$ conjunto dirigidos. Una funci\'on $f: \Gamma \rightarrow \Lambda$ se dice \textit{cofinal} si preserva el orden y $\forall \lambda \in \Lambda$ existe $\omega \in \Gamma$ tal que $f(\omega) \geq \lambda$
	
\end{definition}

\begin{definition}{De subred de un espacio topologico}
	
	Si $f: \Lambda \rightarrow X$ es una red en un espacio topol\'ogico $X$ y $g: \Gamma \rightarrow \Lambda$ es una fuci\'on cofinal, entonces la composici\'on $fg: \Gamma \rightarrow X$ se denomina \textit{subred} de la red $f$ y la denotaremos $(x_{\lambda_{\omega}})_{\omega \in \Gamma}$ donde $x_{\lambda_{\omega}} = fg(\omega)$.
	
\end{definition}


\begin{proposition}
	
	\label{Convergencia de una red por sub-redes}
	
	\begin{enumerate}
		
		\item Si $(x_{\alpha})_{\alpha \in \Lambda}$ es una red en $X$ que converge a $x \in X$, entonces toda subred tambi\'en converge a $x$.
		\item Si $(x_{\alpha})_{\alpha \in \Lambda}$ es una red en $X$, $x \in X$ y toda subred de $x_{\alpha}$ tiene una subred que converge a $x$; entonces $x_{\alpha} \rightarrow x$
		
	\end{enumerate}
	
\end{proposition}

\begin{proof}
	
	Vayamos por partes:
	
	\begin{enumerate}
		
		\item Sea $g : \Gamma \rightarrow \Lambda$ cofinal y consideremos $h = fg$ una subred de $f(\alpha) = x_{\alpha}$, queremos ver que $h \rightarrow x$. Para esto sea $U \ni x$ un entorno abierto de $x$, como $f$ converge a $x$ sabemos que existe $\alpha_0 \in \Lambda$ tal que si $\alpha \geq \alpha_0$ entonces $f(\alpha) \in U$. Como $g$ es cofinal sabemos que existe $\omega_0 \in \Gamma$ tal que $g(\omega_0) \geq \alpha_0$ y ademas si $\omega \geq \omega_0$ entonces $g(\omega) \geq g(\omega_0) \alpha_0$. Por lo tanto si $\omega \geq \omega_0$ tenemos que $h(\omega) = f(g(\omega)) \geq f(g( \omega_0)) \geq f(\alpha_0)$ por lo que $h(\omega) \in U$. Por lo tanto $fg \rightarrow x$.
		
		\item Supongamos que $f \not \rightarrow x$ entonces existe $U_0 \ni x $ entorno abierto de $x$ tal que $\forall \alpha \in \Lambda$ existe $\alpha_0 \geq \alpha$ tal que $f(\alpha_0) \not \in U_0$. Sea $\Gamma = \sett{\gamma \in \Lambda \ / \ f(\gamma) \not \in U_0}$, entonces la suposici\'on implica que este conjunto es dirigido; adem\'as si consideramos $i : \Gamma \rightarrow \Lambda$ dado por $i(\gamma) = \gamma$ esta funci\'on es cofinal. Si consideramos $fi := g$ es una subred de $f$ y por hip\'otesis existe $h: \Pi \rightarrow \Gamma$ cofinal tal que $gh \rightarrow x$. No obstante, $gh(\pi) \not \in U_0$ para todo $\pi \in \Pi$ por definici\'on de $\Gamma$, por lo que $gh \not \rightarrow x$; por lo tanto $f \rightarrow x$ \qed
		
	\end{enumerate}
	
\end{proof}

\subsection{Funciones Continuas}

\begin{definition}{Funci\'on continua}
	
	Sean $X,Y$ espacios topol\'ogicos, una funci\'on $f:X \rightarrow Y$ se dice \textit{continua} si $f^{-1}(U)$ es abierto en $X$ para todo $U \subset Y$ abierto.
\end{definition}

\begin{example}
	\begin{itemize}
		
		\item Si $X$ es cualquiera entonces $1_X : X \rightarrow X$ dado por $x \mapsto x$ es continua.
		
		\item Si $X$ es discreto e $Y$ cualquiera entonces toda funci\'on $f:X \rightarrow Y$ es continua
		
		\item Si $X$ es cualquiera e $Y$ indiscreto entonces toda funci\'on $f:X \rightarrow Y$ es continua
		
		\item Si $X,Y$ son cualesquiera, $c_{y_0} : X \rightarrow Y$ dado por $x \mapsto y_0$ es continua
		
		\item Composici\'on de continuas es continua
		
		\item Si $\tau_1,\tau_2$ son dos topolog\'ias en $X$, entonces $1_X : (X,\tau_1) \rightarrow (X,\tau_2)$ es continua si y s\'olo si $\tau_2 \subseteq \tau_1$
		
	\end{itemize}
\end{example}

\begin{remark}
	
	\label{La indicadora del complemento de U es continua en sierpinsky}
	
	Sea $X$ cualquiera, entonces existe una biyecci\'on entre el conjunto de funciones continuas $f: X \rightarrow \mathfrak{S}$ y el conjunto de abiertos de $X$
\end{remark}

\begin{proof}{De la observaci\'on}
	
	Sea $\psi$ dada por $\psi(f) = f^{-1}(0)$ donde $f: X \rightarrow \mathfrak{S}$ es una funci\'on continua, y sea $\phi (U) = 1_{U^c}$ donde $U$ es un abierto de $X$; veamoso que ambas funciones estan bien definidas y son mutuamente inversas.
	
	Si $f: X \rightarrow \mathfrak{S}$ es una funci\'on continua, entonces $f^{-1}(U)$ es abierto para todo $U$ abierto de $\mathfrak{S}$, en particular $U = \sett{0}$ por lo que efectivamente $f^{-1}(0) \in \tau$ es abierto de $X$. Rec\'iprocamente sea $U \in \tau$ y consideremos $f := 1_{U^c} : X \rightarrow \mathfrak{S}$:
	
	\begin{itemize}
		\item $f^{-1}(\mathfrak{S}) =  X \in \tau$
		\item $f^{-1}(\emptyset) =  \emptyset \in \tau$
		\item $f^{-1}(\sett{0}) =  U \in \tau$
	\end{itemize}
	
	Por lo tanto $f$ es una funci''on continua. y ambas $\phi,\psi$ estan bien definidas. Adem\'as:
	
	\begin{itemize}
		
		\item Si $f : X \rightarrow \mathfrak{S}$ es continua entonces $\phi(\psi(f)) = \phi(f^{-1}(0)) = 1_{f^{-1}(0)^c}$. Si $f(x) = 1$ entonces $x \not\in f^{-1}(0)$ por lo que $x \in f^{-1}(0)^c$ y entonces $1_{f^{-1}(0)^c}(x) = 1$. Es simple ver que si $f(x) = 0$ entonces $1_{f^{-1}(0)^c}(x) = 0$; por lo que $1_{f^{-1}(0)^c} = f$ y entonces $\phi\psi = 1_{A}$ donde $A$ es el conjunto de funciones continuas $f: X \rightarrow \mathfrak{S}$.
		
		\item Si $U \subseteq X$ es abierto entonces $\psi(\phi(U)) =  \psi(1_{U^c}) = 1_{U^c}^{-1}(0) = U$, por lo que $\psi \phi = 1_{\tau}$. \qed
		
	\end{itemize}
		
\end{proof}

\begin{remark}
	
	\label{Ver continuidad solo en la base}
	
	Si $\B$ es una base para la topolog\'ia de un espacio $Y$, una funci\'on $f:(X, \tau_X) \rightarrow Y$ es continua sii $f^{-1}(U) \in \tau_X \ , \ \forall U \in \B$, es m\'as vale solo verlo en los abiertos subbasicos.
	
\end{remark}

\begin{proof}
	Si $V \subset Y$ es abierto entonces $V = \bigcup\limits_{\substack{U \subset V \\ U in \B}}{U}$ por lo que $f^{-1}(V) = \bigcup\limits_{\substack{U \subset V \\ U in \B}}{f^{-1}(U)} \in \tau_X$ \qed
\end{proof}

\begin{proposition}
	
	\label{Equivalencias de continuidad}
	
	Sean $X,Y$ espacios topol\'ogicos arbitrarios y $f: X \rightarrow Y$ una funci\'on, entonces son equivalentes:
	
	\begin{enumerate}
		\item $f$ es continua
		\item $f^{-1}(F)$ es cerrado $\forall F$ cerrado
		\item $f(\overline{A}) \subseteq \overline{f(A)} \ , \ \forall A \subseteq X$
		\item Si $(x_{\alpha})_{\alpha \in \Lambda}$ es una red en $X$  tal que $x_{\alpha} \rightarrow x_0$ entonces $(f(x_{\alpha}))_{\alpha \in \Lambda} \rightarrow f(x)$
	\end{enumerate}
\end{proposition}

\begin{proof}
	Probemos de a partes:
	
	\begin{itemize}
		\item[i) $\Longrightarrow$ iv)] Sea $U \ni x$ un entorno abierto de $f(x)$, entonces como $f$ es continua $f^{-1}(U) \ni x$ es un entorno abierto de $x$. Por lo tanto, existe $\alpha_0 \in \Lambda$ tal que para todo $\alpha \geq \alpha_0$ se tiene $x_{\alpha} \in f^{-1}(U)$, pero esto implica que para todo $\alpha \geq \alpha_0$ se tiene $f(x_{\alpha}) \in U$. Por lo tanto $f(x_{\alpha}) \rightarrow f(x)$
		
		\item[iv) $\Longrightarrow$ ii)] Sea $F \subseteq Y$ cerrado y $x \in \overline{f^{-1}(F)}$, por \ref{Caracterizacion de la clausura por redes} existe $x_{\alpha}$ red en $X$ tal que $x_{\alpha} \in f^{-1}(F) \ , \ \forall \alpha \in \Lambda$ y $x_{\alpha} \rightarrow x$. Por hip\'otesis entonces $f(x_{\alpha}) \rightarrow f(x)$, pero como $f(x_{\alpha}) \in F$ para todo $\alpha \in \Lambda$ entonces por \ref{Caracterizacion de la clausura por redes} tenemos que $f(x) \in \overline{F}=F$; por lo tanto $x \in f^{-1}(F)$ y se tiene que $f^{-1}(F)$ es cerrado.
		
		\item[ii) $\Longrightarrow$ i)] Sea $U \subseteq Y$ abierto, entonces $U^c$ es cerrado y por hip\'otesis $f^{-1}(U^c)=(f^{-1}(U))^c$ es cerrado; por lo tanto $f^{-1}(U)$ es abierto y entonces $f$ es continua.
		
		\item[ii) $\Longrightarrow$ iii)] Sea $A \subseteq X$, entonces $A \subseteq f^{-1}(f(A)) \subseteq f^{-1}(\overline{f(A)})$; luego $\overline{A} \subseteq \overline{f^{-1}(\overline{f(A)})} = f^{-1}(\overline{f(A)})$ por hip\'otesis, por lo tanto $f(\overline{A}) \subseteq \overline{f(A)}$ 
		
		\item[iii) $\Longrightarrow$ ii)] Sea $F \subseteq Y$ cerrado, entonces por hip\'otesis tenemos que $f(\overline{f^{-1}(F)}) \subseteq \overline{f(f^{-1}(F))} \subseteq \overline{F} = F$, por lo tanto $\overline{f^{-1}(F)} = f^{-1}(F)$ es cerrado en $X$ \qed
	\end{itemize}
	
\end{proof}


\begin{definition}
	Sean $X,Y$ espacios topol\'ogicos, una funci\'on $f. X\rightarrow Y$ se dice \textit{abierta} si $f(U) \in \tau_Y$ para todo $U \in \tau_X$. Similarmente se dice \textit{cerrada} si $f(F)^c \in \tau_Y$ para todo $F^c \in \tau_X$.
\end{definition}


\begin{definition}
	Sean $X,Y$ espacios topol\'ogicos, una funci\'on $f. X\rightarrow Y$ se dice \textit{homeomorfismo} si $f$ es continua, biyectiva y $f^{-1}$ continua.
\end{definition}

\begin{remark}
	$f$ es homeomorfismo si y s\'olo si $f$ es continua, biyectiva y abierta; si y s\'olo si $f$ es continua, biyectiva y cerrada
\end{remark}

\begin{definition}{Topolog\'ia subespacio}
	Sea $X$ un espacio topol\'ogico y sea $A \subseteq X$ un sunconjunto. La \textit{topolog\'ia de subespacio} en $A$ es $\tau_A = \sett{U \cap A \ / \ U \in \tau}$ y en ese caso decimos que $(A,\tau_A)$ es un subespacio de $(X,\tau)$.
\end{definition}

\begin{remark}
	La topolog\'ia de subespacio es una topolog\'ia
\end{remark}

\begin{remark}
	
	\label{Restriccion de continua a subespacio es continua}
	
	Si $A$ es un subespacio de $X$ entonces la inclusi\'on $i:A \inc X $ es una funci\'on continua pues $i^{-1}(U) = U\cap A$. Similarmente si $A \subseteq X$ es subespacio y $f:X \rightarrow Y$ es continua, entonces $f|_A := fi : A \rightarrow Y$ es continua pues $(fi)^{-1}(U) = f^{-1}(U) \cap A$ y $f^{-1}(U) \in \tau_X$.
\end{remark}

\begin{proposition}
	
	\label{Propiedad universal del subespacio}
	
	Si $A \subseteq X$ es subespacio y $f : Y \rightarrow A$ es una funci\'on entonces $f$ es continua si y s\'olo si $if : Y \rightarrow X$ es continua
	
\end{proposition}

\begin{proof}
	Para un lado composici\'on de continuas es continua.
	
	Para el otro, si $U \in \tau_X$ entonces $\tau_Y \ni (if)^{-1}(U) = f^{-1}(U \cap A) = f^{-1}(U \cap A)$ y como $U \cap A \in \tau_A$ se tiene que $f$ es continua. \qed
	
\end{proof}

\begin{example}
	Sea $X = [0,1)$ con la topolog\'ia subespacio de $\R$ y $Y = S^1 = \sett{x \in \R^2 \ / \ \norm{x}_2 = 1}$ con la topolog\'ia subespacio de $\R^2$. Sea adem\'as $f:X \rightarrow Y$  dada por $f(t) = (cos(2\pi t) , sin(2 \pi t))$ entonces $f$ es biyectiva.
	
	Consideremos $g : \R \rightarrow S^1$ dada por $t \mapsto (cos(2 \pi t) , sin(2 \pi t))$, entonces $g$ es continua pues $ig$ es continua. por lo tanto $f = g|_X : X \rightarrow Y$ es continua. No obstante $f$ no es homeomorfismo pues $f([0,\frac{1}{2})$ no es abierto.
	
\end{example}

\begin{remark}
	Si $A$ es subespacio entonces los cerrados de $A$ son los cerrados de $X$ intersecados con $A$. En efecto, $F \subseteq X$ es cerrado sii existe $U \subseteq X$ abierto tal que $A \setminus F = U \cap A$ sii existe $U \subseteq X$ abierto tal que $F = A \setminus (U \cap A) = A \cap U^c$
\end{remark}

\begin{proposition}
	
	\label{Lema del pegado}
	
	\begin{enumerate}
		
		\item Si $F_1 , F_2$ son dos subespacios cerrados de un espacio topol\'ogico $X$ tal que $X = F_1 \cup F_2$ y $f : X \rightarrow Y$ tal que $f|_{F_i}$ es continua, entonces $f$ es continua.
		
		\item Si $\sett{U_j}_{j \in J}$ son subespacios abiertos de un espacio topol\'ogico $X$ tal que $X = \bigcup\limits_{j \in J}{U_j}$ y $f : X \rightarrow Y$ tal que $f|_{U_j}$ es continua para todo $j \in J$, entonces $f$ es continua.
		
	\end{enumerate}
\end{proposition}

\begin{proof}
	
	\begin{itemize}
		
		\item[i)] Sea $H \subseteq Y$ cerrado, entonces $f^{-1}(H) = (f^{-1}(H) \cap F_1) \cup (f^{-1}(H) \cap F_2) = (f|_{F_1})^{-1}(H) \cup (f|_{F_2})^{-1}(H)$ y como $f|_{F_i}$ es continua entonces $f^{-1}(H)$ es cerrado.
		
		\item[ii)] Sea $U \subset Y$ abierto, entonces $f^{-1}(U) = \bigcup\limits_{j \in J}{f^{-1}(U) \cap U_j} = \bigcup\limits_{j \in J}{(f|_{U_j})^{-1}(U)}$ y como cada $f|_{U_j}$ es continua entonces $f^{-1}(U)$ es abierto \qed
		
	\end{itemize}
	
\end{proof}

\begin{remark}
	No vale el item anterior de \ref{Lema del pegado} para arbitrarios cerrados.

\end{remark}

\begin{proof} 
	
	En efecto, sean $X = [0,1]$, $F_0 = \sett{0}$ y $F_n= [\frac{1}{n},1]$ para todo $n \in \N$; entonces $X = \bigcup\limits_{n \in \N}{F_n}$ y cada $F_n$ es cerrado. Sea $f : X \rightarrow \R$ dado por:
	
	\[
	f = \left\lbrace 
	\begin{array}{cc}
	0 & \emph{si } x= 0 \\ 
	1 & \emph{si } x \neq 0
	\end{array} 
	\right.
	\]
	
	
	entonces es claro que $f$ no es continua pero $f|_{F_n}$ lo es. \qed
\end{proof}

\section{Topolog\'ias iniciales y finales}

\subsection{Topolog\'ia producto}

\begin{definition}{De topolog\'ia producto para 2 espacios}
	Sean $X,Y$ espacios topol\'ogicos, entonces la \textit{topolog\'ia producto en} $X \times Y$ es la  que tiene por subbase a $\mathcal{S} = \sett{U \times Y \ / \ U \in \tau_X} \cup  \sett{X \times V \ / \ V \in \tau_Y}$, o equivalentemente a la que tiene por base a $\B = \sett{U \times V \ / \ U \in \tau_X \ , \ V \in \tau_Y}$. Notaremos a la topolog\'ia producto $\tau_{X \times Y}$ 
\end{definition}

\begin{remark}
	Notar que la topolog\'ia producto es la topolog\'ia m\'as chica que hace a las proyecciones $p_X, p_Y$ continuas, donde $p_X (x,y) = x$ y $p_Y (x,y) = y$.
\end{remark}

\begin{proof}
	En efecto para una topolog\'ia $\tau$, las proyecciones son continuas sii $\sett{p_{X}^{-1}(U) \ / \ U \in \tau_X} , \sett{p_{Y}^{-1}(V) \ / \ V \in \tau_Y} \subseteq \tau$ sii $\sett{p_{X}^{-1}(U) \ / \ U \in \tau_X} \cup \sett{p_{Y}^{-1}(V) \ / \ V \in \tau_Y} \subseteq \tau$ sii $\mathcal{S} \subseteq \tau$ sii $\sigma(S) \subseteq \tau$ sii $\tau_{X \times Y} \subseteq \tau$. \qed
\end{proof}

\begin{theorem}
	
	\label{Caracterizacion de la topologia producto}
		
	Si $\sett{X_j}_{j \in J}$ es una familia de espacios topol\'ogicos y $\tau$ es una topolog\'ia en $X = \prod\limits_{j \in J}{X_j}$ entonces son equivalentes:
	
	\begin{enumerate}
		
		\item Una subbase de $\tau$ es $\mathcal{S} = \sett{p_{j}^{-1}(U) \ / \ j \in J \ , \ U \in \tau_{X_j}}$
		
		\item Una base de $\tau$ es $\B = \sett{\prod\limits_{j \in J}{U_j} \ / \ U_j \in \tau_{X_j} \ , \ U_j \subsetneq X_j \emph{ sii } j \in J' \emph{ finito}}$
		
		\item $\tau$ es la topolog\'ia m\'as chica que hace a las $p_j$ continuas para todo $j \in J$
		
		\item Dada $f: Y \rightarrow X$ es continua sii $p_jf : Y \rightarrow X_j$ es continua para todo $j \in J$
		
	\end{enumerate}
	
	A la topolog\'ia que cumple estas propiedades en $X = \prod\limits_{j \in J}{X_j}$ la llamaremos \textit{topolog\'ia producto} y la notaremos $(X , \tau_p)$ o $(X , \tau_{\prod X_j})$ 
	
\end{theorem}

\begin{proof}
	
	Es claro que i) vale $\Longleftrightarrow$ ii) $\Longleftrightarrow$ iii) extendiendo la demostraci\'on que hicimos en la observaci\'on. Veamos la \'ultima:
	
	\begin{itemize}
		
		\item[iii) $\Longrightarrow$ iv)] Notemos que para un lado por iii) sabemos que $p_j$ son continuas para todo $j \in J$ y como composici\'on de continuas es continua concluimos que $p_jf$ es continua.
		
		Para el otro lado, por \ref{Ver continuidad solo en la base} podemos tomar un abierto subb\'asico $V = p_{j_0}^{-1}(U)$ de la topolog\'ia producto, donde $j_0 \in J$ y $U \in \tau_{X_{j_0}}$. Notemos que $f^{-1}(V) = f^{-1}(p_{j_0}^{-1}(U)) = (p_{j_0}f)^{-1}(U) \in \tau_Y$ pues $p_{j_0}f$ es continua; concluimos que $f$ es continua.
		
		\item[iv) $\Longrightarrow$ iii)] Supongamos que $\tau$ satisface iv) y llamemos $\tau_p$ la topolog\'ia que cumple iii), luego queremos ver que $\tau = \tau_p$.
		
		Como $1_X : (X, \tau) \rightarrow (X, \tau)$ es continua entonces como $\tau$ cumple iv) tenemos que $p_j 1_X = p_j : X \rightarrow X_j$ es continua para todo $j \in J$. Como $\tau_p$ es la m\'as chica con esa propiedad tenemos que $\tau_p \subseteq \tau$.
		
		Ahora consideremos $1_X : (X, \tau_p) \rightarrow (	X, \tau)$, como $p_j 1_X = p_j : (X, \tau_p) \rightarrow X_j$ es continua para todo $j \in J$, entonces como $\tau$ satisface iv) se tiene que $1_X$ es continua; con lo que $\tau \subseteq \tau_p$. \qed 
		
	\end{itemize}
	
\end{proof}

\begin{example}
	Consideremos $\mathfrak{S}^{\N} := \prod\limits_{n \in \N}{\mathfrak{S}}$ y sea $(x^{n}) \in \mathfrak{S}^{\N}$ dado por $ (x^{n})_i = 1_{\sett{i > n}}$, veamos que $x^{n} \rightarrow 0$
	
	Sea $U \ni 0$ un entorno b\'asico de $0$, entonces existen $\sett{U_i}$ entornos de $0 \in \mathfrak{S}$ tal que $U = \prod\limits_{i=1}^{n_0}{U_i} \times \prod\limits_{i>n_0}{\mathfrak{S}}$. Por lo tanto si $k > n_0$ entonces $x^{k} \in U$, por lo que $x^{n} \rightarrow 0$.
\end{example}

\begin{remark}
	Notemos que tambi\'en $x^{n} \rightarrow 1$ pues si $V \ni 1$ es un entorno abierto de $1$ entonces $V = \mathfrak{S}^{\N}$ y entonces toda $x^{n} \rightarrow 1$.
\end{remark}

\begin{proposition}
	
	\label{Propiedad universal del producto}
	
	Dada una familia $\sett{X_j}_{j \in J}$ de espacios topol\'ogicos existe $X$ espacio topol\'ogico junto con funciones continuas $f_j : X \rightarrow X_j$ para todo $j \in J$ ocn la siguiente propiedad:
	
	Para cualquier espacio $Y$ y funci\'on continua $g_j : Y \rightarrow X_j$ existe una \'unica $f: Y \rightarrow X$ tal que el siguiente diagrama conmuta:
	
	\begin{equation}
	\begin{tikzcd}
	Y \arrow[dashed]{r}{\exists ! f} \arrow[swap, bend right]{dr}{g_j} & X \arrow[swap]{d}{f_j} \\ 
	& X_j
	\end{tikzcd}
	\end{equation}
	
	M\'as a\'un el espacio $X$ es \'unico salvo homeomorifsmo. Esta propiedad la llamaremos \textit{propiedad universal del espacio producto}.
		
\end{proposition}


\begin{proof}
	
	Sea $X = \prod\limits_{j \in J}{X_j}$ y $f_j = p_j : X \rightarrow X_j$, entonces ya sabemos que las $p_j$ son continuas para todo $j \in J$. Sea entonces $Y$ un espacio topol\'ogico y $\sett{g_j}_{j \in J}$ una familia de funciones continuas $g_j :  Y \rightarrow X_j$; entonces consideremos $f : Y \rightarrow X$ dado por $f(y)_j = g_j(y)$ que sabemos que es la \'unica funci\'on de conjuntos que hace el diagrama conmutar. Veamos que $f$ es continua y para esto por \ref{Caracterizacion de la topologia producto} basta ver que $p_jf = g_j$ es continua, por lo tanto $f$ es continua. Esto concluye la existencia de $X$.
	
	Supongamos ahora que $X'$ junto con $f'_j$ es otro espacio que cumple la propiedad. Como $X$ cumple la propiedad entonces existe $f : X' \rightarrow X$ tal que el siguiente diagrama conmuta:
	
		\begin{equation*}
		\begin{tikzcd}
		X' \arrow[dashed]{r}{\exists !f} \arrow[swap, bend right]{dr}{f'_j} & X \arrow[swap]{d}{f_j} \\ 
		& X_j
		\end{tikzcd}
		\end{equation*}
	
	Por otro lado como $X'$ satisface la propiedad existe $g: X \rightarrow X'$ tal que el siguiente diagrama conmuta:
	
		\begin{equation*}
		\begin{tikzcd}
		X \arrow[dashed]{r}{\exists!g} \arrow[swap, bend right]{dr}{f_j} & X' \arrow[swap]{d}{f'_j} \\ 
		& X_j
		\end{tikzcd}
		\end{equation*}
	
	Entonces tenemos que $f_j fg = f'_j g = f_j$ con lo que el siguiente diagrama conmuta:
	
		\begin{equation*}
		\begin{tikzcd}
		X \arrow[shift right = 5]{r}{fg} \arrow[shift left=5]{r}{1_X} \arrow[swap]{dr}{f_j} & X \arrow[swap]{d}{f_j} \\ 
		& X_j
		\end{tikzcd}
		\end{equation*}	
	
	Por unicidad de la funci\'on $h: X \rightarrow X$ tal que $f_j h = f_j$ se tiene que $fg = 1_X$. An\'alogamente se tiene que $gf = 1_X'$ por lo que $X$ es homeomorfo a $X'$ (Lo notaremos $X \simeq X'$). \qed
	
\end{proof}

\begin{definition}{Topolog\'ia caja}
	
	Sea $\sett{X_j}_{j \in J}$ una familia de espacios topol\'ogicos, la \textit{topolog\'ia caja} en $X = \Bigprod{j \in J}{X_j}$ es la que tiene por base a $\B = \sett{\Bigprod{j \in J}{U_j} \ / \ U_j \in \tau_{X_j}}$ y la notaremos $(X,\tau_c)$

\end{definition}

\begin{remark}
	Notemos que $\B$ es base pues $\Bigprod{j \in J}{U_j} \cap \Bigprod{j \in J}{V_j} = \Bigprod{j \in J}{U_j \cap V_j} \in \B$
\end{remark}

\begin{remark}
	Notemos que $\tau_p \subseteq \tau_c$ y si $J$ es finito entonces $\tau_p = \tau_c$.
\end{remark}

\begin{example}
	Sea $(X = \Bigprod{n \in \N}{\mathfrak{S}}, \tau_c)$, entonces $x^{n} = 1_{\sett{ i > n}} \not \rightarrow 0$ pues $V = \Bigprod{n \in \N}{\sett{0}} \ni 0$ es un entorno abierto del $0$ tal que $x^n \not\in V$ para todo $n \in \N$
\end{example}

\subsection{Topolog\'ias iniciales}

\begin{definition}{Topolog\'ia inicial para una familia de funciones}
	
	Sea $X$ un conjunto, $\sett{X_j}_{j \in J}$ una familia de espacios topol\'ogicos y $f_j : X \rightarrow X_j$ funciones para todo $j \in J$. La \textit{topolog\'ia inicial en} $X$ respecto a la familia $\sett{X_j}_{j \in J}$ es la que tiene por subbase a $\mathcal{S} = \sett{f_j^{-1}(U)} \ / \ j \in J \ , \  U \in \tau_{X_j}$ y en este caso decimos que $\sett{f_j}_{j \in J}$ es una \textit{familia inicial de} $X$.
	
\end{definition}

\begin{remark}
	La topolog\'ia inicial respecto a la familia $\sett{f_j}$ es la m\'inima topolog\'ia que hace continuas a todas las $f_j$.
\end{remark}

\begin{example}
	\begin{itemize}
		
		\item Si $\sett{X_j}_{j \in J}$ es una familia de espacios topol\'ogicos, la topolog\'ia inicial respecto a las $\sett{p_i: \Bigprod{j \in J}{X_j} \rightarrow X_i}_{i \in I}$ es la topolog\'ia producto
		
		\item Si $A \subseteq X$ es un subconjunto de un espacio topol\'ogico $X$, entonces la topolog\'ia inicial respecto a $\sett{i : A \inc X}$ es la topolog\'ia subespacio. 
		
	\end{itemize}
\end{example}

\begin{proposition}
	
	\label{Caracterizacion de la topologia inicial}
	
	Sea $X$ un conjunto, $\sett{X_j}_{j \in J}$ una familia de espacios topol\'ogicos y $f_j : X \rightarrow X_j$ una familia de funciones. Entonces la topolog\'ia inicial respecto a las $\sett{f_j}$ es la \'unica que cumple lo siguiente:
	
	\begin{itemize}
		\item 	Para todo $Y$ espacio topol\'ogico y $f : Y \rightarrow X$ se tiene que $f$ es continua si y s\'olo si $f_jf :  Y \rightarrow X_j$ es continua para todo $j \in J$.
	\end{itemize}
	
\end{proposition}

\begin{proof}
	
	Primero veamos que la topolog\'ia inicial  $\tau$ cumple la propiedad:
	
	Para un lado como $\tau$ hace a todas las $p_j$ continuas entonces si $f$ es continua, $p_jf$ es continua para todo $j \in J$.
	
	Para el otro, por \ref{Ver continuidad solo en la base} basta tomar un $V \in \mathcal{S}$ la subbase de $\tau$ para el cual $V = f_{j_0}^{-1}(U)$ para $j_0 \in J$ y $U \in \tau_{X_{j_0}}$. Entonces $f^{-1}(V) = f^{-1}(f_{j_0}^{-1}(U)) = (f_{j_0}f)^{-1}(U) \in \tau_Y$ pues $f_{j_0}f$ es continua, concluimos que $\tau$ cumple la propiedad.
	
	Si $\tau \prime $ es una topolog\'ia en $X$ que cumple la propiedad y $1_{(X, \tau \prime)} : (X, \tau \prime) \rightarrow (X, \tau \prime)$ es continua entonces $f_j 1_{(XX ,\tau \prime)} = f_j$ es continua. Como $\tau$ por definici\'on es la m\'inima que cumple esto, resulta que $\tau \subseteq \tau \prime$.
	
	Por otro lado consideremos:
	
		\begin{equation*}
		\begin{tikzcd}
		(X, \tau) \arrow{r}{f} \arrow[swap]{dr}{f_j} & (X, \tau \prime) \arrow[swap]{d}{f_j} \\ 
		& X_j
		\end{tikzcd}
		\end{equation*}
	
	Como $\tau \prime$ cumple la condici\'on entonces $f$ es continua sii $f_j f = f_j$. Pero como $\tau$ es la topolog\'ia inicial respecto a las $\sett{f_j}$ entonces las $f_j$ son continuas, por lo tanto $f$ es continua y $\tau \prime \subseteq \tau$. Por lo tanto $\tau$ es la \'unica con la propiedad \qed
	
\end{proof}


\begin{definition}{Funciones subespacio}
	Sean $X,Y$ espacios topol\'ogicos entonces una funci\'on $f: X \rightarrow Y$ se dice \textit{inicial} si $\sett{f}$ es una familia inicial. Adem\'as, se dice \textit{subespacio} si $f$ es inyectiva e inicial.
\end{definition}

\begin{example}
	Sea $(A,\tau_A) \subseteq (X,\tau)$ y consideremos $i : A \inc X$, entonces $i$ es subespacio
\end{example}

\begin{proposition}
	
	\label{Si f es subespacio entonces X es homeo a f(X)}
	
	Sea $f: X \rightarrow Y$ una funci\'on subespacio, si consideramos $f(X) \subseteq Y$ con la topolog\'ia subespacio entonces $\widetilde{f}:X \rightarrow f(X)$ es un homeo.
	
\end{proposition}

\begin{proof}
	Como $f$ es inyectiva entonces $\widetilde{f}$ es biyectiva, sea $g$ la inversa y consideremos:
	
	\begin{equation*}
	\begin{tikzcd}
	X \arrow[shift left= 5]{dd}{\widetilde{f}} \arrow[swap, bend left]{dr}{f} & \\ 
	 & Y \\
	f(X) \arrow[shift left = 5]{uu}{g} \arrow[bend right]{ur}{i} & 
	\end{tikzcd}
	\end{equation*}
	
	Como $f(X)$ es subespacio entonces $\widetilde{f}$ es continua sii $i \widetilde{f}$ es continua, pero $i \widetilde{f} = f$; por lo que $\widetilde{f}$ es continua.
	
	Por otro lado, como $f$ es inicial entonces $g$ es continua sii $f g$ es continua, pero $gf = i$ que es claramente continua; por lo que $g$ es continua. \qed
	
\end{proof}

\begin{proposition}
	
	\label{Composicion de iniciales es inicial, y si la composicion es inicial entonces la interna lo es}
	
	\begin{enumerate}
		
		\item Si $\sett{f_j : X \rightarrow X_j}_{j \in J}$ es una familia inicial y para cada $j \in J$ se tiene que $\sett{g_{j,i} : X_j \rightarrow Y_{j,i}}_{i \in I_j}$ es otra familia inicial, entonces $\sett{g_{j,i}f_j}_{\substack{j \in J \\ i \in I_j}}$ es inicial.
		
		\item Si $\sett{f_j : X \rightarrow X_j}_{j \in J}$ y $\sett{g_{j,i} : X_j \rightarrow Y_{j,i}}_{i \in I_j}$ son familias de funciones continuas y $\sett{g_{j,i}f_j : X \rightarrow Y_{j,i}}_{\substack{j \in J \\ i \in I_j}}$  es inicial, entonces $\sett{f_j : X \rightarrow X_j}_{j \in J}$ es inicial
		
	\end{enumerate}
	
\end{proposition}

\begin{proof}
	
	\begin{itemize}
		
		\item[i)] Si $f: Z \rightarrow X$ es continua entonces como $f_j$ y $g_{j,i}$ son iniciales y por ende continuas, se tiene que $g_{j,i}f_j f$ es continua.
		
		Similarmente si $g_{j,i}f_j f$ es continua para todo $i \in I_j \ , \ j \in J$, entonces como $\sett{g_{i,j}}_{\substack{i \in I_j \\ j \in J}}$ es inicial tenemos que $g_{j,i}f_j f$ es continua sii $f_j f$ es continua para todo $j \in J$. Adem\'as como $\sett{f_j}_{j \in J}$ es inicial tenemos que $f_j f$ es continua para todo $j \in J$ sii $f$ es continua. Concluimos que $f$ es continua y por lo tanto $\sett{g_{j,i}f_j}_{\substack{j \in J \\ i \in I_j}}$ es inicial.
		
		\item[ii)] Si $f : Z \rightarrow X$ es tal que $f_jf$ es continua para todo $j \in J$, entonces $g_{j,i}f_jf$ es continua  para todo $i \in I_j \ , \ j \in J$. Como $\sett{g_{j,i}f_j}_{\substack{j \in J \\ i \in I_j}}$ es inicial esto sucede sii $f$ es continua. Como adem\'as si $f$ es continua entonces $f_jf$ es continua para todo $j \in J$ se tiene que $\sett{f_j}_{j \in J}$ es inicial. \qed
		
	\end{itemize}
	
\end{proof}

\begin{corollary}
	
	\label{Producto de tres es conmutativo}
	
	Si $X,Y,Z$ son espacios topol\'ogicos entonces $(X \times Y) \times Z \simeq X \times (Y \times Z) \simeq X \times Y \times Z$
	
\end{corollary}

\begin{proof}
	
	Notemos que:
	
	\begin{itemize}
		
		\item $\sett{p_X , p_Y}$ es inicial para $X \times Y$
		
		\item $\sett{1_Z}$ es inicial para $Z$
		
		\item $\sett{p_{X \times Y}, p_Z}$ es inicial para $(X \times Y) \times Z$
		
	\end{itemize}
	
	Por \ref{Composicion de iniciales es inicial, y si la composicion es inicial entonces la interna lo es} tenemos que $\sett{p_X p_{X \times Y} , p_Y p_{X \times Y} , 1_Z p_Z} = \sett{p_X, p_Y , p_Z}$ es inicial para $(X \times Y) \times Z$. Pero adem\'as esta misma familia es inicial para $X \times Y \times Z$, por la unicidad de la topolog\'ia inicial tenemos que $\tau_{(X \times Y) \times Z} = \tau_{X \times Y \times Z}$ y por lo tanto $(X \times Y) \times Z \simeq X \times Y \times Z$. An\'alogamente con $X \times ( Y \times Z)$. \qed
	
\end{proof}

\begin{proposition}
	
	\label{Subespacio de subespacio es subespacio}
	
	Si $Z \subseteq A$ es subespacio y $A \subseteq X$ es subespacio, entonces $Z \subseteq X$ es subespacio.
	
\end{proposition}

\begin{proof}
	
	Notemos $i_z : Z \inc A$, $i_A : A \inc X$ y $i : Z \inc X$, entonces tenemos que $\sett{i_Z}$ es inicial para $Z$ y $\sett{i_A}$ es inicial para $A$; por lo tanto por \ref{Composicion de iniciales es inicial, y si la composicion es inicial entonces la interna lo es} $\sett{i_Ai_Z} = \sett{i}$ es inicial para $Z$. Pero esto mismo es que $Z \subseteq X$ es subespacio. \qed
	
\end{proof}

\begin{example}
	Si $A \subseteq X$ y $B \subset Y$ son subespacios, entonces $(A \times B , \tau_p) = (A \times B , \tau)$ donde $\tau$ es la topolog\'ia subespacio respecto a $X \times Y$.
	
	En efecto, sea $W = U \times V \in \tau_p$ entonces como $A$ es subespacio $U = \tilde{U} \cap A$ donde $\tilde{U} \in \tau_X$, similarmente $V = \tilde{V} \cap B$ con $\tilde{V} \in \tau_Y$. Por lo tanto $W = U \times V = \left( \tilde{U} \times \tilde{V} \right) \cap \left( A \times B \right) \in \tau$. Por lo tanto $\tau = \tau_p$. \qed
	
	De otra manera, sabemos que $\sett{p_A : A \times B \rightarrow A , p_B : A \times B \rightarrow B}$ es inicial para $A \times B$ y adem\'as que $\sett{i_A : A \rightarrow X}$ es inicial para $A$ y $\sett{i_B : B \rightarrow X}$ es inicial para $B$. Por \ref{Composicion de iniciales es inicial, y si la composicion es inicial entonces la interna lo es} tenemos que $\sett{i_A p_A , i_B p_B}$ es inicial para $A \times B$. Finalmente como $\sett{i_X: X \rightarrow X \times Y}$ es inicial para $X$ y $\sett{i_Y : Y \rightarrow X \times Y}$ es inicial para $Y$ tenemos por \ref{Composicion de iniciales es inicial, y si la composicion es inicial entonces la interna lo es} que $\sett{i_{A \times B}: A \times B \rightarrow X \times Y}$ es inicial. \qed
	
\end{example}

\subsection{Topolog\'ias finales}

\begin{theorem}
	
	\label{Caracterizacion de la topologia final}
	
	Sea $X$ un espacio topol\'ogico, $\sett{X_j}_{j \in J}$ una familia de espacios topol\'ogicos y $f_j : X_j \rightarrow X$ funciones de conjuntos, entonces son equivalentes:
	
	\begin{enumerate}
		
		\item La topolog\'ia de $X$ es la m\'as fina que hace a las $f_j$ continuas
		
		\item $U \subset X$ es abierto sii $f_{j}^{-1}(U) \subseteq X_j$ es abierto para todo $j \in J$
		
		\item $F \subset X$ es cerrado sii $f_{j}^{-1}(F) \subseteq X_j$ es cerrado para todo $j \in J$
		
		\item Dado un espacio topol\'ogico $Y$ y una funci\'on $f:X \rightarrow Y$ entonces $f$ es continua sii $ff_j : X_j \rightarrow Y$ es continua para todo $j \in J$.
		
	\end{enumerate}
	
	En ese caso decimos que $X$ tiene la \textit{topolog\'ia final respecto a la familia} $\sett{f_j}_{j \in J}$ y que dicha familia es \textit{final} para $X$.
	
\end{theorem} 

\begin{proof}
	
	Sea $\tau = \sett{ U \subset X \ / \ f_{j}^{-1}(U) \subseteq X_j}$ notemos que es una topolog\'ia.
	
	\begin{itemize}
		
		\item[ii) $\Longleftrightarrow$ i)] Sea $\tau'$ una topolog\'ia en $X$ que hace a todas las $f_j$ continuas; esto pasa si y s\'olo si para todo $U\in \tau'$ vale que $f_{j}^{-1}(U) $ es abierto, si y s\'olo si para todo $U \in \tau'$ se tiene que $U \in \tau$; si y s\'olo si $\tau \subseteq \tau'$.
		
		\item[ii) $\Longleftrightarrow$ iii)] Es trivial
		
		\item[ii) $\Longrightarrow$ iv)] Sea $f: X \rightarrow Y$ es tal que $ff_j$ es continua para todo $j \in J$ y $U \subseteq Y$ abierto, entonces $f^{-1}(U) \subseteq X$ es abierto si y s\'olo si $f_{j}^{-1}(f^{-1}(U))$ es abierto para todo $j \in J$ por hip\'otesis. No obstante, $f_{j}^{-1}(f^{-1}(U)) = (ff_j)^{-1}(U)$ es abierto pues $ff_j$ es continua para todo $j \in J$; concluimos que $f$ es continua.
		
		\item[iv) $\Longrightarrow$ ii)] Sea $\tau'$ la topolog\'ia en $X$ que cumple iv) y veamos que $\tau' = \tau$. Por un lado si consideramos:
		
		\begin{equation*}
		\begin{tikzcd}
		(X,\tau') \arrow{r}{1_{(X,\tau')}} & (X,\tau')\\ 
		X_j \arrow[bend right]{ur}{f_j} \arrow{u}{f_j} & 
		\end{tikzcd}
		\end{equation*}
		
		Como $\tau'$ cumple iv) y $1_{(X,\tau')}$ es continua, entonces $1_{(X,\tau')}f_j = f_j$ es continua para todo $j \in J$; como $\tau$ es la topolog\'ia m\'as chica que hace a las $f_j$ continuas entonces $\tau \subseteq \tau'$.
		
		Por otro lado si consideramos:

		\begin{equation*}
		\begin{tikzcd}
		(X,\tau') \arrow{r}{f} & (X,\tau)\\ 
		X_j \arrow[bend right]{ur}{f_j} \arrow{u}{f_j} & 
		\end{tikzcd}
		\end{equation*}		
		
		Como $\tau'$ cumple iv) tenemos que $f$ es continua si y s\'olo si $ff_j = f_j : X_j \rightarrow X$ es continua, como todas las $f_j$ son continuas para $\tau$ tenemos que $f$ es continua, resulta que $\tau' \subseteq \tau$. \qed
		 
	\end{itemize}
	
\end{proof}

\begin{example}
	Si $F_1,F_2 \subseteq X$ son cerrados de $X$ tal que $X = F_1 \cup F_2$ entonces $X$ tiene la topolog\'ia final respecto a las inclusiones $\sett{i_j : F_j \inc X}_{j \in \sett{1,2}}$.
	
	En efecto, por \ref{Lema del pegado} sabemos que $f: X \rightarrow Y$ es continua si y s\'olo si $f|_{F_1},f|_{F_2}$ son continuas, si y s\'olo si $\sett{i_1,i_2}$ es final para $X$
	
\end{example}

\begin{definition}{Topolog\'ia coproducto}
	Dada una familia de espacios topol\'ogicos $\sett{X_j}_{j \in J}$ el \textit{espacio coproducto respecto a los} $X_j$ es el que tiene como conjunto subyacente a la uni\'on disjunta de los $X_j$ y notaremos $X = \Bigcoprod{j \in J}{X_j}$ con la topolog\'ia final respecto a las inclusiones $\sett{i_j : X_j \rightarrow X}_{j \in J}$.
\end{definition}

\begin{theorem}
	
	\label{Propiedad universal del coproducto}
	
	Dada una familia de espacios topol\'ogicos $\sett{X_j}$ existe un \'unico (salvo homeomorfismos) espacio $X$ conjunto a funciones continuas $f_j : X_j \rightarrow X$ con la siguiente propiedad universal:
	
	\begin{itemize}
		
		\item Para todo espacio topol\'ogico $Y$ y $g_j: X_j \rightarrow Y$ funciones continuas existe una \'unica $f: X \rightarrow Y$ tal que el siguiente diagrama conmuta:
		
		\begin{equation*}
		\begin{tikzcd}
		X_j \arrow[swap]{d}{f_j} \arrow[swap]{dr}{g_j} & \\ 
		X \arrow[dashed]{r}{\exists! f} & Y 
		\end{tikzcd}
		\end{equation*}
		
	\end{itemize}
	
\end{theorem}

\begin{proof}
	
	Consideremos $X = \Bigcoprod{j \in J}{X_j}$, $f_j = i_j$ y $\tau$ la topolog\'ia final respecto a las inclusiones, veamos que $(X,\tau)$ cumple la propiedad universal.
	
	Para esto definamos:
	\[
	f(x) = g_j(x) \emph{ si } x \in X_j
	\]
	
	Notemos que $f$ esta bien definida y cumple que $fi_j = g_j$ con lo que basta ver que es continua. Para esto notemos que como $X$ tiene la topolog\'ia final respecto a $\sett{i_j}_{j \in J}$ por \ref{Caracterizacion de la topologia final} $f$ es continua si y s\'olo si $fi_j$ es continua, pero $fi_j = g_j$; por lo tanto $X$ cumple la propiedad universal.
	
	Para ver unicidad supongamos que $X'$ junto con $f'_{j}$ es otro espacio topol\'ogico que cumple la propiedad universal. Como $X$ cumple la propiedad entonces existe $f : X \rightarrow X'$ tal que el siguiente diagrama conmuta:
	
	\begin{equation*}
	\begin{tikzcd}
	X_j \arrow[swap]{dr}{f'_j} \arrow[swap]{d}{f_j} & \\ 
	X \arrow[dashed]{r}{\exists! f} & X'
	\end{tikzcd}
	\end{equation*}
	
	Por otro lado como $X'$ satisface la propiedad existe $g: X' \rightarrow X$ tal que el siguiente diagrama conmuta:
	
	\begin{equation*}
	\begin{tikzcd}
	X_j \arrow[swap]{dr}{f_j} \arrow[swap]{d}{f'_j} & \\ 
	X' \arrow[dashed]{r}{\exists! g} & X
	\end{tikzcd}
	\end{equation*}
	
	Entonces tenemos que $gff_j  = gf'_j = f_j$ con lo que el siguiente diagrama conmuta:
	
	\begin{equation*}
	\begin{tikzcd}
	X_j \arrow[swap, bend left]{dr}{f_j} \arrow[swap]{d}{f_j} & \\
	X \arrow[shift right = 5]{r}{gf} \arrow[shift left=5]{r}{1_X} & X
	\end{tikzcd}
	\end{equation*}	
	
	Por unicidad de la funci\'on $h: X \rightarrow X$ tal que $hf_j = f_j$ se tiene que $gf = 1_X$. An\'alogamente se tiene que $fg = 1_X'$ por lo que $X \simeq X'$\qed
	
\end{proof}

\subsection{Topolog\'ia cociente}

\begin{definition}{Topolog\'ia cociente}
	Sea $X$ un espacio topol\'ogico y sea $\sim$ una relaci\'on de equivalencia en $X$, denotamos $[x] = \overline{x}$ a la clase de equivalencia de $x$, $\quotient{X}{\sim}$ al conjunto de clases de equivalencia y $q : X \rightarrow \quotient{X}{\sim}$ a la funci\'on cociente. Definimos la \textit{topolog\'ia cociente en } $\quotient{X}{\sim}$ a la topolog\'ia final respecto a la familia $\sett{q}$ y vamos a notarlo $(\quotient{X}{\sim}, \tau_q)$
\end{definition}

\begin{remark}
	Sea $A \subseteq X$ un subconjunto y sea $\sim_A$ dada por $x \sim_A y$ si y s\'olo si $x,y \in A$, entonces al espacio $\quotient{X}{\sim_A}$ lo vamos a notar $\quotient{X}{A}$
\end{remark}

\begin{example}
	
	Sea $X = \R$ y $A = \R \setminus \sett{0}$, entonces $\quotient{X}{A} = \sett{[0],[1]}$. Notemos que $q^{-1}(\sett{0}) = \sett{0}$ que no es abierto por lo que $\sett{[0]} \not \in \tau_q$, similarmente $q^{-1}(\sett{1}) = \R \setminus \sett{0} \in \tau_X$, por lo tanto $\quotient{X}{A} \simeq \mathfrak{S}$; donde el homeomorfismo es: 
	
	\[f = \left\lbrace
	\begin{array}{cc}
	0 & \emph{si } x=[1] \\ 
	1 & \emph{si } x=[0] 
	\end{array} 
	\right.
	\]
	
\end{example}

\begin{definition}
	Una funci\'on $f: X \rightarrow Y$ se dice \textit{cociente} si es final y sobreyectiva.
\end{definition}

\begin{remark}
	$q: X \rightarrow \quotient{X}{\sim}$ es cociente
\end{remark}

\begin{proposition}
	
	Si $f:X \rightarrow Y$ es cociente, entonces $Y \simeq \quotient{X}{\sim_f}$ donde $x \sim_f x' \ \Longleftrightarrow f(x)=f(x')$
	
\end{proposition}

\begin{proof}
	Consideremos el diagrama:
	
	\begin{equation*}
	\begin{tikzcd}
	Y \arrow[bend left, swap]{dd}{g} & \\ 
	& X \arrow{ul}{f} \arrow{dl}{q} \\
	\quotient{X}{\sim_f}\arrow[swap, bend left]{uu}{h} & 
	\end{tikzcd}
	\end{equation*}
	
	Siendo $g: Y \rightarrow \quotient{X}{\sim_f}$ dado por $g(y) = [x]$ donde $x \in f^{-1}(y)$ y $h : \quotient{X}{\sim_f}$ dada por $h([x]) = f(x)$.
	
	Notemos que $g$ esta bien definida pues $x \sim_f x'$ si y s\'olo si $f(x) = f(x')$, por lo tanto dado $y \in Y$ como $f$ es sobreyectiva existe $x \in X$ tal que $f(x) = y$ y si existe $x' \sim_f x$ entonces $g([x]) = y = g([x'])$. Adem\'as como $f$ es final resulta que $g$ es continua si y s\'olo si $gf$ es continua, pero $gf = q$ por lo que $g$ es continua.
	
	Por otro lado $h$ claramente esta bien definida y como $q$ es final tenemos que $h$ es continua si y s\'olo si $hq$ es continua, pero $hq=f$ por lo que $h$ es continua.
	
	Finalmente $hg(y) \underbrace{=}\limits_{x \in f^{-1}(y)} h([x]) = f(x) = y$ y $gh([x]) = gf(x) = [x]$ por lo que $Y \simeq \quotient{X}{\sim_f}$\qed 
	
\end{proof}

\begin{proposition}
	
	Sea $f: X \rightarrow Y$ continua y  sobreyectiva, si resulta que $f$ es abierta o cerrada; entonces $f$ resulta cociente.
	
\end{proposition}

\begin{proof}
	Dado que $f$ es continua y sobreyectiva basta ver que $f$ es final, para esto supongamos que $f$ es abierta. Si $U \subseteq Y$ es abierto entonces $f^{-1}(U) \subseteq X$ es abierto pues $f$ es continua. Por el otro lado sea $U \subseteq Y$ subespacio tal que $f^{-1}(U)$ es abierto en $X$, como $f$ es sobreyectiva y abierta resulta que $f(f^{-1}(U)) = U$ es abierto; concluimos que $f$ es final. \qed
\end{proof}

\begin{example}
	
	Existe $f$ cociente tal que no es ni abierta ni cerrada.
	
	En efecto, sea $X =  (\sett{0,1,2}, \sett{\sett{0}, \sett{0,1}, X, \emptyset})$ e $Y = \quotient{X}{\sett{0,2}} = \sett{[0], [1]}$ con la topolog\'ia cociente. Entonces $q$ es cociente pero $q^{-1}([0]) = \sett{0,2} \not \in \tau_X$ por lo que $\sett{[0]} \not \in \tau_q$, y adem\'as $q^{-1}(1) = \sett{1} \not \in \tau_X$ por lo que $\sett{[1]} \not \in \tau_q$; por lo tanto $Y = (\sett{[0],[1]} , \tau_i)$ y $q$ no es abierta pues $q(\sett{0})$ no es abierto y $q$ no es cerrada pues $q(\sett{2}) = \sett{0}$ no es cerrado.
	
\end{example}

\begin{proposition}
	
	\label{Extension de funciones al cociente}
	
	Si $q : X \rightarrow Y$ es cociente y $f: X \rightarrow Z$ es uan funci\'on continua tal que $q(x) = q(x')$ implica que $f(x) = f(x')$ entonces existe una \'unica $g : Y \rightarrow Z$ continua tal que el siguiente diagrama conmuta:
	
	\begin{equation*}
	\begin{tikzcd}
	X \arrow[bend right, swap]{dr}{f} \arrow {r} {q} & Y \arrow[swap, dashed] {d} {\exists ! g}\\ 
	& Z 
	\end{tikzcd}
	\end{equation*}
	
\end{proposition}

\begin{proof}
	
	Sea $y \in Y$, como $q$ es sobreyectiva sabemos que existe $x \in X$ tal que $q(x) = y$, definamos $g(y) = f(x)$. Entonces $g$ esta bien definida pues si $x,x' \in q^{-1}(y)$ entonces $f(x) = f(x') = g(y)$. Claramente es la \'unica que hace conmutar el diagrama y adem\'as como $q$ es final sabemos que $g$ es continua si y s\'olo si $gq$ es continua, pero $gq = f$, por lo tanto $g$ es continua y es la funci\'on buscada. \qed
	
\end{proof}

\begin{proposition}
	
	Sean $X$ un espacio topol\'ogico, $\sim$ una relaci\'on de equivalencia en $X$ y $\sim'$ una relaci\'on de equivalencia en $X/_{\sim}$ Entonces el espacio $\left(X/_{\sim} \right)/_{\sim'}$ es homeomorfo a $X/_{\sim''}$, donde $x \sim'' y$ si y s\'olo si $q(x) \sim' q(y)$, con $q: X\rightarrow X/_{\sim}$ la proyecci\'on al cociente.
	
\end{proposition}

\begin{proof}
	
	Sea $q : X \rightarrow X/ \sim$, $q_1: X \rightarrow X / \sim'$ y $q_2 : X \rightarrow X / \sim''$. Consideremos $q : X \rightarrow X/ \sim$ y veamos el diagrama:
	
	\[
	\begin{tikzcd}
	X \arrow{r}{q} \arrow[swap]{d}{q_2} & X / \sim \arrow[swap]{d}{q_1} \\
	X / \sim'' \arrow[dashed]{r}{\widetilde{q}} & X / \sim / \sim' \\ 
	\end{tikzcd}
	\]
	
	Notemos que $q_1 \circ q : X \rightarrow X / \sim / \sim'$ es continua y que si $x \sim'' y $ entonces $q(x) \sim' q(y)$ y entonces $q_1 \circ q (x) = q_1 \circ q (y)$. Por ende por la PU del cociente el diagrama de arriba conmuta. Es claro que $\widetilde{q}$ es continua y sobreyectiva e inyectiva (esto\'ultimo por la cuenta de arriba), nos bastar\'ia ver que es abierta. Sea $U_2 \subset X / \sim''$ abierto, entonces $q_2^{-1}(U_2) =  U$ es abierto. Entonces $\widetilde{q}(U_2)$ es abierto sii $q_1^{-1}(\widetilde{q} (U_2))$ es abierto sii $q^{-1}(q_1 ^{-1}(\widetilde{q}(U))) = q_1^{-1}(U_2) = U$ es abierto. Por ende $\widetilde{q}$ es abierta y entonces $\widetilde{q}$ es homeo. \qed 
	
\end{proof}

\begin{corollary}
	
	Valen los siguientes homeomorfismos:
	
	\begin{enumerate}
		
		\item El toro $T$ cumple que $T \simeq \mathbb{S}^1 \times I /_{[(z,0) \sim (z, 1)]}$
		
		\item La botella de Klein $K$ cumple que $K \simeq \mathbb{S}^1 \times I /_{[(z,0) \sim (\bar{z}, 1)]}$.
		
	\end{enumerate}
	
\end{corollary}

\begin{proposition}
	
	Sean $\{X_n\}_{n\in \mathbb{N}}$ una sucesi\'on de espacios topol\'ogicos,
	y $f_n : X_n \rightarrow X_{n-1}$ funciones continuas. Consideramos $X= \{(x_n)\in \prod X_n: f_n(x_n) = x_{n-1} ~ \forall n\in \mathbb{N}\}$,
	y $p_n : X\rightarrow X_n$ las funciones definidas por $p_n((x_k)) = x_n$. Le damos a $X$ la topolog\'ia inicial inducida por $\{p_n\}_{n\in \mathbb{N}}$. $X$ es el \textit{l\'imite inverso} o \textit{l\'imite proyectivo} de $\{X_n\}$, y se denota $X=\varprojlim X_n$. Entonces el l\'imite proyectivo cumple la siguiente propiedad universal:
	
	\begin{enumerate}
		
		\item Dados $Y$ espacio topol\'ogico y $g_n: Y \rightarrow X_n$ familia de funciones continuas tal que $f_n g_n = g_{n-1}$, existe una \'unica $g:Y\rightarrow X$ funci\'on continua tal que $p_n g = g_n$.
	\end{enumerate}
	
\end{proposition}

\begin{proof}
	
	Dados $Y$ y $\sett{g_n}$ tal que el siguiente diagrama conmuta $\forall n$:
	
	\[
	\begin{tikzcd}
	Y \arrow{r}{g_n} \arrow{dr}{g_{n-1}} & X_n \arrow[swap]{d}{f_n} \\
	& X_{n-1} \\
	\end{tikzcd}
	\]
	
	Debemos hallar una $g : Y \rightarrow X$ continua tal que el siguiente diagrama conmute:
	
	
	\[
	\begin{tikzcd}
	Y \arrow{r}{g} \arrow{dr}{g_{n}} & X \arrow[swap]{d}{p_n} \\
	& X_{n} \\
	\end{tikzcd}
	\]
	
	Definamos $g : Y \rightarrow X$ dado por $y \mapsto (g_n(y))_{n \in \N}$. Y veamos que cumple la propiedad universal:
	
	\begin{itemize}
		\item {$g$ esta bien definida}
		
		Sea $y \in Y$ entonces como $f_n(g(y)_n) = f_n(g_n(y)) =  g_{n-1}(y) = g(y)_{n-1}$  se sigue que $g(y) \in X$
		
		\item {$g$ cumple el diagrama conmutativo de arriba}
		
		Trivial por construcci\'on, pues $p_n(g(y)) = p_n( (g_k(y))_k )=g_n(y)$
		
		\item {$g$ es continua}
		
		Como las $\sett{p_n}$ son familia inicial, entonces $g$ es continua sii $p_n g = g_n$ es continua $\forall n \in \N$, pero esto vale por hip\'otesis. Por ende $g$ es continua.
		
		\item {Unicidad}
		
		Sea $h : Y \rightarrow X$ otra funci\'on que hace conmutar el diagrama, entonces $p_n(g) = g_n = p_n(h) \ \forall n \in \N$, y por ende $p_n(g-h) = 0 \ \forall n$ y como las $p_n$ son iniciales entonces $g = h$.\qed
		
	\end{itemize}
	
\end{proof}

\begin{corollary}
	
	Sea $f_n : \mathbb{R}^n \rightarrow \mathbb{R}^{n-1}$ la proyecci\'on a las primeras $n-1$ coordenadas, entonces $\varprojlim \mathbb{R}^n$ es homeomorfo a $\mathbb{R}^\mathbb{\omega}$.
	
\end{corollary}

\begin{proof}
	
	Veamos que $\R^{\omega}$ cumple la propiedad universal. Sean $Y$ y $g_n : Y \rightarrow \R^n$ tal que:
	
	\[
	\begin{tikzcd}
	Y \arrow{r}{g_n} \arrow{dr}{g_{n-1}} & \R^n \arrow[swap]{d}{f_n} \\
	& \R^{n-1}
	\end{tikzcd}
	\]
	
	Y sea $g : Y \rightarrow \R^{\omega}$ dada por $y \mapsto ((g_n(y))_n)_{n \in \N}$ O sea en el lugar n-\'esimo tenemos a la coordenada n-\'esima de $g_n$. Entonces $g$ cumple y por ende $\R^{\omega} \simeq \varprojlim \R^n$ \qed
	
	
\end{proof}

\section{Producto fibrado}

\begin{definition}
	
	\label{PU del pullback}
	
	Sean $X,Y,Z$ espacios topol\'ogicos y $f: X \rightarrow Z$, $g: Y\rightarrow Z$ funciones continuas, entonces el \textit{producto fibrado o pullback} del diagrama:
	
	\begin{equation*}
	\begin{tikzcd}
	& X \arrow[swap] {d} {f}\\ 
	Y \arrow {r} {g} & Z 
	\end{tikzcd}
	\end{equation*}	
	
	es un espacio topol\'ogico $P$ junto con funciones continuas $\tilde{g} : P \rightarrow X$ y $\tilde{f} : P \rightarrow Y $ tal que el siguiente diagrama conmute:
	
	\begin{equation*}
	\begin{tikzcd}
	P \arrow[swap] {d}{\tilde{f}} \arrow {r} {\tilde{g}} & X \arrow[swap] {d} {f}\\ 
	Y \arrow {r} {g} & Z 
	\end{tikzcd}
	\end{equation*}	
	
	Y cumpla la siguiente propiedad universal:
	
	\begin{itemize}
		
		\item Si $W$ es un espacio topol\'ogico junto con funciones continuas $s: W \rightarrow X$ y $t : W \rightarrow Y$ tal que $fs = gt$, entonces existe una \'unica $h : W \rightarrow P$ continua tal que el siguiente diadrama conmuta:
	
		\begin{equation*}
		\begin{tikzcd}
		W \arrow[dashed] {dr} {\exists ! h} \arrow[swap, bend right] {ddr} {t} \arrow[bend left] {drr} {s} & & \\
		& P \arrow[swap] {d}{\tilde{f}} \arrow {r} {\tilde{g}} & X \arrow[swap] {d} {f} \\ 
		& Y \arrow {r} {g} & Z 
		\end{tikzcd}
		\end{equation*}	
		 
				
	\end{itemize}
	
\end{definition}

\begin{example}
	
	Sea el diagrama:
	
	\begin{equation}
	\label{Diagrama pullback 1}
	\begin{tikzcd}
	& X \arrow[swap] {d} {1_x}\\ 
	Y \arrow {r} {g} & X 
	\end{tikzcd}
	\end{equation}		
	
	entonces el pullback de \ref{Diagrama pullback 1} resulta $Y$ junto con $\tilde{g} = g$ y $\tilde{f} = 1_Y$. En efecto, el tr\'io $(Y,1_Y,g)$ hace conmutar el diagrama \ref{Diagrama pullback 1}, por otro lado sea el diagrama:
	
		\begin{equation*}
		\begin{tikzcd}
		W \arrow[dashed] {dr} {\exists ! t} \arrow[swap, bend right] {ddr} {t} \arrow[bend left] {drr} {s} & & \\
		& P \arrow[swap] {d}{1_Y} \arrow {r} {g} & X \arrow[swap] {d} {1_X} \\ 
		& Y \arrow {r} {g} & Z 
		\end{tikzcd}
		\end{equation*}
		
	Entonces sea $h : W \rightarrow Y$ dado por $h(w) = t(w)$, entonces $1_Y t = t$ y $gt = 1_X s = s$, por lo tanto $Y$ cumple la propiedad universal y resulta el pullback buscado.
	
\end{example}

\begin{theorem}
	
	\label{Caracterizacion del pullback}
	
	Dado un diagrama de espacios topol\'ogicos y funciones continuas:

	\begin{equation*}
	\begin{tikzcd}
	& X \arrow[swap] {d} {f}\\ 
	Y \arrow {r} {g} & Z 
	\end{tikzcd}
	\end{equation*}	
	
	Existe un producto fibrado $(P,\tilde{f}, \tilde{g})$. Adem\'as si $(P', \tilde{f'}, \tilde{g'})$ es otro producto fibrado, existe $h: P' \rightarrow P$ homeomorfismo tal que $\tilde{g} h = \tilde{g'}$ y $\tilde{f} h = \tilde{f'}$
	
\end{theorem}

\begin{proof}
	
	Sea $P = \sett{(x,y) \in X \times Y \ / \ f(x) = g(y)}$ visto como subespacio de $X \times Y$ y sean $\tilde{f} = p_Y\vert_{P}$, $\tilde{g} = p_X\vert_{P}$. Notemos que $\tilde{f} , \tilde{g}$ son continuas pues $P$ es subespacio y \ref{Restriccion de continua a subespacio es continua}.
	
	Notemos adem\'as que $f \tilde{g} (x,y) = f(x) = g(y) = g \tilde{f}(x,y)$ por lo que hacen conmutar el diagrama.
	
	Sea ahora $(W,s,t)$ tal que el siguiente diagrama conmuta:
	
		\begin{equation}
		\label{Construccion producto fibrado}
		\begin{tikzcd}
		W \arrow[dashed] {dr} {(s,t)} \arrow[swap, bend right] {ddr} {t} \arrow[bend left] {drr} {s} & & \\
		& P \arrow[swap] {d}{\tilde{f}} \arrow {r} {\tilde{g}} & X \arrow[swap] {d} {f} \\ 
		& Y \arrow {r} {g} & Z 
		\end{tikzcd}
		\end{equation}	
		
	Si $h : W \rightarrow P$ es tal que $h_{2}(w) = \tilde{f}h(w) = t(w)$ y $ h_{1}(w) = \tilde{g} h (w) = s (w)$, por lo que $h = (s,w)$ hace conmutar el diagrama. 
	
	Sea ahora $w \in W$, entonces $(x,y) = h(w) = (s(w) , t(w))$ por lo tanto $f(x) = f(s(w)) = g(t(w)) = g(y)$; de aqu\'i se deduce que $h(W) \subseteq P$ y $h$ est\'a bien definida. 
	
	Para ver que $h$ es continua por \ref{Composicion de iniciales es inicial, y si la composicion es inicial entonces la interna lo es} como $\sett{i : P \rightarrow X \times Y}$ es inicial y $\sett{p_X : X \times Y \rightarrow X , p_Y : X \times Y \rightarrow Y}$ es inicial, obtenemos que $\sett{p_X i , p_Y i} = \sett{\tilde{f} , \tilde{g}}$ es inicial. Por lo tanto, $h$ es continua si y s\'olo si $\tilde{g}h = s$ y $\tilde{f}h = t$ son continuas; de esto se concluye que $(P , \tilde{f} , \tilde{g})$ es el producto fibrado buscado.
	
	Supongamos que $(P',\tilde{g'}, \tilde{f'})$ sea otro producto fibrado que satisface la propiedad universal, entonces por la PU de $P$ se tiene que existe una \'unica $h: P' \rightarrow P$ tal que $\tilde{g}h = \tilde{g'}$ y $\tilde{f}h = \tilde{f'}$, es decir el siguiente diagrama conmuta:
	
		\begin{equation*}
		\begin{tikzcd}
		P' \arrow[dashed] {dr} {\exists ! h} \arrow[swap, bend right] {ddr} {\tilde{f'}} \arrow[bend left] {drr} {\tilde{g'}} & & \\
		& P \arrow[swap] {d}{\tilde{f}} \arrow {r} {\tilde{g}} & X \arrow[swap] {d} {f} \\ 
		& Y \arrow {r} {g} & Z 
		\end{tikzcd}
		\end{equation*}		
	
	Por otro lado por la PU de $P'$ existe una \'unica $k: P \rightarrow P'$ tal que $\tilde{g'}k = \tilde{g}$ y $\tilde{f'}k = \tilde{f}$, es decir el siguiente diagrama conmuta:

		\begin{equation*}
		\begin{tikzcd}
		P \arrow[dashed] {dr} {\exists ! k} \arrow[swap, bend right] {ddr} {\tilde{f}} \arrow[bend left] {drr} {\tilde{g}} & & \\
		& P' \arrow[swap] {d}{\tilde{f'}} \arrow {r} {\tilde{g'}} & X \arrow[swap] {d} {f} \\ 
		& Y \arrow {r} {g} & Z 
		\end{tikzcd}
		\end{equation*}		
	
	Finalmente entonces por la PU de $P$ al siguiente diagrama:
	
		\begin{equation*}
		\begin{tikzcd}
		P \arrow[dashed, bend right] {dr} {hk} \arrow[dashed, bend left] {dr} {1_P} \arrow[swap, bend right] {ddr} {\tilde{f}} \arrow[bend left] {drr} {\tilde{g}} & & \\
		& P \arrow[swap] {d}{\tilde{f}} \arrow {r} {\tilde{g}} & X \arrow[swap] {d} {f} \\ 
		& Y \arrow {r} {g} & Z 
		\end{tikzcd}
		\end{equation*}	
	
	Sabemos por un lado que $1_P$ cumple que $\tilde{f} 1_P = \tilde{f}$ y que $\tilde{g} 1_P = \tilde{g}$; pero por el otro lado $\tilde{f} hk = \tilde{f'}k = \tilde{f}$ y $\tilde{g} hk = \tilde{g'}k = \tilde{g}$, por lo tanto $hk = 1_P$ por la unicidad de la funci\'on. An\'alogamente $kh = 1_{P'}$ y conclu\'imos que $P \simeq P'$. \qed
	
\end{proof}

\begin{example}
	
	\label{Pullback contra el singleton}
	
	El pullback del siguiente diagrama:
	
	\begin{equation*}
	\begin{tikzcd}
	& X \arrow[swap] {d} {f}\\ 
	Y \arrow {r} {g} & * 
	\end{tikzcd}
	\end{equation*}	
	
	es $(X \times Y, p_X p_Y)$ por \ref{Caracterizacion del pullback}.
	
	
\end{example}

\begin{remark}
	
	\label{Las inyectivas son estables por cambio de base}
	
	Si $f : X \rightarrow Z$ es inyectiva, entonces $\tilde{f} : P \rightarrow Y$ es inyectiva.
	
\end{remark}

\begin{proof}
	
	En efecto, supongamos que $\tilde{f}(x,y) = \tilde{f}(x',y')$, entonces por la definici\'on en \ref{Caracterizacion del pullback} de $\tilde{f}$ se tiene que $y = \tilde{f}(x,y) = \tilde{f}(x',y') = y'$
	
	Por otro lado, como $(x,y) \in P$ se tiene que $f(x) = g(y) = g(y') = f)(x')$ y como $f$ es inyectiva tenemos que $x = x'$. \qed
	
\end{proof}

\begin{definition}
	
	Decimos que una clase de funciones continuas $\mathcal{A}$ es \textit{estable por cambio de base} si cada vez que se tiene un pullback:

	\begin{equation*}
	\begin{tikzcd}
	P \arrow[swap] {d} {\tilde{f}} \arrow {r} {\tilde{g}}  & X \arrow[swap] {d} {f}\\ 
	Y \arrow {r} {g} & Z 
	\end{tikzcd}
	\end{equation*}		
	
	donde $f \in \mathcal{A}$ vale que $\tilde{f} \in \mathcal{A}$.
	
\end{definition}

\begin{remark}
	Acabamos de ver que las funciones inyectivas son estables por cambio de base.
\end{remark}

\begin{proposition}
	\label{Los homeomorfismos son estables por cambio de base}
	
	Los homeomorfismos son estables por cambio de base
	
\end{proposition}

\begin{proof}
	
	Sea un pullback:
	
	\begin{equation*}
	\begin{tikzcd}
	P \arrow[swap] {d} {\tilde{f}} \arrow {r} {\tilde{g}}  & X \arrow[swap] {d} {f}\\ 
	Y \arrow {r} {g} & Z 
	\end{tikzcd}
	\end{equation*}	
	
	donde $f$ es homeomorfismo. Como $ff^{-1}g = g$ por la PU de $P$ sabemos que existe una \'unica $h : Y \rightarrow P$ tal que el siguiente diagrama conmuta:
	
		\begin{equation*}
		\begin{tikzcd}
		Y \arrow[dashed] {dr} {\exists ! h}  \arrow[swap, bend right] {ddr} {1_Y} \arrow[bend left] {drr} {f^{-1}g} & & \\
		& P \arrow[swap] {d}{\tilde{f}} \arrow {r} {\tilde{g}} & X \arrow[swap] {d} {f} \\ 
		& Y \arrow {r} {g} & Z 
		\end{tikzcd}
		\end{equation*}	
		
	Por lo que se obtiene que $\tilde{f}h = 1_Y$. Por otro lado por la PU de $P$ existe una \'unica $r: P \rightarrow P $ tal que el siguiente diagrama conmuta:
	
		\begin{equation*}
		\begin{tikzcd}
		P \arrow[dashed, bend right] {dr} {h\tilde{f}} \arrow[dashed, bend left] {dr} {1_P} \arrow[swap, bend right] {ddr} {\tilde{f}} \arrow[bend left] {drr} {\tilde{g}} & & \\
		& P \arrow[swap] {d}{\tilde{f}} \arrow {r} {\tilde{g}} & X \arrow[swap] {d} {f} \\ 
		& Y \arrow {r} {g} & Z 
		\end{tikzcd}
		\end{equation*}		
	
	
	Como $r = 1_P$ cumple y por $\tilde{f}h \tilde{f} = 1_Y \tilde{f} = \tilde{f}$ y $\tilde{g}h \tilde{f} = f^{-1}g \tilde{f} = f^{-1}f \tilde{g} = \tilde{g}$ tambi\'en $r = h\tilde{f}$ cumple, se sigue que $h\tilde{f} = 1_P$; conclu\'imos que $\tilde{f}$ es homeomorfismo \qed
	
	
\end{proof}

\begin{example}
	Las funciones cerradas no son estables por cambio de base.
	
	En efecto, si consideramos $X = Y = \R$, $Z = *$ entonces por \ref{Pullback contra el singleton} sabemos que el pullback resulta:
	
	\begin{equation*}
	\begin{tikzcd}
	\R \times \R \arrow[swap] {d} {p_2} \arrow {r} {p_1}  & \R \arrow[swap] {d} {f}\\ 
	\R \arrow {r} {g} & * 
	\end{tikzcd}
	\end{equation*}		 
	
	y trivialmente $f$ es cerrada. No obstante, si consideramos $F^{P} = \sett{(x,y) \ / \ xy = 1}$ entonces $\tilde{f}(F^P) = \R_{>0}$ que no es cerrado.
	
\end{example}

\begin{proposition}
	
	Sea $A \subseteq X$ subespacio y $f: Y \rightarrow X$ continua, entonces el siguiente es un pullback:
	
	\begin{equation}
	\label{Pullback de subespacio}
	\begin{tikzcd}
	f^{-1}(A) \arrow[swap, hook] {d} {i_{f^{-1}(A)}} \arrow {r} {f|_{f^{-1}(A)}}  & A \arrow[swap, hook] {d} {i_A} \\ 
	Y \arrow {r} {f} & X 
	\end{tikzcd}
	\end{equation}		
	
\end{proposition}

\begin{proof}
	
	Vamos que cumple la propiedad universal, sea $(W,s,t)$ tal que $i_As = ft$, entonces si $h: W \rightarrow f^{-1}(A)$ es tal que $i_{f^{-1}(A)}h = t$ entonces nesariamente debemos definir $h(w) = t(w)$.
	
	Sea $w \in W$, entonces $f(t(w)) = i_A(s(W)) = s(w) \in A$ por lo que $h(w) = t(w) \in f^{-1}(A)$ y concluimos que $h$ est\'a bien definida. dem\'as por la misma cuenta es claro que es la \'unica que  hace conmutar el diagrama; basta ver que es continua pero esto es claro pues $h = t$ que ya era continua. Por lo tanto conclu\'imos que \ref{Pullback de subespacio} era un pullback. \qed
	
\end{proof}

\begin{corollary}
	
	\label{Funciones subespacio son estables por cambio de base}
	
	Las funciones subespacio son estables por cambio de base
	
\end{corollary}

\begin{proposition}
	
	\label{Pullback de pullback es pullback}
	
	Dado un diagrama conmutativo de funciones continuas:
	
	
	\begin{equation*}
	\begin{tikzcd}
	W \Center{\circled{1}}{dr} \arrow {r}{g_2} \arrow[swap] {d}{f_3} & X \Center{\circled{2}}{dr} \arrow[swap] {d} {f_2} \arrow {r} {g_1}  & R \arrow[swap] {d} {f_1}\\ 
	Y \arrow {r} {h_2} & Z \arrow {r} {h_1} & T
	\end{tikzcd}
	\end{equation*}		
	
	Entonces valen:
	
	\begin{enumerate}
		\item Si $\circled{1}, \circled{2}$ son pullback, entonces el diagrama entero es pullback.
		\item Si $\circled{2}$ y el diagrama entero es pullback, entonces $\circled{1}$ es pullback.
	\end{enumerate}
	
\end{proposition}

\begin{proof}
	
	\begin{enumerate}
		
		\item Para ordenarnos, sabemos que $(X,f_2,g_1)$ es el pullback de $\circled{2}$, por lo tanto si notamos $f_1 = f, h_1 = h$ sabemos que $f_2 = \tilde{f}, g_1 = \tilde{h}$.
		
		Asimismo como $(W,f_3,g_2)$ es pullback de $\circled{1}$ entonces sabemos que $f_3 = \tilde{\tilde{f}}, h_2 = g,  g_2 = \tilde{g}$ donde notamos $h_2$ como $g$. Finalmente vamos a notar a $X = Z \times_T R$ y entonces a $W = Y \times_Z (Z \times_T R)$, por lo tanto tenemos:
		
		\begin{equation}
		\begin{tikzcd}
		Y \times_Z (Z \times_T R) \Center{\circled{1}}{dr} \arrow {r}{\tilde{g}} \arrow[swap] {d}{\tilde{\tilde{f}}} & Z \times_T R \Center{\circled{2}}{dr} \arrow[swap] {d} {\tilde{f}} \arrow {r} {\tilde{h}}  & R \arrow[swap] {d} {f}\\ 
		Y \arrow {r} {g} & Z \arrow {r} {h} & T
		\end{tikzcd}
		\end{equation}
		
		Y lo que queremos ver es que $Y \times_Z (Z \times_T R) = Y \times_T R$, para ver eso debemos ver que si tenemos un tr\'io $(P,s,t)$ en el siguiente diagrama:
		
		\begin{equation}
		\label{Diagrama pullback 2}
		\begin{tikzcd}
		P \arrow[dashed] {dr} { \exists ! r}  \arrow[bend left] {drrr}{s} \arrow[swap, bend right] {ddr} {t} & & . & \\
		 & Y \times_Z (Z \times_T R) \Center{\circled{4}}{ur} \Center{\circled{3}}{dl} \Center{\circled{1}}{dr} \arrow {r}{\tilde{g}} \arrow[swap] {d}{\tilde{\tilde{f}}} & Z \times_T R \Center{\circled{2}}{dr} \arrow[swap] {d} {\tilde{f}} \arrow {r} {\tilde{h}}  & R \arrow[swap] {d} {f}\\ 
		. & Y \arrow {r} {g} & Z \arrow {r} {h} & T
		\end{tikzcd}
		\end{equation}
		
		Entonces podemos encontrar una \'unica $r: P \rightarrow Y \times_Z (Z \times_T R)$ tal que el diagrama conmute. Si $\tilde{\tilde{f}}r (p) = t (p)$ entonces como $\tilde{\tilde{f}}r (p) = \tilde{\tilde{f}}(r_1(p),r_2(p),r_3(p) )= r_1(p)$ se tiene que $r_1 = t$. 
		
		Similarmente si $\tilde{h} \tilde{g} r = s$ entonces como $\tilde{h} \tilde{g} r (p) = \tilde{h} \tilde{g} (r_1,r_2,r_3)(p) = r_3(p)$ se concluye que $r_3 = s$.
		
		Finalmente como $\tilde{f} \tilde{g} r = gt$ conclu\'imos que $r_2 = gt$. Definimos entonces $r : P \rightarrow Y \times_Z (Z \times_T R)$ como $r = (t , gt ,s)$. Veamos que esta bien definida, que hace conmutar el diagrama, que es la \'unica y que es continua, ve\'amoslo por partes:
		
		\begin{enumerate}
			
			\item Sea $p \in P$, entonces $g(r_1(p)) = g(t(p)) \underbrace{=}\limits_{\emph{1 y 3}} \tilde{f}(\tilde{g}(r(p))) = \tilde{f}((r_2(p), r_3(p)) $, por lo tanto tenemos que $r(p) \in Y \times_Z (Z \times_T R)$.
			
			\item Es claro por la construcci\'on que $r$ hace conmutar $\circled{3}, \circled{4}$, y $\circled{1}, \circled{2}$ ya conmutaban, por lo que \ref{Diagrama pullback 2} conmuta y adem\'as $r$ es la \'unica que hace esto.
			
			\item De la demostraci\'on de \ref{Caracterizacion del pullback} sabemos que $\sett{\tilde{\tilde{f}}, \tilde{g}}$ es inicial y similarmente $\sett{\tilde{h}, \tilde{f}}$ es inicial. Por \ref{Composicion de iniciales es inicial, y si la composicion es inicial entonces la interna lo es} entonces $\sett{\tilde{\tilde{f}}, \tilde{f}\tilde{g}, \tilde{h}\tilde{g}}$ es inicial para $Y \times_Z (Z \times_T R)$. Por lo tanto $r$ es continua si y s\'olo si $\tilde{\tilde{f}}r, \tilde{f}\tilde{g}r, \tilde{h}\tilde{g}r$ son continuas, pero estan justamente son $t,gt,s$ que son continuas por hip\'otesis. Por lo tanto $r$ es continua.
			
		\end{enumerate}
		
		Entonces dado $(P,s,t)$ un tr\'io que hac\'ia conmutar a \ref{Diagrama pullback 2} constru\'imos una $r : P \rightarrow Y \times_Z (Z \times_T R)$ \'unica y continua tal que todo conmuta, entonces $Y \times_Z (Z \times_T R)$ cumple la propiedad universal del pullback y conclu\'imos que $Y \times_Z (Z \times_T R) \simeq Y \times_T R$ \qed
		
		\item Siguiendo la notaci\'on anterior tenemos el siguiente diagrama:
		
		\begin{equation}
		\label{Diagrama pullback 3}
		\begin{tikzcd}
		Y \times_T R \Center{\circled{1}}{dr} \arrow {r}{a} \arrow[swap] {d}{b} & Z \times_T R \Center{\circled{2}}{dr} \arrow[swap] {d} {\tilde{f}} \arrow {r} {\tilde{h}}  & R \arrow[swap] {d} {f}\\ 
		Y \arrow {r} {g} & Z \arrow {r} {h} & T
		\end{tikzcd}
		\end{equation}
		
		Donde $b$ (que no notaremos $\tilde{\tilde{f}}$ para no confundirnos con que es el pullback de $\tilde{f}$) cumple que $hgb = f\tilde{h}a$, y $a$ cumple eso mismo y no notaremos $\tilde{g}$ para no confundirnos con lo que queremos probar. Con esto queremos ver que $Y \times_T R \simeq Y \times_Z (Z \times_T R)$.
		
		Sea entonces $(P,s,t)$ un tr\'io tal que hace conmutar al siguiente diagrama:

		\begin{equation}
		\label{Diagrama pullback 4}
		\begin{tikzcd}
		P \arrow[dashed] {dr} { \exists ! r}  \arrow[bend left] {drr}{s} \arrow[swap, bend right] {ddr} {t} & & . & \\
		& Y  \times_T R \Center{\circled{4}}{ur} \Center{\circled{3}}{dl} \Center{\circled{1}}{dr} \arrow {r}{a} \arrow[swap] {d}{b} & Z \times_T R \Center{\circled{2}}{dr} \arrow[swap] {d} {\tilde{f}} \arrow {r} {\tilde{h}}  & R \arrow[swap] {d} {f}\\ 
		. & Y \arrow {r} {g} & Z \arrow {r} {h} & T
		\end{tikzcd}
		\end{equation}		
		
		Y queremos ver que existe una \'unica $r : P \rightarrow Y \times_T R$ continua tal que $br = t$, que $ar = s$ y que \ref{Diagrama pullback 4} conmute.
		
		Para ello, sea $p \in P$ entonces $br(p) = r_1(p)$ por lo que definimos $r_1 = t$. Por el otro lado como $\circled{1} + \circled{2}$ es pullback si consideramos $r_2 = \tilde{h}ar = \tilde{h}s = s_2$. Por lo tanto definimos $r : P \rightarrow Y \times_T R$ como $r = (t,s_2)$ y veamos todos los items anteriores:
		
		\begin{enumerate}
			
			\item Sea $p \in P$, entonces $g(r_1(p)) =  gt(p) = \tilde{f}s(p) = s_1$ y no me sale conciliar que $\tilde{f}$ recibe dos coordenadas...
			 
		\end{enumerate}		
		
	\end{enumerate}
	
\end{proof}

\section{Conexi\'on y arcoconexi\'on}

\begin{definition}
	
	Un espacio topol''ogico $X$ se dice \textit{disconexo} si existen $U,V$ abiertos disjuntos no vac\'ios tal que $X = U \cup V$, en este caso $\sett{U,V}$ se dice una desconexi\'on de $X$. Un espacio topol\'ogico $X$ se dice \textit{ conexo} si no es disconexo.
	
\end{definition}

\begin{proposition}
	
	\label{Conexion es invariante topologico}
	
	Si $X$ es conexo y $f : X \rightarrow Y$ es un homeomorfismo, entonces $Y$ es conexo.
	
\end{proposition}

\begin{proof}
	
	En efecto, sea $U,V$ una disconexi\'on de $Y$, entonces $f^{-1}(U), f^{-1}(V)$ es una disconexi\'on de $X$ pues $f$ es homeo; por lo tanto $U = f(f^{-1}(U)) = f(\emptyset) = \emptyset $, concluimos que $Y$ es conexo. \qed
	
\end{proof}

\begin{proposition}
	
	\label{La conexion se tira para adelante por continuas}
	
	Si $f: X \rightarrow Y$ es continua y $X$ es conexo entonces $f(X)$ es conexo
\end{proposition}

\begin{proof}
	
	Consideremos $\tilde{f} : X \rightarrow f(X)$ que es continua pues $i : f(X) \rightarrow Y$ es inicial y $i\tilde{f} = f$ que es continua. Si $\sett{U,V}$ es una disconexi\'on de $f(X)$ entonces $\sett{\tilde{f}^{-1}(U), \tilde{f}^{-1}(V)}$ cumplen que son abiertos pues $\tilde{f}$ es continua y son no vac\'ios pues $\tilde{f}$ es sobreyectiva; por lo tanto alguno es vac\'io y entonces $f(X)$ es conexo. \qed
	
\end{proof}

\begin{remark}
	
	$X$ es conexo sii los \'unicos subespacios abiertos y cerrados son $\emptyset$ y $X$.
	
\end{remark}

Con esta observaci\'on ya podemos contruir varios ejemplos de espacios conexos.

\begin{example}
	
	\begin{itemize}
		
		\item $\emptyset$ es conexo
		
		\item Si $X$ es indiscreto entonces es conexo
		
		\item Si $X$ es discreto entonces es conexo si y s\'olo si $|X| = 1$
		
		\item $\mathfrak{S} $ es conexo
		
		\item $\R$ con la topolog\'ia usual es conexo (Avanzado)
		
	\end{itemize}
	
\end{example}

\begin{remark}
	
	Sea $X$ un espacio topol\'ogico y $A \subseteq X$ un subespacio disconexo. No necesariamente existen $U,V \in \tau_X$ disjuntos tales que $A \subseteq U \cup V$ y $A \cap U \neq \emptyset, A \cap V \neq \emptyset$ (En espacios m\'etricos si vale)
	
	Consideremos $X = (\sett{0,1,2}, \sett{\emptyset, X, \sett{0}, \sett{0,1}, \sett{0,2}})$ y $A = \sett{1,2}$, entonces $A = (A , \tau_d)$ y tiene m''as de un elemento por lo que es disconexo. No obstante, si $U,V \in \tau_X$ son abiertos entonces $x \in U \cap V$.
	
\end{remark}

\begin{proposition}
	
	\label{Si A es denso en B y A conexo entonces B conexo}
	
	Sea $X$ un espacio topol\'ogico y $A,B \subseteq X$ subespacios tal que $A \subseteq B \subseteq \overline{A}$; si $A$ es conexo entonces $B$ es conexo.
	
\end{proposition}

\begin{proof}
	
	Supongamos que $\sett{U,V}$ es una disconexi\'on de $B$, entonces por \ref{Subespacio de subespacio es subespacio} sabemos que $A$ es subespacio de $B$, luego $A \cap U = i^{-1}(U) , A \cap V = i^{-1}(V)$ son abiertos de $A$, disjuntos y cubren $A$; por lo tanto alguno, supongamos sin p\'erdida de generalidad $A \cap V$, es vac\'io por lo que $A \subseteq U$. Como $U$ es cerrado en $B$ tenemos que existe $F \subseteq X$ cerrado tal que $U = F \cap B \subseteq F$, por lo tanto $B \subseteq \overline{A} \subseteq F$ y se obtiene que $B =  B \cap F = U$; conclu\'imos que $V = \emptyset$. \qed
	
\end{proof}

\begin{definition}
	Un espacio topol\'ogico $X$ se dice \textit{totalmente disconexo} si para todos $x \neq y \in X$ existe una desconexi\'on $\sett{U,V}$ de $X$  tal que $x \in U, y \in V$ 
\end{definition}

\begin{example}
	
	Si $X$ es discreto, entonces es totalmente disconexo pero la vuelta no vale. Por ejemplo, $Q$ no es discreto pero es totalmente disconexo.
	
\end{example}

\begin{corollary}
	
	\label{Una funcion de un conexo en un totalmente disconexo es constante}|
	
	Si $X$ es conexo, $Y$ totalmente disconexo y $f: X \rightarrow Y$ es continua, entonces $f$ es constante.
	
\end{corollary}

\begin{proof}
	
	Si $f$ no es constante existen $y \neq y' \in f(X)$, como $Y$ es totalmente disconexo existe una desconexi\'on $\sett{U,V}$ tal que $y \in U, y' \in V$. Entonces $\sett{f^{-1}(U), f^{-1}(V)}$ es una desconexi\'on de $X$. \qed
	
\end{proof}

\begin{proposition}
	
	\label{Union de conexos con interseccion no vacia es conexo}
	
	Sea $\sett{X_j}_{j \in J}$ una familia de subespacios de un espacio topol\'ogico $X$. Si $X_j$ resulta conexo para todo $j \in J$ y adem\'as $\Bigcap{j \in J}{X_j} \neq \emptyset$, entonces $\Bigcup{j \in J}{Xj} \subseteq X$ es conexo.
	
\end{proposition}

\begin{proof}
	
	Sea $x_0 \in \Bigcap{j \in J}{X}$ y sea $\sett{U,V}$ una desconexi\'on de $\Bigcup{j \in J}{X_j}$, sin p\'erdida de generalidad podemos suponer que $x_0 \in U$. Como $X_j$ es subespacio de $\Bigcup{j \in J}{X_j}$ entonces $U \cap X_j, V \cap X_j$ son una desconexi\'on de $	X_j$. Como $X_j$ es conexo tenemos que $V \cap X_j = \emptyset$ pues $x_0 \in U \cap X_j$; por lo tanto $V = \Bigcup{j \in J}{V \cap X_j} = \emptyset$ y entonces $\Bigcup{j \in J}{X_j}$ es conexo. \qed
	
\end{proof}

\begin{definition}
	
	Un espacio topol\'ogico $X$ se dice \textit{arcoconexo} si para todos $x,x' \in X$ se tiene que existe $\gamma : I \rightarrow X$ tal que $\gamma{0} = x, \gamma(1) = x'$. A $\gamma$ lo llamaremos un \textit{camino o arco de } $x \emph{ a } x'$ y lo notaremos $x \xrightarrow{\gamma} x'$.
	
\end{definition}

\begin{proposition}
	
	\label{Arcoconexo implica conexo}
	
	Si $X$ es arcoconexo entonces es conexo
	
\end{proposition}

\begin{proof}
	
	Sea $x_0 \in X$, entonces para todo $x \in X$ existe $x_0 \xrightarrow{\gamma_x} x$. Como $I$ es conexo se tiene que $\gamma_x(I)$ es conexo para todo $x \in X$, como $x_0 \in \Bigcap{x \in X}{\gamma_x(I)}$ entonces por \ref{Union de conexos con interseccion no vacia es conexo} se tiene que $X = \Bigcup{x \in X}{\gamma_x(I)}$ es conexo \qed
	
\end{proof}

\begin{example}{El peine}
	
	En $I \times I$ consideramos el subespacio $P = (I \times \sett{0}) \cup \left( \Bigcup{n \in \N}{\left(\sett{\frac{1}{n}} \times I\right)} \right) \cup \sett{(0,1)}$, entonces afirmo que $P$ es conexo pero no arcoconexo.
	
	En efecto, si consideramos $A = (I \times \sett{0}) \cup \left( \Bigcup{n \in \N}{\left(\sett{\frac{1}{n}} \times I\right)} \right)$ entonces $A$ es arcoconexo y por \ref{Arcoconexo implica conexo} conexo, adem\'as se tiene que $\overline{A} = A \cup (I \times \sett{0})$ por lo que $A \subset P \subset \overline{A}$ con $A$ conexo, por \ref{Si A es denso en B y A conexo entonces B conexo} tenemos que $P$ es conexo.
	
	Sin embargo, sea $p=(0,1) \neq x \in P$ y supongamos que $x_0 \xrightarrow{\gamma} x$. Sea $V = \gamma^{-1}(p) \subseteq I$ que es cerrado por continuidad de $\gamma$ y no vac\'io, veamos que $V$ es abierto y entonces $V = I$ por la conexi\'on de $I$. 
	
	Para eso sea $t_0 \in V$, por la continuidad de $\gamma$ existe un $\delta > 0 $ tal que si $\norm{t-t_0} < \delta$ entonces $\norm{\gamma(t) - p} < \frac{1}{2}$ y por lo tanto si $\norm{t-t_0} < \delta$ se tiene que $\gamma(t)$ no toca el eje-x. Llamemos $J = I \cap \sett{t \in I \ / \ \norm{t - t_0} < \delta}$ que es conexo y sea $f : J \rightarrow \R$ dado por $f(t) = p_X \gamma$, como $f(J) \subseteq \sett{0} \cup \sett{\frac{1}{n} \ / \ n \in \N}$ que es totalmente disconexo, por \ref{Una funcion de un conexo en un totalmente disconexo es constante} entonces $f = cte$ y como $f(t_0) = 0$ tenemos que $f(J) = 0$. Por lo tanto $J \subset V$ y $V$ es abierto, conclu\'imos que el \'unico camino continuo desde $p \in P$ es el constante y por lo tanto $P$ no es arco conexo \qed
	
\end{example}

\begin{definition}
	
	 Sea $X$ un espacio topol\'ogico arbitrario y se define $\sim$ en $X$ dado por $x \sim y$ si y s\'olo si existe un subespacio conexo $C$ tal que $x,y \in C$.
	
\end{definition}

\begin{remark}
	Esta relaci\'on es de equivalencia
\end{remark}

\begin{definition}
	
	A las clases de equivalencia de $\sim$ se llaman \textit{componentes conexas} y decimos que la clase de un $x \in X$ es la \textit{componente conexa de $x$}.
	
\end{definition}

\begin{remark}
	
	Las componentes conexas son conexas, es m\'as, es el subespacio conexo m\'as grande que contiene a $x$.
	
	En efecto, sea $y \in C_x$ donde $C_x = [x] \ , \ x \in X$, entonces como $y \sim x$ se tiene que $x,y \in C_y$ por lo que $C_x \subseteq \Bigcup{y \sim x}{C_y}$y $x \in \Bigcap{y \sim x}{C_y}$, entonces por \ref{Union de conexos con interseccion no vacia es conexo} tenemos que $C_x$ es conexo. Trivialmente se da la segunda condici\'on.\qed
	
\end{remark}

\begin{proposition}
	
	Dado un espacio topol\'ogico $X$ se define la relaci\'on $x \sim y$ si y s\'olo si existe $x \xrightarrow{\gamma} y$ y esta relaci\'on es de equivalencia
	
\end{proposition}

\begin{proof}
	
	En efecto, es reflexiva pues $x \xrightarrow{C_{x}} x$ y  es sim\'etrica  pues si $x \xrightarrow{\gamma} y$ entonces definiendo $\overline{\gamma} := \gamma(1-t)$ tenemos que $y \xrightarrow{\overline{\gamma}}x$.
	
	Para la transitividad si $x \xrightarrow{\gamma} y$ y $y \xrightarrow{\beta} z$ entonces definimos
	
	\[
	\gamma * \beta = \left\lbrace
	\begin{array}{cc}
	\gamma(2t) & \emph{si } t \leq \frac{1}{2} \\ 
	\beta(2t-1) &  \emph{si } y \geq \frac{1}{2}
	\end{array} 
	\right.
	\]
	
	Como $\gamma * \beta|_{[0,\frac{1}{2}]}$ y $\gamma * \beta|_{[\frac{1}{2},1]}$ son continuas entonces por \ref{Lema del pegado} tenemos que $\gamma * \beta$ es continua y $x \xrightarrow{\gamma * \beta} z$. \qed
	
\end{proof}

\begin{definition}
	A las clases de equivalencia dadas por la relaci\'on anterior las llamaremos \textit{componentes arcoconexas de } $x$ y las notaremos $A_x$; a $\quotient{X}{\sim} = \Pi_0(X)$
\end{definition}

\begin{remark}
	Se observa que si $f: X \rightarrow Y$ y $y \in C_x$ entonces $f(y) \in C_{f(x)} = f(C_x)$. Similarmente si $y \in A_x$ entonces $f(y) \in A_{f(x)}$.
\end{remark}

\begin{definition}
	
	Un espacio topol\'ogico se dice \textit{localmente conexo (resp arcoconexo)}si para todo $x \in X$ y para todo entorno $U \ni x$ existe un abierto $V$ conexo (resp arcoconexo) tal que $x \in V \subseteq U$
	
\end{definition}

\begin{example}
	
	Sea $P' = P \cup (\sett{0} \times I)$ entonces $P'$ es arcoconexo (f\'acil) pero no localmente conexo.
	
	En efecto, sea $p = (0,1)$ y $V \ni x$ un entorno que no toque al eje x como el construido en el ejemplo del peine anterior, probemos que no existe un abierto conexo $U$ tal que $p \in U \subseteq V$.
	
	Sea $U$ un abierto con esa propiedad entonces por construcci\'on existe $m \in \N$ tal que para todo $n \geq m$ se tiene que $(\frac{1}{n}, 1) \in U$. Sea entonces $\frac{1}{m+2} < x < \frac{1}{m+1}$ y consideremos $V = (\R_{<x} \times \R) \cap U$ y $H = (\R_{<x} \times \R) \cap U$. Es claro que son abiertos del subespacio $U$, no vac\'ios y como $(\sett{x} \times \R) \cap U = \emptyset$ tenemos que $U = V \cup H$ por lo que $U$ no es conexo \qed
	
\end{example}

\begin{example}
	
	Si consideramos $X = (\sett{0,1}, \tau_d)$ entonces es localmente arcoconexo pero no es conexo.
	
\end{example}

\section{Los primeros tres axiomas de separaci\'on}

\begin{definition}
	
	Un espacio topol\'ogico $X$ se dice $T_0$ si para cualquier par de puntos $x,y \in X$ existe $U$ abierto tal que $x \in U \not \ni y$. 
	
	Se dice $T_1$ si para cualquier par de puntos $x,y \in X$ existen $U,V$ abiertos tal que $x \in U \not \ni y$ y $y \in V \not \ni x$
	
	Finalmente se dice $T_2$ o \textit{Haussdorff} si para cualquier par de puntos $x,y \in X$ existen $U,V$ abiertos disjuntos tal que $x \in U$ y $y \in V$.
	
\end{definition}

\begin{remark}
	
	Tivialmente se da que $T_2 \Longrightarrow T_1 \Longrightarrow T_0$.
	
\end{remark}

\begin{example}
	
	\begin{itemize}
		
		\item Si consideramos $X$ indiscreto con m\'as de un punto entonces no es $T_0$
		
		\item Si consideramos $\mathfrak{S}$ entonces es $T_0$ pero no es $T_1$
		
		\item Si $X$ es infinito entonces $X$ con la topolog\'ia del complemento finito es $T_1$ pero no es $T_2$.
		
		En efecto, si $x \neq y \in X$ entonces $y \in \sett{x}^c \not \ni x$ y $x \in \sett{y}^c \not \ni y$. Pero si $U,V$ son abiertos disjuntos entonces $V \subseteq U^c$ y por lo tanto $V^c$ es infinito; se concluye que $V = \emptyset$ porque era abierto.
		
		\item Si $(X,\tau_m)$ es un espacio m\'etrico entonces es $T_2$
		
	\end{itemize}
	
\end{example}

\begin{proposition}
	
	\label{Subespacio y producto de T0 es T0}
	
	Subespacio y productos de $T_0$ son $T_0$
	
\end{proposition}

\begin{proof}
	
	Sean $x \neq y \in U \subseteq X$ puntos distintos de un subespacio, entonces como $X$ es $T_0$ sabemos que existe $V \subseteq X$ abierto tal que $x \in V \not \ni y$. Por lo tanto si consideramos $V' = V \cap U$ es un abierto de $U$ que cumple que $x \in V' \not \ni y$; se concluye que $U$ es $T_0$.
	
	Sean ahora $x \neq y \in \Bigprod{j \in J}{X_j}$ por lo que existe $j_0$ tal que $x_{j_0} \neq y_{j_0} \in X_{j_0}$; como $X_{j_0}$ es $T_0$ sabemos que existe $U_{j_0} \subseteq X_{j_0}$ tal que $x_{j_0} \in U_{j_0} \not \ni y_{j_0}$. Si tomamos $U = p_{j_0}^{-1}(U_{j_0})$ entonces es un abierto que cumple $x \in U \not \ni y$, por lo que $\Bigprod{j \in J}{X_j}$ es $T_0$. \qed 
	
\end{proof}

\begin{theorem}
	
	\label{Caracterizacion de T0}
	
	Un espacio topol\'ogico es $T_0$ si y s\'olo si existe $f: X \rightarrow \Bigprod{}{\mathfrak{S}}$ subespacio.
	
\end{theorem}

\begin{proof}
	
	Para un lado es simplemente notar que $\mathfrak{S}$ es $T_0$ y aplicamos \ref{Subespacio y producto de T0 es T0}.
	
	Para el otro consideremos $f : X \rightarrow \Bigprod{U \in \tau}{\mathfrak{S}}$ dado por:
	
	\[
	f(x)_U = \left\lbrace
	\begin{array}{cc}
	0 & \emph{si } x \in U  \\ 
	1 & \emph{si } x \not \in U
	\end{array} 
	\right.
	\]
	
	Notemos que $P_U f = 1_{U^c}$ que ya vimos que es continua para todo $U\in \tau$ en la demostraci\'on de \ref{La indicadora del complemento de U es continua en sierpinsky}, por lo tanto como $\sett{P_U}_{U \in \tau}$ es inicial para $\Bigprod{U \in \tau}{\mathfrak{S}}$ se tiene que $f$ es continua.
	
	Para ver que $f$ es inicial basta ver que $\sett{P_u f}_{U \in \tau} = \sett{1_{U^c}}_{U \in \tau}$ es inicial por \ref{Composicion de iniciales es inicial, y si la composicion es inicial entonces la interna lo es}, pero esto es claro por definici\'on.
	
	Finalmente si $x \neq y$ entonces como $X$ es $T_0$ entonces existe $x \in U \not \ni y$ abierto que separa, entonces $f(x)_U \neq f(x)_U$ por lo que $f(x) \neq f(y)$.
	
	Por lo tanto $f$ es continua, inyectiva e inicial; o sea subespacio \qed.
	
\end{proof}

\begin{proposition}
	
	\label{Caracterizacion de T1 por puntos cerrados}
	
	Sea $X$ un espacio topol\'ogico, entonces es $T_1$ si y s\'olo si $\sett{x}$ es cerrado para todo $x \in X$
	
\end{proposition}

\begin{proof}
	
	Para un lado, si dados $x,y \in X$ se tiene que $\sett{x}, \sett{y}$ son cerrados entonces $x \in \sett{y}^c \not \ni y$ y $y \in \sett{x}^c \not \ni x$; por lo tanto $X$ es $T_1$
	
	Para el otro lado, sea $x \in X$, si $y \in \sett{x}^c$ como $X$ es $T_1$ existe $U \ni y$ tal que $U \subseteq \sett{x}^c$, por lo tanto $\sett{x}^c$ es abierto \qed.
	
\end{proof}

\begin{proposition}
	
	\label{Subespacio y producto de T1 es T1}
	
	Subespacio y productos de $T_1$ son $T_1$
	
\end{proposition}

\begin{proof}
	
	Sea $x \neq y \in U \subseteq X$ puntos distintos de un subespacio, entonces como $X$ es $T_1$ sabemos que existen $V,W \subseteq X$ abiertos tal que $x \in V \not \ni y$ y $y \in W \not \ni x$. Por lo tanto si consideramos $V' = V \cap U$ y $W' = W \cap U$ son abiertos de $U$ que cumplen que $x \in V' \not \ni y$ y $y \in W' \not \ni x$; se concluye que $U$ es $T_1$.
	
	Sean ahora $x \neq y \in \Bigprod{j \in J}{X_j}$ por lo que existe $j_0$ tal que $x_{j_0} \neq y_{j_0} \in X_{j_0}$; como $X_{j_0}$ es $T_1$ sabemos que existen $U_{j_0},V_{j_0} \subseteq X_{j_0}$ tal que $x_{j_0} \in U_{j_0} \not \ni y_{j_0}$ y $y_{j_0} \in V_{j_0} \not \ni x_{j_0}$. Si tomamos $U = p_{j_0}^{-1}(U_{j_0})$ y $V =p_{j_0}^{-1}(V_{j_0}) $ entonces son abiertos que cumplen $x \in U \not \ni y$ y $y \in V \not \ni x$, por lo que $\Bigprod{j \in J}{X_j}$ es $T_1$. \qed 
	
\end{proof}

\begin{proposition}
	
	\label{Caracterizacion de T2 y equivalencias}
	
	Sea $X$ un espacio topol\'ogico, entonces son equivalentes:
	
	\begin{enumerate}
		
		\item $X$ es $T_2$
		
		\item Si $\Delta: X \rightarrow X \times X$ es dada por $\Delta(x) = (x,x)$ entonces $\Delta(X)$ es cerrado
		
		\item Toda red convergente en $X$ tiene l\'imite \'unico 
		
		
	\end{enumerate}

\end{proposition}

\begin{proof}
	
	Vayamos de a partes:
	
	\begin{itemize}
		
		\item[i) $\Longrightarrow$ ii)] Sea $(x,y) \not \in \Delta(X)$, entonces $x \neq y$ y como $X$ es $T_2$ existen $U,V$ abiertos disjuntos tal que $(x,y) \in U \times V$ y $(U \times V) \cap \Delta(X) = \emptyset $ pues $U \cap V = \emptyset$; por lo tanto $(x,y) \in U \times V \subseteq \Delta(X)^c$. Se concluye que $\Delta(X)$ es cerrado.
		
		\item[ii) $\Longrightarrow$ i)] Sean $x \neq y \in X$, entonces $(x,y) \in \Delta(X)^c $ y por hip\'otesis existe $U \times V \subseteq X \times X$ tal que $(x,y) \in U \times  V \subseteq \Delta(X)^c$. Por lo tanto esto quiere decir que existen $U,V$ abiertos disjuntos tal que $x \in U, y \in V$, o sea $X$ es $T_2$.
		
		\item[i) $\Longrightarrow$ iii)] Sea $\sett{x_{\alpha}}_{\alpha \in \Lambda}\in \Bigprod{\alpha \in \Lambda}{X}$ una red en $X$ y sea $x \neq y$ dos l\'imites de $x_{\alpha}$. Entonces como $X$ es $T_2$ se tiene que existe $U,V$ abiertos disjuntos tal que $x \in U,y \in V$.
		
		Como $x_{\alpha} \rightarrow x$ existe un $\alpha_0 'in \Lambda$ tal que $x_{\alpha} \in U$ para todo $\alpha \geq \alpha_0$, por otro lado como $x_{\alpha} \rightarrow y$ existe $\alpha_1 \in \Lambda$ tal que $x_{\alpha} \in V$ para todo $\alpha \geq \alpha_1$.
		
		Como $\Lambda$ es dirigido existe $\alpha_2 \geq \alpha_1,\alpha_0$ por lo que si $\alpha \geq \alpha_2$ se tiene que $x_{\alpha} \in U \cap V = \emptyset$. Conclu\'imos que $x = y$ y el l\'imite es \'unico.
		
		\item[iii) $\Longrightarrow$ i)] Supongamos que $X$ no es $T_2$, entonces existen $x \neq y$ tal que para todos $U,V$ entornos de $x,y$ respectivamente se tiene que $U \cap V \neq \emptyset$. Sea $\Lambda = \sett{U \cap V \ / \ U \ni x \ , \ V \ni y \ , \ U,V\emph{ abiertos}}$ y lo ordenamos por la inclusi\'on inversa, entonces es claro que es dirigido y definimos $f : \Lambda \rightarrow X$ por $f(U \cap V) \in U \cap V$.
		
		Por un lado $f \rightarrow x$ pues dado $U \ni x$ entorno abierto tenemos que si $\alpha \geq U \cap X \in \Lambda$ entonces $f(\alpha) \in \alpha \subseteq U \cap X = U$, similarmente $f \rightarrow y$ y por hip\'otesis $x = y$; conclu\'imos que $X$ es $T_2$. \qed



	\end{itemize}
	
\end{proof}

\begin{proposition}
	
	\label{Subespacio y producto de T2 es T2}
	
	Subespacio y productos de $T_2$ son $T_2$
	
\end{proposition}

\begin{proof}
	
	Sea $x \neq y \in U \subseteq X$ puntos distintos de un subespacio, entonces como $X$ es $T_2$ sabemos que existen $V,W \subseteq X$ abiertos disjuntos tal que $x \in V$ y $y \in W$. Por lo tanto si consideramos $V' = V \cap U$ y $W' = W \cap U$ son abiertos disjuntos  de $U$ que cumplen que $x \in V'$ y $y \in W'$; se concluye que $U$ es $T_2$.
	
	Sean ahora $x \neq y \in \Bigprod{j \in J}{X_j}$ por lo que existe $j_0$ tal que $x_{j_0} \neq y_{j_0} \in X_{j_0}$; como $X_{j_0}$ es $T_2$ sabemos que existen $U_{j_0},V_{j_0} \subseteq X_{j_0}$ disjuntos tal que $x_{j_0} \in U_{j_0}$ y $y_{j_0} \in V_{j_0}$. Si tomamos $U = p_{j_0}^{-1}(U_{j_0})$ y $V =p_{j_0}^{-1}(V_{j_0}) $ entonces son abiertos disjuntos que cumplen $x \in U$ y $y \in V$, por lo que $\Bigprod{j \in J}{X_j}$ es $T_2$. \qed 
	
\end{proof}

\section{Compacidad}

\begin{definition}
	
	Un espacio topol\'ogico $X$ se dice \textit{compacto} si dado una familia de abiertos $\sett{U_j}_{j \in J}$ tal que $\Bigcup{j \in J}{U_j} = X$ existe $J' \subset J$ finito tal que $\Bigcup{j \in J'}{U_j} = X$. A esta familia con la propiedad de cubrir la llamaremos un \textit{cubrimiento por abiertos}.
	
\end{definition}

\begin{example}
	
	\begin{enumerate}
		
		\item Si $X$ es finito entonces trivialmente $X$ es compacto
		
		\item Si $X$ es m\'etrico entonces $X$ es compacto si y s\'olo si $X$ es completo y totalmente acotado (Avanzado)
		
		\item Si $X$ tiene la topolog\'ia del complemento finito, entonces $X$ es compacto.
		
		En efecto, si $\Bigcup{j \in J}{U_j} = X$ como $U_{1}^c$ es finito entonces existen finitos $U_2, \dots U_n \in \sett{U_j}_{j \in J}$ tal que $U_j \cap U_1 \neq \emptyset$, por lo tanto $\Bigcup{1 \leq j \leq n}{U_j} = X$.
		
		
	\end{enumerate}
	
\end{example}

\begin{proposition}
	
	\label{Lema del tubo}
	
	Sean $X,Y$ espacios topol\'ogicos con $X$ copmpacto. Si $y_0 \in Y$ y $W \subseteq X \times Y$ son tales que $X \times \sett{y_0} \subseteq W$ entonces existe $V \ni y_0$ abierto de $Y$ tal que $X \times V \subseteq W$.
	
\end{proposition}

\begin{proof}
	
	Sea $x \in X$, luego como $(x,y_0) \in W$ existe $U_x \times V_x$ abierto b\'asico tal que $(x,y_0) \in U_x \times V_x \subseteq W$; por lo tanto se tiene que $\Bigcup{x \in X}{U_x} = X$. Como $X$ es compacto existe $J= \sett{x_1 , \dots, x_n}$ finito tal que $\Bigcup{x \in J}{U_x} = X$, si tomamos $V = \Bigcap{x \in J}{V_x}$ se tiene que $V$ es abierto, $y_0 \in V$ y $X \times V \subseteq \Bigcup{x \in J}{U_x \cap V_x} \subseteq W$. \qed
	
\end{proof}

\begin{definition}
	
	Decimos que una familia de cerrados $\sett{F_j}_{j \in J}$ de $X$ tiene la \textit{propiedad de intersecci\'on finita} si para cada  $J' \subset J$ finito se tiene que $\Bigcap{j \in J'}{F_j} \neq \emptyset$
	
\end{definition}

\begin{theorem}
	
	\label{Caracterizacion de compacidad}
	
	Sea $X$ un espacio topol\'ogico, entonces son equivalentes:
	
	\begin{enumerate}
		
		\item $X$ es compacto
		
		\item Para toda familia de cerrados con la propiedad de intersecci\'on finita se tiene que $\Bigcap{j \in J}{F_j} \neq \emptyset$
		
		\item Para todo espacio topol\'ogico $Y$ se tiene que $p_Y : X \times Y \rightarrow Y$ es cerrada
		
		\item Toda red en $X$ tiene una subred convergente.
	
	\end{enumerate}
	
\end{theorem}

\begin{proof}
	
	Vayamos de a partes:
	
	\begin{itemize}
		
		\item[ii) $\Longrightarrow$ i)] Si $\Bigcup{j \in J}{U_j} = X$ entonces $\Bigcap{j \in J}{U_{j}^{c}} = \emptyset$, por hip\'otesis existe entonces $J' \subset J$ finito tal que $\Bigcap{j \in J'}{U_{j}^{c}} = \emptyset$, por lo tanto $\Bigcup{j \in J'}{U_{j}} = X$.
		
		\item[i) $\Longrightarrow$ iii)] Sea $Y$ espacio topol\'ogico y $F \subseteq X \times Y$ cerrado, consideremos $y \in p_Y(F)^c$  entonces $X \times y \subseteq F^c$. Por \ref{Lema del tubo} existe $V \subseteq Y$ abierto tal que $(x,y) \in X \times V \subseteq F^c$, por lo tanto $y \in V \subseteq P_y(F)^c$ y se concluye que si $F$ es cerrado entonces $p_Y(F)$ es cerrado.
		
		\item[iii) $\Longrightarrow$ iv)] Sea $f : \Lambda \rightarrow X$ una red y consideremos $\Lambda' = \Lambda \cup \sett{\infty}$ con al topolog\'ia dada por la base $\B = \sett{\sett{\alpha} \ , \ \alpha \in \Lambda} \cup \sett{\sett{\beta \geq \alpha} \cup \sett{\infty} \ , \ \alpha \in \Lambda }$, o sea la "compactificaci\'on de Alexandroff sobre $\Lambda$".
		
		Si $A = \sett{\beta \geq \alpha_0} \cup \sett{\infty}$ es un entorno b\'asico de $\infty$ entonces si llamamos $P := P_{\Lambda'}$se tiene que  $\alpha_0 \in A \cap P(\overline{\sett{\alpha, f(\alpha) \ / \ \alpha \in \Lambda}})$, es decir que $\infty \in \overline{P(\overline{\sett{\alpha, f(\alpha) \ / \ \alpha \in \Lambda}})} = P(\overline{\sett{\alpha, f(\alpha) \ / \ \alpha \in \Lambda}}) $ por hip\'otesis. Entonces existe $x_0 \in X$ tal que $(\infty,x_0) \in B = \overline{\sett{\alpha, f(\alpha) \ / \ \alpha \in \Lambda}}$
		
		Sea $\Gamma = \sett{(\alpha, U) \ / \ \alpha \in \Lambda \ , \ x_0 \in U \in \tau_X \ , \ f(\alpha) \in U }$ y lo ordenamos de la siguiente manera:
		
		\[ (\alpha_1 , U_1) \geq (\alpha_2 , U_2)  \ \Longleftrightarrow \alpha_1 \geq \alpha_2 \ , \ U_1\subseteq U_2 \]
		
		Veamos que $\Gamma$ es dirigido. Sean $(\alpha_1 , U_1), (\alpha_2 , U_2) \in \Gamma$, como $\Lambda$ es dirigido existe $\alpha_3 \in \Lambda$ tal que $\alpha_3 \geq \alpha_1,\alpha_2$; por lo tanto $C = (\sett{\beta \geq \alpha_3} \cup \sett{\infty}) \times (U_1 \cap U_2)$ es un abierto de $\Lambda' \times X$ que cumple que $(\infty, x_0) \in C$.
		
		Pero como $(\infty,x_0) \in B$ existe $\alpha_4 \in \Lambda$ tal que $\alpha_4 \geq \alpha_3$ y $f(\alpha_4) \in U_1 \cap U_2$, por lo tanto $(\alpha_4 , U_1 \cap U_2) \in \Lambda$ y  $(\alpha_4 , U_1 \cap U_2) \geq (\alpha_1 , U_1), (\alpha_2 , U_2)$. Definimos $g : \Gamma \rightarrow \Lambda$ dada por $g(\alpha,U) = \alpha$ que es cofinal.
		
		Consideremos la subred $fg$ y veamos que $fg \rightarrow x_0$. Sea $U \ni x_0$ un entorno abierto, entonces $\Lambda' \times U$ es un abierto de $\Lambda' \times X$ que cumple que $(\infty , x_0) \in B \cap (\Lambda' \times U)$, por lo tanto $(\Lambda' \times U) \cap \sett{\alpha, f(\alpha) \ / \ \alpha \in \Lambda} \neq \emptyset$ y existe $\alpha_0 \in \Lambda$ tal que $f(\alpha_0) \in U$.
		
		Si $(\alpha,V) \geq (\alpha_0 , U)$ entonces $fg(\alpha, V)= f(\alpha) \in V \subseteq U$ y se concluye que $fg \rightarrow x_0$.
		
		\item[iv) $\Longrightarrow$ ii)] Sea $\sett{F_j}_{j \in J}$ una familia de cerrados de $X$ con la PIF y sea entonces $\Lambda = \sett{J' \emph{finitos} \ / \ J' \subset J}$ ordenado por la inclusi\'on, y sea $f: \Lambda \rightarrow X$ dada por $f(J') \in \Bigcap{j \in J'}{F_j}$ que est\'a bien definida pues la familia tiene la PIF.
		
		Sea $\Gamma \xrightarrow{g} \Lambda \xrightarrow{f} X$ una subred convergente a cierto $x_0 \in X$ y supongamos que existe $j_0 \in  J$ tal que $x_0 \in F_{j_0}^c$, luego existe $\gamma_1 \in \Gamma$ tal que si $\gamma \geq \gamma_1$ entonces $fg(\gamma) \in F_{j_0}^{c}$. Por otro lado, como $g$ es cofinal existe $\gamma_2 \in \Gamma$ tal que $g(\gamma_2) \geq \sett{j_0}$; como $\Gamma$ es dirigido sea $\gamma_3 \geq \gamma_1,\gamma_2$.
		
		Por un lado, como $\gamma_3 \geq \gamma_1$ se tiene que $fg(\gamma_3) \in F_{j_0}^c$; pero por el otro como $\gamma_3 \geq \gamma_2$ se tiene que $fg(\gamma_3) \in \Bigcap{j \in g(\gamma_2)}{F_j} \subseteq F_{j_0}$ pues $\sett{j_0} \subseteq g(\gamma_2)$; concluimos que $fg(\gamma_3) \in F_{j_0} \cap F_{j_0}^{c} = \emptyset$ de lo que sale que no exist\'ia tal $j_0$. Por lo tanto $x_0 \in \Bigcap{j \in J}{f_j} \neq \emptyset$. \qed
		 

	\end{itemize}
	
\end{proof}

\begin{definition}
	
	Decimos que una clase $\A$ de funciones continuas es \textit{buena} si:
	
	\begin{enumerate}
		\item $\A$ contiene a todos los homeomorfismos
		\item $f,g \in \A$ y tiene sentido componer, entonces $gf \in \A$
		\item $\A$ es estable por cambio de base
	\end{enumerate}
	
\end{definition}

\begin{example}
	
	\begin{itemize}
		
		\item Por \ref{Los homeomorfismos son estables por cambio de base} los homeomorfismos son una clase buena
		
		\item Por \ref{Las inyectivas son estables por cambio de base}  las funciones inyectivas son una clase buena
		
		\item Por \ref{Funciones subespacio son estables por cambio de base} las funciones subespacio son una clase buena
		
		\item Las funciones cerradas no son una clase buena pues no son estables por cambio de base
	
	\end{itemize}
	
\end{example}

\begin{example}
	
	Una funci\'on $f : X \rightarrow Y$ si cada vez que se tiene un pullback:
	
	\begin{equation*}
	\begin{tikzcd}
	P \arrow[swap] {d}{\tilde{f}} \arrow {r} {\tilde{g}} & X \arrow[swap] {d} {f}\\ 
	Y \arrow {r} {g} & Z 
	\end{tikzcd}
	\end{equation*}		
	
	entonces $\tilde{f}$ es cerrada.
	
\end{example}

\begin{proposition}
	
	\label{Propia implica cerrada}
	
	Si una funci\'on $f: X \rightarrow Y$ es propia entonces es cerrada.
	
\end{proposition}

\begin{proof}
	
	Basta con considerar el siguiente pullback:
	
	\begin{equation*}
	\begin{tikzcd}
	X \arrow[swap] {d}{{f}} \arrow {r} {1_X} & X \arrow[swap] {d} {f}\\ 
	Y \arrow {r} {1_Y} & Y 
	\end{tikzcd}
	\end{equation*}	
	
	\qed
	
\end{proof}

\begin{proposition}
	
	\label{Propias son clase buena}
	
	Las funciones propias son una clase buena.
	
\end{proposition}

\begin{proof}
	
	\begin{itemize}
		
		\item Los homeomorfismos al ser estables por cambio de base y cerrados son propias.
		
		\item Sean $f : X \rightarrow Y, g : Y \rightarrow Z, h:W \rightarrow Z$  tal que $f,g$ son propias y veamos el siguiente pullback:
		
		\begin{equation*}
		\begin{tikzcd}
		P \arrow[swap] {d}{\tilde{f}} \arrow {r} {\tilde{\tilde{h}}}  & X \arrow[swap] {d} {f}\\ 
		T \Center{\circled{1}}{ur}  \arrow[swap] {d}{\tilde{g}} \arrow{r}{\tilde{h}} & Y \arrow[swap] {d} {g} \\
		W \Center{\circled{2}}{ur} \arrow{r} {h} & Z 
		\end{tikzcd}
		\end{equation*}	
		
		Como tanto $f$ como $g$ son propia entonces $\tilde{f},\tilde{g}$ son cerradas y como $\circled{1}, \circled{2}$ son pullback entonces por \ref{Pullback de pullback es pullback} tenemos que existe $t : W \times_Z X \rightarrow T$ homeomorfismo tal que $\tilde{g}\tilde{f}t = \tilde{fg}$, como $t,\tilde{f}, \tilde{g}$ son cerradas se concluye que $\tilde{fg}$ es cerrada y entonces $fg$ es propia.
		
		\item Sea el diagrama:
	
		\begin{equation*}
		\begin{tikzcd}
		W \times_Z (Z \times_Y X) \arrow {r}\arrow[swap] {d} {\tilde{\tilde{f}}} & Z \times_Y X \arrow[swap] {d}{\tilde{f}} \arrow {r} {1_X} & X \arrow[swap] {d} {f}\\ 
		W \Center{\circled{1}}{ur} \arrow {r}& Z \Center {\circled{2}}{ur} \arrow {r} {1_Y} & Y 
		\end{tikzcd}
		\end{equation*}		
		
		Como tanto $\circled{1}, \circled{2}$ son pullback entonces por \ref{Pullback de pullback es pullback} diagrama completo es pullback y por lo tanto como $f$ es propia se tiene que $\tilde{\tilde{f}}$ es cerrada, o sea que $\tilde{f}$ es propia y las propias son estables por cambio de base. \qed
		
	\end{itemize}
	
\end{proof}

\begin{proposition}
	
	\label{Compacidad es que la funcion al singleton es propia}
	
	Sea $X$ un espacio topol\'ogico, entonces $X$ es compacto si y s\'olo si $X \rightarrow * $ es propia.
	
\end{proposition}

\begin{proof}
	
	$X \xrightarrow{f} * $ es propia si y s\'olo si para todo $Y$ espacio topol\'ogico se tiene que:
	
	\begin{equation*}
	\begin{tikzcd}
	X \times Y \arrow {r}{} \arrow {d} {p_Y} & X \arrow[swap] {d} {f}\\ 
	Y \arrow {r}  & * 
	\end{tikzcd}
	\end{equation*}	
	
	$p_Y = \tilde{f}$ es cerrada, si y s\'olo si $X$ es compacto. \qed
	
\end{proof}

\begin{proposition}
	
	\label{Subespacio mas cerrada es propia}
	
	Las funciones subespacio y cerradas son propias
	
\end{proposition}

\begin{proof}
	
	Por \ref{Pullback de subespacio} sabemos como es el pullback contra un subespacio, por  \ref{Funciones subespacio son estables por cambio de base} sabemos que $\tilde{i}$ es subespacio. Adem\'as si $i$ es cerrada entonces $F $ es cerrada por lo que $f^{-1}(F)$ es cerrado, lo que implica que $\tilde{i} $ es cerrada. \qed
	
\end{proof}

\begin{corollary}
	
	\label{Cerrado en un compacto es compacto}
	
	Sea $X$ un espacio topol\'ogico y $F\subseteq X $ un subespacio cerrado, entonces $F$ es compacto.
	
\end{corollary}

\begin{proof}
	
	En efecto, consideremos $F \inc X \rightarrow *$ entonces por \ref{Subespacio mas cerrada es propia} sabemos que $i: F \rightarrow X$ es propia y como $X$ es compacto $X \rightarrow *$ es propia, por \ref{Propias son clase buena} sabemos entonces que $F \rightarrow *$ es propia, conclu\'imos con \ref{Compacidad es que la funcion al singleton es propia}. \qed
	
\end{proof}

\begin{proposition}
	
	\label{Producto de 2 compactos es compacto}
	
	Sean $X,Y$ espacios compactos, entonces $X \times Y$ es compacto.
	
\end{proposition}

\begin{proof}
	
	Consideremos el siguiente diagrama:
	
	\begin{equation*}
	\begin{tikzcd}
	X \times Y \arrow {r}{} \arrow {d} {p_Y} & X \arrow[swap] {d} {f}\\ 
	Y \Center{\circled{1}}{ur} \arrow {r} \arrow {d} {}  & * \\
	* & 
	\end{tikzcd}
	\end{equation*}		
	
	Como $X \rightarrow *$ es propia por \ref{Compacidad es que la funcion al singleton es propia}, $\circled{1}$ es pullback y por \ref{Propias son clase buena} (Propias son estables por cambio de base) sabemos que $p_Y : Y \times X \rightarrow Y $ es propia; como adem\'as $Y \rightarrow *$ es propia por \ref{Compacidad es que la funcion al singleton es propia}, de \ref{Propias son clase buena} (Composici\'on de propias es propia) se tiene que $X \times Y \rightarrow *$ es propia. \qed
	
\end{proof}

\begin{proposition}
	
	\label{Si f continua y X compactoentonces f(X) compacto}
	
	Sea $f: X \rightarrow Y$ y $X$ compacto, entonces $f(X) \subseteq Y$ es compacto
	
\end{proposition}

\begin{proof}
	
	Sea $\sett{U_j}_{j \in J}$ un cubrimiento de $f(X)$ por abiertos de $Y$, entonces $\sett{f^{-1}(U_j)}_{j \in J}$ es un cubrimiento por abiertos de $X$. Como $X$ es compacto existe $J\subset J$ finito tal que $X = \Bigcup{j 'in J'}{f^{-1}(U_j)}$, por lo tanto $f(X) = \Bigcup{j \in J'}{f(f^{-1}(U_j))} \subseteq \Bigcup{j \in J'}{U_j}$ y se sigue que $f(X)$ es compacto. \qed
	
\end{proof}

\begin{corollary}
	
	La compacidad es un invariante topol\'ogico
	
\end{corollary}

\begin{proposition}
	
	\label{Compacto en un haussdorff es cerrado}
	
	Si $X$ es $T_2$ y $K \subseteq X$ es compacto entonces $K$ es cerrado.
	
\end{proposition}

\begin{proof}
	
	En efecto, sea $x \in \overline{K} $ y sea $x_{\alpha}$ la red en $K$ tal que $x_{\alpha} \rightarrow x$, por \ref{Caracterizacion de compacidad} existe una subred $(x_{\alpha_{\gamma}})_{\gamma \in \Gamma}$ tal que $x_{\alpha_{\gamma}} \rightarrow y \in K$, por otro lado por \ref{Convergencia de una red por sub-redes} se tiene que $x_{\alpha_{\gamma}} \rightarrow x$ y por \ref{Caracterizacion de T2 y equivalencias} se tiene que $x = y \in K$, se finaliza con \ref{Caracterizacion de la clausura por redes} \qed
	
\end{proof}

\begin{corollary}
	
	\label{Continua de un compacto en un haussdorff es cerrada}
	
	Sea $f : X \rightarrow Y$ donde $X$ es compacto e $Y$ es $T_2$, entonces $f$ es cerrada.
	
\end{corollary}

\begin{proof}
	
	Sea $F \subseteq X$ cerrado, por \ref{Cerrado en un compacto es compacto} se tiene que $F$ es compacto, luego por \ref{Si f continua y X compactoentonces f(X) compacto} sabemos que $f(F)$ es compacto, pero por \ref{Compacto en un haussdorff es cerrado} tenemos que $f(F)$ es cerrado; luego $f$ es cerrada. \qed
	
\end{proof}


\begin{corollary}
	
	\label{Continua y biyectiva de un compacto en un haussdorff es homeomorfismo}
	
		Sea $f : X \rightarrow Y$ biyectiva donde $X$ es compacto e $Y$ es $T_2$, entonces $f$ es homeomorfismo.
	
\end{corollary}

\begin{example}
	
	Sea $f: \quotient{I}{0 \sim 1} \rightarrow S^1$ dada por $f([t]) = (cos(2 \pi t), sin(2 \pi t)$, entonces ya vimos previamente que $fq$ es continua por lo que como $q$ es final, $f$ es continua; adem\'as es trivialmente biyectiva como vimos previamente. Como $I$ es compacto y $q : I \rightarrow \quotient{I}{\sim}$ es continua se tiene que $\quotient{I}{0 \sim 1}$ es compacto, y finalmente $S^1$ es $T_2$ por ser subespacio de $\R^2$ y \ref{Subespacio y producto de T2 es T2}. Por \ref{Continua y biyectiva de un compacto en un haussdorff es homeomorfismo} de tiene que $f$ es homeomorfismo.
	
\end{example}

\begin{theorem}
	
	\label{Equivalencias de propia}
	
	Sea $f:X \rightarrow Y$ continua, entonces son equivalentes:
	
	\begin{enumerate}
		
		\item $f$ es cerrada y $f^{-1}(\sett{y})$ es compacto para todo $y \in Y$.
		
		\item $f$ es cerrada y $f^{-1}(K)$ es compacto para todo $K \subseteq  Y$ compacto.
		
		\item Para todo $Z$ espacio topol\'ogico, $id_Z � f : Z \times X \rightarrow Z � Y$ es cerrada
		
		\item $f$ es propia
		
	\end{enumerate}
	
\end{theorem}

\begin{proof}
	
	Veamoslo por partes:
	
	
	\begin{itemize}
		
		\item[iv $\Longrightarrow$ iii)] Consideremos el siguiente diagrama:
		
		\begin{equation*}
		\begin{tikzcd}
		Z \times X \arrow {r}{p_X} \arrow {d} {1_Z \times f} & X \arrow[swap] {d} {f}\\ 
		Z \times Y \arrow {r}{p_Y}  & Y 
		\end{tikzcd}
		\end{equation*}	
		
		Veamos que es un pullback. Consideremos $(P,s,t)$ un tr\'io, entonces si $h : P \rightarrow X \times Y$ cumple que $(1_Z \times f)h = s$ entonces $h_1 = s_1$ y $fh_2=s_2$, por otro lado $h_2 = p_X h = t$; por lo tanto consideremos $h = (s_1,t)$.
		
		Entonces $h$ est\'a bien definida, $(1_Z \times f) h = s_1 \times ft = s_1 \times (p_Y s) = s_1 \times s_2 = s$, $p_Xh = t$ y adem\'as como $\sett{p_X,p_Z}$ es inicial se tiene que $h$ continua si y s\'olo si $s_1,t$ son continuas; conclu\'imos que $Z \times X$ es un pullback y como $f$ es propia entonces $1_Z \times f$ es cerrada.
		
		\item[iii) $Longrightarrow$ ii)] Sea $K \subseteq Y$ compacto y sea $Z$ un espacio topol\'ogico entonces veamos que $p_Z : Z \times f^{-1}(K) \rightarrow Z$ es cerrada pues por \ref{Caracterizacion de compacidad} esto es equivalente a que $f^{-1}(K)$ es compacto. 
		Consideremos $1_Z \times f : Z \times X \rightarrow Z \times Y$ que es cerrada y veamos $1_Z \times f |_{f^{-1}(K)} : Z \times f^{-1}(K) \rightarrow Z \times K $, si $F \subseteq Z \times f^{-1}(K)$ es cerrado entonces $1_Z \times f (F) \subseteq Z \times Y$ es cerrado. 
		
		Si $(F_1,f(F_2)) \not \in Z \times K$ entonces existe $y_2 = f(x) \in f(f^{-1}(K))$ tal que $y_2 \not \in K$, por lo que $x \not \in f^{-1}(K)$, por lo tanto $(1_Z \times f) (F) \subseteq Z \times K$ es cerrado y conclu\'imos que $1_Z \times f|_{f^{-1}(F)} : Z \times f^{-1}(K) \rightarrow Z \times K$ es cerrada.
		
		Como $K$ es compacto entonces $p_Z : Z \times K \rightarrow Z$ es cerrada y entonces $p_Z \circ (1_Z \times f|_{f^{-1}(K)}) : Z \times f^{-1}(K) \rightarrow Z$ es cerrada, por \ref{Caracterizacion de compacidad} $f^{-1}(K)$ es compacto.
		
		\item[ii) $Longrightarrow$ i)] Dado $y \in Y$ como $\sett{y}$ es compacto tenemos que $f^{-1}(\sett{y})$ es compacto.
		
		\item[i) $Lngrightarrow$ iv)] Consideremos la construcci\'on dada en \ref{PU del pullback} del pullback como $P = \sett{(z,x) \in Z \times X \ / \ g(z) = f(x)}$ y $\tilde{f} = p_Z$, entonces sea $F \subseteq P$ cerrado y queremos ver que $p_Z(F) \subseteq Z$ es cerrado, sea para esto $z_0 \in p_Z(F)^c$.
		
		Como $F \subseteq P$ es cerrado existe un cerrado $H \subseteq Z \times X$ tal que $F = H \cap P$ y como $z_0 \not \in p_Z(F)$ entonces no existe $(z_0 , x) \in H$ tal que $g(z_0) = f(x)$, o equivalentemente que $\sett{z_0} \times f^{-1}(g(z_0)) \subseteq H^c$. Como $f^{-1}(g(z_0))$ es compacto por hip\'otesis entonces por \ref{Lema del tubo} existen $U \subseteq Z,V \subseteq X$ abiertos talque $\sett{z_0} \times f^{-1}(g(z_0)) \subseteq U \times V \subseteq H^c$
		
		Sea entonces $(z,x) \in F$ y supongamos que $p_Z(z,x) \in U$ luego $x \not \in V$ lo que implica que $g(z) = f(x) \in f(X \setminus V)$, luego $g(z) \not \in Y \setminus f(X \setminus ) := W$ y por lo tanto $z \not \in g^{-1}(W)$; es m\'as como $f(g(z_0)) \in V$ entonces $g(z_0) \in f^{-1}(V)$ y entonces $z_0 \in W$. Luego tenemos que $z_0 \in g^{-1}(W) \cap U \subseteq p_Z(F)^c$, conclu\'imos que $\tilde{f}$ es cerrada y entonces $f$ es propia \qed
		
		
	\end{itemize}
	
\end{proof}



\begin{definition}
	
	Un  espacio topol\'ogico  $X$ se dice \textit{localmente compacto} si paraa todo $x \in X$ existe $K \ni x$ entorno compacto
	
\end{definition}

\begin{remark}
	
	\begin{itemize}
		
		\item Si $X$ resulta compacto entonces es localmente compacto
		
		\item $\R$ con la topolog\'ia usual es localmente compacto
		
		
	\end{itemize}
	
\end{remark}

\begin{proposition}
	
	\label{Caracterizacion localmente compacto}
	
	Sea $X$ un espacio topol\'oogico $T_2$, entonces son equivalentes:
	
	\begin{enumerate}
		
		\item $X $ es localmente compacto
		
		\item Para todo $x \in X$ y todo entorno abierto $U \ni x$ existe un abierto $V$ tal que $x \in V \subseteq \overline{V} \subseteq U$ y $\overline{V}$ es compacto. 
		
		
	\end{enumerate}
	
\end{proposition}

\begin{proof}
	
	Sea $x \in X$ y $U \ni x$ entorno abierto, como $X$ es localmente compacto exise adem\'as $K$ entorno compacto de $x$. Por \ref{Compacto en un haussdorff es cerrado} se tiene que $K$ es cerrado y por lo tanto $K \setminus U$ es compacto por \ref{Cerrado en un compacto es compacto}.
	
	Usemos el siguiente lema \'util en esta situaci\'on:
	
	\begin{lemma}
		
		\label{Si X es compacto y haussdorf entonces es T3}
		
		Sea $X$ un espacio topol\'ogico $T_2$,  $K$ compacto y $x \in X$ tal que $x \not \in K$, entonces existen $U,V \subseteq X$ abiertos disjuntos tal que $x \in U$ y $K \subseteq V$.
		
	\end{lemma}
	
	Luego por \ref{Si X es compacto y haussdorf entonces es T3} existen abiertos $W,W'$ disjuntos tal que $x \in W$, $K \setminus U \subseteq W'$. Sea $V = W \cap \interior{K}$ y afirmamos este es el abierto que busc\'abamos:
	
	\begin{itemize}
		
		\item $x \in V$
		
		\item $V$ es abierto
		
		\item Como $V \subseteq K$ entonces $\overline{V} \subseteq \overline{K} = K$ por \ref{Compacto en un haussdorff es cerrado}, luego por \ref{Cerrado en un compacto es compacto} se tiene que $\overline{V}$ es compacto.
		
		\item Como $V = W \cap \interior{K}$ entonces $\overline{V} \subseteq \overline{W} \cap K$, pero como $W \subseteq (W')^c$ que es cerrado se tiene que $\overline{W} \subseteq (W')^c \subseteq (K \setminus U)^c = K^c \cup U$.
		
		Por lo tanto $\overline{V} \subseteq (K^c \cup U) \cap K = U \cap K \subseteq U$. \qed
	
	\end{itemize}
	
\end{proof}

\begin{proof}{Del lema}
	
	Sea $y \in K$, entonces como $X$ es $T_2$ sabemos que existen $U_y,V_y$ abiertos disjuntos tal que $x \in U_y$, $y \in V_y$, por lo tanto se tiene que $K = \Bigcup{y \in K}{V_y}$. Como $K$ es compacto existe $J= \sett{y_1 , \dots , y_n}$ finito tal que $K = \Bigcup{y \in J}{V_y}$, tomemos $U = \Bigcap{y \in J}{U_y}$ y $V = \Bigcup{y \in J}{V_y}$. Entonces como $x \in U_y$ para todo $y \in K$ se tiene que $x \in U$, adem\'as por la construcci\'on $K \subseteq V$ y ambos son abiertos.
	
	Finalmente $U \cap V = \Bigcup{y \in J}{U \cap V_y} = \emptyset$ pues $V_y \cap U_y = \emptyset$ para todo $y \in K$. \qed
	
\end{proof}

\begin{corollary}
	
	Sea $X,Y$ espacios topol\'ogicos tal que $Y$ es localmente compacto y $T_2$. Sea $f : X \rightarrow Y$ continua tal que $f^{-1}(K)$ es compacto para todo $K$ compacto, entonces $f$ es propia.
	
\end{corollary}

\begin{proof}
	
	Sea $F \subseteq X$ cerrado y consideremos $f|_F$, entonces si $K \subseteq Y$ es compacto entonces $f|_{F}^{-1}(K) = f^{-1}(K) \cap F$ que por \ref{Cerrado en un compacto es compacto} es compacto; luego por \ref{Equivalencias de propia} basta ver que $f(X)$ es cerrado. Como $Y$ es localmente compacto si consideramos $y \in \overline{f(X)}$ existe un entorno compacto $K \ni y$ y por hip\'otesis $f^{-1}(K)$ es compacto, entonces por \ref{Continua de un compacto en un haussdorff es cerrada} tenemos que $f|_{f^{-1}(K)}$ es cerrada.
	
	Por lo tanto, $f(f^{-1}(K)) = f(X )\cap K \ni y$ es cerrado; luego $y \in f(X) \cap Y \subseteq f(X)$, y por \ref{Equivalencias de propia} conclu\'imos que $f$ es propia.
	
\end{proof}

\subsection{Teorema de Tychonoff}

\begin{definition}
	
	Decimos que una familia $\sett{A_j}_{j \in J}$ de subconjuntos de $X$ un espacio topol\'ogico tiene la \textit{propiedad de intersecci\'on finita} si para cada  $J' \subset J$ finito se tiene que $\Bigcap{j \in J'}{A_j} \neq \emptyset$
	
\end{definition}


\begin{remark}
	
	De \ref{Caracterizacion de compacidad} se tiene que si $X$ es un espacio topol\'ogico compacto y $\sett{A_j}_{j \in J}$ es una familia con la PIF entonces $\Bigcap{j \in J}{\overline{A_j}} \neq \emptyset$
	
\end{remark}

\begin{lemma}
	
	\label{Lema 1 para Tchonoff}
	
	Sea $X$ un espacio topol\'ogico y sea $\F$ una familia con la PIF, entonces existe una familia $\mathcal{D}$ que cumple:
	
	\begin{enumerate}
		
		\item  $\F \subseteq \mathcal{D}$
		
		\item $\mathcal{D}$ tiene la PIF
		
		\item $\mathcal{D}$ es maximal respecto de la PIF
		
	\end{enumerate}
	
\end{lemma}

\begin{proof}
	
	Sea $\Lambda$ el conjunto de familia de subconjuntos de $X$ con la PIF y que contienen a $\F$ ordenados por la inclusi\'on. Entonces $\F \in \Lambda \neq \emptyset$ y si $\sett{F_\gamma}_{\gamma \in \Lambda}$ es una cadena en $\Lambda$ entonces $\epsilon = \Bigcup{\gamma \in \Lambda}{F_{\gamma}}$ es una cota superior.
	
	Finalmente si $A_1 , \dots , A_n \in \epsilon$ son tales que $A_i \subseteq F_{\gamma_i}$ entonces $A_1, \dots , A_n \in \F_{max \gamma_i}$ y como $\F_{max \gamma_i}$ tiene la PIF se da que $\Bigcap{1 \leq i \leq n}{A_i} \neq \emptyset$; por lo tanto $\epsilon$ tiene la PIF. Por \ref{Lema de Zorn} existe un elemento m\'aximal $\mathcal{D} \in \Lambda$. \qed
	
\end{proof}

\begin{lemma}
	
	\label{Lema para Tychonoff 2}
	
	Sea $\mathcal{D}$ una familia maximal  con respecto a la PIF. Entonces:
	
	\begin{enumerate}
		
		\item $\mathcal{D}$ es cerrado por intersecciones finitas
		
		\item Si $A \subseteq X$ es tal que $A \cap B \neq \emptyset$ para todo $B \in \mathcal{D}$ entonces $A \in \mathcal{D}$. 
		
		
	\end{enumerate}
	
\end{lemma}

\begin{proof}
	
	\begin{itemize}
		
		\item Si $A_1 , \dots, A_n \in \mathcal{D}$ entonces $\mathcal{D'} = \mathcal{D} \cup \sett{\Bigcap{1 \leq i \leq n}{A_i}}$ es una familia con la PIF, como $\mathcal{D}$ es maximal entonces $\mathcal{D} = \mathcal{D'}$.
		
		\item Sea $\mathcal{D'} = \mathcal{D} \cup A$y veamos que esta familia tiene la PIF. En efecto, si $B_1, \dots , B_n \in \mathcal{D}$ entonces por el item anterior $\Bigcap{1 \leq i \leq n}{B_i} \in D$ y por hip\'otesis $\Bigcap{1 \leq i \leq n}{B_i} \cap A \neq \emptyset$. Como $\mathcal{D}$ es maximal respecto a la PIF conclu\'imos que $A \in \mathcal{D}$. \qed
		
	\end{itemize}
	
\end{proof}

\begin{theorem}
	
	\label{Teorema de Tychonoff}
	
	Sea $\sett{X_j}_{j \in J}$ una familai de espacios topol\'ogicos compactos, entonces $\Bigprod{j \in J}{X_j}$ es compacto.
	
\end{theorem}

\begin{proof}
	
	Sea $F = \sett{B_j}_{j \in J}$ una familia de subconjuntos de $X = \Bigprod{j \in J}{X_j}$ con la PIF y sea $\mathcal{D}$ la familia dada por \ref{Lema 1 para Tchonoff}. Para cada $j \in J$ tenemos que $\sett{p_j(A) \ / \ A \in \mathcal{D}}$ es una familia con la PIF pues si $A_1 , \dots ,  A_n \in \mathcal{D}$ entonces $\Bigcap{1 \leq i \leq n}{p_j(A_i)} \subseteq p_j(\Bigcap{1 \leq i \leq n}{A_i}) \neq \emptyset$; por lo tanto como $X_j$ es compacto existe $x_j \in \Bigcap{A \in \mathcal{D}}{\overline{p_j(A)}}$.
	
	Definimos $x \in X$ dado por $(x)_j = x_j$ y veamos que $x \in \Bigcap{A \in \mathcal{D}}{\overline{A}}$. Para esto sea $V \ni x$ un entorno b\'asico de $x$ y tenemos que ver que $V \cap A \neq \emptyset$ para todo $A \in \mathcal{D}$, y por \ref{Lema para Tychonoff 2} basta ver que $V \in \mathcal{D}$. Por \ref{Lema 1 para Tchonoff} basta tomar $V$ entorno subb\'asico, luego sea $V = p_{j}^{-1}(U)$ con $U \subseteq X_j$ abierto tal que $x_j \in V$; como $x_j \in \Bigcap{A \in \mathcal{D}}{\overline{p_{j}(A)}}$ entonces $V \cap p_j(A) \neq \emptyset$ para todo $A \in \mathcal{D}$. Por lo tanto $p_{j}^{-1}(V) \cap A \neq \emptyset$ para todo $A \in \mathcal{D}$, por \ref{Lema para Tychonoff 2} se tiene que $p_{j}^{-1}(V) \in \mathcal{D}$, luego $x \in \Bigcap{A\in \mathcal{D}}{\overline{A}} \subseteq \Bigcap{A \in \F}{\overline{A}} \neq \emptyset$ y por \ref{Caracterizacion de compacidad} se tiene que $X$ es compacto. \qed
	
\end{proof}

\subsection{Compactificaci\'on de Alexandroff}

\begin{definition}
	
	Sea $X$ un espacio topol\'ogico $T_2$, localmente compacto y no compacto; entonces se define la \textit{compactificaci\'on de Alexandroff o compactificaci\'on de un punto} de $X$ como el espacio $X^* = X \cup \sett{\infty}$ y con la topolog\'ia $\tau = \tau_X \cup \sett{X^* \setminus C \ / \ C \subseteq X \emph{ compacto}}$
	
\end{definition}

\begin{remark}
	
	Es simple ver que $\tau$ es una topolog\'ia.
	
\end{remark}

\begin{proposition}
	
	\label{Compactificacion de un punto}
	
	Sea $X$ un espacio topol\'ogico $T_2$, localmente compacto y no compacto; entonces $X^*$ es $T_2$, compacto y $X$ es un subespacio denso de $X^*$
	
\end{proposition}

\begin{proof}
	
	Vayamos por partes:
	
	\begin{itemize}
		
		\item Si $x \neq y \in X$ entonces se pueden separar pues $X$ es $T_2$; por otro lado si $x \in X$ como $X$ es localmente compacto existe $C \subseteq X$ compacto y $U \subseteq C$ abierto tal que $x \in U \subseteq C$, por lo tanto $\sett{U , X^* \setminus C}$ separan a $x$ de $\infty$.
		
		\item Si $\mathcal{U}$ es un cubrimiento abierto de $X^*$ entonces existe $X^* \setminus C \in \mathcal{U}$; por lo tanto $C \subseteq \Bigcup{\substack{U \in \mathcal{U} \\ U \neq X^* \setminus C}}{U}$ y como $C$ es compacto existe $\mathcal{U} ' $ subcubrimiento finito. Se concluye que $X^* = \Bigcup{U \in \mathcal{U'}}{U} \cup X^* \setminus C$.
		
		\item Por un lado si $U \subseteq X$ es abierto entonces $U \subseteq X^*$ es abierto y $U = U \cap X$; por el otro lado si $U \subseteq X^*$ es abierto de $X^*$ y no de $X$ entonces $U = X^* \setminus C$ por lo que $U \cap X = X \setminus C$ es abierto de $X$. De esto conclu\'imos que si $U \subseteq X$ es abierto de $X$ entonces $U = i^{-1}(V)$ con $V \subseteq X^*$ abierto; entonces $X$ es subespacio.
		
		\item Sea $X^* \setminus C$ un entorno b\'asico de $\infty$, como $X$ no es compacto existe $y \in X \setminus C$ y por lo tanto $y \in X \cap (X^{*} \setminus C) \neq \emptyset$ y se sigue que $\infty \in \overline{X}$, conclu\'imos que $\overline{X} = X^*$. \qed
		
	\end{itemize}
	
\end{proof}

\begin{example}
	Se tiene f\'acilmente que $\R^* = S^1$
\end{example}

\section{Axiomas de separaci\'on}

\begin{definition}
	
	Un espacio topol\'ogico $X$ se dice \textit{regular} o $T_3$ si es $T_1$ y adem\'as dados $F \subseteq X$ cerrado y $x \not \in F$ entonces existen $U,V$ abiertos disjuntos tal que $x \in U$, $F \subseteq V$
	
\end{definition}

\begin{definition}
	
	Un espacio topol\'ogico $X$ se dice \textit{completamente regular} o $T_4$ si es $T_1$ y adem\'as dados $F \subseteq X$ cerrado y $x \not \in F$ entonces existe $f: X \rightarrow I$ continua tal que $f(x) = 0$ y  $f(F) =1$
	
\end{definition}

\begin{definition}
	
	Un espacio topol\'ogico $X$ se dice \textit{normal} o $T_5$ si es $T_1$ y adem\'as dados $F,H \subseteq X$ cerrados disjuntos entonces existen $U,V$ abiertos disjuntos tal que $H \subseteq U$, $F \subseteq V$
	
\end{definition}

\begin{example}
	
	Sea $X = (\sett{0,1,2}, \sett{\sett{\emptyset}, X, \sett{0}, \sett{1,2}})$ entonces $X$ no es $T_1$ pero al ser disconexo cumple los otros 3 requerimientos. 
	
\end{example}

\begin{remark}
	
	Notemos que $T_4 \Longrightarrow T_3 \Longrightarrow T_2$.
	
	En efecto, si $X$ es $T_4$ y $x \not \in F$ entonces tomemos $U= f^{-1}([0,\frac{1}{4}))$ y $V = f^{-1}([\frac{3}{4},1))$ y se tiene que $U \cap V = \emptyset$ y $F \subseteq V, x \in U$; como adem\'as por \ref{Subespacio y producto de T1 es T1} se tiene que $X$ es $T_3$. Similarmente si $x \neq y$ entonces como $X$ es $T_1$ los puntos son cerrados y por ser $T_3$ existen $V,W$ disjuntos tal que $x \in W$, $y \in V$, por lo que $X$ es $T_2$.
	
\end{remark}

\begin{proposition}
	
	\label{Caracterizacion de T3}
	
	Sea $X$ un espacio $T_1$, entonces son equivalentes:
	
	\begin{enumerate}
		
		\item X es $T_3$ 
		
		\item Para todo $x \in X$ y todo entorno $U \ni x$ existe un abierto $V$ tal que $x \in V \subseteq \overline{V} \subseteq U$
		
		
	\end{enumerate}
	
	
\end{proposition}

\begin{proof}
	
	Hagamos por partes:
	
	\begin{itemize}
		
		\item[i) $\Longrightarrow$ ii)] Supongamos que el entorno $U$ es abierto, entonces $x \not \in ^C$, como $X$ es $T_3$ existen $V,W \subseteq X$ abiertos tal que $x \in V$ y $U^c \subseteq W$. Como $V \cap W = \emptyset$ entonces $V \subseteq W^c$ que es cerrado, por lo tanto tenemos que $\overline{V} \subseteq W^c$; conclu\'imos que $x \in V \subseteq \overline{V} \subseteq W^c \subseteq U$. 
		
		\item[ii) $\Longrightarrow$ i)] Sea $F \subseteq X$ cerrado y $x \not \in F$, entonces $F^c$ es un entorno abierto de $x$ por lo que existe $V \subseteq X$ abierto tal que $x \in V \subseteq \overline{V} \subseteq F^c$. Es claro que si tomamos $U = V$ y $W = (\overline{V})^c$ entonces estos son abiertos disjuntos tal que $x \in U$, $F \subseteq W$. \qed
		
	\end{itemize}
	
\end{proof}

\begin{proposition}
	
	\label{Sespacio y producto de T3 es T3}
	
	Subespacio de un espacio $T_3$ es $T_3$ y produtos de $T_3$ es $T_3$.
	
\end{proposition}

\begin{proof}
	
	\begin{itemize}
		
		\item Sea $F \subseteq Y \subseteq X$ un cerrado en $Y$ que es subespacio de $X$ y consideremos $x \in Y \setminus F$, entonces como $F$ es cerrado en el subespacio existe $H \subseteq X$ cerrado tal que $F = Y \cap H$, por lo tanto $y \not \in H$. Como $X$ es $T_3$ existen $U,V \subseteq X$ abiertos disjuntos tal que $x \in U$, $H \subseteq V$; si tomamos $U' = U \cap Y$ y $V' = V \cap Y$ entonces estos son abiertos disjuntos en $Y$ tal que $y \in U'$ y $F \subseteq V'$. Como adem\'as por \ref{Subespacio y producto de T1 es T1} $Y$ es $T_1$ conclu\'imos que $Y$ es $T_3$.
		
		\item Sea $\sett{X_j}_{j \in J}$ una familia de espacios $T_3$, entonces $X = \Bigprod{j \in J}{X_j}$ es $T_1$ por \ref{Subespacio y producto de T1 es T1}. Sea ahora $x \in U \subseteq X$ un entorno abierto, entonces tenemos que $x_j \in U_j \subseteq X_j$ para todo $j \in J$ y como $X_j$ es $T_3$ existe $V_j$ abierto tal que $x_j \in V_j \subseteq \overline{V}_j \subseteq U_j$. Sea $V = \Bigprod{j \in J}{V_j}$ donde tomaremos $V_j = X_j$ si $U_j = X_J$, entonces $V$ es abierto de $X$ y cumple que $x \in V$, usemos el siguiente lema:
		
		\begin{lemma}
			
			\label{Clausura en topologia producto}
			
			Si $V \subseteq X$ en la topolog\'ia producto, entonces $\overline{V} = \overline{\Bigprod{j \in J}{V_j}} = \Bigprod{j \in J}{\overline{V_j}}$
			
		\end{lemma}
		
		Entonces usando \ref{Clausura en topologia producto} tenemos que $x \in V \subseteq \overline{V} = \Bigprod{j \in J}{\overline{V_j}} \subseteq \Bigprod{j \in J}{U_j} = U$; conclu\'imos que $X$ es $T_3$. \qed
		
	\end{itemize}
	
\end{proof}

\begin{proposition}
	
	\label{Si X es compacto y Haussdorf entonces es normal}
	
	Sea $X$ un espacio topol\'ogico compacto y $T_2$, entonces es normal.
	
\end{proposition}

\begin{proof}
	
	Por un lado como $X$ es $T_2$ entonces es $T_1$. Por el otro lado sean $F,H \subseteq X$ cerrados disjuntos, entonces por \ref{Cerrado en un compacto es compacto} tenemos que son ambos compactos. Sea $x \in F$, entonces por \ref{Si X es compacto y haussdorf entonces es T3} existe $U_x,V_x \subseteq X$ abiertos disjuntos tal que $x \in U_x, H \subseteq V_x$; por lo tanto se tiene que $F = \Bigcup{x \in F}{\sett{x}} \subseteq \Bigcup{x \in F}{U_x}$. Como $F$ es compacto existe $J = \sett{x_1 , \dots, x_n}$ finito tal que $F \subseteq \Bigcup{x \in J}{U_x}$, por lo tanto si tomamos $U = \Bigcup{x \in J}{U_x}$ y $V = \Bigcap{x \in J}{V_x}$ veamos que estos sirven para separar $F$ de $H$:
	
	\begin{itemize}
		
		\item Ambos son abiertos pues $J$ es finito
		
		\item $F \subseteq U$ por construcci\'on
		
		\item $ H \subseteq V_x$ para todo $x \in F$, luego $H \subseteq \Bigcap{x \in F}{V_x} \subseteq V$
		
		\item $U \cap V = \Bigcup{x \in J}{U_x \cap V} \subseteq \Bigcup{x \in J}{U_x \cap V_x} = \emptyset$.
		
	\end{itemize}
	
	Cnoclu\'imos que $X$ es normal. \qed
	
\end{proof}

\begin{proposition}
	
	\label{Subespacio y producto de T4 es T4}
	
	Subespacio y productos de $T_4$ es $T_4$
	
\end{proposition}

\begin{proof}
	
	\begin{itemize}
	
		\item Sea $Y \subseteq X$ un subespacio de $X$ que es $T_4$, luego por \ref{Subespacio y producto de T1 es T1} se tiene que $Y$ es $T_1$. Sea $F \subseteq Y$ un cerrado e $y \in Y \setminus F$, luego existe $H \subseteq X$ cerrado tal que $F = H  \cap Y$ y se sigue que $y \not \in H$. Como $X$ es $T_4$ existe una funci\'on continua $f : X \rightarrow I$ tal que $f(y) = 0$ y $f(H) = \sett{1}$; como $Y$ es subespacio se tiene por \ref{Restriccion de continua a subespacio es continua} que $f|_Y : Y \rightarrow I$ es continua y $f|_Y(y) = f(y) = 0$ y $f|_Y(F) = f(F) \subseteq f(H) = \sett{1}$, y como $F \neq \emptyset$ se tiene que $f(F) = \sett{1}$.
	
		\item Sea $\sett{X_j}_{j \in J}$ una familia de espacios $T_4$, $x \in X = \Bigprod{j \in J}{X_j}$ y $F \subseteq X$ cerrado tal que $x \not \in F$. Como $x \in F^c = U$ abierto entonces $x_j \in U_j \subseteq X_j$ para todo $j \in J$, luego $x_j \not \in U_{j}^c$ que es cerrado de $X_j$. Consideremos $J' \subset J$ el conjunto finito donde $U_{j}^c \neq \emptyset$, luego como $X_j$ es $T_4$ para todo $j \in J \subset J$ existe una funci\'on continua $f_j : X_j \rightarrow I$ tal que $f_j(x_j) = 1$ y $f_j(U_{j}^c) = \sett{0}$ (Basta tomar $f'_j = 1 - f_j$ para tener la usual). Consideremos entonces $f : X \rightarrow I$ dada por $f(y) = \Bigprod{j \in J'}{f_j(p_j(y))}$ que es continua por ser un producto finito de continuas; entonces $f(x) = \Bigprod{j \in J'}{f_j(x_j)} = 1$ y si $y \in F$ entonces $f(y) = \Bigprod{j \in J'}{f_j(y_j)} = 0$ pues $y \not \in U$ por lo que $y \in U_{j_0}^c$ para alg\'un $j_0$. Basta tomar $f' = 1-f$ para obtener la $f$ como en la definici\'on y conclu\'imos que $X$ es $T_4$. \qed
	
	\end{itemize}
	
\end{proof}

\begin{proposition}
	
	\label{Si X es localmente compacto y Haussdorf entonces es T4}
	
	Sea $X$ un espacio topol\'ogico localmente compacto y $T_2$, entonces es $T_4$.	
	
\end{proposition}

\begin{proof}
	
	Sea $X^*$ la compactificaci\'on de un punto de $X$, luego es compacto y $T_2$ por \ref{Compactificacion de un punto} por lo que por \ref{Si X es compacto y Haussdorf entonces es normal} $X^*$ es normal, por lo tanto por \ref{Si X es T5 entonces es T4} se tiene que $X^*$ es $T_4$ y como $X$ es un subespacio de $X^*$ por \ref{Subespacio y producto de T4 es T4} conclu\'imos que $X$ es $T_4$. \qed
	
\end{proof}



\begin{proposition}
	
	\label{Todo conjunto bien ordenado es normal}
	
	Todo conjunto $X$ bien ordenado es normal
	
\end{proposition}

\begin{proof}
	
	Es m\'as, vale que todo conjunto en la topolog\'ia del orden es normal, pero para esto es m\'as complicado y se usa el Axioma de Elecci\'on (Demo entendible: Counterexamples in Topology). Vayamos por partes:
	
	\begin{enumerate}
		
		\item Primero afirmamos que todo conjunto $(x,y] \subseteq X$ es abierto.
		
		En efecto, si $y = max(X)$ entonces $(x,y] \in \tau_O$ es un elemento b\'asico; y si no entonces como $X$ es bien ordenado existe un elemento estrictamente siguiente $y' > y$ por lo que $(x,y] = (x,y') \in \tau_O$.
		
		\item Si $A,B$ son cerrados disjuntos tal que ninguno contiene a $a_0 = min(X)$ entonces existen $U,V$ abiertos disjuntos que los separan.
		
		En efecto, sea $a \in A$ entonces como $a \not \in \overline{B}$ existe un abierto b\'asico $U_a \ni a$ tal que $U_a \cap B = \emptyset$, luego como $a$ no es m\'inimo existe un abierto $a \in  (x,a] \subseteq U_a$; similarmente para cada $b \in B$ tomemos un intervalo $(y_b,b]$  disjunto de $A$. Entonces si consideramos:
		
		\[
		U = \Bigcup{a \in A}{(x_a,a]} \quad V = \Bigcup{b \in B}{(y_b,b]}
		\]
		
		Es claro que $U,V$ son abiertos y $A \subseteq U, B \subseteq V$, veamos que son disjuntos.
		
		Supongamos $a < b$ y consideremos $z \in (x_a,a] \cap (y_b,b]$ entonces tenemos que $a \in (y_b,b]$ que es contrario a la elecci\'on de $y_b$ (Aca nuevamente estamos usando el buen orden para ver que existe tal $y_b$), luego $U \cap V = \emptyset$.
		
		\item Si $A,B$ son cerrados disjuntos entonces existen $U,V$ abiertos disjuntos que los separan.
		
		En efecto si $a_0 \in A$ entonces $\sett{a_0}$ es cerrado y abierto en $X$, por lo tanto $A - \sett{a_0},B$ son cerrados disjuntos y entoces por el item anterior existen $U,V$ tal que los separan, basta considerar $U \cup \sett{a_0} , V$. \qed
		
		
	\end{enumerate}
	
\end{proof}

\begin{corollary}
	
	\begin{itemize}
		
		\item Subespacio de $T_5$ no necesariamente es $T_5$
		
		\item Producto de $T_5$ no necesariamente es $T_5$
		
	\end{itemize}
	
\end{corollary}

\begin{proof}
	
	Consideremos $S_{\Omega}$ que es $T_5$ por \ref{Todo conjunto bien ordenado es normal} , $\overline{S_\Omega}$ que es $T_5$ por \ref{Si X es compacto y Haussdorf entonces es normal}
	y $\overline{S_{\Omega}} \times \overline{S_{\Omega}}$ que es compacto por \ref{Teorema de Tychonoff} y por lo tanto normal por \ref{Si X es compacto y Haussdorf entonces es normal}. Afirmo que $X = S_{\Omega} \times \overline{S_{\Omega}}$ (que es producto de $T_5$) visto como subespacio de $\overline{S_{\Omega}} \times \overline{S_{\Omega}}$ (Y por lo tanto es subespacio de $T_5$ adem\'as) no es normal.
	
	En efecto sea $A = \Delta \setminus \Omega \times \Omega$ que es cerrado pues $A = \Delta \cap X$ donde $\Delta \subseteq \overline{S_{\Omega}} \times \overline{S_{\Omega}}$ es cerrado pues $\overline{S_{\Omega}}$ es $T_2$ y \ref{Caracterizacion de T2 y equivalencias}; por otro lado sea $B = S_{\Omega} \times \sett{\Omega} = (\overline{S_{\Omega}} \times \sett{\Omega}) \cap X$ que es cerrado pues $\overline{S_{\Omega}} \times \sett{\Omega}$ es producto de cerrados de $\overline{S_{\Omega}} \times \overline{S_{\Omega}}$ y adem\'as cumple que $A \cap B = \emptyset$.
	
	Supongamos que existen $U,V$ abiertos disjuntos tal que $A \subseteq U$, $B \subseteq V$ y dado $x \in S_{\Omega}$ consideremos el conjunto $\sett{x} \times \overline{S_{\Omega}}$ y veamos que existe $x < \beta < \Omega$ tal que $(x,\beta) \not \in U$. Si para todo $\beta > x$ se tuviese que $(x,\beta) \in U$ entonces se tiene f\'acilmente que $V \ni (x,\Omega) \in \overline{U}$ y contradicir\'iamos la hip\'otesis que $U \cap V = \emptyset$; por lo tantopara cada $x \in S_{\Omega}$ llamemos $\beta(x) = min(\sett{\beta \in S_{\Omega} \ / \  \Omega > \beta > x \ / \ (x,\beta) \not \in U})$ que existe pues probamos que es un conjunto no vac\'io. 
	
	Sea $x_1 = x$ y definamos recursivamente $x_n = \beta(x_{n-1})$, entonces $\sett{x_n}_{n \in \N}$ es una sucesi\'on creciente de elementos y por \ref{Si A es numerable en SOmega entonces tiene cota superior} tiene cota superior en $S_{\Omega}$. Sea $b = min(\sett{ b \in S_{\Omega} \ / \ b \geq x_n \ \forall n \in \N})$, entonces como $x_n$ es creciente se tiene que $x_n \rightarrow b$ y $\beta{x_n}=x_{n+1} \rightarrow b$, por lo que:
	
	\[
	x_n \times \beta(x_n) \rightarrow b \times b
	\]
	
	Pero tenemos que $b \times b \in A$ pues ninguna sucesi\'on converge a $S_{\Omega}$ pero $x_n \times \beta(x_n) \not \in U$ para todo $n \in \N$; conclu\'imos que no exist\'ian tales $U,V$ abiertos que separen $A$ y $B$, por lo tanto $S_{\Omega} \times \overline{S_{\Omega}}$ no es normal. \qed
	
\end{proof}

\begin{proposition}
	
	Sea $X$ un espacio $T_1$, entonces son equivalentes:
	
	\begin{enumerate}
		
		\label{Caracterizacion de T5}
		
		\item $X$ es $T_5$
		
		\item para todo $F \subseteq X$ cerrado y $U \subseteq X$ abierto tal que $F \subseteq U$ entonces existe $	V \subseteq X$ abierto tal que $F \subseteq V \subseteq \overline{V} \subseteq U$		
		
	\end{enumerate}
	
\end{proposition}

\begin{proof}
	
	\begin{itemize}
		
		\item[i) $\Longrightarrow$ ii)]  Sea $F \subseteq U$, luego $F,U^c$ son cerrados disjuntos y como $X$ es $T_5$ se tiene que existen $V,W \subseteq X$ abiertos disjuntos tal que $F \subseteq V , U^c \subseteq W$, luego se tiene que $F \subseteq V \subseteq \overline{V} \subseteq W^c \subseteq U$.
		
		\item [ii) $\Longrightarrow$ i)] Sean $F,H$ cerrados disjuntos, luego $F \subseteq H^c$ que es abierto y entonces existe $V \subseteq X$ abierto tal que $F \subseteq V \subseteq \overline{V} \subseteq H^c$; basta considerar $W = \overline{V}^c$ para ver que $V,W$ son abiertos disjuntos y que separan. \qed
		
	\end{itemize}

\end{proof}

\begin{theorem}{Lema de Urysohn}
	
	\label{Lema de Urysohn}
	
	Sea $X$ un espacio topol\'ogico $T_5$ y $F,H \subseteq X$ cerrados disjuntos no vac\'ios, entonces existe una funci\'on continua $f: X \rightarrow I$ tal que $f(F) = \sett{0}$ y $f(H) = \sett{1}$
	
	
\end{theorem}

\begin{proof}
	
	Sea $\sett{q_0,q_1, \dots}$ una numeraci\'on de $\Q$ de modo que $q_0 = 0$ y $q_1 = 1$, y definimos $U_1 = H^c$; luego como $F \subseteq U_1$ y $X$ es normal, por \ref{Caracterizacion de T5} existe $U_0 \subseteq X$ abierto tal que $F \subseteq U_0 \subseteq \overline{U_0} \subseteq U_1$. En particular como $\overline{U_0} \subseteq U_1$ y $X$ es normal existe $U_{q_2} \subseteq X$ abierto tal que $\overline{U_0} \subseteq U_{q_2} \subseteq \overline{U_{q_2}} \subseteq U_1$; recursivamente definimos $U_{q_i}$ abiertos tal que si $q_i < q_j$ entonces $\overline{U_{q_i}} \subseteq U_{q_j}$. De esta manera tenemos definido $U_q$ para todo $q \in \Q \cap I$ tal que $F \subseteq U_0$, $H = U_{1}^c$ y si $r < q$ entonces $\overline{U_r} \subseteq U_q$.
	
	Definimos por completitud $U_q = \emptyset$ si $q \in \Q \cap \R_{<0}$ y $U_q = X$ si $q \in \Q \cap \R_{>1}$ y finalmente sea $f : X \rightarrow \R$ dada por $f(x) = inf(\sett{r \in \Q \ / \ x \in U_r})$, veamos que esta $f$ nos sirve:
	
	\begin{itemize}
		
		\item $f$ est\'a bien definida:
		
		En efecto, se tiene que $2 \in \sett{r \in \Q \ / \ x \in U_r} \neq \emptyset$ para todo $x \in X$ y adem\'as est\'a acotado inferiormente pues $r \not \in \sett{r \in \Q \ / \ x \in U_r}$ para todo $r < 0$. Adem\'as se sigue que $f(X) \subseteq I$
		
		\item $f(F) = \sett{0}$
		
		Si $x \in F$ entonces $x \in U_0 \subseteq U_r$ para todo $r \in \Q \cap \R_{\geq 0}$, por lo tanto $\sett{r \in \Q \ / \ x \in U_r} = [0, \infty) \cap \Q$ y se tiene que $f(x) = 0$. Como adem\'as $F \neq \emptyset$ se tiene que $f(F) = \sett{0}$
		
		\item $f(H) = \sett{1}$
		
		Si $x \in H$ entonces $x \not \in U_1$ y por lo tanto $x \not \subseteq U_r$ para todo $r \in \Q \cap \R_{<1}$, por lo tanto $\sett{r \in \Q \ / \ x \in U_r} = (1, \infty) \cap \Q$ y se tiene que $f(x) = 1$. Como adem\'as $H \neq \emptyset$ se tiene que $f(H) = \sett{1}$
		
		\item $f$ es continua
		
		Sea $(a,b) \subseteq \R$ y sea $x \in f^{-1}(a,b)$, sean $q,r \in \Q$ ta que $a < q < f(x) < r < b$ y consideremos $U = U_r \setminus \overline{U_q}$, luego $U$ es abierto.
		
		Como $f(x) < r$ existe $r' \in \Q$ tal que $f(x) < r' < r$ y $x \in U_{r'}$, por lo que $x \in U_r$. Por otro lado, como $q < f(x)$ si $q < q' < f(x)$ entonces $x \not \in U_{q'}$ y como $\overline{U_q} \subseteq U_{q'}$ se tiene que $x \not \in \overline{U_{q'}}$; conclu\'imos que $x \in U$.
		
		Finalmente, sea $y \in U$ por lo que $y \not \in U_{q'}$ para todo $q' < q$ y eso implica que $f(y) \geq q > a$; por otro lado $y \in U_r$ por lo que $f(y) \leq r < b$. Juntando todo se tiene que $U \subseteq f^{-1}(a,b)$ y por lo tanto es abierto y $f$ continua. \qed
		
	\end{itemize}
	
\end{proof}

\begin{corollary}
	
	\label{Si X es T5 entonces es T4}
	
	Si $X$ es un espacio normal, entonces es $T_4$
	
\end{corollary}

\begin{proof}
	
	Como $X$ es $T_1$ se tiene que $\sett{x}$ es cerrado, entonces si $x \not \in F$ por \ref{Lema de Urysohn} dados existe $f : X \rightarrow I$ continua tal que $f(\sett{x}) = 0$ y $f(F) = 1$. Lego $X$ es $T_4$ \qed
	\qed	

\end{proof}

\begin{remark}
	
	Notemos que $T_2 \not \Longrightarrow T_3 \not \Longrightarrow T_4 \not \Longrightarrow T_5 \not \Longrightarrow \emph{Espacio m\'etrico}$.
	
	En efecto veamos uno por uno:
	
	\begin{itemize}
		
		\item  Sea $(\R, \tau_K)$ la \textit{$K$ topolog\'ia en $\R$} dada por la base $\B_K = \sett{(a,b) \ / \ a < b} \cup \sett{(a,b) \setminus K \ / \ a < b \ , \ K = \sett{\frac{1}{n}}}$. Entonces se tiene que $\tau \subseteq \tau_K$ trivialmente pues una base contiene a la otra, y por lo tanto como $(\R , \tau)$ es $T_2$ (es m\'etrico) se tiene que $\R_K = (\R , \tau_K)$ es $T_2$.
		
		No obstante, sean $\sett{0}$  que es cerrado pues al ser $T_2$ es $T_1$, y sea $K$ que es cerrado pues $K = (\R \setminus K) ^c$ y sea $U \ni 0$ un entorno abierto de $0$. Entonces existe $n_0 \in \N$ tal que $\frac{1}{n} \in U$ para todo $n \geq n_0$ y conclu\'imos que $U \cap K \neq \emptyset$. Panto no podemos separar por abiertos disjuntos a $\sett{0}$ y a $K$ y se tiene que $\R_K$ no es $T_3$.
		
		\item Un espacio $T_3$ pero no $T_4$ es muy dif\'icil
		
		\item Si consideramos ${S_{\Omega}} \times \overline{S_{\Omega}}$ ya vimos que no es normal, pero ${S_{\Omega}} \times \overline{S_{\Omega}} \subset \overline{S_{\Omega}} \times \overline{S_{\Omega}}$ y como $\overline{S_{\Omega}} \times \overline{S_{\Omega}}$ es compacto y $T_2$ por \ref{Si X es compacto y Haussdorf entonces es normal} es normal, entonces por \ref{Si X es T5 entonces es T4} es $T_4$ y por \ref{Subespacio y producto de T4 es T4} se tiene que ${S_{\Omega}} \times \overline{S_{\Omega}}$ es $T_4$.
		
		\item Ya vimos que $\overline{S_{\Omega}}$ es $T_5$ pero no es m\'etrico.
		
	\end{itemize}
	
\end{remark}

\begin{theorem}
	
	\label{Caracterizacion de T4 por insercion en copias de I}
	
	Sea $X$ un espacio topol\'ogico, entonces $X$ es $T_4$ si y s\'olo si es subespacio de un producto de copias de $I$.
	
\end{theorem}

\begin{proof}
	
	Para un lado es trivial que como $I$ es compacto y $T_2$ en la topolog\'ia usual entonces $\Bigprod{j \in J}{I}$ es compacto por \ref{Teorema de Tychonoff} y $T_2$ por \ref{Subespacio y producto de T2 es T2}, luego normal por \ref{Si X es compacto y Haussdorf entonces es normal} y entonces $T_4$ por \ref{Si X es T5 entonces es T4}; conclu\'imos por \ref{Subespacio y producto de T4 es T4}.
	
	Para el otro consideremos $f : X \rightarrow \Bigprod{\substack{h : X \rightarrow I \\ h  \emph{continua}}}{I} = K$ dada por $f(x)_h = h(x)$. Es claro que $f$ es continua pues $P_h f = h$ es continua para todo $h:X \rightarrow I$ continua y $\sett{P_h}_{\substack{h : X \rightarrow I \\ h  \emph{continua}}}$ es inicial para $K$.
	
	Sea ahora $\tau_X$ la topolog\'ia de $X$ y $\tau'$ otra topolog\'ia en $X $ que hace continua a las $h$; sea $U \in \tau_X$ y $x \in U$ como $X$ es $T_4$ sabemos que existe $h_x : (X,\tau) \rightarrow I$ continua tal que $h_x(x) = 0$ y $h_x(U^c) = 1$. Por lo tanto por hip\'otesis $h_{x}^{-1}([0,\frac{1}{2})) \in \tau'$ y conclu\'imos que $U \subseteq \Bigcup{x \in U}{h_{x}^{-1}([0,\frac{1}{2}))} \in \tau'$. Por lo tanto $\tau \subseteq \tau'$ y $f$ es inicial.
	
	Finalmente si $x \neq y$ entonces como $X$ es $T_4$ existe $h: X \rightarrow I$ continua tal que $h(x) = 0$ y $h(y) = 1$, por lo tanto $f(x) \neq f(y)$ pues $f(x)_h \neq f(y)_h$; se sigue que $f$ es inyectiva y por lo tanto subespacio. \qed
	
	
\end{proof}

\section{Compactificaci\'on de Stone-Cech}

\begin{lemma}
	
	\label{Dos funciones que coinciden en un denso con codominio T2 coinciden en todos lados}
	
	Sean $f,g : X\rightarrow $ funciones continuas tal que $f|_A = g|_A$ para alg\'un $A \subseteq X$ denso, entonces si $Y$ es $T_2$ se tiene que $f = g$.
	
\end{lemma}

\begin{proof}
	
	Sea $x \in X$, luego como $A$ es denso por \ref{Caracterizacion de la clausura por redes} existe $\sett{x_{\alpha}}_{\alpha \in \Lambda}$ red en $A$ tal que $x_{\alpha} \rightarrow x$, luego por \ref{Equivalencias de continuidad} se tiene que $g(x) \leftarrow g(x_{\alpha}) = f(x_{\alpha}) \rightarrow f(x)$ y por \ref{Caracterizacion de T2 y equivalencias} se tiene que $f(x) = g(x)$. \qed
	
\end{proof}

\begin{definition}
	Sea $X$ un espacio topol\'ogico e $i : X \rightarrow \Bigprod{\substack{h : X \rightarrow I \\ h  \emph{continua}}}{I}$, definimos $\beta(X) = \overline{i(X)} \subseteq \Bigprod{}{I}$ la \textit{compactificaci\'on de Stone-Cech} de $X$.
	
\end{definition}

\begin{remark}
	
	Notar que $\beta(X)$ es $T_2$ por \ref{Subespacio y producto de T2 es T2} y adem\'as como es cerrado en $\Bigprod{\substack{h : X \rightarrow I \\ h  \emph{continua}}}{I}$ tenemos por \ref{Teorema de Tychonoff} y \ref{Cerrado en un compacto es compacto} que es compacto.
	
\end{remark}

\begin{lemma}
	
	\label{Lemma para Stone-Cech1}
	
	Supongamos $f: X \rightarrow I$ una funci\'on continua, entonces existe una \'unica $g : \beta(X) \rightarrow I$ tal que el siguiente diagrama conmute:
	
	\begin{equation*}
	\begin{tikzcd}
	X \arrow{r}{i} \arrow[swap, bend right]{dr}{f} & \beta(X) \arrow[swap, dashed]{d}{\exists ! g} \\ 
	& I
	\end{tikzcd}
	\end{equation*}	
	
\end{lemma}

\begin{proof}
	
	En efecto, si consideramos $P_f i (X) = (i(X))_f = f(X)$ por lo tanto tenemos que $g = P_f|_{\beta(X)}$ cumple lo pedido. Es m\'as si $g'$ es tal que $g' i =f$ entonces como $g|_{i(X)} = g'|_{i(X)}$ y $\overline{i(X)} = \beta(X)$ se tiene por la observaci\'on que $g' = g$. \qed
	
\end{proof}

\begin{theorem}
	
	\label{Propiedad de la compactificacion Stone-Cech}
	
	Sea $X$ un espacio topol\'ogico e $i : X \rightarrow \beta(X)$ su compactificaci\'on de SC, entonces para todo $K$ espacio compacto y $T_2$ y $f : X \rightarrow K$ continua existe una \'unica $g : \beta(X) \rightarrow K$ tal que el siguiente diagrama conmuta:
	
	\begin{equation*}
	\begin{tikzcd}
	X \arrow{r}{i} \arrow[swap, bend right]{dr}{f} & \beta(X) \arrow[swap, dashed]{d}{\exists ! g} \\ 
	& K
	\end{tikzcd}
	\end{equation*}		
	
\end{theorem}

\begin{proof}
	
	Consideremos el diagrama:
	
	\begin{equation*}
	\begin{tikzcd}
	X \arrow{r}{i} \arrow[swap, bend right]{dr}{f} & i(X) \arrow[hook]{r}{} & \beta(X) \arrow[dashed, swap, bend left]{dr}{\exists ! g_h} \\ 
	& K \arrow{r}{i_K} & \Bigprod{\substack{h : K \rightarrow I \\ h  \emph{continua}}}{I} \arrow{r}{P_h} & I  
	\end{tikzcd}
	\end{equation*}	
	
	Luego como $P_hi_kf$ es continua por \ref{Lemma para Stone-Cech1} existe una \'unica funci\'on $g_h : \beta(X) \rightarrow I$ tal que $P_hi_kf = g_hi$. Luego por \ref{Propiedad universal del producto} existe una \'unica $g : \beta(X) \rightarrow \Bigprod{\substack{h : K \rightarrow I \\ h  \emph{continua}}}{I}$ continua tal que $P_hg = g_h$ para todo $h: K \rightarrow I $ continua.
	
	Por lo tanto hasta ahora el siguiente diagrama conmuta:
	
	\begin{equation*}
	\begin{tikzcd}
	X \arrow{r}{i} \arrow[swap, bend right]{dr}{f} & i(X) \arrow[hook]{r}{} & \beta(X) \arrow[swap, dashed]{d}{\exists ! g} \arrow[dashed, swap, bend left]{dr}{g_h} \\ 
	& K \arrow{r}{i_K} & \Bigprod{\substack{h : K \rightarrow I \\ h  \emph{continua}}}{I} \arrow{r}{P_h} & I  
	\end{tikzcd}
	\end{equation*}
	
	Y de aqu\'i se ve claro que $P_hi_Kf = P_hgi$, luego como $\sett{P_h}$ es inicial resulta que $i_Kf = gi$.
	
	Por otro lado, como $K$ es compacto se tiene que $i_K(K)$ es compacto en un $T_2$, por ende por \ref{Compacto en un haussdorff es cerrado} es cerrado. Entonces se tiene:

	
	\begin{equation*}
	g(\beta(X)) = g(\overline{i(X)}) \subseteq \overline{g(i(X))} \subseteq \overline{i_K(K)} = i_K(K)
	\end{equation*}
		
	Adem\'as como $K$ es compacto y $T_2$, por \ref{Si X es compacto y Haussdorf entonces es normal} es $T_5$, luego $T_4$ y entonces por \ref{Caracterizacion de T4 por insercion en copias de I} se tiene que $i_K$ es subespacio. Por lo tanto por \ref{Si f es subespacio entonces X es homeo a f(X)} se tiene que $i_K : K \rightarrow i_K(K)$ es homeo y existe $r : i_k(K) \rightarrow K$ homeomorfismo tal que $ri_K = 1_K$, luego como $g(\beta(X)) \subseteq i_K(K)$ se tiene que $rg : \beta(X) \rightarrow K$ esta bien definida  y es continua. Es m\'as el siguiente diagrama conmuta:
	
	\begin{equation*}
	\begin{tikzcd}
	X \arrow{r}{i} \arrow[swap, bend right]{dr}{f} & i(X) \arrow[hook]{r}{} & \beta(X) \arrow[dashed , swap] {dl} {\exists! rg}  \arrow[swap, dashed]{d}{g} \arrow[dashed, swap, bend left]{dr}{g_h} \\ 
	& K \arrow{r}{i_K} & \Bigprod{\substack{h : K \rightarrow I \\ h  \emph{continua}}}{I} \arrow{r}{P_h} \arrow[bend left, dashed] {l} {r} & I  
	\end{tikzcd}
	\end{equation*}	
	
	Finalmente si $g' : \beta(X) \rightarrow K$ fuese otra funci\'on que conmutase el diagrama, entonces $g'|_{i(X)} = rg|_{i(X)}$ que es denso en $\beta(X)$ y luego por \ref{Lemma para Stone-Cech1} como $K$ es $T_2$ se tiene que $g' = rg$. \qed
	
\end{proof}

\begin{proposition}{Propiedad universal de la compactificaci\'on de Stone-Cech}
	
	\label{Propiedad universal de Stone Cech}
	
	Dado un espacio topol\'ogico $X$ existe un \'unico (salvo homeomorfismo) espacio topol\'ogico $\beta(X)$ compacto	y $T_2$ y una funci\'on $i : X \rightarrow \beta(X)$ tal que cumple la siguiente propiedad universal:
	
	\begin{itemize}
		\item Para todo espacio compacto y $T_2$ $K$ y toda funci\'on continua $f : X \rightarrow K$ existe una \'unica $g : \beta(X) \rightarrow K$ tal que el siguiente diagrama conmuta:
		
		\begin{equation*}
		\begin{tikzcd}
		X \arrow{r}{i} \arrow[swap, bend right]{dr}{f} & \beta(X) \arrow[swap, dashed]{d}{\exists ! g} \\ 
		& I
		\end{tikzcd}
		\end{equation*}	
		
	\end{itemize}
	
	
\end{proposition}

\begin{proof}
	
	Ya vimos que $\beta(X) = \overline{i(X)}$ donde $i : X \rightarrow \Bigprod{\substack{h : X \rightarrow I \\ h \emph{ continua}}}{I}$ cumple la propiedad en \ref{Propiedad de la compactificacion Stone-Cech}. Supongamos que $Y,j$ tambi\'en la cumple, entonces como $Y$ es compacto y $T_2$ existe una \'unica $g : \beta(X) \rightarrow Y$ tal que el siguiente diagrama conmuta:
	
	\begin{equation*}
	\begin{tikzcd}
	X \arrow {r}{i} \arrow[swap, bend right] {dr} {j} & \beta(X) \arrow[dashed, swap] {d} {\exists ! g} \\
	& Y  
	\end{tikzcd}
	\end{equation*}
	
	Por otro lado como $\beta(X)$ es compacto y $T_2$ se tiene que existe una \'unica $h : Y \rightarrow \beta(X)$ tal que el siguiente diagrama conmuta:
	
	\begin{equation*}
	\begin{tikzcd}
	X \arrow {r}{j} \arrow[swap, bend right] {dr} {i} & Y \arrow[dashed, swap] {d} {\exists ! h} \\
	& \beta(X)  
	\end{tikzcd}
	\end{equation*}	
	
	Finalmente si consideramos el siguiente diagrama:
	
	\begin{equation*}
	\begin{tikzcd}
	X \arrow {r}{i} \arrow[swap, bend right] {drr} {i} & \beta(X) \arrow[dashed, swap, bend right] {dr}{hg} \arrow[swap, dashed, bend left]{dr}{1_{\beta(X)}}   \\
	& & \beta(X)  
	\end{tikzcd}
	\end{equation*}		
	
	Entonces $hgi = hj = i $ por lo que $hg$ conmuta el diagrama, pero $1_{\beta(X)}i = i$ por lo que por unicidad de la funci\'on se tiene que $hg = 1_{\beta(X)}$; similarmente $1_Y = gh$ por lo que $g$ es homeomorfismo y resulta que $\beta(X) \simeq Y$. \qed
	
\end{proof}

\begin{example}
	
	\begin{itemize}
		
		\item $\beta(S_{\Omega}) = \overline{S_{\Omega}}$
		
		Como $\overline{S_{\Omega}}$ es compacto y $T_2$ basta ver que $i : S_{\Omega} \rightarrow \overline{S_{\Omega}}$ tiene la propiedad universal. Sea $K$ compacto y $T_2$ y $f : S_{\Omega} \rightarrow K$ continua y veamos que podemos extender $f$, para eso veamos que existe $\alpha \in S_{\Omega}$ tal que $f|_{(\alpha , \Omega]} = cte$.
		
		Para esto afirmo que existe una sucesi\'on $a_n \in S_{\Omega}$ tal que para todo $n \in \N$ vale que:
		
		\begin{equation}
		\label{resulta auxiliar 1}
		\sup\limits_{\beta > a_n} \vert f(\beta) - f(a_n) \vert \leq 2^{-n}
		\end{equation}
		
		Supongamos que no existe dicha $a_n$ entonces existir\'ia $k \in \N$ tal que para todo $\alpha \in S_{\Omega}$ existe $\beta > \alpha$ con $|f(\beta) - f(\alpha)| \geq 2^{-k}$; luego en particular tomemos $\omega_n$ una sucesi\'on creciente tal que $|f(\omega_{n+1}) - f(\omega_n)| \geq 2^{-k}$. Por \ref{Si A es numerable en SOmega entonces tiene cota superior} sabemos que $\sett{\omega_n}$ es acotada y creciente, por lo que converge a su supremo $\omega \in S_{\Omega}$.
		
		Luego por un lado se tiene que $|f(\omega_{n+1}) - f(\omega_n)| \geq 2^{-k}$ pero por el otro como $f$ es continua y $\omega_n \rightarrow \omega$ entonces existe $n_0$ tal que si $m \geq n_0$ entonces $|f(\omega_{m+1}) - f(\omega_m)| \leq |f(\omega_{m+1}) - f(\omega)| + |f(\omega) - f(\omega_m)| \leq 2^{-k}$; luego existe $a_n$ con la propiedad mencionada. Sea $\alpha = \sup\limits_{n}{a_n} $ entonces $f|_{(\alpha, \Omega]}$ es constante por \ref{resulta auxiliar 1}.
		
		Finalicemos definiendo $g: \overline{S_{\Omega}} \rightarrow K$ dado por:
		
		\[
		g(x) = \left\lbrace
		\begin{array}{cc}
		g(x) & \emph{si } x \in S_{\Omega} \\ 
		g(\beta) & \emph{ donde } \beta \in (\alpha, \Omega] \emph{ y } x = \Omega
		\end{array} 
		\right.
		\]
		
		Luego se tiene que $gi = f$ y  $g|_{[a_0 , \beta]} = f|_{[a_0 , \beta]}$ es continua por hip\'otesis, con $a_0 = min(S_\Omega)$ y $\beta \in (\alpha, \Omega]$ fijo, y $g|_{[\beta, \Omega]}$ es constante y por lo tanto continua. Luego como $\overline{S_{\Omega}} = [a_0 , \beta] \cup [\beta, \Omega]$ y son cerrados entonces $g$ es continua por \ref{Lema del pegado}. 
		
		Como $\overline{S_{\Omega}}$ cumple la propiedad universal, por \ref{Propiedad universal de Stone Cech} se tiene que $\beta(S_{\Omega}) \simeq \overline{S_{\Omega}} = S_{\Omega} ^*$.
		
		\item Si $X$ es compacto y $T_2$ entonces trivialmente $\beta(X) = X$
		
		\item Si $X$ es indiscreto, entonces $\beta(X) = *$
		
		En efecto, sea $K$ compacto y $T_2$ y $f:X \rightarrow K$ continua, como $X$ es indiscreto entonces $f = cte$. Por ende si consideramos $j : X \rightarrow *$ la \'unica funci\'on (continua!) de $X$ en el singleton y definimos $g(*) = f(x)$ donde $x \in X$ es arbitrario, entonces se tiene que $f(x) = g(j(x))$ y adem\'as $g$ es continua pues $*$ es discreto.
		
		Luego, se concluye que $*$ cumple la propiedad universal y $\beta(X) \simeq *$.

	
	\end{itemize}
	
\end{example}

\section{Espacio de funciones y ley exponencial}

\textbf{Notaci\'on:} Dados $X,Y$ espacios topol\'ogicos notaremos $\Cont(X,Y)$ al conjunto de funciones continuas de $X$ a $Y$.

\begin{remark}
	
	Notar que $\Cont(X,Y) \subseteq Y^X = \Bigprod{x \in X}{Y}$
	
\end{remark}

\begin{definition}
	
	La \textit{topolog\'ia de convergencia puntual} en $\Cont(X,Y)$ es la topolog\'ia subespacio de $Y^X$ considerado con la topolog\'ia producto.
	
\end{definition}

\begin{remark}
	
	Si $p_x : Y^X \rightarrow Y$ es la proyecci\'on y $ev_x : \Cont(X,Y) \rightarrow Y$ la evaluaci\'on en $x$, entonces $ev_x = p_x|_{\Cont(X,Y)}$. Luego como $\sett{i : \Cont(X,Y) \rightarrow Y^X}$ es inicial y $\sett{p_x}_{x \in X}$ es inicial, entonces $\sett{ev_x = p_xi}_{x \in X}$ es inicial por \ref{Composicion de iniciales es inicial, y si la composicion es inicial entonces la interna lo es}.
	
\end{remark}

\begin{remark}
	
	Si $A,B,C$ son conjuntos entonces existe una biyecci\'on natural entre $C^{A \times B}$ y $C^{B^A}$ dado por $\phi : C^{A \times B} \rightarrow C^{B^A}$ definida por $\phi(f)(a)(b) = f(a,b)$ con inversa $\psi(g)(a,b) = g(a)(b)$. A esta biyecci\'on se la llama \textit{ley exponencial para conjuntos} y queremos replicar esta idea.
	
\end{remark}

\begin{proposition}
	
	Sean $X,Y,Z$ espacios topol\'ogicos y sea $f: Z \times X \rightarrow Y$ continua, entonces $\phi(f) : Z \rightarrow \Cont(X,Y)$ esta bien definida.
	
\end{proposition}

Sea $X \inc X \times Z \xrightarrow{f} Y$ entonces $fi(x) = f(x,z) = \phi(f)(z)(x)$ y como $i,f$ es continua entonces $\phi(f)(z)$ es continua para todo $z \in Z$. \qed

\begin{definition}
	
	\label{Definicion de topologia exponencial}
	
	Decimos que una topolog\'ia en $\Cont(X,Y)$ es \textit{exponencial} si $\phi,\psi$ restringidas inducen una biyecci\'on entre $\Cont(Z \times X,Y)$ y $\Cont(Z , \Cont(X,Y))$. Equivalentemente si vale:
	
	\begin{enumerate}
		\item \label{Condicion 1 de topologia exponencial} Si $f : Z \times X \rightarrow Y$ es continua entonces $\phi(f) : Z\rightarrow \Cont(X,Y)$ es continua.
		
		\item \label{Condicion 2 de topologia exponencial} Si $g : Z \rightarrow \Cont(	X,Y)$ es continua entonces $\psi(g) : Z \times X \rightarrow Y$ es continua.
		
	\end{enumerate}
\end{definition}

\begin{definition}
	
	Dados $X,Y$ espacios topol\'ogicos notaremos $ev : \Cont(X,Y) \times X \rightarrow$ a la \textit{evaluaci\'on} dada por $ev(f,x) = f(x)$.
	
\end{definition}

\begin{proposition}
	
	\label{Equivalencia para cumplir 2 de topologia exponencial}
	
	Una topolog\'ia en $\Cont(X,Y)$ cumple \ref{Condicion 2 de topologia exponencial} si y s\'olo si $ev$ es continua.
	
\end{proposition}

\begin{proof}
	
	Consideremos el siguiente diagrama conmutativo para una $g : Z \rightarrow \Cont(X,Y)$ continua:
	
	\begin{equation*}
	\begin{tikzcd}
	Z \times X \arrow {r}{g \times 1_X} \arrow[swap, bend right] {dr} {\psi(g)} & \Cont(X,Y) \times X \arrow[dashed, swap] {d} {ev} \\
	& Y  
	\end{tikzcd}
	\end{equation*}		
	
	Entonces para un lado es simplemente notar que si $ev,g,1_X$ son continuas entonces $ev (g\times 1_X) = \psi(g)$ es continua.
	
	Para el otro como para todo $Z$ y $g : Z \rightarrow \Cont(X,Y)$ continua se tiene que $\psi(g)$ es continua, tomemos $Z = \Cont(X,Y)$ y $g = 1_{\Cont(X,Y)}$, luego $\psi(g) = ev (1_{\Cont(X,Y)} \times 1_X) = ev$ es continua. \qed
	
\end{proof}

\begin{remark}
	
	Notemos que la topolog\'ia de convergencia puntual en $\Cont(X,Y)$ tiene como subbase $\sett{ev_{x}^{1}(U)}_{x \in X}$ pues $\sett{ev_x}_{x \in X}$ es inicial. Pero $ev_{x}^{-1}(U) = \sett{f \in \Cont(X,Y) \ / \ f(x) \in U}_{\substack{x \in X} \\  U \subseteq Y \emph{ abierto}} := S(x,U)$.
	
\end{remark}

\begin{definition}
	
	Sean $X,Y$ espacios topol\'ogicos, $K \subseteq X$ compacto y $U \subseteq Y$ abierto, denotamos $W(K,U) = \sett{f \in \Cont(X,Y) \ / \ f(K) \subseteq U}$ y la \textit{topolog\'ia compacto-abierta} es la que tiene por subbase a $\B = \sett{W(K,U)}_{\substack{K \subseteq X \emph{ compacto} \\ U \subseteq Y \emph{ abierto}}}$ y la notaremos $\tau_{CA}$
	
\end{definition}

\begin{remark}
	
	Como $\sett{x}$ es compacto entonces $S(x ,U) = W(\sett{x} , U) \in \B$ y por lo tanto la topolog\'ia compacto-abierta es m\\as fina que la de la convergencia puntual.
	
\end{remark}

\begin{proposition}
	
	Sea $\tau$ una topolog\'ia en $\Cont(X,Y)$ que satisface \ref{Condicion 2 de topologia exponencial}, entonces $\tau_{CA} \subseteq \tau$. 
	
\end{proposition}

\begin{proof}
	
	Como vale \ref{Condicion 2 de topologia exponencial} entonces por \ref{Equivalencia para cumplir 2 de topologia exponencial} $ev : \Cont(X,Y) \times X \rightarrow Y$ es continua; sea $K \subseteq X$ compacto y $U \subseteq Y$ abierto y veamos que $W(K,U) \in \tau_{CA}$.
	
	Como $ev$ es continua entonces $ev : \Cont(X,Y) \times K \rightarrow Y$ tambi\'en es continua y entonces $ev^{-1}(U) \subseteq \Cont(X,Y) \times K$ es abierto. Como $K$ es compacto $p_{\Cont(X,Y)} : \Cont(X,Y) \times K \rightarrow \Cont(X,Y)$ es cerrada por \ref{Caracterizacion de compacidad} y entonces $V = p(ev^{-1}(U)^c)^c \subseteq \Cont(X,Y)$ es abierto. Sea $f \in V$, entonces :
	
	\[
	\begin{array}{cc}
	f \in V &  \Longleftrightarrow  f \not \in p(ev^{-1}(U)^c) \\ 
	& \Longleftrightarrow (f,k) \not \in ev^{-1}(U)^c \ \forall k \in K \\ 
	& \Longleftrightarrow (f,k)  \in ev^{-1}(U) \ \forall k \in K \\ 
	& \Longleftrightarrow f(k) \in U \ \forall k \in K  \\ 
	& \Longleftrightarrow f \in W(K,U) \qed
	\end{array} 
	\] .
	
	
\end{proof}

\begin{proposition}
	
	\label{La topo CA cumple 1}
	
	La topolog\'ia compacto-abierta satisface \ref{Condicion 1 de topologia exponencial}.
	
\end{proposition}

\begin{proof}
	
	Sea $f : Z \times X \rightarrow Y$ continua y veamos que $\phi(f)^{-1}(W(K,U)) \subseteq Z$ es abierto para todo $K \subseteq X$ compacto y $U \subseteq U$ abierto.
	
	Luego $V = \phi(f)^{-1}(W(K,U)) = \sett{z \in Z \ / \ f(z,k) \in U \ / \ \forall k\in K}$ por lo que $V^c = \sett{z \in Z \ / \ \exists k\in K \ , \ f(z,k) \not \in U} = \sett{z \in Z \ / \ \exists k\in K \ , \ (z,k)\in f^{-1}(U^c)} = p_Z(f^{-1}(U^c) \cap (Z \times K)$. Pero como $U$ es abierto entonces $U^c$ es cerrado y como $f$ continua entonces $f^{-1}(U^c)$ es cerrado; por lo que $f^{-1}(U^c) \cap (Z \times K)$ es un cerrado de $Z \times K$ y como $K$ es compacto por \ref{Caracterizacion de compacidad} $p_Z(f^{-1}(U^c) \cap (Z \times K) =  V^c$ es cerrado. Conclu\'imos que $V$ es abierto y $'phi(f)$ es continua. \qed
	
\end{proof}

\begin{definition}
	
	Sea $X$ un espacio topol\'ogico decimos que todo punto de $X$ tiene una \textit{base de entornos compactos} si para todo $x \in X$ y todo entorno $U \ni x$ existe un entorno compacto $K \ni x$ tal que $K \subseteq U$.
	
\end{definition}

\begin{proposition}
	
	\label{Si X tiene una base de entornos compactos entonces la topo CA cumple 2}
	
	Sean $X,Y$ espacios topol\'ogicos tal que $X$ tiene una base de entornos compactos, entonces $\tau_{CA}$ cumple \ref{Condicion 2 de topologia exponencial}. 
	
\end{proposition}

\begin{proof}
	
	Sea $U \subseteq Y$ abierto y $(f,x) \in ev^{-1}(U)$, luego $x \in f^{-1}(U)$ y como $X$ tiene una base de entornos compactos existe $K \subseteq X$ compacto y $V \subseteq X$ abierto tal que $x \in V \subseteq K\subseteq f^{-1}(U)$. Consideremos $W(K,U) \times V$ que es un abierto de $\Cont(X,Y) \times X$ y sea $(g,y) \in W(K,U) \times V$, entonces $ev((g,y)) = g(y) \in U$ pues $y \in V \subseteq K$, luego $(f,x) \in W(K,U) \times V \subseteq ev^{-1}(U)$ y conclu\'imos que $ev$ es continua, finalizamos con \ref{Equivalencia para cumplir 2 de topologia exponencial}. \qed
	
\end{proof}

\begin{theorem}
	
	\label{La topo compacto abierta es exponencial}
	
	Sean $X,Y$ espacios topol\'ogicos tal que todo punto de $X$ tiene una base de entornos compactos, entonces $(\Cont(X,Y), \tau_{CA})$ es exponencial.
	
\end{theorem}

\begin{proof}
	
	Por \ref{La topo CA cumple 1} y \ref{Si X tiene una base de entornos compactos entonces la topo CA cumple 2} se tiene el resultado \qed
	
\end{proof}

\begin{corollary}
	
	\label{Si X es localmente compacto y T2 entonces la topo CA es exponencial}
	
	Si $X$ es localmente compacto y $T_2$, entonces  $(\Cont(X,Y), \tau_{CA})$ es exponencial.
	
\end{corollary}

\begin{proof}
	
	Sea $x \in X$ y $U \ni x$ un entorno abierto entonces por \ref{Caracterizacion localmente compacto} existe $V \subseteq X$ entorno abierto tal que $x \in V \subseteq \overline{V} \subseteq U$ con $\overline{V}$ compacto, por lo tanto tomando $K = \overline{V}$ se tiene que todo punto de $X$ tiene una base de entornos compactos, luego por \ref{La topo compacto abierta es exponencial} vale el resultado. \qed
	
\end{proof}

\begin{corollary}
	
	Sean $Z,Y$ espacios topol\'ogicos, entonces $f: Z \times I \rightarrow Y$ es continua si y s\'olo si $\phi(f) : Z \rightarrow \Cont(I,Y)$ es continua. Similarmente $g :  Z \rightarrow \Cont(I,Y)$ es continua si y s\'olo si $\psi(g) : Z \times I \rightarrow Y$ es continua.
	
\end{corollary}

\begin{proof}
	
	Como $I$ es compacto y $T_2$ vale \ref{Si X es localmente compacto y T2 entonces la topo CA es exponencial}. \qed
	
\end{proof}

\begin{proposition}
	
	$Y$ es $T_0$ si y s\'olo si $(\Cont(X,Y), \tau_{CA})$ es $T_0$
	
\end{proposition}

\begin{proof}
	
	Para un lado, sean $f \neq g \in (\Cont(X,Y), \tau_{CA})$, por lo tanto existe $x \in X$ tal que $f(y) \neq g(y)$; como $Y$ es $T_0$ entonces existe $U \subseteq Y$ abierto tal que $f(x) \in U \not \ni g(x)$; conclu\'imos que $f \in W(\sett{x}, U) \not \ni g$ y entonces $\Cont(X,Y)$ es $T_0$
	
	Para el otro sean $x \neq y \in Y$, luego $C_x \neq C_y \in \Cont(X,Y)$ y como es $T_0$ existe un compacto $K \subseteq X$ y $U \subseteq Y$ tal que $C_x \in W(K,U) \not \ni C_y$, en particular esto dice que $x \in U \not \ni y$; entonces $Y$ es $T_0$. \qed
	
\end{proof}

\begin{proposition}
	
	$Y$ es $T_1$ si y s\'olo si $(\Cont(X,Y), \tau_{CA})$ es $T_1$
	
\end{proposition}

\begin{proof}
	
	Para un lado, sean $f \neq g \in (\Cont(X,Y), \tau_{CA})$, por lo tanto existe $x \in X$ tal que $f(y) \neq g(y)$; como $Y$ es $T_1$ entonces existen $U,V \subseteq Y$ abiertos tal que $f(x) \in U \not \ni g(x)$ y $g(x) \in V \not \ni f(x)$; conclu\'imos que $f \in W(\sett{x}, U) \not \ni g$ y $g \in W(\sett{x}, V) \not \ni f$ entonces $\Cont(X,Y)$ es $T_1$
	
	Para el otro sean $x \neq y \in Y$, luego $C_x \neq C_y \in \Cont(X,Y)$ y como es $T_1$ existen dos compactos $K,H \subseteq X$ y $U,V \subseteq Y$ abiertos tal que $C_x \in W(K,U) \not \ni C_y$ y $C_y \in W(H,V) \not \ni C_x$, en particular esto dice que $x \in U \not \ni y$ y $y \in V \not \ni x$; entonces $Y$ es $T_1$. \qed
	
\end{proof}

\begin{proposition}
	
	$Y$ es $T_2$ si y s\'olo si $(\Cont(X,Y), \tau_{CA})$ es $T_2$
	
\end{proposition}

\begin{proof}
	
	Para un lado, sean $f \neq g \in (\Cont(X,Y), \tau_{CA})$, por lo tanto existe $x \in X$ tal que $f(y) \neq g(y)$; como $Y$ es $T_2$ entonces existen $U,V \subseteq Y$ abiertos disjuntos tal que $f(x) \in U$,$g(x) \in V$; conclu\'imos que $f \in W(\sett{x}, U)$ y $g \in W(\sett{x}, V)$ y adem\'as como $U \cap V = \emptyset$ necesariamente $ W(\sett{x}, U) \cap  W(\sett{x}, V) = \emptyset$, luego $\Cont(X,Y)$ es $T_2$
	
	Para el otro sean $x \neq y \in Y$, luego $C_x \neq C_y \in \Cont(X,Y)$ y como es $T_2$ existen dos compactos $K,H \subseteq X$ y $U,V \subseteq Y$ abiertos tal que $C_x \in W(K,U)$ y $C_y \in W(H,V)$ con $W(K,U) \cap W(H,V) = \emptyset$, en particular esto dice que $x \in U$,$y \in V$ y $U \cap V 0 \emptyset$; entonces $Y$ es $T_2$. \qed
	
\end{proof}

\begin{proposition}
	
	$Y$ es $T_3$ si y s\'olo si $(\Cont(X,Y), \tau_{CA})$ es $T_3$
	
\end{proposition}

\begin{proof}
	
	Para un lado, sean $f \not \in F \subseteq (\Cont(X,Y), \tau_{CA})$ con $F$ cerrado, luego $f \in F^c$ que es abierto y entonces existe $K \subseteq X$ compacto y $U \subseteq Y$ abierto tal que $f \in W(K,U) \subseteq F^c$ queremos ver que existe $V \subseteq Y$ abierto tal que se cumple que $f \in W(K,V) \subseteq W(K, \overline{V}) \subseteq W(K,U) \subseteq F^c$, donde a�n debemos ver adem\'as (pero tiene sentido) que si $\overline{V} \subseteq U$ entonces dado $K \subseteq X$ compacto vale que $ \overline{W(K,V)} \subseteq W(K,U)$.
	
	Sea $k \in K$, luego se tiene que $f(k) \in U$ entorno abierto y como $Y$ es $T_3$ entonces por \ref{Caracterizacion de T3} existe $V \subseteq Y$ abierto tal que $f(k) \in V \subseteq \overline{V} \subseteq U$. Como $k$ era arbitrario conclu\'imos que $f \in W(K,V) \subseteq \overline{W(K,V)}$. Sea ahora $f \in \overline{W(K,V)}$, luego existe $f_{\alpha}$ red en $W(K,V)$ tal que $f_{\alpha} \rightarrow f$, entonces $f_{\alpha}(k) \rightarrow f(k)$ pues $ev_k$ es continua para $\tau_{CA}$. Se sigue que $f(k) \in \overline{V} \subseteq U$ y por lo tanto como $k \in K$ era arbitrario tenemos que $f \in W(K,U)$.
	
	Juntando todo, tenemos que existe $V \subseteq Y$ abierto tal que $f \in W(K,V) \subseteq \overline{W(K,V)} \subseteq W(K,U) \subseteq F^c$, luego por \ref{Caracterizacion de T3} tenemos que $\Cont(X,Y)$ es $T_3$.
	
	Para el otro sean $x \in F^c \subseteq Y$ con $F \subseteq Y$ cerrado, luego si consideramos $H = \sett{C_y}_{y \in F}$ entonces $H^c = \sett{f \in \Cont(X,Y) / f(k) \in F^c \ , \ k \in X} = \Bigcup{k \in X}{S(k,F^c)} \supseteq S(k,F^c)$ por lo que $H$ es cerrado en $(\Cont(X,Y) , \tau_{CA})$. Como $\Cont(X,Y)$ es $T_3$ se tiene que existen $K \subseteq X$ y $V \subseteq Y$ abierto tal que $C_x \in W(K,V) \subseteq \overline{W(K,V)} \subseteq W(K,F^c)$. 
	
	Ahora sea $g \in W(K,\overline{V})$ y $k \in K$, luego existe $v^{k}_{\alpha}$ tal que $v^{k}_{\alpha} \rightarrow g(k)$, por lo tanto si consideramos $g_{\alpha} = C_{v^{k}_{\alpha}}$ tenemos que dado $k \in K$ y $g$ existe $g_{\alpha} $ red en $W(K,V)$ tal que $g_{\alpha} \rightarrow g$ y por lo tanto $g \in \overline{W(K,V)}$. Conclu\'imos que $C_x \in W(K,V) \subseteq W(K,\overline{V}) \subseteq \overline{W(K,V)} \subseteq W(K,F^c)$ y esto es equivalente a que $x \in V \subseteq \overline{V} \subseteq F^c$; conclu\'imos que $Y$ es $T_3$ por \ref{Caracterizacion de T3}. \qed
	
	
\end{proof}

\begin{remark}
	
	Veamos ejemplos de funciones continuas entre espacios de funciones:
	
	\begin{enumerate}
		\item Sean $X,Y$ espacios topol\'ogicos y $A \subseteq X$ subespacio, entonces si dotamos a $\Cont(X,Y)$ y $\Cont(A,Y)$ de la topolog\'ia compacto-abierta entonces $r_A$ dada por $r_A(f) = f|_A$ es continua.
		
		En efecto, sea $W(K,V) \subseteq \Cont(A,Y)$ un abierto subb\'asico y consideremos $r_{A}^{-1}(W(K,V)) = \sett{f \in \Cont(X,Y) \ / \ f(K) \subseteq V}$ pues $K \subseteq A$ es compacto en $A$. Pero $K \subseteq X$ es compacto tambi\'en, por lo tanto $r_{A}^{-1}(W(K,V)) = W(K,V)$ y entonces $r_A$ es continua.
		
		\item Sean $X,Y,Z$ espacios topol\'ogicos y consideremos $\circ : \Cont(Y,Z) \times \Cont(X,Y) \rightarrow \Cont(X,Z)$ con las topolog\'ias compacto-abierta, entonces $\circ$ es continua.
		
		Sea $W(K,U) \subseteq \Cont(X,Z)$ y consideremos $\circ^{-1}(W(K,U)) = \sett{f \in \Cont(Y,Z) \times \Cont(X,Y) \ / \ f(K) \subseteq U} = \sett{(f_1,f_2) \in \Cont(Y,Z) \times \Cont(X,Y) \ / \ f_1(f_2(K)) \subseteq U}$. Cor lo tanto tenemos que $f_2(K) \subseteq f_{1}^{-1}(U)$ donde $f_2(K)$ es compacto pues $f_2$ es continua y $f_{1}^{-1}(U)$ es abierto pues $f_1$ es continua; si hallamos $V \subseteq Y$ abierto tal que $f_2(K) \subseteq  V \subseteq \overline{V} \subseteq f_{1}^{-1}(U)$ con $\overline{V}$ compacto entonces tendr\'iamos que $W(\overline{V}, U) \times W(K,V) \subseteq \circ^{-1}(W(K,U))$ y entonces $\circ$ ser\'ia continua.
		
		Para esto consideremos $Y^*$ la compactificaci\'on de un punto de $Y$ que existe por \ref{Compactificacion de un punto}, entonces $H = Y^* \setminus f_{1}^{-1}(U)$ es cerrado en $Y^*$ y por \ref{Cerrado en un compacto es compacto} compacto, adem\'as es disjunto de $f_2(K)$, luego por \ref{Si X es compacto y Haussdorf entonces es normal} existen $V,G \subseteq Y^*$ abiertos tal que $f_2(K) \subseteq V$, $H \subseteq G$. Como adem\'as $\overline{V} \cap G = \emptyset$ y $\overline{V}$ es cerrado entonces $\overline{V}$ es compacto y cumple que $\overline{V} \subseteq f_{1}^{-1}(U)$. \qed
		
		
	\end{enumerate}
	
	
\end{remark}

\begin{proposition}
	
	\label{Producto de cocientes es cociente si X el loc compacto y T2}
	
	Si $p: E \rightarrow B$ es cociente y $X$ es localmente compacto y $T_2$, entonces $p \times 1_X : E \times X \rightarrow B \times X$ es cociente
	
\end{proposition}

\begin{proof}
	
	Hay que ver que $p \times 1_X$ es continua, sobreyectiva y final:
	
	\begin{itemize}
		
		\item Como $p$ es cociente es sobreyectiva y como $1_X$ es sobreyectiva se tiene que $p \times 1_X$ es sobreyectiva. 
		
		\item Como $\sett{p_B, p_X}$ es inicial para $B \times X$ entonces $p \times 1_X$ es continua si y s\'olo si $p_B (p \times 1_X) = p, p_X(p \times 1_X) = 1_X$ son continuas, pero esto es trivial por hip\'otesis.
		
		\item Para ver que es final por \ref{Caracterizacion de la topologia final} sea $Y$ un espacio topol\'ogico y $f : B \times X \rightarrow Y$  tal que $f(p \times 1_X) : E \times X \rightarrow Y$ es continua. 
		
		Como $X$ es localmente compacto y $T_2$ por \ref{Si X es localmente compacto y T2 entonces la topo CA es exponencial} se tiene que $\phi(f) : E \rightarrow \Cont(X,Y)$ es continua, pero $\phi(f) (e)(x) = f(e,x)$, luego $f$ es continua; conclu\'imos que $p \times 1_X$ es final.\qed
		
	\end{itemize}
	
\end{proof}

\newpage

\part{Teor\'ia de homotop\'ia}

Teoremas y ejercicios varios porque no llegue a pasar la carpeta:

\section{Revestimientos y levantamientos}

\begin{theorem}
	
	\label{Lema del levantamiento}
	
	Sea $p : E\rightarrow B$ un revestimiento, $X$ un espacio topol\'ogico conexo y $x_0 \in X$; entonces si $f,g : X \rightarrow E$ cumplen que $pf = pg$ y $f(x_0) = g(x_0)$ entonces $f=g$.
	
\end{theorem}

\begin{proof}
	
	Sea $A = \sett{x \in X \ / \ f(x) = g(x)} \subseteq X$, entonces $x_0 \in A \neq \emptyset$. 
	
	Sea ahora $x \in A$ y $U \ni p(f(x))$ un entorno parejamente cubierto, luego $p^{-1}(U) = \Bigcoprod{j \in J}{V_j}$ donde cada $V_j \subseteq E$ es abierto y cumple que $p|_{V_j} \rightarrow B$ es homeomorfismo. Como $f(x) \in p^{-1}(U)$ entonces existe $j_0 \in J$ tal que $f(x) = g(x) \in V_{j_0}$y consideremos $V = f^{-1}(V_{j_0}) \cap g^{-1}(V_{j_0})$, si $y \in V$ entonces $f(y),g(y) \in V_{j_0}$ pero $pf(y) = pg(y)$ y $p|_{V_{j_0}}$ es homeomorfismo, por lo tanto $f(y) = g(y)$; conclu\'imos que $x \in V \subseteq A$.
	
	Finalmente sea $x \in A^c$ y sea $U \ni pf(x)$ un entorno parejamente cubierto de $pf(x)$, luego existe $j_0$ tal que $f(x) \in V_{j_0}$ y $p|_{V_{j_0}}$ es homeomorfismo; por lo tanto como $x \not \in A$ entonces $g(x) \in V_{j_1} \neq V_{j_0}$. Consideremos $V = f^{-1}(V_{j_0}) \cap g^{-1}(V_{j_1})$ y tomemos $y \in V$, luego $f(y) \in V_{j_0}, g(y) \in V_{j_1}$ y $V_{j_0} \cap V_{j_1} = \emptyset$ por lo que $y \in V \subseteq A^c$.
	
	Como $A$ es cerrado, abierto y no vac\'io entonces $A = X$ y conclu\'imos que $f=g$. \qed
	
\end{proof}

\begin{theorem}
	
	\label{Levantamiento unico de caminos}
	
	Sea $p : E \rightarrow B$ un revestimiento, $b_0 \in B$ y $e_0 \in p^{-1}(b_0)$; luego si $\gamma : I \rightarrow B$ es un camino que empieza en $b_0$, entonces existe un \'unico $\widetilde{\gamma}:I \rightarrow E$ camino levantado que empieza en $e_0$
	
\end{theorem}

\begin{proof}
	
	Sea $\mathcal{U}$ un cubrimiento de $B$ por abiertos parejamente cubiertos, luego $\B = \sett{\gamma{-1}(U) \ / \ U \in \mathcal{U}}$ es un cubrimiento por abiertos de $I$; por lo tanto como $I$ es un espacio m\'etrico compacto admite un n\'umero de Lebesgue $\delta > 0$. 
	
	Sea $\sett{t_i}_{1 \leq i \leq n}$ una suceci\'on creciente de $I$ tal que $t_0 = 0, t_n = 1$ y $| [t_i , t_{i+1}] | < \delta$, entonces consideremos $[t_0,t_1]$ que al tener di\'ametro menor a $delta$ se tiene que $\gamma([t_0,t_1]) \subseteq U_0 \in \mathcal{U}$. Como $U_0$ esta parejamente cubierto entonces existe un \'unico $V_{j_0} \subseteq p^{-1}(U_0)$ abierto tal que $p|_{V_{j_0}}$ es homeomorfiamo y $e_0 \in V_{j_0}$, luego definimos $\widetilde{\gamma}_{0} : [t_0,t_1] : E$ dado por $\widetilde{\gamma}_0 = (p|_{V_{j_0}})^{-1} \gamma|_{[t_0,t_1]}$ que es continua.
	
	Sea ahora $b_1 = \gamma(t_1)$ y $e_1 = \widetilde{\gamma}_0 (t_1)$, luego $p(e_1) = p|_{V_{j_0}} (e_1) = p|_{V_{j_0}} (p^{-1}|_{V_{j_0}}(\gamma(t_1))) = \gamma(t_1) = b_1$ y por lo tanto $e_1 \in p^{-1}(b_1)$. Recursivamente definimos $\widetilde{\gamma}_i = (p|_{V_{j_i}})^{-1} \gamma_{[t_i,t_{i+1}]}$ donde $\gamma([t_i , t_{i+1}]) \subseteq U_i$ y $V_{j_i} \subseteq p^{-1}(U_i)$ es el \'unico elemento tal que $\widetilde{\gamma}_{i-1} (t_i) = e_i \in V_{j_i}$.
	
	Definimos $\widetilde{\gamma} : I \rightarrow E$ dado por $\widetilde{\gamma}(t) = \widetilde{\gamma}_{i}(t)$ si $t \in [t_i , t_{i+1}]$. Luego $p \widetilde{\gamma} = \gamma$ y adem\'as $\widetilde{\gamma}|_{[t_i,t_{i+1}]}$ es continua, luego por \ref{Lema del pegado} tenemos que $\widetilde{\gamma}$ es continua.
	
	Si $\gamma'$ es otro levantado de $\gamma$ que empieza en $e_0$, entonces por \ref{Lema del levantamiento} se tiene que $\gamma' = \widetilde{\gamma}$. \qed

\end{proof}


\begin{theorem}
	
	\label{Levantamiento unico de hmotopias}
	
	Sea $p : E \rightarrow B$ un revestimiento, $b_0 \in B$ y $e_0 \in p^{-1}(b_0)$; luego si $H : I \times I \rightarrow B$ es una homotop\'ia tal que $H(0,0) = b_0$, entonces existe un \'unico $\widetilde{H}:I \times I \rightarrow E$ homotop\'ia levantada  tal que $\widetilde{H}(0,0) = e_0$
	
\end{theorem}

\begin{proof}
	
	Sea $\mathcal{U}$ un cubrimiento de $B$ por abiertos parejamente cubiertos, luego $\B = \sett{H{-1}(U) \ / \ U \in \mathcal{U}}$ es un cubrimiento por abiertos de $I \times I$; por lo tanto como $I \times I$ es un espacio m\'etrico compacto admite un n\'umero de Lebesgue $\delta > 0$. 
	
	Sea $\sett{R_{i,j}}_{\substack{0 \leq i \leq n \\ 0 \leq j \leq m}}$  una partici\'on de $I \times I$  de di\'ametro menor a $\delta$, entonces consideremos $R_{0,0}$ que al tener di\'ametro menor a $delta$ se tiene que $H(R_{0,0}) \subseteq U_{0,0} \in \mathcal{U}$. Como $U_{0,0}$ esta parejamente cubierto entonces existe un \'unico $V_{0,0} \subseteq p^{-1}(U_{0,0})$ abierto tal que $p|_{V_{0,0}}$ es homeomorfismo y $e_0 \in V_{0,0}$, luego definimos $\widetilde{H}_{0,0} : R_{0,0} \rightarrow E$ dado por $\widetilde{H}_{0,0} = (p|_{V_{0,0}})^{-1} H|_{R_{0,0}}$ que es continua.
	
	Sea ahora $b_{1,0} = H(s_1,t_0)$ donde $(s_1,t_0)$ es el extremo inferior derecho de $R_{0,0}$ y $e_{1,0} = \widetilde{H}_{0,0} (s_1,t_0)$, luego $p(e_{1,0}) = p|_{V_{0,0}} (e_{1,0}) = p|_{V_{0,0}} (p^{-1}|_{V_{0,0}}(H(s_1,t_0))) = H(s_1,t_0) = b_{1,0}$ y por lo tanto $e_{1,0} \in p^{-1}(b_{1,0})$. Recursivamente definimos $\widetilde{H}_{i,j} = (p|_{V_{i,j}})^{-1} H|_{[R_{i,j}}$ donde $H(R_{i,j}) \subseteq U_{i,j}$ y $V_{i,j} \subseteq p^{-1}(U_{i,j})$ es el \'unico elemento tal que $\widetilde{H}_{i-1,j-1} (s_i,t_i) = e_{i,j} \in V_{i,j}$.
	
	Definimos $\widetilde{H} : I \times I \rightarrow E$ dado por $\widetilde{H}(t) = \widetilde{H}_{i,j}(t)$ si $t \in R_{i,j}$. Luego $p \widetilde{H} = H$ y adem\'as $\widetilde{H}|_{R_{i,j}}$ es continua, luego por \ref{Lema del pegado} tenemos que $\widetilde{H}$ es continua.
	
	Si $H'$ es otro levantado de $H$ que empieza en $e_{0,0}$, entonces por \ref{Lema del levantamiento} se tiene que $H' = \widetilde{H}$. 
	
	Finalmente para ver que $\widetilde{H}$ es una homotop\'ia de caminos usemos el siguiente lema:
	
	\begin{lemma}
		
		\label{Las fibras de un revestimiento son un subespacio discreto}
		
		Si $p:E\to B$ es un revestimiento, la fibra $E_b=p^{-1}(b)$ es un subespacio discreto de $E$ para todo $b\in B$.
		
	\end{lemma}
	
	\begin{proof}(Del lema)
		
		Sea $p:E \rightarrow B$ y $b \in B$, entonces como $p$ es revestimiento $\exists U \ni b$ abierto de $B$ tal que $p^{-1}(U) = \coprod _{i \in I}{V_i}$ con $V_i$ abiertos disjuntos y $p|_{V_i}:V_i \rightarrow B$ es homeo. 
		
		Supongamos que $\exists j_0 \in I$ tal que $\left| p^{-1}(b) \cap V_{j_0} \right| > 1$ y sean $v_1,v_2 \in V_{j_0}$ dichos elementos tal que $p(v_1)=p(v_2) = b$, pero entonces $p|_{V_{j_0}}$ no es inyectiva y por ende no es homeo! Abs! 
		
		Por ende $p^{-1}(b) \cap V_i = \{v_i\} \ \forall i \in I$, y si $E_b$ tiene la topolog\'ia subespacio entonces de la ecuaci\'on anterior se ve que es discreto. \qed
		
	\end{proof}
	
	Ahora notemos que $H(\sett{0} \times I) = b_0$ y por lo tanto $\widetilde{H}(\sett{0} \times I) \subseteq p^{-1}(b_0)$; como $(\sett{0} \times I$ es conexo, $\widetilde{H}$ es continua, y por \ref{Las fibras de un revestimiento son un subespacio discreto} entonces $\widetilde{H}(\sett{0} \times I) = e_0$ y conclu\'imos que $\widetilde{H}$ es una homotop\'ia de caminos.	\qed
	
\end{proof}

\begin{corollary}
	
	\label{Dos caminos homotopicos al levantarse por un revestimiendo siguen siendo homotopicos}
	
	Sea $p: E \rightarrow B$ un revestimiento, $b_0,b_1 \in B$ y $e_0 \in E_{b_0}, e_1 \in E_{b_1}$ y $\gamma,\omega : I \rightarrow B$ caminos de $b_0$ a $b_1$ tal que $\gamma \simeq_c \omega$; entonces se tiene que $\widetilde{\gamma} \simeq_c \widetilde{\omega}$.
	
\end{corollary}

\begin{proof}
	
	Sea $H: \gamma \simeq_c \omega$ y $\widetilde{H}$ es levantado de $H$ desde $e_0$ por \ref{Levantamiento unico de hmotopias}, luego $p \widetilde{H}(- , 0) = H(-,0) = \gamma$ y $\widetilde{H}(0,0) = e_0 = \widetilde{\gamma}(0)$; por lo tanto por \ref{Levantamiento unico de caminos} se tiene que $\widetilde{H}(-,0) = \widetilde{\gamma}$.
	
	An\'alogamente se tiene que $\widetilde{H}(0, - ) = C_{e_0} , \widetilde{H}(-,1) = \widetilde{\omega}$ y $\widetilde{H}(1,-) = C_{e_1}$; conclu\'imos que $\widetilde{H}: \widetilde{\gamma} \simeq_c \widetilde{\omega}$ \qed
	
\end{proof}


\end{document}















